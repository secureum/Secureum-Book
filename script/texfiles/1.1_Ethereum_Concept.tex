\section{Ethereum: Concept, Infrastructure and
Purpose}\label{ethereum-concept-infrastructure-and-purpose}

Ethereum is a \textbf{next generation blockchain} that supports
\textbf{smart contracts} to allow \textbf{decentralized applications} to
be built on itself. Ethereum was one of the first blockchains to put
forth this idea and enter into this concept of a \emph{next generation
smart contract based decentralized application platform}.

One of the fundamental aspects of Ethereum is the fact that it is
\textbf{Turing complete}: Ethereum supports a \textbf{Turing complete
programming language}. Turing completeness is a fundamental concept in
computer science, which refers to the \textbf{expressiveness of a
programming language}: what can you do with it, is the logic that you
can express with that language arbitrary, is it bounded, is it
unbounded\ldots{}

Many of the high level languages that you may be familiar today (like
\texttt{C}, \texttt{C++}, \texttt{Java}, \texttt{Python}, \texttt{Rust},
\texttt{Golang}\ldots) are Turing complete.

Therefore, the language supported by Ethereum is expressive enough in
arbitrary and unbounded ways. This property is very powerful and it
affects both the design and security of Ethereum, smart contracts and
the decentralized applications governed by them.

Smart contracts, given that the programming language they are written
with is Turing complete, are also Turing complete. This subsequently
means that these smart contracts, and the applications they govern, can
encode arbitrary rules over arbitrary states in such a way that said
states can be read and written using those arbitrary rules. This
contributes to what is known as a \textbf{state transition}.

Think about \textbf{finite state automatons} from computer science:

\begin{enumerate}
\def\labelenumi{\arabic{enumi}.}
\tightlist
\item
  You have a \texttt{state\ rule}
\item
  Said \texttt{state\ rule} is applied to a \texttt{state}
\item
  The \texttt{state} is read and modified (which means that it is taken
  from \texttt{state} to \texttt{state\textquotesingle{}})
\end{enumerate}

The fundamental state transition rule can be done with a Turing complete
programming language in arbitrary ways without any constraints on it.
These aforementioned rules can be of any kind: rules for ownership,
transaction formats, state transition functions\ldots{} So any state/any
rule allows Ethereum to support any application on it without any
artificial constraints coming from the programming language or the
platform.

At a high level, Ethereum is an \textbf{open source globally
decentralized computing infrastructure} that executes smart contracts.
By design, everything in the space is open source (the protocol, the
specification of the protocol itself and all the code that that actually
implements that protocol) so that everything is transparent. This has
big implications to security.

Ethereum uses a \textbf{blockchain} (namely the Ethereum blockchain) to
\textbf{store the various states from the smart contracts}, and given
that it's a blockchain, it's \textbf{decentralized}: there are many
nodes which to agree upon and synchronize the ``single view'' (global
state) that every node agrees on and works with.

So, what is the purpose of Ethereum as a platform? What is the vision
that is being worked towards? Due to the decentralization aspect
(there's not one central entity controlling the vision), a lot of these
can be thought of as narratives.

Ethereum's initial purpose, put forth in the white paper, was for it
\textbf{not to be just a currency} nor just a payment network. This
becomes clearer if you are aware of how Bitcoin works: Bitcoin is a
predecessor of Ethereum and a large source of inspiration, but
Ethereum's vision was to go beyond it being a currency or a payment
network.

There is a \textbf{native currency} in Ethereum called \textbf{Ether}
($\Xi$). Ether is divisible up to 18 decimals. The smallest unit of
ether is known as wei: $10^{18}$ wei add up to one Ether. There are
other units as well: 1 a Babbage is $10^3$ wei; 1 Lovelace is $10^6$
wei\ldots{} These names are in honour of Charles Babbage and Ada
Lovelace, which are important people that contributed a lot to computer
science.

Ether is used to measure the amount of \textbf{resources} that is being
used when smart contracts are run. This allows to constrain how long and
how many resources the smart contracts use up. It is an important
property because it ties with Turing completeness: since smart contracts
are Turing complete, the \textbf{resources and time of execution of a
smart contract must be bounded} so that it does not take over all the
resources of the network, and consecutively collapse it.

While being integral to Ethereum, Ether is not the ``\emph{be-all}'' or
``\emph{end-all}'' goal of Ethereum. The idea for Ether was for it to be
a utility token: you need the Ether token to utilize the benefits of the
Ethereum platform, so if somebody wants to use Ethereum they need to pay
using Ether. This is the high level purpose.

You have probably been reading about narratives of Ether being a store
of value in a medium of exchange, or a digital gold or a world computer
productive asset, things like that. These are all the narratives that
are being discussed in the community. \textbf{The vision of Ethereum
being a world computer is enabled by its rich infrastructure}.
