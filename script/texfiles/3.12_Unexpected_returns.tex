\section{Unexpected Returns}\label{unexpected-returns}

\subsection{\texorpdfstring{\texttt{transfer()}}{transfer()}}\label{transfer}

This security pitfall is related to the \texttt{transfer} function of
\texttt{ERC20} tokens. The \texttt{ERC20} specification says that a
\texttt{transfer} function should return a boolean value, however a
token contract might not adhere to the specification completely and may
not return a boolean value may not return any value.

This was okay until the service compiler version \texttt{0.4.22}, but
any contract compiled with a more recent \texttt{Solidity} compiler
version will \texttt{revert} in such scenarios. So the recommended best
practice for dealing with this scenario is to use the OpenZeppelin's
\texttt{SafeERC20} wrappers for such interactions.

\subsection{\texorpdfstring{\texttt{ownerOf()}}{ownerOf()}}\label{ownerof}

This pitfall is similar to the previous one and applies to the
\texttt{ownerOf()} function of the \texttt{ERC721} token standard.

The specification says that this function should return an address value
however contracts that did not adhere to this specific aspect would
return a boolean value.

It used to be okay until the \texttt{Solidity} version \texttt{0.4.22},
but with any newer compiler version returning a boolean value would
cause a revert. So the best practice again is to use the \texttt{ERC721}
contract from \texttt{OpenZeppelin}.

\subsection{Low-level Calls \& Account
Existence}\label{low-level-calls-account-existence}

Checking the return values of functions at \texttt{call} sites is a
classic software engineering best practice that's been recommended over
several decades, and in the case of \texttt{Solidity} this specifically
applies to return values of function calls made using the low level
\texttt{call} primitives.

These are the \texttt{call}, \texttt{delegateCall} and \texttt{send}
parameters that do not revert under exceptional behavior, but instead
return a \texttt{bool} indicating either success or failure.

So because of this particular characteristic it becomes critical for the
call sites in contracts that use these primitives to check the return
values and act accordingly, if not it could lead to unexpected failure.

Another related security concern is checking for the existence of a
smart contract account at a particular address before making a call.

The reason for that is because when such calls are made using low level
call primitives \texttt{call}, \texttt{delegatecall} or
\texttt{staticcall} these functions return \texttt{true} even if the
account does not exist at that address.

So if the contract making such a call looked at the return value, saw it
was \texttt{true} and assumed that the target contract existed at the
address that it called and also assumed that the contract executed
successfully, then that would be a faulty assumption.

This as you can imagine could have some serious implications to
security, so the best practice here is before making low-level calls to
external contract addresses one should check that those accounts do
indeed exist at those addresses.

\subsection{Unused Return Value}\label{unused-return-value}

There are security risks associated with function return values in
\texttt{Solidity}. Remember that in \texttt{Solidity}, functions may
take arguments, implement logic that uses those arguments along with
some local and global state, create some side effects due to all of that
logic and then may return values that reflect the impact of that logic.
For functions that return such values, the call sites are expected to
look at those values and use them in some fashion.

The reason for this is because those return values could reflect some
error codes that are indicative of some issues that happen during the
processing within that function, or they could reflect the data that is
produced as a side effect of that execution of the logic within the
function.

If these return values are not used at the call sites, then that could
be indicative of some missed error checking that needs to happen at the
call sites. Or in cases where there was data that was being returned
without any errors, not using that data at the call sites could result
in unexpected behavior.

In both these scenarios, these could affect the security of the contract
if that error checking or the missing logic (due to not using the data
returned) affected the security aspects of the smart contract logic.

The best practice here is to see if a function needs to return values
and if functions are returning values, then all their call sites should
be checking those return values and using them in appropriate ways. If
any of those call sites are not using the return values, and it does not
affect the security, then the developers (or the auditors) need to
evaluate if the functions need to return any value at all and remove the
values from being returned from those functions.
