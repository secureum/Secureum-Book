\section{Modifiers}\label{modifiers}

\subsection{Side-effects in Modifiers}\label{side-effects-in-modifiers}

Modifiers in \texttt{Solidity} smart contracts are typically used to
implement different kinds of security checks (for example access control
checks), or accounting checks on fund balances and so on. Such modifiers
should not have any side-effects, they should not be making any state
changes to the contract or external calls to other contracts.

The reason for that is any such side-effects made by the modifiers, may
go unnoticed both by the developers as well as the smart contract
security auditors evaluating the security of these contracts.

They go unnoticed not only because developers and auditors assume that
modifiers don't make side-effects, but also because the modified code is
typically declared in a different location from the function
implementation itself. Remember that the best practice is for the
modifiers to be declared in the beginning of the contract and function
implementations in the later part of the contract.

So as a security check, one should make sure that modifiers declared in
contract should not have any side-effects and they should be only
enforcing checks on different aspects of the contract.

\subsection{Incorrect Modifiers}\label{incorrect-modifiers}

Incorrect modifiers are a security risk. Modifiers should not only
implement the correct access control or accounting checks as relevant to
the smart contract logic, but they should also execute the code in
``\texttt{\_}'' or revert along all the control flow paths within that
modifier. Remember that in the context of \texttt{Solidity},
``\texttt{\_}'' inlines the function code on which the modifier is
applied.

So, if this does not happen along any particular control flow path
within the modifier, then the default value for that function is return.

This may be unexpected from the context of the caller who called this
function on which this modifier is applied, so the security check is to
make sure that all the control flow paths within the modifier either
execute ``\texttt{\_}'' or revert.
