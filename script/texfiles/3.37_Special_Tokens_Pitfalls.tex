\section{Special Tokens Pitfalls}\label{special-tokens-pitfalls}

So far we have looked at different security aspects of ERC20 tokens
let's now take a look at few other tokens that are nowhere as widely
used as ERC20 tokens, but introduce some concepts that are interesting
from a security perspective.

\subsection{ERC1400 Addresses}\label{erc1400-addresses}

One of such token standards is ERC1400. This token standard was driven
by PolyMath and was related to the concept of security tokens (tokens
that represent ownership in a financial security, and note that the
security has nothing to do with the program or application security we
are talking about).

This token standard introduced the notion of permissioned addresses,
which could block transfers from certain addresses. This is interesting
from a security perspective because, if those addresses are malicious or
if they can be compromised, then it leads to a denial of service (DoS)
risk where transfers to and from such addresses can be blocked. This is
a risk that we need to keep in mind if our smart contract application
ever has to deal with ERC1400 tokens.

\subsection{ERC1400 Transfers}\label{erc1400-transfers}

Related to the notion of permissioned addresses, ERC1400 also introduced
the concept of forced transfers where there are trusted actors within
the context of the standard that can perform unbounded transfers. These
trusted actors can transfer arbitrary amounts of funds to whichever
addresses that they choose. This introduces a transfer risk that needs
to be kept in mind when dealing with such tokens.

\subsection{ERC1644 Transfers}\label{erc1644-transfers}

A related token standard to ERC1400 is ERC1644 that allows the concept
of forced transfers that we just discussed. This is again in the context
of a controller role, which is a trusted actor in this standard that is
allowed to perform arbitrary transfers of funds to arbitrary addresses.
The trusted actor, if malicious or compromised, can steal funds. In this
ERC standard, there is a risk from the controller address that needs to
be kept in mind.

\subsection{\texorpdfstring{ERC621
\texttt{totalSupply()}}{ERC621 totalSupply()}}\label{erc621-totalsupply}

ERC621 token standard allows a different way to control the total supply
of tokens. In this standard, there is a notion of trusted actors who can
change the total supply after the contract is deployed. This is allowed
using the \texttt{increaseSupply()} and \texttt{decreaseSupply()}
functions that are specified by the standard. This introduces what is
known as a token supply risk, where the token supply of such tokens can
be changed arbitrarily after the contract is deployed.

\subsection{ERC884 Reissue}\label{erc884-reissue}

ERC884 is another token standard that introduces yet another interesting
security aspect. In this case, this token standard introduces the notion
of cancelling and re-issuing. What this means is that the standard
defines actors known as token implementers who can cancel addresses in
the context of a contract that implements the standard.

In that process, what these implementers do is that they move any tokens
owned or held by those addresses to a new address while cancelling the
older address. This, from a user's perspective, introduces token holding
risk because if you are holding certain number of tokens in a particular
address, then the token implementers could move those to a new address
and cancel your existing address.

\subsection{ERC884 Whitelisting}\label{erc884-whitelisting}

ERC884 also introduces the concept of whitelisting addresses, where a
certain set of addresses may be whitelisted by a contract implementing
the standard. Token transfers are allowed only to such whitelisted
addresses and not to addresses that don't exist in this whitelist. This
again, as you can imagine, is a token transfer risk because a user might
want to transfer tokens to a particular address but, if that is not
whitelisted, then that token transfer is not allowed
