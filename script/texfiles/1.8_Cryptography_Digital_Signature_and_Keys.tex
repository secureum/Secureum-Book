\section{Cryptography, Digital Signature and
Keys}\label{cryptography-digital-signature-and-keys}

Most of you know tat there are two classes of cryptography:

\begin{itemize}
\tightlist
\item
  \textbf{symmetric cryptography}: there is a single key shared between
  parties.
\item
  \textbf{asymmetric cryptography}: there is a key pair; public key and
  private key.
\end{itemize}

In the case of Ethereum, the cryptography that is used is all about
digital signatures and not as much about encryption at a protocol level.
These digital signatures however depend on the concept of public key and
private key.

\subsection{Private Key}\label{private-key}

The private key is a secret and the owner has to keep it in a safe
place. In the case of Ethereum, it's a 256 bit private key. It's
effectively a random number and it's used to derive the public key.

\subsection{Public Key}\label{public-key}

The public key, however, is not secret. It is a point on the elliptic
curve calculated from the private key using elliptic curve
multiplication. The public key is used then to derive the address of an
Ethereum account (by hashing the public key by means of the keccak-256
cryptographic hash function and taking the last 20 bytes of the output;
it is a very simple calculation) and it is also used by others to engage
in cryptographic protocols with the owner of the private key.

It is important to remember that the public key cannot be used to derive
the private key. This is should be something obvious to security,
because otherwise if the public key could be used to derive the private
key, then this key pair system would not deliver any kind of security.

This is the high-level aspect that you need to remember: there's a
private key, which is used to obtain the public key, and from the public
key we derive the address of the Ethereum account.

\subsection{keccak-256}\label{keccak-256}

We mentioned earlier that the keccak-256 cryptographic hash function is
used in the steps of computing the EOA address from the public key.

keccak-256 is actually the cryptographic hash function that is used by
Ethereum. It is very closely related to the SHA3 (the secure hash
function). The latter was finalized as the standard by MIST (National
Institute of Standards and Technology) and in the case of Keccak-256, it
was the winning candidate for the SHA3. However, the SHA3 standard was
adopted instead (because some minor modifications were applied).

keccak-256 is critical to a lot of the functioning of the Ethereum
protocol and smart contracts as it's a fundamental primitive to how
computation in many ways is done on Ethereum.

\subsection{Digital Signature: ECDSA}\label{digital-signature-ecdsa}

The digital signature algorithm used by Ethereum is the same one that is
used by Bitcoin. It is known as ECDSA: \textbf{Elliptic Curve Digital
Signature Algorithm}.

Elliptic Curve Cryptography is an approach to public key cryptography
based on a particular algebraic structure of elliptic curves over finite
fields.

In the case of Ethereum, the particular elliptic curve used is known as
Secp-256k1 (this refers to the parameters that are used for the elliptic
curve).

Digital signatures are fundamental to how Ethereum works, are powered by
public key cryptography (asymmetric cryptography) and have three main
purposes:

\begin{enumerate}
\def\labelenumi{\arabic{enumi}.}
\tightlist
\item
  \textbf{Authorization}: inclusion of the signature proves that the
  owner of the private key who created the signature (and who by
  implication is the owner of the sending Ethereum account) has
  authorized the transaction to spend the ether or to execute the
  contract that it is targeted.
\item
  \textbf{Non-repudiation}: once the signature has been included, it
  cannot be later denied that authorization was granted for that
  transaction to execute.
\item
  \textbf{Integrity}: it proves that the transaction data has not been
  modified or cannot be modified by anyone after the transaction has
  been signed. This is one of the fundamental security properties.
\end{enumerate}
