\section{Ethereum State \& Account
Types}\label{ethereum-state-account-types}

Ethereum state is a mapping from the address of the Ethereum accounts to
the state contained within it as a data structure. It is implemented as
a \textbf{Modified Merkle-Patricia Tree}: a combination of a Merkle tree
and a Patricia tree with some changes that are specific to Ethereum.

Each of the Ethereum accounts has a unique 20 byte address associated
with it, which is used by accounts to ``talk to each other''. Addresses
are critical to how messaging works within the Ethereum protocol and how
the accounts engage in transfer of value or information, since accounts
need to be able to refer to each other using their addresses.

In addition, accounts have four fields:

\begin{itemize}
\tightlist
\item
  \texttt{nonce}: a counter that's used to make sure that each
  transaction can only be processed once used to prevent replay attacks.
\item
  \texttt{balance}: a number representing the amount of Ether that the
  account has at any point in time.
\item
  \texttt{code}: the smart contract code (absent in Externally Owned
  Accounts).
\item
  \texttt{storage}: the associated smart contract storage (absent in
  Externally Owned Accounts).
\end{itemize}

\subsection{Account types}\label{account-types}

Ethereum has two account types:

\begin{itemize}
\item
  \textbf{Externally Owned Account (EOA)}: it is an account that is
  controlled by a private key.\\

  Anyone who has a private key can create a digital signature that can
  be used to control the Ether that is present in an EOA. These
  signatures can be used to sign transactions from the EOA, which in
  turn can trigger messages from the EOA to other accounts.\\

  These messages can result in a transfer of value or they can trigger
  smart contracts. An EOA does not have any associated code or storage.
\item
  \textbf{Contract account}: it is an account that is controlled by the
  code that is contained within that account.\\

  Unlike EOAs, contract accounts have an associated smart contract code
  and storage. Whenever the contract account receives a message, it
  triggers the code present and accesses any internal storage associated
  with it. When the code runs it can send messages to other accounts or
  even create new contracts.
\end{itemize}

In this sense, smart contracts can be thought of autonomous agents as
they're always present in the execution environment of the Ethereum
blockchain. They're always ready to be triggered by a transaction or a
message that is sent to them.

Through their contract account they have access to the Ether balance and
the contract storage. The execution of the code results in manipulation
of this balance and the contract storage.
