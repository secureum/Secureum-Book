\section{Typographical Errors}\label{typographical-errors}

\subsection{Unary Expressions}\label{unary-expressions}

Unary expressions are where an operator is used on a single operand as
opposed to two operands, in which case it would be a binary expression.
Such unary expressions are susceptible to typographical errors by
developers.

For example let's take a look at the scenario where there's a variable
\texttt{x}. The developer wants to increment it by one, so the way to do
that is to say \texttt{x\ +=\ 1} which effectively is
\texttt{x\ =\ x\ \ +\ 1}. But if the developer interchanges the order of
\texttt{+} and \texttt{=} and instead uses \texttt{x\ =\ +1}, then this
would result in re-initializing the value of \texttt{x} to \texttt{+1}.
The reason for this is that \texttt{+\ 1} is a unary expression whose
value is \texttt{1} and \texttt{x} would get initialized to that value.

As you can imagine such typographical errors are likely to be made by
developers it's very easy to make these switching the order and, if they
are considered as valid by the compiler, then it's very hard to notice
such errors both by the developer as well as by the security auditor.

So in order to prevent some of the most common usages that result in
such typographical errors the unary \texttt{+} was deprecated as of
compiler \texttt{Solidity} version \texttt{0.5.0}.

\subsection{Long Number Literals}\label{long-number-literals}

There is a security risk in the use of long number literals within
\texttt{Solidity} contracts. These number literals may require many
digits to represent their high values (as constants) or many decimal
digits of precision, which as you can imagine is error prone.

For example, if one were to define a variable representing Ether, then
it would need to be assigned a number literal that has 18 zeros to
represent the 18 decimals of precision.

So the developer may accidentally use an extra zero or miss a zero in
which case the Ether precision is different, thus the logic using this
variable will be broken.

The best practice here is to use the Ether or time suffixes supported by
\texttt{Solidity} as applicable or to use the Scientific Notation which
is also supported by \texttt{Solidity}.
