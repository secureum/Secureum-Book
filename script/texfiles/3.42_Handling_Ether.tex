\section{Handling Ether}\label{handling-ether}

Let's now talk about another fundamental aspect of smart contracts and
Ethereum which is the way they handle Ether. Contracts that accept,
manage or transfer Ether should take care of several things.

\begin{itemize}
\tightlist
\item
  They should ensure that functions handling Ether are using
  \texttt{msg.value} appropriately, remember that \texttt{msg.value} is
  a global variable in the context of a transaction which, for example
  when used or accounted multiple times (say inside loops) have led to
  critical vulnerabilities.
\item
  They should ensure that logic that depends on Ether value accounts for
  either less or more Ether set via \texttt{payable} functions.
\item
  Logic that depends on contract Ether balance, accounts for the
  different direct or indirect ways of receiving Ether such as
  \texttt{coinbase} transaction or \texttt{selfDestruct} recipient that
  we have discussed earlier.
\item
  Logic that handles withdrawal balance and transfers does so correctly
  in any accounting logic.
\item
  Transfers should be reentrancy safe.
\item
  Ether can't accidentally get locked within a contract.
\end{itemize}

Functions handling Ether should also be checked extra carefully for
access control input validation and error handling all these various
aspects of Ether handling should be reviewed for correctness.
