\section{Criticals}\label{criticals}

\subsection{ConsenSys Diligence}\label{consensys-diligence}

\subsubsection{1inch}\label{inch}

This finding was a ConsenSys Diligence audit of 1inch where it was a
critical severity finding related to access control and input data
validation in which anyone could steal all the funds that belong to the
referral fee receiver.

For context, any token or ETH that belonged to the \texttt{feeReceiver}
was at risk and could be drained by any user by providing a custom
\texttt{UniswapPool} contract that referenced existing token holdings,
because none of the functions in the \texttt{feeReciever} verified that
the user provided \texttt{UniswapPool} address was actually deployed by
the linked \texttt{UniswapFactory}.

The recommendations were:

\begin{enumerate}
\def\labelenumi{\arabic{enumi}.}
\tightlist
\item
  To enforce that the user provided \texttt{Uniswap} contract was
  actually deployed by the linked factory because other contracts can't
  be trusted.
\item
  To consider implementing token sorting and deduplication in the pool
  contract constructor as well.
\item
  To consider employing a re-entrancy guard to safeguard the contract
  from reentrancy attacks.
\item
  To improve testing because the vulnerable functions were not covered
  at all.
\item
  To improve documentation and provide a specification that outlined how
  this contract was supposed to be used.
\end{enumerate}

This is related to system specification and documentation in 136, 137
access control specification and implementation in 148, 149 and broader
aspects of testing in 155 data validation issues in 169 and access
control issues in 172 that we discussed in security pitfalls and best
practices 201 model.

\subsubsection{DeFi Saver}\label{defi-saver}

This finding was a ConsenSys Diligence audit of the DeFi Saver protocol.
It was a critical vulnerability of the reentrancy type which allowed for
a random task execution in the context of the protocol. Specifically, in
a scenario where a user took a flash loan, one of the functions gave the
flash loan wrapper contract permission to execute functions on behalf of
the users \texttt{DSProxy}.

This permission was revoked only after the entire recipe execution
finished, which meant that in a case that any of the external calls
along the recipe execution was malicious, it could perform a reentrancy
attack by injecting any task of choice leading to users funds being
transferred out or draining approved tokens.

This vulnerability was due to potential reentrancies from malicious
external calls and therefore the recommendation was to add a reentrancy
guard, such as the one from OpenZeppelin, where the
\texttt{NonReentrant} modifiers are used on functions that may be
vulnerable to reentrancies. We have discussed these aspects in
\href{../2.Solidity/2.28_Open_Zeppelin_Libraries.md\#ReentrancyGuard}{OpenZeppelin
Libraries for Security} and
\href{../3.Security_Pitfalls_and_Best_Practices/3.6_Reentrancy.md}{the
Reentrancy Security pitfall}.

\subsubsection{DAOFI}\label{daofi}

This finding was a ConsenSys Diligence audit of the DAOFI protocol where
it was a critical severity finding in the input validation category. The
finding here was that token approvals can be stolen in the
\texttt{addLiquidity} function of the protocol where the function
created the desired contract, if it did not already exist, then
transferred tokens into the pair. However, there was no validation of
the address to transfer tokens from and so, an attacker could have
passed in any address with non-zero token approvals to the DAOFI V1
route. This could have been used to add liquidity to a pair contract for
which the attacker was the pair Owner allowing the stolen funds to be
retrieved using the \texttt{withdrawal} function.

The recommendation was to transfer tokens from \texttt{msg.sender}
instead of \texttt{lp.sender}. We have discussed the importance of
access control checks on correct addresses in number 148, 149, 160, 172,
180, 181 and 183 of security pitfalls and best practices 201 module, and
also the importance of input validation specifically on function
parameters tokens and addresses in 138 and 159 of security pitfalls and
best practices 201 module.

\subsubsection{Fei}\label{fei}

\begin{itemize}
\item
  This finding was a ConsenSys Diligence audit of the Fei protocol where
  it was a critical severity finding in the application logic where the
  \texttt{GenesisGroup.commit} function overrode previously committed
  values. The amount stored in the recipient's \texttt{committedFGEN}
  balance overrode any previously committed value, including allowing
  anyone to commit an amount of zero to any account, deleting their
  commitment entirely.\\

  The recommendation was to ensure that the committed amount is added to
  the existing commitment instead of overwriting it. This finding is
  related to the numerical and accounting issues we discussed in number
  170 and 171 or security pitfalls and best practices 201 module and
  also the general challenges of detecting application specific business
  logic issues in number 191 of that same module.
\item
  Another critical severity finding from ConsenSys Diligence audit of
  the Fei protocol was related to the timing category. Here, purchasing
  and committing was still possible after launch, which meant that even
  after the \texttt{GenesisGroup.launch} had successfully been executed,
  it was still possible to invoke \texttt{GenesisGroup.purchase} and
  commit functions.\\

  The recommendation was to consider adding validation by ensuring that
  these functions could not be called after launch. This finding is
  related to the ordering issues we discussed in number 145 and 178 and
  also the timing issues discussed in 143 and 177 of the security
  pitfalls and best practices 201 module.
\end{itemize}

\subsubsection{Bancor V2}\label{bancor-v2}

This finding was a ConsenSys Diligence audit of Bancor V2 protocol where
it was a critical severity finding related to the typing category. This
issue was about Oracle updates that could be manipulated to arbitrage
rate changes by sandwiching the Oracle update between two transactions.
The attacker could send two transactions at the moment the Oracle update
appeared in the mempool. The first transaction sent with a higher Gas
Price than the Oracle update transaction, so as to front run it would
convert a very small amount to lock in the conversion rate, so that the
stale Oracle price would be used in the following transaction. The
second transaction sent at a slightly lower Gas Price than the
transactions that updated the Oracle, so as to back run it and
effectively sandwich it would perform a large conversion at the old
scale weight, add a small amount of liquidity to trigger rebalancing and
then convert back at the new rate. The attacker could obtain liquidity
for step two using a flash loan and use that to deplete the reserves.

The recommendation was to not allow users to trade at a stale Oracle
rate and trigger an Oracle update in the same transaction. This finding
is related to the transaction order dependence aspect discussed in
number 21 of security pitfalls and best practices 101 module, ordering
aspect discussed in number 178 and freshness aspect discussed in number
185 of the security pitfalls and best practices 201.

\subsubsection{Lien}\label{lien}

This finding was a ConsenSys Diligence audit of Lien protocol where it
was a critical severity finding related to denial of service, where a
reverting fallback function would lock up all payouts in the context of
the \texttt{transferEth} function. If any of the Ether recipients of
such batch transfers were to be a smart contract that reverted, then the
entire payout would fail and be unrecoverable.

The recommendation was to implement a pull-withdrawal pattern or ignore
a failed transfer leaving the responsibility then up to the recipients.
We have discussed denial of service in number 176 and concerns with
Ether handling functions in number 158 of security pitfalls and best
practices 201 module. We have discussed concerns with calls within loops
leading to denial of service In number 43 oF security pitfalls and best
practices 101 module. We've also reviewed OpenZeppelin's PullPayment
library which specifically addresses this pull versus push aspect of
Ether transfers in number 158 of \texttt{Solidity} 201 module.

\subsubsection{LAO protocol}\label{lao-protocol}

\begin{itemize}
\item
  This finding was a ConsenSys Diligence audit of LAO protocol where it
  was a critical severity finding related to denial of service. The
  issue was related to \texttt{safeRagequit()} and \texttt{ragequit()}
  functions used for withdrawing funds from the LAO.\\

  The difference between them was that while \texttt{ragequit()} tried
  to withdraw all the allowed tokens, \texttt{safeRagequit()} only
  withdrew some subset of those tokens as defined by the user. The
  problem was that, even though one could quit, they would lose the
  remaining tokens. The tokens were not completely lost, but they would
  belong to the LAO and could potentially still be transferred to the
  user who quit. However, that required a lot of trust, coordination and
  time, and anyone could steal some of those tokens.\\

  The recommendation was to implement a pull-pattern for token
  withdrawals. We have discussed denial of service in numbers 176 of
  security pitfalls and best practices 201 module.
\item
  Another critical severity finding from ConsenSys Diligence audit of
  the LAO protocol was again related to denial of service. The issue was
  that, if someone submitted a proposal and transferred some amount of
  tribute tokens, these tokens were transferred back if the proposal was
  rejected. But if the proposal was not processed before the emergency
  processing, these tokens would not be transferred back.\\

  The proposal tokens were not completely lost, but belong to the LAO
  shareholders who may try to return that money back, but that required
  a lot of coordination and time, and everyone who \texttt{ragequit}
  during that time would take a part of those tokens.\\

  The recommendation again was to use a pull pattern for token transfers
  this is again related to the derivatives of this aspect we discussed
  in number 176 of security pitfalls and best practices 201 module.
\item
  Yet another critical severity finding from ConsenSys Diligence audit
  of the LAO protocol was again related to denial of service. The
  specific issue here was that emergency processing could be blocked.\\

  The rationale for emergency processing mechanism was that there was a
  chance that some token transfers may be blocked, and in such a
  scenario emergency processing would help by not transferring tribute
  tokens back to the user and rejecting the proposal.\\

  The problem was that there was still a deposit transferred back to the
  sponsor that could potentially be blocked too. So if that were to
  happen, the proposal couldn't be processed and the allowed will be
  blocked.\\

  The recommendation again was to use a pull-pattern for token
  transfers, this is again related to the denial of service aspect we
  discussed in number 176 of security pitfalls and best practices 201
  module.
\end{itemize}

\subsection{Sigma Prime}\label{sigma-prime}

\subsubsection{Infinigold}\label{infinigold}

This finding was a Sigma Prime audit of Infinigold where it was a
critical severity finding related to configuration, in which there was
an incorrect Proxy implementation that prevented contract upgrades.

The token implementation contract initialized order, name, symbol and
decimal state variables in a constructor instead of an initialize
function. Therefore, when token Proxy made a \texttt{delegateCall} to
token implementation, it would not be able to access any of the state
variables of the token implementation contract.

Instead, the token Proxy would access its local storage which would not
contain the variables set in the constructor of the token implementation
contract and so, the Proxy call to the implementation was made.

Variables such as order would be uninitialized and effectively sent to
their default values without access to the implementation state
variables. The Proxy contract was rendered unusable.

The recommendation was:

\begin{enumerate}
\def\labelenumi{\arabic{enumi}.}
\tightlist
\item
  To set fixed constant parameters as constants because then, the Proxy
  contract wouldn't need to initialize anything.
\item
  implement a standard Proxy implementation which uses an initialize
  function instead of a constructor and a few other recommendations as
  well.
\end{enumerate}

This is related to OpenZeppelin's OZ Initializable library in number 192
and other Proxy related aspects we discussed in \texttt{Solidity} 201
module, the aspect of initializing state variables in Proxy-based
upgradable contracts in number 96 of security pitfalls and best
practices 101 module along with the broader aspects of configuration in
165 and initialization in 166 that we discussed in security pitfalls and
best practices 201 module.

\subsubsection{Synthetix's Unipool}\label{synthetixs-unipool}

This finding was a Sigma Prime audit of Synthetix's Unipool where it was
a critical severity finding related to ordering, in which the wrong
order of operations led to exponentiation of reward per token stored
value because reward per token stored was mistakenly used in the
numerator of a fraction instead of being added to the function.

This would allow users to withdraw more funds than allocated to them or
being unable to withdraw their funds at all because of insufficient SNX
balance.

The recommendation was to fix the operand ordering in the expression. As
expected, this is related to numerical issues of 170 and accounting
issues of 171 that we discussed in the security pitfalls and best
practices 201 modules.

\subsection{OpenZeppelin}\label{openzeppelin}

\subsubsection{MCDEX Mai Protocol}\label{mcdex-mai-protocol}

\begin{itemize}
\item
  This finding was a OpenZeppelin audit of MCDEX Mai protocol where it
  was a critical severity finding related to access control, in which
  anyone could liquidate on behalf of another account. For context, the
  perpetual contract had a public \texttt{liquidateFrom()} function that
  bypassed the checks in the \texttt{liquidate()} function, which meant
  that it could be called to liquidate a position and with any user
  being able to set an arbitrary \texttt{from} address would cause a
  third party to confiscate an undercollateralized trader's position. So
  effectively, this meant that any trader could unilaterally rearrange
  another account's position and also liquidate on behalf of the
  perpetual Proxy which could break down the automated market maker
  invariants.\\

  The recommendation was to consider restricting \texttt{liquidateFrom}
  to \texttt{internal} visibility from \texttt{public} visibility. This
  is related to aspects of function visibility specifiers in number 23
  of \texttt{Solidity} 101 in current access control and number four of
  security pitfalls and best practices 101 module and aspects of
  function visibility in 140 along with broader aspects of access
  control in 148 149 and 172 and trust issues in 181 that we discussed
  in security pitfalls and best practices 201 modules.
\item
  Another critical severity finding from OpenZeppelin audit of MCDEX Mai
  protocol was again related to denial of service, in which orders could
  not be cancelled. For context, when a user or broker called
  \texttt{cancelOrder()}, the \texttt{cancel} mapping was updated but
  that had no subsequent effects because \texttt{validateOrderParam} did
  not check if the order had been cancelled.\\

  The recommendation was to consider adding that check to order
  validation to ensure that cancelled orders would not be filled. This
  is related to broader aspects of data validation in 169 and denial of
  service issues in 176 that we discussed in security pitfalls and test
  practices 201 module.
\end{itemize}
