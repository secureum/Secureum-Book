\section{PRNG and Time}\label{prng-and-time}

\subsection{PRNG}\label{prng}

This security pitfall is related to pseudo-random number generation on
the blockchain within smart contracts applications that require such
random numbers.

Remember that these values could be influenced to a certain extent by
miners who are mining the blocks that contain these values. So if the
stakes in those applications using these as sources of randomness is
high, then such actors could use their influence to a certain extent to
gain advantage.

So this is a risk from randomness that needs to be paid attention to
something to be aware of and, if the stakes are high for the
applications where you desire a much better source of randomness then,
there are some alternatives such as the verifiable random function
provided by Chainlink.

\subsection{Time}\label{time}

Similar to randomness, the notion of getting the time on-chain is also
tricky. Often smart contracts resort to using \texttt{block.timestamp}
or \texttt{block.number} as sources for inferring the time within the
application's logic.

Again, what needs to be paid attention to is that this notion of time
can be influenced to a certain extent by the miners. There are issues
with synchronization across the different blockchain nodes and there are
also aspects of the block times that change by a certain degree over
time.

This is again a risk that needs to be paid attention to and, there are
some alternatives to this using the concept of Oracles.
