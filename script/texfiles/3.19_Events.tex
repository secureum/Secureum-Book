\section{Events}\label{events}

Events should be emitted within smart contracts for all critical
operations. Emission of events that are missing for such critical
operations is a security concern.

The reason for this is because it affects off-chain monitoring remember
that events emitted from smart contracts end up storing the parameters
of such events in the log part of the blockchain.

These logs either the topics part or the data part can be queried by
off-chain monitoring tools or off-chain interfaces to understand what is
happening in the smart contracts. This is an easier way to understand
the state of the smart contracts without having to query the contracts
themselves.

These events become very important from a transparency and user
experience perspective. So the best practice is to recommend the
addition of events in all places within the smart contracts where
critical operations are happening, these could be updates to critical
parameters from the smart contract applications perspective this could
be operations that are being done only by the owner or privileged roles
within the smart contract. So in all such cases events should be emitted
to allow transparency and a better user experience.

\subsection{Event Parameters}\label{event-parameters}

Having talked about events, let's now focus on the event parameters.
Event parameters not being indexed may be a concern in certain
situations.

Remember that event parameters may be considered as either indexed or
not depending on the use of the \texttt{indexed} keyword. This results
in those parameters being stored either in the topics part of the log or
the data part of the log. Being stored in the topics part of the log
allows for those parameters to be accessed or queried faster due to the
use of the bloom filter. If they're stored in the data part, then it
results in a much slower access.

There are certain parameters for certain events that are required to be
indexed as per specifications. let's take the \texttt{ERC20} token
standard for example: it has transfer and approval events that require
some of their parameters to be indexed.

Not doing it will result in the off-chain tools that are looking for
such index events to be confused or thrown off track.

So the best practice here is to add the \texttt{indexed} keyword to
critical parameters in an event. Especially if the specification
requires them to be in text, this comes at cost of some additional Gas
usage, but allows for faster query.

\subsection{Event Signatures}\label{event-signatures}

The concern here was that of incorrect event signature in libraries. The
reason for this happening was because, if events used in libraries had
parameters of contract types, then because of a compiler bug, the actual
contract name was used to generate the signature hash instead of using
their address type.

This resulted in a wrong hash for such events being used in the logs.
The mitigation here was to fix the compiler bug which happened in
version \texttt{0.5.8} where the address type was used instead of using
the contract name incorrectly.
