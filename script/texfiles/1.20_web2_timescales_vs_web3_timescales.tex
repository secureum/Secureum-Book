\section{web2 Timescales vs.~web3
timescales}\label{web2-timescales-vs.-web3-timescales}

The timescale of innovation in web2, although it is seemingly fast
(exponential in some ways: smartphones in just 15 years, PCs, Moore's
law\ldots), those timescales are really long when you compare that to
the compressed timescales of innovation that happen in the web3 world
and specifically Ethereum, which again is driven by a lot of these
interrelated concepts we talked about: everything being open source by
design, composable, permissionless and borderless. Plus combine that
with the mechanism design where a lot of this is incentivized by
tokenomics.

As a side effect, unfortunately, security has in some sense taken a back
seat: it hasn't been really thought of as much as it should be in the
design and development of a lot of these smart contracts (and hence, the
applications they support). This was what contributed to a lot of the
vulnerabilities we have seen within smart contracts or web3
applications, which led to exploits causing losses of millions or tens
of millions of dollars overnight in a fraction of a second within a few
transactions.

And remember this is all irreversible: all these aspects of pseudonymous
teams in some cases, the presenting threat model, the use of keys, the
use of tokens, the lack of any centralized third party that can reverse
the negative side effects of some of these exploits\ldots{} All these
interrelated concepts affect the security aspects of Ethereum and web3
in general.
