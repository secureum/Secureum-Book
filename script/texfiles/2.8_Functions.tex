\section{Functions}\label{functions}

Functions are the executable units of code. In the case of
\texttt{Solidity}, they are usually defined inside a smart contract, but
they can also be defined outside of the contracts in which case they are
specified at a file level. Such functions are referred to as
``\emph{free functions}''.

Functions are what allow modifications to the state that is encapsulated
as part of the contract, so they are how logic manifests itself within
the smart contracts and the state transitions from one initial state to
the modified state, as a result of any of the transactions or messages
that interact with the smart contract.

\subsection{Parameters}\label{parameters}

Functions typically specify parameters. These are declared just like
variables within the function. Parameters are how the caller of the
function sends in data into the function for it to work on.

Parameters are used and assigned in a very similar manner to local
variables within the function, and the nomenclature that the function
specifies the parameter and the caller sends in arguments that get
assigned to these parameters in the context of the function.

\subsection{Return Variables}\label{return-variables}

Functions typically also return values. These are returned using the
\texttt{return} keyword. \texttt{Solidity} functions can return single
variables or they can return multiple variables. The return variables
can also be of 2 types:

\begin{itemize}
\tightlist
\item
  \textbf{Named return variables}: they have a specific name or names.
  They are treated just like local variables within the context of the
  function.
\item
  \textbf{Unnamed return variables}: an explicit return statement needs
  to be used to return that variable a return value to the context of
  the function caller.
\end{itemize}

The caller specifies arguments that get assigned to the respective
parameters of the callee function. The caller function works with these
parameters (in the context of that function), does something with them
along with all the local variables that might be defined within that
function (it can also use the state variables that are declared within
that contract) and once it is done with that logic, it can return values
back to the caller.

\subsection{Modifiers}\label{modifiers}

Function modifiers are something unique and specific to
\texttt{Solidity}. They are declared using the \texttt{modifier} keyword
and the format is something like this

\begin{lstlisting}[language=Solidity,numbers=none]
modifier mod() {
    Checks;
    _;
}
\end{lstlisting}

As you can see, they are very similar to a function where, because
modifiers have some logic encapsulated within them. The underscore acts
as a placeholder for the function that we're attempting to modify;
because \textbf{modifiers are used along with functions}.

So in this case if there is a function \texttt{foo()} on which this
modifier is applied, then whenever this function is called, it goes
first to the modifier and depending on any of the checks (any of the
logic implemented within that modifier), the function's logic gets
called at the point where the underscore is placed within that modifier.

So, if there are a bunch of checks in the modifier prior to the
underscore, then those checks implement some preconditions that are
evaluated before the function's logic is executed. Similarly, if the
underscore precedes the checks in the modifier, the function's logic
gets executed first and then the modifier executes its checks.

Examples for the usage here could be access control checks that are
implemented as preconditions on the function in the modifier, and they
could be post-conditions that could be evaluated if the underscore
happens to be before the checks in the modifier, and these could
implement some sort of accounting checks in the context of the contract.

Function modifiers play a critical role because they're very often used
to implement access control checks, things that allow a contract to
specify only certain addresses for example, to call the function where
the modifier is applied\ldots{} This is something that becomes critical
when you evaluate the security of smart contracts.

\subsection{Function Visibility}\label{function-visibility}

It is similar to the visibility for state variables functions. Functions
have the 4 different visibility specifiers

\begin{itemize}
\tightlist
\item
  \textbf{\texttt{public}}: these functions are part of the contract
  interface and they can be called either internally (within the
  contract) or via messages.
\item
  \textbf{\texttt{external}}: these functions are also part of the
  contract interface, which means they can be called from other
  contracts and via transactions, but they cannot be called internally.
\item
  \textbf{\texttt{internal}}: these functions on the other hand can only
  be accessed internally (from within the current contract or contracts
  deriving from it).
\item
  \textbf{\texttt{private}}: these functions can be accessed only from
  within the contract where they are defined and not even from the
  derived contracts.
\end{itemize}

\subsection{Function Mutability}\label{function-mutability}

Similar to the state variable mutability, functions also have the
concept of mutability. This affects what state can they read or modify.
Depending on that there are two function mutability specifiers:

\begin{itemize}
\item
  \textbf{\texttt{view}}: these functions are allowed only to read the
  state but not modifying it. This is enforced at the EVM level using
  the \texttt{STATICCALL} opcode.\\

  There are various actions that are considered as state modifying that
  are not allowed for view functions, these include:

  \begin{itemize}
  \tightlist
  \item
    Writing to state variables (as should be obvious)
  \item
    Emitting events
  \item
    Creating other contracts
  \item
    Using self-destruct
  \item
    Sending ether to other contracts
  \item
    Calling other functions not marked \texttt{view} or \texttt{pure}
  \item
    Using low level calls
  \item
    Using inline assembly that contain certain opcodes
  \end{itemize}
\item
  \textbf{\texttt{pure}}: these on the other hand are allowed to neither
  read contract state nor modify it.\\

  The not modification part can be enforced at the EVM level, but the
  reading part cannot because there are no specific opcodes that allow
  that. There are various actions that are considered as reading from
  the state:

  \begin{itemize}
  \tightlist
  \item
    Reading from state variables (obviously)
  \item
    Accessing the balance of contracts
  \item
    Accessing members of \texttt{block}
  \item
    Transactions or messages
  \item
    Calling other functions not marked as \texttt{pure}
  \item
    Using inline assembly that contain certain opcodes
  \end{itemize}
\end{itemize}

The read/write mutability aspect of functions again has security
implications as you can imagine.

\subsection{Function Overloading}\label{function-overloading}

This is something fundamental to object oriented programming. It means
that it supports multiple functions within a contract to have the same
name but with different parameters or different parameter types.
Overloaded functions are selected by matching the function declarations
within the current scope to the arguments supplied in the function call,
so depending on the number and the type of arguments the correct
function is correctly chosen.

Note that return variables are not considered for the process of
resolving overloading, so this notion of overloading is an interesting
one that is supported by \texttt{Solidity} given that it is an
object-oriented programming language.

\subsection{Free Functions}\label{free-functions}

They are functions that are defined at the file level (i.e.~outside the
scope of contracts) and thus these are different from the contract
functions (defined within the scope of the contract). Free functions
always have implicit \texttt{internal} visibility and their code is
included in all the contracts that call them, similar to internal
library functions. These functions are not very commonly seen.

\subsection{Special Functions}\label{special-functions}

\subsubsection{Constructor}\label{constructor}

This concept is specific and unique to \texttt{Solidity} because it
applies to smart contracts and the way they are created on Ethereum.
Recall contracts on ethereum can be created from outside the blockchain
via transactions, or from within the \texttt{Solidity} contracts
themselves. When a contract is created, you can imagine that one would
want to initialize the contract state in some manner. This is made
possible by the constructor.

So the constructor is a special function that gets triggered when a
contract is created. A constructor is optional and there can be only one
constructor for every contract. These special functions are specified by
using the \texttt{constructor} keyword; some of the syntax and semantics
have changed over the course of time, but this is how it has been in the
most recent versions of \texttt{Solidity}

\begin{lstlisting}[language=Solidity,numbers=none]
contract Base {
   uint data;
   constructor(uint _data) public {
      data = _data;   
   }
}
\end{lstlisting}

So constructors are used to initialize the state of a contract when they
are created and deployed on the blockchain. They're triggered when a
contract is created and it's run only once. Once the constructor has
finished executing, the final code of the contract is stored on the
blockchain and this deployed code does not include the constructor code
or any of the internal functions that are called from within the
constructor.

From a security perspective, constructors are very interesting because
they allow one to examine what initializations are being done to the
contact state because if not, the default values of the specific types
of state variables are used instead. For example it could be used in the
context of the various contract functions, which is an interesting and
important aspect when it comes to evaluating the security of smart
contracts.

\subsubsection{Receive Function}\label{receive-function}

Another special function in the context of \texttt{Solidity} is the
\texttt{receive()} function. This function gets triggered automatically
whenever there is an Ether transfer made to this contract via
\texttt{send()} or \texttt{transfer()} primitives.

It also gets triggered when a transaction targets the contract but with
empty \texttt{CALLDATA}. Recall that a transaction that targets a
contract specifies which function needs to be called in that contract
and what arguments need to be used within the data portion of the
transaction, but if that data is empty then the receive function is the
function that gets automatically triggered in the contract.

There can only be one \texttt{receive()} for every contract and this
function cannot have any arguments, it cannot return anything and it
must also have external visibility and a \texttt{payable} state
mutability.

\emph{\texttt{payable} state mutability is something we haven't
discussed so far but what it specifies is that the function that has
this \texttt{payable} specifier can receive Ether as part of a
transaction and that applies to the receive function as well because it
is triggered when Ether transfers happen.}

The send and transfer primitives are designed in \texttt{Solidity} to
transfer only 2300 gas. The rationale behind this was to prevent the
risk, or mitigate the risk of what are known as ``\emph{reentrancy
attacks}'' which we'll talk more in the security chapter. This minimal
amount of gas does not allow a \texttt{receive()} function to do
anything much more than some basic logging (using events).

From a security context, the \texttt{receive()} function becomes
interesting to evaluate because it affects the Ether balance of a
contract and any assumptions in the contract logic that depends on the
contract's Ether balance.

\subsubsection{Fallback Function}\label{fallback-function}

Yet another special function in \texttt{Solidity}. It is very similar to
the \texttt{receive()} function with some particularities: the
\texttt{fallback()} function gets triggered automatically on a call to
the contract if none of the functions in the contract match the function
signature specified in the transaction. It also gets triggered if there
was no data supplied at all in the transaction and there is no
\texttt{receive()} function.

Similar to \texttt{receive()}, there can be only one \texttt{fallback()}
for every contract, however this \texttt{fallback()} function can
receive and return data if required. The visibility is \texttt{external}
and if the \texttt{fallback()} function is meant to receive Ether, then
it needs to have the \texttt{payable} modifier specified (similar to the
\texttt{receive()} function).

The \texttt{fallback()} function cannot assume that more than 2300 gas
can be supplied to it because it can be triggered via the \texttt{send}
or \texttt{transfer} primitives and, similar to \texttt{receive()}, the
security implications of \texttt{fallback()} have to consider that the
Ether balance can be changed via this function, so any assumptions in
the contract logic specific to the Ether balance need to be examined.
