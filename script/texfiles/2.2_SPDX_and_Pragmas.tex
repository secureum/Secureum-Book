\section{SPDX \& Pragmas}\label{spdx-pragmas}

\subsection{SPDX}\label{spdx}

One of the things that you will often see specified at the top of every
\texttt{Solidity} file is what is known as the SPDX license identifier.
\textbf{SPDX} stands for \textbf{Software Package Data Exchange}. In the
case of \texttt{Solidity} it's a comment that indicates its license and
it is specified as a best practice to be at the top of every file. An
example looks like this

\begin{lstlisting}[language=Solidity,numbers=none]
// SPDX-License-Identifier: AGPLv3
\end{lstlisting}

The specific license obviously depends on what the developer intends for
the particular smart contract. This identifier (i.e.~the license) is
included by the compiler in the byte code metadata that is generated, so
it becomes machine readable.

\subsection{Pragma}\label{pragma}

The \texttt{pragma} keyword in \texttt{Solidity} is used to enable
certain compiler features or compiler checks. An example looks something
like this

\begin{lstlisting}[language=Solidity,numbers=none]
pragma solidity ^0.8.0;
\end{lstlisting}

At a high level, there are two types of pragmas:

\begin{enumerate}
\def\labelenumi{\arabic{enumi}.}
\item
  The first kind specifies the version. There are, again, two types of
  versions that can be specified:\\

  \begin{enumerate}
  \def\labelenumii{\alph{enumii}.}
  \tightlist
  \item
    \textbf{The version \texttt{pragma}}, which indicates the specific
    \texttt{Solidity} compiler version that the developer expects to be
    used for that source file, and it looks like\\
  \end{enumerate}

  \begin{lstlisting}[language=Solidity,numbers=none]
  pragma solidity x.y.z;
  \end{lstlisting}

  where \texttt{x}, \texttt{y} and \texttt{z} are numerals that specify
  that compiler version.\\

  This does not change the version of the compiler used nor enables or
  disables any features of the compiler. All it does is instructing the
  compiler at compilation time to check whether its version matches the
  one specified by the developer. This could be of several formats: it
  could be a very \textbf{simple} format, a \textbf{complex} one or even
  a \textbf{floating} one (which has some security implications).\\

  The latest \texttt{Solidity} compiler versions as of now are in the
  \texttt{0.8} range with a different \texttt{z} in the \texttt{pragma}
  directive. If you look at \texttt{x.y.z}; a different \texttt{z}
  indicates bug fixes and a different \texttt{y} indicates breaking
  changes between the compiler versions.\\

  So if we have compiler versions in the \texttt{0.5} range, then by
  looking at the \texttt{0.6} range it means that the \texttt{0.6.z}
  range has at least one or more breaking changes compared to the
  previous versions.\\

  A floating \texttt{pragma} is a \texttt{pragma} that has a caret
  symbol (\texttt{\^{}}) prefixed to \texttt{x.y.z} in the directive.
  This indicates that the contract can be compiled with versions
  starting with \texttt{x.y.z} all the way until \texttt{x.(y\ +\ 1).z}.
  So, as an example, consider\\

  \begin{lstlisting}[language=Solidity,numbers=none]
  pragma solidity ^0.8.3;
  \end{lstlisting}

  \hfill\break
  It indicates that the source file can be compiled with any compiler
  version starting from \texttt{0.8.3} going to \texttt{0.8.4},
  \texttt{0.8.5} and whatever else has been released; but not
  \texttt{0.9.z}, so the transition from \texttt{0.8} to \texttt{0.9} is
  what is prevented by this floating platform.\\

  This allows the developer to specify a range of compiler versions that
  can be used with a particular contract, and that has some security
  implications similar to the floating \texttt{pragma}.\\

  A range of compiler versions can be indicated with a complex practice,
  where you have \texttt{\textgreater{}}, \texttt{\textgreater{}=},
  \texttt{\textless{}}, \texttt{\textless{}=} symbols that are used to
  combine multiple versions of the \texttt{Solidity} compiler.\\

  This affects the compiler version, which in turn brings in different
  features that are implemented by said version. Some of those could be
  security features, others could be security bug fixes or
  optimizations.\\

  All these aspects affect the security posture of the bytecode that is
  generated from a particular smart contract.\\

  \begin{enumerate}
  \def\labelenumii{\alph{enumii}.}
  \setcounter{enumii}{1}
  \tightlist
  \item
    \textbf{The ABI coder \texttt{pragma}}. This directive allows a
    developer to specify the choice between \textbf{Version 1} or
    \textbf{Version 2} ABI coder.\\
  \end{enumerate}

  The newer Version 2 was considered experimental for a while, but is
  now activated by default and allows the encoding/decoding of nested
  arrays and structs.\\

  You might encounter old \texttt{Solidity} source code using the old
  directive, such as shown below\\

  \begin{lstlisting}[language=Solidity,numbers=none]
  pragma experimental ABIEncoderV2;
  \end{lstlisting}

  \hfill\break
  Version 2 is a strict superset of Version 1: contracts that use
  Version 2 can interact with other contracts that do not use it without
  any concern or limitations.\\

  This \texttt{pragma} also applies to the code defined in the file
  where it is activated, regardless of where that code ends up
  eventually; what this means is that a contract whose file is using
  Version 1 can still contain code that uses Version 2 by inheriting it
  from another contract. An example of ABI Coder \texttt{pragma}
  statement is\\

  \begin{lstlisting}[language=Solidity,numbers=none]
  pragma abicoder v1; // or v2, which is the default from version 0.8.z onwards
  \end{lstlisting}

  \hfill\break
  The ABI coder affects encoding and decoding. The optimizations it does
  have certain security implications.
\item
  The second \texttt{pragma} directive helps the developer to specify
  features that are considered experimental as of that point in time.\\

  These features are not enabled by default and have to be explicitly
  specified as part of this \texttt{pragma} directive and within every
  file where it is required. As of now there is only one experimental
  feature, which is known as \texttt{SMTChecker}.\\

  \begin{lstlisting}[language=Solidity,numbers=none]
  pragma experimental SMTChecker;
  \end{lstlisting}

  \hfill\break
  \textbf{SMT} stands for \textbf{Satisfiability Modulo Theory} which is
  an approach to formal verification, and in the case of
  \texttt{Solidity} it is used to implement safety checks by what is
  known as querying an SMT solver.\\

  There are various security checks performed by the SMT checker. The
  first one is where it uses the \texttt{require} and \texttt{assert}
  statements that are included as part of the smart contract.\\

  The checker considers all the required statements specified as
  assumptions by the developer and it tries to prove that the conditions
  inside the \texttt{assert} statements are \texttt{true}.\\

  If a failure can be established, then the checker provides what is
  known as a counter example that shows the user how this assertion can
  fail.\\

  There are various other checks that have been added to this empty
  checker over time. These include the arithmetic overflow, underflow,
  division by zero, unreachable code and so on.\\

  So SMT checker is a critical security feature that comes packaged as
  part of \texttt{Solidity}. It's implemented in the compiler itself.\\

  Formal verification is considered as a fundamental part of programming
  languages' security, so we can imagine that this particular
  \texttt{pragma} directive affects the security and optimizations of
  the smart contracts that use them.
\end{enumerate}

What needs to be kept in mind with the \texttt{pragma} directives is
that they are local to the files where they are specified. So if you
have a \texttt{Solidity} file that imports other files, the pragmas from
the imported files do not automatically carry over to the file that is
of concern.
