\section{Solidity Version Changes}\label{solidity-version-changes}

Every new \texttt{Solidity} version introduces bug fixes and sometimes
breaking changes. Remember that breaking versions are versions that are
not backwards compatible, or in other words they've introduced
significant changes to the syntax, to the underlying semantics, that are
not compatible with the previous changes.

These breaking versions increment the number that you see in the middle
of the version, so for a \texttt{Solidity} version \texttt{x.y.z}, the
next breaking version would be \texttt{x.(y\ +\ 1).z}.

In this section we are going to revise recent \texttt{Solidity} versions
and their most impactful changes.

\subsection{\texorpdfstring{\texttt{solc\ 0.6.0}}{solc 0.6.0}}\label{solc-0.6.0}

\subsubsection{Breaking Changes}\label{breaking-changes}

In \texttt{Solidity\ 0.6.0} a breaking semantic feature that was
introduced changed the behavior of the existing code without changing
the syntax itself. It was specifically related to the exponentiation.
The type of the result until this version was the smallest type that
could hold both the type of the base and the type of the exponent. With
this change the resulting type was always the type of the base.

\subsubsection{Explicitness}\label{explicitness}

\texttt{Solidity\ 0.6.0} also introduced a set of explicitness
requirements. Explicitness, as you can imagine, is good for security
because it reduces ambiguity and any vulnerabilities that result because
of that ambiguity.

\begin{enumerate}
\def\labelenumi{\arabic{enumi}.}
\tightlist
\item
  In this case, keywords \texttt{virtual} and \texttt{override} were
  introduced for functions in base and derived classes. Functions and
  base classes can now only be overridden when they are marked with the
  \texttt{virtual} keyword, and their corresponding overriding functions
  need to use the \texttt{override} keyword.
\item
  Array length is read-only: it's no longer possible with version
  \texttt{0.6.0} to resize storage arrays by assigning a new value to
  their length.
\item
  An \texttt{abstract} keyword was introduced for what became abstract
  contracts or contracts where at least one function is not defined.
\item
  Libraries have to implement all their functions, not only the internal
  ones, as of this version and there are various restrictions
  (explicitness restrictions) brought forward for the assembly
  variables.
\item
  State variable shadowing being removed as this led to confusing
  results in ambiguity and has impacted the security of smart contracts.
\end{enumerate}

\subsubsection{Changes}\label{changes}

There were many other syntactic and semantic changes brought forward by
\texttt{Solidity\ 0.6.0}.

\begin{itemize}
\tightlist
\item
  \texttt{external} function type conversions to \texttt{address} types
  are not allowed. Instead they have an address member that allows
  similar functionality.
\item
  Dynamic storage arrays have now their \texttt{push(x)} return nothing,
  while until then it returned the length of the array.
\item
  Until \texttt{0.6.0} there was a concept of \texttt{unnamed}
  functions. This version split the functionality implemented by such a
  functions into a \texttt{fallback} function and a separate
  \texttt{receive} function. As you know, there are differences between
  these two functions and specific use cases where one of them is
  applicable versus the other.
\end{itemize}

\subsubsection{New Features}\label{new-features}

\texttt{0.6.0} also introduced several new features:

\begin{itemize}
\tightlist
\item
  \texttt{try}/\texttt{catch} blocks for exception handling.
\item
  \texttt{struct} and \texttt{enum} types can be declared at a file
  level with this version. Until then, it was only possible at contract
  level.
\item
  Array slices can be used for all data arrays.
\item
  \texttt{NatSpec}, as of this version supports multiple return
  parameters for developer documentation; it enforces the same naming
  checks as the param tag.
\item
  The inline assembly language \texttt{YUL} introduced a new statement
  called \texttt{leave} to help exit the current function.
\item
  Conversions from \texttt{address} type to \texttt{address\ payable}
  type are now possible via the \texttt{payable(x)} primitive, where
  \texttt{x} is of type \texttt{address}.
\end{itemize}

\subsection{\texorpdfstring{\texttt{solc\ 0.7.0}}{solc 0.7.0}}\label{solc-0.7.0}

\subsubsection{Breaking Changes}\label{breaking-changes-1}

The next breaking release was \texttt{Solidity\ 0.7.0}. With this
version, exponentiation and shift of literals by non-literals will
always use \texttt{uint256} or \texttt{int256} to perform the operation.
Until this version the operation was performed using the type of the
shift amount or the type of the exponent, which can be misleading, so
this became very explicit.

This again is a breaking semantic change because the behavior of
exponentiation and shifts changed underneath without any changes to the
syntactic aspect.

\subsubsection{Changes}\label{changes-1}

This version also introduced several syntactic changes that could cause
existing contracts to not compile anymore and therefore considered a
breaking change. Examples of such changes are:

\begin{itemize}
\tightlist
\item
  The syntax for specifying the Gas and Ether values applied during
  external calls.
\item
  The \texttt{now} keyword for time management within contracts was
  deprecated in favor of \texttt{block.timestamp} because \texttt{now}
  gave the perception that time could change within the context of a
  transaction whereas it is a property of the block, correctly indicated
  by \texttt{block.timestamp}.
\item
  The \texttt{NatSpec} aspect for variables was also changed to allow
  that for only \texttt{public} state variables and not for
  \texttt{local} or \texttt{internal} variables.
\item
  \texttt{gwei} was declared as a keyword and therefore can't be used
  for identifiers.
\item
  string literals can contain only printable \texttt{ASCII} characters.
  As of this version \texttt{unicode} string literals were also
  supported with the use of the \texttt{unicode} prefix.
\item
  The state mutability of functions during inheritance was also allowed
  to be restricted with this version, so functions with the default
  state mutability can be overridden by \texttt{pure} and \texttt{view}
  functions while the \texttt{view} functions can be overridden by
  \texttt{pure} functions.
\end{itemize}

There are also multiple changes introduced to the assembly support
within \texttt{Solidity}.

\subsubsection{Removed}\label{removed}

This version also removed some features that were considered as unused
or unsafe and therefore beneficial for security.

\begin{itemize}
\tightlist
\item
  Struct or arrays that contained mappings were allowed to be used only
  in storage and not in memory, the reason for this was that mapping
  members within such structural arrays in memory were silently skipped.
  This as you can imagine would be error prone.
\item
  The visibility of constructors, either \texttt{public} or
  \texttt{external} is not needed anymore.
\item
  The \texttt{virtual} keyword is disallowed for library functions,
  because libraries can never be inherited from and therefore the
  library functions should not need to be \texttt{virtual}.
\item
  Multiple events with the same name and parameter types in an
  inheritance hierarchy are disallowed, again to reduce confusion.
\item
  The directive \texttt{using\ A\ for\ B} with respect to library
  functions and types affects only the contract it is specified in as of
  this version. Previously this was inherited, now it has to be repeated
  in all the derived contracts that require this feature.
\item
  Shifts by sign types are disallowed as of this version. Until now
  shift by negative amounts were allowed, but they caused a
  \texttt{revert} runtime.
\item
  The Ether denominations of \texttt{fini} and \texttt{szabo} were
  considered as rarely used and therefore were removed as of this
  version.
\item
  The keyword \texttt{var} was also removed because this would until now
  pass, but result in a type error as of this version.
\end{itemize}

\subsection{\texorpdfstring{\texttt{solc\ 0.8.0}}{solc 0.8.0}}\label{solc-0.8.0}

\subsubsection{Breaking Changes}\label{breaking-changes-2}

\texttt{Solidity\ 0.8.0} is the latest of the breaking versions of
\texttt{Solidity}. This version introduced several breaking changes:

\begin{itemize}
\item
  The biggest perhaps is the introduction of default checked arithmetic.
  This is the overflow and underflow arithmetic checks that are so
  commonly used in \texttt{Solidity} contracts to prevent the wrapping
  behavior that results in overflows and has resulted in several
  security vulnerabilities.\\

  Until this version, the best practice was to rely on the
  \texttt{OpenZeppelin} \texttt{SafeMath} libraries or their equivalents
  to make sure that there are runtime checks for overflows and
  underflows. These never result in vulnerabilities. This is so commonly
  used that \texttt{Solidity\ 0.8.0} introduced the concept of checked
  arithmetic by default, so all the arithmetic that happens with
  increment, decrements, multiplication and division is all checked by
  default.\\

  This might come at a slight increase of Gas Cost, but it also
  increases the default security level significantly and it also
  improves the readability of code because now one doesn't have to use
  or see the use of calls to the \texttt{SafeMath} libraries in the form
  of \texttt{.add()}, \texttt{.sub()} and, so on\ldots{}\\

  And as an escape hatch where the developer knows for sure that certain
  arithmetic is safe from such underflows and overflows,
  \texttt{Solidity} provides the \texttt{unchecked} primitive that is
  allowed to be used on blocks of arithmetic expressions where this
  default underflow and overflow checks are not done by the
  \texttt{Solidity} compiler.\\
\item
  ABI coder version \texttt{v2} is activated by default. As of this
  version, it doesn't have to be explicitly specified but if the
  developer wants to fall back on the older \texttt{v1} version that has
  to be specified.
\item
  Exponentiation is right associative as opposed to being left
  associative that was the case. This is the common way to parse
  exponentiation operator in other languages, so this was fixed.
\item
  As of this version the use of the \texttt{REVERT} opcode versus the
  use of the \texttt{INVALID} opcode for failing asserts and internal
  checks was removed. Now both use the \texttt{REVERT} opcode and static
  analysis tools are allowed to distinguish these two differing
  situations by noticing the use of the panic error in the case of
  failing asserts and internal checks.\\

  When storage byte arrays are accessed where the length is encoded
  incorrectly a panic is raised that's another change introduced.
\item
  The \texttt{byte} type which used to be an alias of \texttt{bytes1}
  has been removed as of this version.
\end{itemize}

\subsubsection{Restrictions}\label{restrictions}

\texttt{Solidity\ 0.8.0} also introduced several restrictions:

\begin{itemize}
\tightlist
\item
  Explicit conversions of multiple types are disallowed. Remember that
  explicit conversions are where the user forces conversions between
  certain types without the compiler necessarily thinking those are safe
  from a type safety perspective, so these explicit conversions being
  disallowed may be considered as a good thing from a security
  perspective.
\item
  Address literals now have the type \texttt{address} instead of
  \texttt{address} \texttt{payable}, these have to be explicitly
  converted to \texttt{address\ payable}.
\item
  If required, the function call options for specifying the Gas and
  Ether value passed can only be provided once and not multiple times.
\item
  The global functions \texttt{log0}, \texttt{log1} all the way to
  \texttt{log4} that may be used for specifying events or logs have been
  removed because they were low level functions that were considered as
  largely unused by \texttt{Solidity} contracts and, if a developer
  wants to use them, they need to resort to inline assembly.
\item
  \texttt{enum} definitions now can't contain more than 256 members.
  This makes it safer because the underlying type is always
  \texttt{uint8}, so 8 bits that allows only 256 members to be
  represented by that type.
\item
  Declarations with name \texttt{this}, \texttt{super} and \texttt{\_}
  are disallowed.
\item
  Transaction origin (\texttt{tx.origin}) and message sender
  (\texttt{msg.sender}) global variables now have the type
  \texttt{address} instead of \texttt{address\ payable}, and again
  require an explicit conversion where \texttt{address} \texttt{payable}
  is needed.
\item
  The mutability of \texttt{chainId} is now considered \texttt{view}
  instead of \texttt{pure}.
\end{itemize}

All these different types of restrictions were introduced in this
version that have an impact on security.
