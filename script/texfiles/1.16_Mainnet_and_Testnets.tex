\section{Mainnet and Testnets}\label{mainnet-and-testnets}

Mainnet refers to the main Ethereum network. There is a distinction
because there also exist several testnets. These testnets are test
Ethereum networks where protocol and smart contract developers can test
their protocol upgrades and smart contracts prior to final deployment in
mainnet.

While mainnet uses real Ether, testnets use what is known as
``\emph{test Ether}'\,', so that you can simulate the Gas, the transfer
of value and so on\ldots{} These test Ether can be obtained from
faucets.

Some of the popular Ethereum testnets are \textbf{Goerli} (a proof of
authority testnet that allows one to look at a lot of the Ethereum
concepts and test them as if they are happening on mainnet). This
particular testnet works across all the clients.It's called proof of
authority because there are a small number of nodes that are allowed to
create the blocks and validate them. After the Ropsten testnet reached a
Terminal Total Difficulty (TTD) of $5\times 10^{16}$, the Goerli
testnet transitioned to a proof-of-stake consensus mechanism to mimic
Ethereum mainnet.

Then we have the \textbf{Kovan} testnet, which is again a proof of
authority testnet specifically for those running OpenEthereum clients.

There is also the \textbf{Sepolia} Testnet. Sepolia was a
proof-of-authority testnet created in October 2021. Similarly to Goerli,
after the Ropsten testnet reached a Terminal Total Difficulty (TTD) of
$5\times 10^{16}$, the Sepolia testnet transitioned to a
proof-of-stake consensus mechanism. Sepolia was designed to simulate
harsh network conditions, and has shorter block times, which enable
faster transaction confirmation times and feedback for developers.
Compared to other testnets like Goerli, Sepolia's total number of
testnet tokens is uncapped, which means it is less likely that
developers using Sepolia will face testnet token scarcity like Goerli.

As you can see, these testnets are also evolving, new ones are being
added over time trying to make it as easy as possible for the developers
to simulate the real mainnet Ethereum blockchain and all its
dependencies. This again becomes very critical to security because
\textbf{testing is fundamental}: if you do not test, or if the testing
environment is not very similar to the production environment, then the
assumptions (the dependencies and other aspects that are tested) will be
very different from what happens when you deploy the contract, which
could end up causing a lot of security issues.

\subsection{Deprecated testnets}\label{deprecated-testnets}

Testnets are also removed over time. Some of the deprecated testnets are

\begin{itemize}
\tightlist
\item
  \textbf{Rinkeby} testnet. It was a proof of authority based testnet
  which was specifically for the Geth clients. It shut down in 2023.
\item
  \textbf{Ropsten} testnet. It was a proof of work testnet. It shut down
  at the end of 2022.
\end{itemize}
