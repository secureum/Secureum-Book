\section{Error Handling}\label{error-handling}

Error handling is one of the most important fundamental and critical
aspects of programming languages' security. The reason is that errors
during program execution are what result in security vulnerabilities.
These could be errors resulting from user inputs when they interact with
the smart contract and the inputs are not as expected by the developer
during the coding of the contract.

In the EVM, an exception undoes or reverses all the changes made to the
state of the smart contract in the context of the current transaction:
the calls and all the subcalls that may be several levels deep. In
addition, an error is also flagged to the caller so that they can take
appropriate action.

They could also be due to assumptions made within the smart contract
that are not really valid for the various control and data flows that
happen during program execution. They could also be related to the
programming variants that are expected from a specification perspective
and these invariants might not hold good during certain control and data
flows.

\subsection{Exceptions}\label{exceptions}

So when exceptions happen within subcalls in that call hierarchy, during
runtime they bubble up, and What this means is that exceptions are
rethrown at the higher level calls automatically.

There are some exceptions to this rule. There are some differences here
in the context of the \texttt{send} primitive, and the low level
function calls: \texttt{call}, \texttt{delegatecall} and
\texttt{staticcall} which we talked about earlier. These primitives
(\texttt{send}, \texttt{call}, \texttt{delegatecall} and
\texttt{staticcall}) return a boolean \texttt{true} or \texttt{false} as
their first return value instead of an exception bubbling up.

This is an important distinction to be kept in mind when analyzing smart
contracts because the exception behavior is different for these
primitives, compared to the standard message calls. Exceptions that
happen in external calls made during the contact execution can be caught
with the \texttt{try\ catch} statement.

These exceptions can contain data that is passed back to the caller and
this data consists of a function selector indicating which function the
exception happened in, and also some other ABI encoded data that gives
more information about the exception.

\subsection{\texorpdfstring{Error Signatures: \texttt{error} and
\texttt{panic}}{Error Signatures: error and panic}}\label{error-signatures-error-and-panic}

\texttt{Solidity} supports two error signatures: \texttt{error} and
\texttt{panic}. \texttt{error} takes a \texttt{string} parameter whereas
\texttt{panic} takes an unsigned parameter. \texttt{error} is meant to
be used for ``\emph{regular error conditions}'', such as input
validation and so on. \texttt{panic} is used for errors that should not
be present in bug free code.

\subsubsection{\texorpdfstring{\texttt{panic}}{panic}}\label{panic}

The \texttt{panic} exception is generated in various situations in
\texttt{Solidity}, and the error code supplied with the error data
indicates the kind of panic.

There are many of these error codes. Some of them are

\begin{itemize}
\tightlist
\item
  \texttt{0x01}: indicates that \texttt{assert} has an argument that
  evaluated to \texttt{false}.
\item
  \texttt{0x11}: an overflow or underflow happened in arithmetic.
\item
  \texttt{0x12}: division by zero or modulus by zero occured.
\item
  \texttt{0x31}: \texttt{pop()} of an empty array occurred.
\item
  \texttt{0x32}: out of bounds access for an array.
\end{itemize}

There are numerous error codes for \texttt{panic}\ldots{}

This error reverts all the state changes made to the contract logic so
far in the context of the transaction that triggered it.

\subsubsection{\texorpdfstring{\texttt{error}}{error}}\label{error}

Error string exception, as discussed earlier, are generated when
\texttt{require} (which we'll see shortly) executes and its argument
evaluates to \texttt{false}.

The error string is also generated in other situations such as an
external call made to a contact that contains no code, or if the
contract receives ether via public function without the payable
modifier. Or if the contract receives ether via a public getter
function.

\subsection{Error Handling Primitives}\label{error-handling-primitives}

\texttt{Solidity} supports multiple primitives for error handling, being
the first set of primitives are functions that let the developer assert
or require certain conditions to be held.

\subsubsection{\texorpdfstring{\texttt{assert}}{assert}}\label{assert}

The \texttt{assert(x)} primitive for example, specifies a condition
\texttt{x} as its argument, and if that condition is not met (if it
evaluates to \texttt{false} during runtime), The \texttt{assert}
primitive is meant to be used for internal errors for program invariants
that should never be violated within the smart contract if it does not
have bugs as intended by the developer.

These asserts result in the \texttt{panic} errors that take the
\texttt{uint256} type, to reiterate they should be used for internal
errors for checking invariants, normal code bug free code should never
cause \texttt{panic}s.

\subsubsection{\texorpdfstring{\texttt{require}}{require}}\label{require}

It is another error handling primitive supported by \texttt{Solidity}.
Similarly to \texttt{assert} it also specifies a condition that gets
evaluated at runtime, and if that condition evaluates to \texttt{false},
then it again raises a \texttt{revert}, that reverses all the state
changes made to the contract so far. The \texttt{require} primitive
takes an optional string as an argument: this is a message that gets
printed if the required condition is not met.

Therefore, it either creates an error of type error string or an error
without any error data.

The \texttt{require} primitive is meant to be used for errors in inputs
from users which should be validated to make sure they are within the
thresholds of what is acceptable with the smart contract logic, so some
sanity checks on those values are necessary (this is a fundamental
pillar of security).

\texttt{require} is also used for checking the return values from calls
that are made to external contracts. So any type of external
interaction, be it inputs from users or return values from external
contact calls, are what require is meant to be used with. Require takes
in an optional message string that is output when the condition fails.

The thing to be kept in mind is the difference between \texttt{assert}
and \texttt{require}. There were some historical differences as well in
the use of a particular opcode: a different opcode for \texttt{assert}
versus \texttt{require} that affects the Gas consumption, but some of
these have been changed in the recent \texttt{Solidity} versions.

\subsubsection{\texorpdfstring{\texttt{revert}}{revert}}\label{revert}

Finally there is a \texttt{revert} primitive that unconditionally aborts
execution when triggered, reverting all the state changes similar to
when the conditions are not met for a certain \texttt{require}
primitive.

There are two ways to explicitly trigger a \texttt{revert} in
\texttt{Solidity}: using the \texttt{revert\ CustomError(arg1,...)}
primitive or the \texttt{revert({[}string{]})} function, where the
\texttt{string} parameter is optional.

In both these cases, the execution is aborted and all the state changes
made. as part of the transaction, are reversed. This distinction between
\texttt{CustomError} and the \texttt{string}, is interesting from an
optimization and usecase perspective.

\subsection{try/catch}\label{trycatch}

These primitives supported by \texttt{Solidity} are fundamental error
handling. The syntax is

\begin{lstlisting}[language=Solidity,numbers=none]
try Expr [returns()] {...}
catch <Block> {...}
\end{lstlisting}

So, we have the \texttt{try} and \texttt{catch} keywords coupled with an
expression that contains an external function call or creation of a
contract. These are coupled with code blocks corresponding to the
success blocks or the catch blocks. These are code segments within the
curly braces shown above. Which block gets executed depends on whether
there was a failure or not in that external call within that expression.

If there were no errors then the success block gets executed (the block
that immediately follows the \texttt{try} expression in the syntax shown
earlier). But if there was an error in that external call then the catch
block, or one of the catch blocks, gets executed. Which catch block gets
executed depends on the error type, and there are multiple of them.

\subsubsection{\texorpdfstring{\texttt{catch}
Blocks}{catch Blocks}}\label{catch-blocks}

As mentioned, \texttt{Solidity} supports different kinds of
\texttt{catch} blocks depending on the error type. There is a
\texttt{catch} block that supports an error string.
\texttt{catch\ Error(\textless{}string\ reason\textgreater{})}. This is
executed if the error was caused by \texttt{revert} with a reason
string, or \texttt{require} where the condition evaluated to
\texttt{false}.

Then there is a \texttt{catch} kind that supports \texttt{panic} error
code \texttt{catch\ Panic(uint\ \textless{}error\ code\textgreater{})}.
If this error was caused by a panic failing \texttt{assert}, (remember:
division by zero, outer bound array accesses, arithmetic
underflow/overflow) this is the \texttt{catch} block that will be run.

In addition there is a \texttt{catch} that specifies the low level data
\texttt{catch\ (bytes\ \textless{}LowLevelData\textgreater{})}. This one
gets executed if the error signature does not match any other clause
shown above. Or if there was an error while decoding the error message
itself, or if no error data was provided with that exception. This
variable that is declared the low-level data gives us access to the
error data in that case.

Finally if the developer is not interested in the type of error data,
one can simply use \texttt{catch} as is. These give various options to
deal with different types of exceptions, that might come from the
external call that is used within the \texttt{try} \texttt{catch}
permit.

\subsubsection{try/catch State Change}\label{trycatch-state-change}

As we just discussed, there exists the concept of the success block,
that gets executed when there are no exceptions in that external call.
There are also error blocks that correspond to the different
\texttt{catch} blocks which get executed when there are exceptions
encountered in that external call.

If execution reaches the success block, it means that there were no
exceptions in that external call, and all the state changes that are
done in the context of the external call are committed to the state of
the contract.

But if execution reaches one of the \texttt{catch} or error blocks, then
it means that the state changes in that external call context have been
reverted, because of the exception. There could also be a context where
the \texttt{try} \texttt{catch} statement itself reverts for reasons of
decoding or low level failures.

\subsubsection{External Call Failure}\label{external-call-failure}

These failures of the external call made in the context of the catch
primitive, could happen for a variety of reasons and one cannot always
assume that the error message is coming directly from the contract that
was called in that external call, because the error could happen deeper
down in the call chain resulting from that call. It was forwarded to the
point where it was received.

This could also be due to an out of Gas (OOG) situation, in which case
the caller still has a bit of Gas to deal with that exception because
not all of it is forwarded to the callee.
