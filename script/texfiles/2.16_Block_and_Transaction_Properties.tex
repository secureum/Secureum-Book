\section{Block \& Transaction
Properties}\label{block-transaction-properties}

\texttt{Solidity} allows accessing various block and transaction
properties within smart contracts. These allow developers to perform
interesting logic that are dependent on different aspects of the current
block or the transaction.

\subsection{Block}\label{block}

In the case of \texttt{block}, we have the following members:

\begin{itemize}
\tightlist
\item
  \textbf{\texttt{blockhash}}: gives the hash of the specified block,
  but only works for the most recent 256 ones, otherwise it returns
  zero.
\item
  \textbf{\texttt{chain\ id}}: gives the current id of the chain that
  this is executing on.
\item
  \textbf{\texttt{number}}: gives the sequence number of the block
  within the blockchain.
\item
  \textbf{\texttt{timestamp}}: gives the number of seconds since the
  unix epoch.
\item
  \textbf{\texttt{coinbase}}: it is controlled by the miner and gives
  the beneficiary address where the block rewards and transaction fees
  go to.
\item
  \textbf{\texttt{difficulty}}: block's difficulty related to the proof
  of work.
\item
  \textbf{\texttt{gaslimit}}: Gas limit for the block.
\end{itemize}

\subsubsection{Randomness Source}\label{randomness-source}

\textbf{The block timestamp and block hash that we just discussed are
not good sources of randomness}, that's because both these values can be
influenced by the miners mining the blocks to some degree. The only
aspects of timestamps that are guaranteed, is that the \textbf{current
blocks timestamp must be strictly larger than the timestamp of the last
block} and the other guarantee is that \textbf{it will be somewhere
between the timestamps of two consecutive blocks in the canonical
blockchain}. Therefore smart contract developers should not rely on
either the block timestamp or the block hash as a source of good
randomness.

\subsubsection{Message and Transaction}\label{message-and-transaction}

There are also fields related to the message (\texttt{msg}):

\begin{itemize}
\tightlist
\item
  \textbf{\texttt{value}}: represents the amount of Ether that was sent
  as part of the transaction.
\item
  \textbf{\texttt{data}}: gives access to the complete call data sent in
  this transaction.
\item
  \textbf{\texttt{sender}}: gives the sender of the current call or
  message.
\item
  \textbf{\texttt{signature}}: gives the function identifier or the
  first four bytes of the call data representing the function selector
  that we talked about earlier.
\end{itemize}

The thing to be kept in mind is that every external call made, changes
the \texttt{sender}. Every external call made can also change the
\texttt{value}.

So if we have three contracts \texttt{A}, \texttt{B} and \texttt{C}
where \texttt{A} calls \texttt{B} and \texttt{B} calls \texttt{C}, in
the context of the contract \texttt{B} the \texttt{msg.sender} is
\texttt{A}, but in the context of the contract \texttt{C} the
\texttt{msg.sender} is contract \texttt{b} and not \texttt{a}.

These aspects should be kept in mind when analyzing the security of
smart contract because the developers could have made incorrect
assumptions about some of these that could result in security issues.

Then there are transaction (\texttt{tx}) components: -
\textbf{\texttt{gasprice}}: the Gas price used in the transaction. There
is an interesting \texttt{Solidity} native function called
\textbf{\texttt{gasleft()}} which returns the amount of Gas left in the
transaction after all the computation so far.- \textbf{\texttt{origin}}:
gives the sender of the transaction, representing the EOA.
