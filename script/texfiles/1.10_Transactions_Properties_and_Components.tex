\section{Transactions: Properties \&
Components}\label{transactions-properties-components}

Transactions are \textbf{signed messages} that originate outside of the
Ethereum blockchain. They are \textbf{triggered by EOA}s (that are
managed or controlled by a private key). The trigger happens to be the
digital signature, derived from the private key.

These transactions are transmitted by the Ethereum network and they
trigger state changes on the blockchain. In fact, Ethereum is
fundamentally a transaction based state machine, as only transactions
are capable of triggering state changes.

\subsection{Properties}\label{properties}

\begin{enumerate}
\def\labelenumi{\arabic{enumi}.}
\item
  Transactions are \textbf{atomic}: they run from the beginning to the
  end completely, it's either all or nothing.\\

  So the side effects of the transactions are only reflected in the
  blockchain if they run to completion. If they don't, nothing that they
  do is reflected on the blockchain and it's as if the transaction never
  happened.\\

  In other words: transactions cannot be divided or interrupted with
  some of the partial state being reflected on the blockchain and the
  rest of it not.\\

  This contrasts withtraditional computing environments where a
  particular process or a particular control might be interrupted, gets
  context washed out and then something else (a different process or a
  thread) executes and then the original context is brought back. None
  of that happens within the context of Ethereum.
\item
  Transactions are \textbf{serial}: they're executed one after the
  other, sequentially without any overlapping. There is no parallelism
  when it comes to the execution of transactions.
\item
  Transaction \textbf{inclusion}. When a user submits a transaction,
  what is the warranty that it gets included within one of the blocks on
  the Ethereum blockchain?\\

  This property is controlled by entities on the Ethereum blockchain
  known as miners. They run Ethereum nodes and decide which transactions
  are included within a block. This depends on multiple factors, the key
  ones being the congestion on the Ethereum network (or in other words,
  the other transactions that are competing for the same block space)
  and the Gas price that the user decides to use for the particular
  transaction.
\item
  \textbf{Inclusion order}. It refers to the specific order of the
  transactions included within a block.\\

  Again, this chosen by the miners and, similarly to inclusion, is
  determined by factors of congestion and Gas price. The key takeaway of
  properties 3 and 4 is that there are entities known as miners on the
  blockchain who get to decide which transactions get included within a
  certain block and the specific order of the transactions within that
  block.
\end{enumerate}

\subsection{Components}\label{components}

Transactions contain seven components:

\begin{enumerate}
\def\labelenumi{\arabic{enumi}.}
\item
  \texttt{nonce}: the name is an abbreviation for ``a number used only
  once''. It's a sequence number that, as part of the protocol, is
  incremented in a particular fashion (it changes for every
  transaction).\\

  The application of the nonce is prevention of replay attacks
  (i.e.~replaying the same transaction over and over again). In the case
  of an EOA, the nonce value is equal to the number of transactions sent
  from that account. In the case of a contract, it is equal to the
  number of other contracts created by this contract account.
\item
  \texttt{gasPrice}: the price for every Gas unit that the sender is
  willing to pay for a particular transaction. It's measured in
  wei/gas.\\

  \texttt{gasPrice} is not fixed by the Ethereum protocol. The higher
  the Gas price that the the sender is willing to pay for this
  transaction, the faster the particular transaction gets included by
  the miner into a block in the blockchain. This price depends on the
  demand for the block space at the point in time when the transaction
  is submitted.\\

  The reason for this is that there is a limited amount of space in the
  block, so there's only a limited number of transactions (as determined
  by the Gas used by each one of them) that can be included within this
  block.
\item
  \texttt{gasLimit}: the maximum number of Gas units that the sender is
  willing to pay for a particular transaction. This depends on the type
  of transaction that is being sent.\\

  If it is a simple Ether transfer then it costs 21000 Gas units. But if
  it is a transaction that is targeting a particular contract (or a
  particular function of the contract) then the required amount of gas
  is higher. If sufficient Gas (in the form of Gas limit) is not set for
  the transaction (if it's less than what is required to), then it
  results in what is known as an \textbf{out of Gas exception} (OOG
  Error) and that transaction fails.\\

  The way it works is that for any transaction that is being sent by a
  sender, there is an estimated Gas that needs to be sent as part of the
  transaction. If that estimated amount of Gas is not sent then it leads
  to the exception. If there is excess gas, then the remaining Gas is
  sent back to the sender.
\item
  \texttt{recipient}: the destination 20 byte Ethereum address for a
  transaction (i.e.~the destination account that this transaction is
  targeting).\\

  This could be an EOA address or a contract address, it depends on the
  target of that particular transaction. It could be any address on the
  Ethereum blockchain, and the protocol itself does not validate these
  recipient addresses in the transactions. So one can send a transaction
  to any address and that address might not even have a corresponding
  private key, nor the contract that the sender expects to have.\\

  Thus, all such validation should be done at the user interface level.
  That validation is critical for security reasons (more on that in
  later chapters). Note that this recipient is really the target
  address. There is no ``from address'' that is a component of the
  transaction. The reason for this is that the ``from address'' can be
  derived from the ECDSA signature components $v$, $r$ and $s$:
  they can be used to derive the public key, which in turn can be used
  to derive said address.
\item
  \texttt{value}: the amount of Ether (in wei) that the sender is
  sending to the recipient address.\\

  What happens with such funds depends on the recipient: if it happens
  to be in EOA, then the balance of that account will be increased by
  this value and the sender's balance correspondingly decreases. If it
  hapens to be a contract account, then what happens depends on any
  other data present as part of this transaction (i.e.~the contract
  function being invoken with the transaction data).\\

  If there is no data being sent as part of this transaction, and the
  destination happens to be a contract account, then the contracts'
  \texttt{receive} or \texttt{fallback} functions (if they were defined
  or if they are present; more on that in the upcoming chapters) are
  triggered and thus, what happens with the received Ether depends on
  their implementation.\\

  If there is no \texttt{fallback} function, then the transaction
  results in an exception and the Ether, that is sent as part of the
  transaction, remains with the sender account.
\item
  \texttt{data}: payload of variable length and binary encoded (as per
  the format required by Ethereum) that is sent as part of this
  transaction.\\

  This field is relevant when the recipient is a contract account. As
  mentioned previously, the data in that case contains the contract
  function that is being targeted by the transaction plus the specific
  arguments that are relevant for said function.
\item
  \texttt{v}, \texttt{r} \& \texttt{s}: The ECDSA signature is 65 bytes
  in length and has three subcomponents: \texttt{v}, \texttt{r} and
  \texttt{s}.

  \begin{itemize}
  \tightlist
  \item
    \texttt{r} and \texttt{s} represent the signature components. They
    are 32 bytes in length each (adding up to 64 bytes).
  \item
    The final subcomponent, \texttt{v}, is the recovery identifier. It's
    just one byte and its value can be either 27 or 28, or it can be
    twice the value of chain ID ($2\times\text{ID}_\text{chain}$) plus
    either 35 or 36. The chain ID is the identifier of the blockchain.
    In the case of the Ethereum mainnet chain, $\text{ID}=1$.
  \end{itemize}
\end{enumerate}

For a particular transaction, the Ether used to purchase Gas is credited
to the beneficiary address that was specified in the block header (more
details on this are found in the upcoming sections). Then there's also a
concept of a Gas refund: the difference between the Gas limit and the
Gas Used is refunded back to the sender of the transaction. This is done
at the same Gas price as indicated in that transaction.
