\section{Control Structures}\label{control-structures}

These are fundamental to any programming language because there is a
control flow to the sequence of instructions specified in the high-level
language that get translated into machine code by the compiler.

In the case of \texttt{Solidity}, the control structures supported are
\texttt{if}, \texttt{else}, \texttt{while}, \texttt{do}, \texttt{for},
\texttt{break}, \texttt{continue} and \texttt{return}. These are very
similar to the ones found in any programming language. Although are some
differences in \texttt{Solidity}: paranthesis for example cannot be
omitted for conditionals that some of the other languages support,
however curly braces can be omitted around single statement bodies for
such conditionals.

Also note that there is no type conversion from a non-boolean to a
boolean type. As an example, \texttt{if(1)} is not allowed in
\texttt{Solidity} because 1 is not convertible to the boolean
\texttt{true}, which is supported by some of the other languages.

So control structures play a critical role in the security analysis of
smart contracts whether you're doing a manual review or whether you're
writing a tool to pass the \texttt{Solidity} smart contract. These are
some things to be understood really well because that's how control
flows, and any analysis depends critically on making sure that the
control flow is accurately followed and representative of what really
happens at runtime.
