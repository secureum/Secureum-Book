\section{Properties of the Ethereum
Infrastructure}\label{properties-of-the-ethereum-infrastructure}

\subsection{High availability and High
auditability}\label{high-availability-and-high-auditability}

\textbf{High availability} refers to the fact that Ethereum is
\textbf{always up and running} (24 hours, 7 days a week, 365 days of the
year): there's \textbf{never a downtime} that is expected because of
upgrades or because of any issues (that's the goal) which, again,
contrasts with most web2 services where they might be taken down for
maintenance, upgrades or any other reasons.

High availability is given by \textbf{decentralization}, because of the
absence of centralized infrastructure choke points that can go down and
bring the whole infrastructure down with them.

\textbf{High auditability} refers to the fact that everything that
happens on Ethereum (everything that happens on a blockchain) is
auditable (it can be examined, analyzed and reasoned about).

\subsection{Transparency and
neutrality}\label{transparency-and-neutrality}

The fundamental vision of Ethereum is that \textbf{applications are
permissionless}: if somebody was to build any part of the infrastructure
for Ethereum (the protocol, the Ethereum client, smart contracts\ldots),
they can do so \textbf{without permission from any centralized entity}
within Ethereum.

The tools and the infrastructure are \textbf{open sourced}: you can look
at it, extend it, deploy and use it without anybody's permission.

This is what lends to \textbf{permissionless applications}
(decentralized applications), in contrast with how you build, deploy and
use nowadays' mobile applications (say on the Apple platform or the
Google platform) where you have a centralized entity that you have to
register with, get the permission from, test with, follow the regulation
set forth by the by that entity (by the Play store or the Apple store)
and then deploy it while being subjected to certain ``rules'' that
govern how you can use those apps.

This is the so-known \textbf{contrast} between the \textbf{web3} space
and the current existing \textbf{web2} space. This is a key aspect of
permissionless interaction, development and innovation. All this is
\textbf{incentivized} because of the built-in economics (crypto
economics) which makes people run the Ethereum nodes, deploy and use the
applications.

Additionally, as it is built on a blockchain, Ethereum has a high degree
of \textbf{transparency}: nothing is meant to be proprietary (the source
code, the design of the protocols, the transactions that interact with
it\ldots) and behind pay walls, or hidden in such a way that you cannot
reason about the security or transparency aspects of it.

That's, at least, the high level design goal and all these properties
lend themselves to make the platform and everything that's built on it,
highly neutral. We'll see more about how decentralization really
contributes to neutrality because there's no centralized entity that can
change the availability, auditability or transparency aspects of the
platform or applications built on top of it.

\subsection{Censorship Resistance}\label{censorship-resistance}

The aforementioned properties lend themselves to a very high degree of
censorship resistance. This is may be something you're familiar with in
existing platforms: if an app, a website or anything else does not
subscribe itself to the compliance aspects of the platform or any other
entity, then it might be taken away from the platform at any time by the
entity that is controlling it. There are many many stories of this being
done extremely wrong. There have been accidents where unintentionally
some of these apps were taken down because they fit into some larger
category type of applications that were being de-platformed. Blockchains
in general make this very hard to do at a platform level.

Censorship leads to what is known as ``\emph{lowering the counterparty
risk}'': there is always a risk associated with the party, the platform,
the application on top of it or the logic that you're interacting with.
Thanks to the transparency, neutrality and censorship resistance, that
risk is much much lower.

So none of this is black or white: it's all on a spectrum. We're going
towards what is known as ``\emph{progressive decentralization}'' where
some of these properties might not be completely there yet. There might
be elements of centralization that over time and by design are removed,
so that we reach a point where these applications or platforms are
completely decentralized with no single entity or groups of entities
that can really manipulate the platform and abuse it. That is where we
are headed towards.
