\section{Style and Conventions}\label{style-and-conventions}

\subsection{Programming Style}\label{programming-style}

So far, we have reviewed all the aspects of the various basics of
\texttt{Solidity}: syntax, semantics, \ldots{} are rules that are
enforced in the \texttt{Solidity} grammar.

Programming style on the other hand is coding convention, and these are
different across different developers. Different styles are adopted
based on what the developer is comfortable with based on what they
believe is an optimal way of programming, \textbf{but fundamentally the
programming style is about consistency}.

\textbf{The reason is that programming style affects the readability and
maintainability of the code}. So when anyone other than the developer
looks at the code to evaluate the security, to audit it or to make fixes
or extend those smart contracts, the consistency of the programming
style becomes important.

If different styles are used within the same function, within the same
module or within the same project, because of the same developer not
being consistent or because multiple developers are involved in that
project having different styles, then this significantly affects the
readability and maintainability of the code, impacting both
significantly on the security life cycle of the code.

Programming style is subjective in nature: different developers,
different teams might have different philosophies as to what works best
for them. The key is consistency within the function modules, contracts
or projects, as it affects readability and maintainability, which are
critical for security, specifically with respect to smart contract
audits.

There are two main categories of programming style, that of code
\textbf{layout} and that of \textbf{naming}.

\subsubsection{Code Layout}\label{code-layout}

\textbf{Code layout refers to the physical layout of the various
programming elements within a source code file}. There are many
programming style aspects related to code layout: those related to
indentation, where the best practice and \texttt{Solidity} is to
recommend 4 spaces per indentation level and to prefer spaces over tabs
and to definitely not mix them.

There are also style guidelines with respect to blank lines used to
surround declarations, with the max line length that's recommended to be
79 or 99 characters for best readability. There are also recommendations
for wrapping lines, for the encoding used in the source files
(\texttt{ASCII} or \texttt{UTF-8}), and for keeping the import
statements at the top of the file and not anywhere else.

Finally, the ordering of the functions within the contract, where the
recommendation is to have the constructor as the first in that order
followed by functions of different visibilities. Grouping all the
external functions first, followed by public functions and then internal
and then private functions.

Strings are recommended to be used with double quotes instead of single
quotes operators, spaces have some guidelines as well. Finally the
ordering of different program elements within a \texttt{Solidity} file
also have a guideline, which is to have the \texttt{pragma} declaratives
all the way at the top, followed by the \texttt{import} directives and
then the contract library or interface definition itself.

Within each of the contracts libraries and interfaces, the guideline
suggests using the types first, followed by declarations of state
variables, then events and finally all the various functions.

\subsection{Naming Conventions}\label{naming-conventions}

The next aspect of programming style is naming convention. This refers
to the names that are given to various variables, events, contracts,
libraries and all the different program elements used within smart
contracts.

There are different types of names: lower case names, lower case with
underscores, all upper case, upper case with underscores, capitalized
words, mixed case, capitalized words with underscores and so on. All
these different types are recommended to be used for different program
elements, and as a general rule the guideline is to avoid letters that
can be confused with the different numerals, like the lowercase
letter\texttt{"l"}, or the uppercase letter \texttt{"O"} and uppercase
letter \texttt{"I"} that could be confused with \texttt{0} and
\texttt{1} numerals.

Contract and libraries should be named with cap word style, they should
also match their file names and if a contact file includes multiple
contracts and libraries then the file name should match the core
contract (or what is considered as a core contract for that file by the
developer).

Structs should be named using the cab word style. Event should be named
using cap word style again. Functions should be named using mixed case.

This is something that you'll encounter often within contracts
developers are sometimes consistent with these naming, sometimes they're
mixed up because of multiple developers or the style just not being
consistent or being confused with that of some external libraries.

Again all these aspects affect the readability and maintainability, they
do not have any impact on the syntax or the semantics of the contract
itself.

The bytecode is just the same but it has an effect to a certain extent,
on the security audit aspect when you look at this code, and when the
naming convention is different or not consistent then it could lead to
some assumptions being made on these variables being the same ones or
different ones.

Some more naming conventions are: function arguments should be in mixed
case, local state variables again in mixed case, constants however
should be with all capital letters and underscores separating the
multiple words if they are present.

Modifiers in mixed case, enums in cap word style and finally one should
avoid naming collisions where the desired names, variables or functions
collides with that of a built-in within \texttt{Solidity} or any other
reserve name. So those should be resolved using single trailing
underscores in those names.
