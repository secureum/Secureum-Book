\section{Reentrancy}\label{reentrancy}

The reentrancy security pitfall is perhaps unique to smart contracts
where external calls made to contracts can result in what can be thought
of as callbacks to the called contract itself.

So for example, if there is a contract \texttt{C1}, that makes a call to
an external contract \texttt{C2}, where \texttt{C2} could potentially be
untrusted could be malicious because it is not developed by the same
team or within the same project as \texttt{C1}, then that external
contract \texttt{C2} could call back into \texttt{C1} to the same
function that called it or to any other function of \texttt{C1} that
allows such a call.

This could be exploited to do malicious things, such as multiple
withdrawals or something less harmful, such as out of order emission of
events. There have been multiple exploits that have taken advantage of
this class of reentrancy attacks, some of them are historical in nature
such as the DAO hack on Ethereum.

So this class of security vulnerabilities, that is specific to smart
contracts needs to be paid attention to, the best practice to prevent
such reentrancy vulnerabilities from being exploited is to follow what
is known as the \textbf{Checks-Effects-Interactions} pattern (the CEI
pattern for short), where the interactions with external potentially
untrusted contracts is only made after performing all the checks and all
the effects where effects are nothing, but changes to the state of the
calling contract, so that any anticipated side-effects of interactions
with the external contracts are already reflected in the state of the
calling contract.

So this CEI pattern is recommended as a best practice to be followed in
all functions that are making external contract calls specifically to
contact calls that could be malicious because they're untrusted. The
other best practice is to use what are known as reentrancy guards, we
talked about this in the context of the reentrancy guard library from
OpenZeppelin where a a \texttt{nonReentrant} modifier is provided. This
modifier when applied to specific functions prevents them from being
called within a callback, so it avoids any reentrances to that function
itself.

\subsection{Reentrancy via ERC777}\label{reentrancy-via-erc777}

This security pitfall is related to the use of \texttt{ERC777} standard,
the potential for re-entrancy vulnerabilities due to the callbacks it
supports.

Remember that \texttt{ERC777} standard is considered as an extension to
the \texttt{ERC20} standard it's considered as making improvements to
it. One improvement is the notion of hooks that it supports during token
transfers, if such an \texttt{ERC777} token contract is potentially
malicious, then it could use these hooks to cause reentrancy into the
calling contract.

So for example, if there's a contract \texttt{C1} that calls an
\texttt{ERC777} token contract that is malicious, then that contract
could use the hook functionality to cause a reentrancy into the calling
contract \texttt{C1} and take advantage of it as we just mentioned.

The best practice again is to follow the Checks Effects Interaction
(CEI) pattern in the calling contract and also to consider the use of
reentrancy guards.

\subsection{\texorpdfstring{OOG in \texttt{transfer()} \&
\texttt{send()}}{OOG in transfer() \& send()}}\label{oog-in-transfer-send}

This security pitfall is related to the use of the \texttt{transfer} and
\texttt{send} primitives in \texttt{Solidity}.

These primitives were introduced as reentrancy mitigations, because they
only forward 2300 Gas to the called contract, which is typically
sufficient only for basic processing such as emitting a few logs or
something even simpler, this Gas is not enough to make a real currency
call back to the calling contract which requires more than 2300 gas.

So this has been recommended for a long time as a security best practice
for preventing reentrancy attacks however over time some of the opcodes
have been reprised when it comes to their Gas usage, so their Gas Cost
has increased in some of the recent hard forks on Ethereum and because
of that the use of these primitives that enforce the Gas subsidiary 2300
Gas could break the contract because it might not allow the called
contract to even do the basic processing that we just talked about.

So the latest security best practice recommendation is to not rely on
\texttt{transfer()} and \texttt{send()} as reentrancy mitigations, but
instead to use the low-level \texttt{call()} directly that does not have
those hard-coded Gas Limits and couple that with a CEI pattern or
reentrancy guard or both for re-entrance mitigation.
