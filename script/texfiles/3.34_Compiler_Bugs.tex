\section{Compiler Bugs}\label{compiler-bugs}

Let's now discuss a set of security risks that manifested themselves
because of compiler bugs: these were bugs in older versions of the
\texttt{Solidity} compiler that have been since fixed.

They are very specific to certain complex data structures or very
specific conditions that one may not often encounter in typical smart
contracts. Because of that, we will not be able to get into the details
of these compiler bugs and their security risks.

Nevertheless, taking a look at the higher level aspects of these
compiler bugs would hopefully let us appreciate the complexity of some
of them and the security risks that they may pose.

\subsection{\texorpdfstring{Storage Array with \texttt{int} under
\texttt{ABIEncoderV2}}{Storage Array with int under ABIEncoderV2}}\label{storage-array-with-int-under-abiencoderv2}

This specific compiler bug was related to storage arrays and signed
integers, and their usage was enabled by the \texttt{ABIEncoderV2},
which was a \texttt{pragma} directive, that needed to be explicitly
specified until the latest versions (as it is now used by default).

This specific bug arose when assigning an array of signed integers to a
storage array of a different type \texttt{Type{[}{]}\ =\ int{[}{]}}.
Under such assignments, it led to data corruption in that array.

This bug was present in \texttt{Solidity} versions \texttt{0.4.7} until
\texttt{0.5.10} (which are much older versions than the latest one that
we often encounter), so it's very unlikely that we'll look at smart
contracts using these much older versions, but it is something to be
kept in mind.

\subsection{\texorpdfstring{Dynamic Constructor Arguments Clipped under
\texttt{ABIEncoderV2}}{Dynamic Constructor Arguments Clipped under ABIEncoderV2}}\label{dynamic-constructor-arguments-clipped-under-abiencoderv2}

A contract's constructor that takes structs or arrays that contain
dynamically sized arrays (made possible because of
\texttt{ABIEncoderV2}) reverted or decoded to invalid data.

This compiler bug was present in \texttt{Solidity} versions
\texttt{0.4.16} to \texttt{0.5.9}.

\subsection{\texorpdfstring{Storage Array with Multi-slot Element under
\texttt{ABIEncoderV2}}{Storage Array with Multi-slot Element under ABIEncoderV2}}\label{storage-array-with-multi-slot-element-under-abiencoderv2}

There was a compiler bug related to storage arrays in \texttt{Solidity},
specifically those with multi-slot elements, again made possible because
of \texttt{ABIEncoderV2}.

Such storage arrays containing structs or other statically sized arrays
were not read properly when they were directly encoded in external
function calls or using the \texttt{abi.encode()} primitive.

This bug was present in \texttt{Solidity} versions \texttt{0.4.16} to
\texttt{0.5.10}.

\subsection{\texorpdfstring{Calldata Structs with Statically Sized and
Dynamically Encoded Members under
\texttt{ABIEncoderV2}}{Calldata Structs with Statically Sized and Dynamically Encoded Members under ABIEncoderV2}}\label{calldata-structs-with-statically-sized-and-dynamically-encoded-members-under-abiencoderv2}

Another compiler bug was related to the \texttt{struct} type
(specifically calldata structs) which consisted in reading from calldata
structs that contained dynamically encoded, but statically sized
members, could result in incorrect values being read.

This again was limited to the \texttt{Solidity} compiler versions
\texttt{0.5.6} to \texttt{0.5.11}.

\subsection{\texorpdfstring{Packed Storage under
\texttt{ABIEncoderV2}}{Packed Storage under ABIEncoderV2}}\label{packed-storage-under-abiencoderv2}

There was a compiler bug related to packed storage. Storage structs and
arrays with types smaller than 32 bytes when encoded directly from
storage using \texttt{ABIEncoderV2} could cause data corruption.

This occurred with \texttt{Solidity} compiler versions \texttt{0.5.0} to
\texttt{0.5.7}.

\subsection{\texorpdfstring{Incorrect Loads with \texttt{Yul} Optimizer
and
\texttt{ABIEncoderV2}}{Incorrect Loads with Yul Optimizer and ABIEncoderV2}}\label{incorrect-loads-with-yul-optimizer-and-abiencoderv2}

This is another compiler bug specifically coming from the \texttt{Yul}
optimizer, part of it resulted in incorrect loads being done.

When the experimental \texttt{Yul} optimizer was activated manually in
addition to \texttt{ABIEncoderV2}, it resulted in memory loads and
storage loads via \texttt{MLOAD} and \texttt{SLOAD} instructions to be
replaced by values that were already written.

So effectively, the \texttt{Yul} optimizer replaced the \texttt{MLOAD}
and \texttt{SLOAD} calls with stale values which is a serious bug. This
occurred with \texttt{Solidity} compiler versions \texttt{0.5.14} to
\texttt{0.5.15}.

\subsection{\texorpdfstring{Array Slice Dynamically Encoded Base Type
under
\texttt{ABIEncoderV2}}{Array Slice Dynamically Encoded Base Type under ABIEncoderV2}}\label{array-slice-dynamically-encoded-base-type-under-abiencoderv2}

There was a compiler bug specifically related to array slices, which
there are views of the arrays that lets us access specific ranges of
those arrays in a very efficient manner.

Accessing such array slices for arrays that had dynamically encoded base
types resulted in invalid data being read for the \texttt{Solidity}
compiler versions \texttt{0.6.0} to \texttt{0.6.8}.

\subsection{\texorpdfstring{Missing Escaping in Formatting under
\texttt{ABIEncoderV2}}{Missing Escaping in Formatting under ABIEncoderV2}}\label{missing-escaping-in-formatting-under-abiencoderv2}

This compiler bug was related to missed escaping. Escaping is relevant
to string literals where certain characters can be escaped using the
double backslash.

String literals that contained double backslash characters for escaping,
that were passed directly to \texttt{external}, or encoding function
calls, could result in a different string being used when
\texttt{ABIEncoderV2} was enabled.

Notice that this compiler bug was present across many \texttt{Solidity}
compiler versions all the way from \texttt{0.5.14} to \texttt{0.6.8}.

\subsection{Double Shift Size
Overflow}\label{double-shift-size-overflow}

If multiple conditions were true, then the shifting operations resulted
in overflows resulting in unexpected values being output.

Some of those conditions were that the optimizer needed to be enabled.
These had to be double bitwise shifts where large constants were being
used whose sum overflowed 256 bits.

Under such conditions the shifting operations overflowed for the
\texttt{Solidity} compiler versions \texttt{0.5.5} to \texttt{0.5.6}.

\subsection{Incorrect Byte Instruction
Optimization}\label{incorrect-byte-instruction-optimization}

This was a compiler bug originating from incorrect optimization of byte
instructions.

The optimizer, when dealing with byte codes whose second argument was
\texttt{31} or a constant expression that evaluated to \texttt{31},
incorrectly optimized it which resulted in unexpected values being
produced.

This was possible when doing an index access on the \texttt{bytesNN}
types (so all the types like \texttt{bytes1}, \texttt{bytes2} to
\texttt{bytes32}) or when using the \texttt{BYTES} opcode in assembly.

Unexpected values were produced when these conditions were met, from
\texttt{Solidity} versions \texttt{0.5.5} to \texttt{0.5.7}.

\subsection{\texorpdfstring{Essential Assignments Removed with
\texttt{Yul}
Optimizer}{Essential Assignments Removed with Yul Optimizer}}\label{essential-assignments-removed-with-yul-optimizer}

There was another compiler bug coming from the \texttt{Yul} optimizer.
In this case, the \texttt{Yul} optimizer removed essential assignments
for variables that were specifically declared inside \texttt{for} loops.

This would happen while using \texttt{Yul}'s \texttt{continue} or
\texttt{break} statements, and again limited to the \texttt{Solidity}
compiler versions \texttt{0.5.8}/\texttt{0.6.0} to
\texttt{0.5.16}/\texttt{0.6.1}.

\subsection{Privat Methods Overriden}\label{privat-methods-overriden}

Remember that function visibilities in \texttt{Solidity} can be
\texttt{private}, \texttt{internal}, \texttt{public} or
\texttt{external}.

\texttt{private} functions are specific to the contract in which they
are defined: they can't be called from any other contract, even those
deriving from it.

While this is true, it was still possible for a derived contract to
declare a function of the same name and type as a \texttt{private}
function in one of the base contracts. And by doing so, change the
behavior of the base contracts function.

What is interesting to note here, from a security perspective, is that
this compiler bug was present across multiple \texttt{Solidity} versions
all the way from \texttt{0.3.0} to \texttt{0.5.17}.

\subsection{Tuple Assignment Multi-stack Slot
Components}\label{tuple-assignment-multi-stack-slot-components}

Tuple assignments where the components occupied several stack slots, for
example in the case of nested tuples, resulted in invalid values because
of a compiler bug.

Notice again that this compiler bug lasted across 5 breaking
\texttt{Solidity} versions: all the way from \texttt{0.1.6} to
\texttt{0.6.6}.

\subsection{Dynamic Array Cleanup}\label{dynamic-array-cleanup}

When dynamically sized arrays were being assigned with types whose size
was at most 16 bytes in storage, it would cause the assigned array to
shrink to reduce their slots.

However, some parts of the deleted slots were not being zeroed out by
the compiler. This would lead to stale or dirty data being used. This
bug was fixed in \texttt{Solidity} version \texttt{0.7.3}.

\subsection{Empty Byte Array Copy}\label{empty-byte-array-copy}

This bug is related to byte arrays, from \texttt{memory} or
\texttt{calldata}, that were empty were copied to \texttt{storage} and
they could result in data corruption.

This only occurred if the target array's length was subsequently
increased, but without storing new data in it. Notice how specific the
conditions are for this bug to be triggered. Nevertheless, this bug was
discovered and fixed in \texttt{Solidity} version \texttt{0.7.4}.

\subsection{Memory Array Creation
Overflow}\label{memory-array-creation-overflow}

When memory arrays were being created, if they were very large in size,
then they would result in overlapping memory regions, which would lead
to corruption.

In this case, this compiler bug was introduced in \texttt{Solidity}
version \texttt{0.2.0} and fixed in version \texttt{0.6.5}.

\subsection{\texorpdfstring{Calldata \texttt{using\ for} compiler
bug}{Calldata using for compiler bug}}\label{calldata-using-for-compiler-bug}

Remember, \texttt{using\ for} primitive is used for calling library
functions on specific types used within the smart contract.

In this case the bug was specific to when the parameters used in such
function calls were in the calldata portion of the EVM. In such cases,
the reading of such parameters would result in invalid data being read.
This bug existed accross \texttt{Solidity} versions \texttt{0.6.9} to
\texttt{0.6.10}.

\subsection{Free Function
Redefinition}\label{free-function-redefinition}

Remember, free functions in \texttt{Solidity} are functions that are
declared outside contracts (i.e.~at file level).

This compiler bug allowed free functions to be declared with the same
name and parameter types. This redefinition or collision was not
detected by the compiler as an error. This bug was present in one of the
recent \texttt{Solidity} versions: \texttt{0.7.1}, and fixed in
\texttt{0.7.2}.

Compiler bugs should be taken very seriously because, unlike smart
contracts that may differ from each other in the logic implemented, in
the data structures or other aspects used, the compiler is a common
dependency or perhaps a single point of failure for all the smart
contracts compiled with that version.

Having said that, let's also recognize that a compiler is another
software, so just like any software it is bound to have bugs and perhaps
even more, because the compiler is significantly more complex than a
smart contract or any other general software application.

From a security perspective, the things to be kept in mind when looking
at a compiler version that's being used in smart contracts is to know
which features of that compiler version are considered as being
extensively used, and which are considered as experimental and perhaps
staying away from them, so that one is not susceptible or vulnerable to
any bugs in them.

It is also important to recognize the bugs that have been fixed in the
compiler version, the bugs that have been reported and perhaps fixed in
later versions (if those are available). These aspects should dictate
the choice of the compiler version for the smart contracts and the
specific features that are available within those compiler versions.

So the takeaway is that some of these compiler bugs may be so deep down
in the compiler code and may be triggered only under specific
conditions, that they might not be discovered very soon after the
compiler is released. So while the test of time is true, there are no
guarantees that a much older compiler version has most of its bugs
discovered, reported and fixed.
