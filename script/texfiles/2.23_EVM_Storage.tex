\section{EVM Storage}\label{evm-storage}

Let's see how some of the \texttt{Solidity} concepts map to the EVM
storage. Remember: it is a \texttt{(key,\ value)} store that maps 256
bit words to 256 bit words, so the \texttt{key} and \texttt{value} are
both considered to be the word size supported by the EVM. The
instructions used to access the storage are \texttt{SLOAD} to load from
storage and \texttt{SSTORE} to write to storage from the stack. Remember
that all locations in the storage are initialized to zero.

\subsection{Storage Layout}\label{storage-layout}

State variables are stored in the different storage slots. Each slot in
the EVM storage corresponds to a word size of 256 bits. The various
state variables declared within the smart contracts are mapped to these
storage slots in the EVM, and if there are multiple state variables that
can fit within the same storage slot depending on their types, then they
are done so to maintain a compact representation of the state variables
within that storage slot.

The mapping is done in the same order as the declaration of the state
variables, so state variables that are declared within a contract are
stored contiguously in their declaration order in the different storage
slots of the EVM, which means that the first state variable is stored in
slot 0 the second one in slot 1 or maybe the same slot 0\ldots{}

\subsubsection{Storage Packing}\label{storage-packing}

If the first variable was of a size smaller than 256 bits, the second
one could fit as well within that slot, so \textbf{except for dynamic
arrays and mappings, all the other types of state variables are stored
contiguously item after item starting with the first state variable}.
This is known as \textbf{storage packing}.

Remember that \texttt{Solidity} supports different types and each type
has a default size and bytes, so it all depends on the types of the
state variables declared and their underlying sizes. If there are
multiple contiguous state variables that need less than 32 bytes, then
those are packed into the single storage slot where possible.

There are some rules that are followed: the first item in a storage slot
is stored lower-order aligned value types that only use as many bytes
that are necessary to store them, and when a value type does not fit the
remaining part of a storage slot, it is stored in the next storage slot.
This concept of storage packing becomes important when we are looking at
a smart contract code and trying to determine which storage slot a
particular state variable fits in, which depends on the other state
variables that are declared around it.

\subsubsection{Layout, Types \& Ordering}\label{layout-types-ordering}

Storage packing allows us to optimize the storage slot layout depending
on the types of the state variables. So state variables can be made to
have a reduced size type depending on the values that they're supposed
to hold, then storage packing allows such state variables to share a
storage slot. This allows the service compiler to combine multiple reads
or writes into a single operation when it generates the corresponding
bytecode.

However, if those state variables sharing the same slot are not read or
written at the same time, depending on the contract logic, this can have
an opposite effect, which results in more Gas being used than expected.
This is because when one such value of a state variable that shares that
slot with other state variables is being read or written, then the
entire slot is read or written because that is the size that the EVM and
\texttt{Solidity} work with.

Now the specific state variable within that slot has to be separated out
for reading or writing, this is done by masking out all the other state
variables that share that slot.

This masking results in additional instructions being generated which
lead to additional Gas being used in this case, so depending on the
specific sizes of the types and on the pattern of reading or writing,
the types of state variables that are adjacent to each other in the
declarations should be bid for efficient optimization from storage
packing.

To summarize: the ordering of the state variable declarations within a
smart contract impact the layout of their storage slots and affects if
multiple state variables declared contiguously can be packed within the
same storage slot or if they need separate storage slots. This packing
has a huge impact on the Gas Cost because the instructions that read and
write state variables (if you remember are \texttt{SLOAD}s and
\texttt{SSTORE}s) are the most expensive instructions from a Gas Cost
perspective supported by EVM.

\begin{itemize}
\tightlist
\item
  \texttt{SSLOAD}s costs as much as 2100 Gas or 100 Gas depending on how
  many times the state variables has been read. So far in the context of
  the transaction.
\item
  \texttt{SSTORE}s cost as much as 20000 Gas in the most recent EVM
  versions.
\end{itemize}

As an example, if we have three state variables of types
\texttt{uint128}, \texttt{uint128} and \texttt{uint256} that are
declared within the same smart contract contiguously, then these
variables would use 2 storage slots because the first 2 storage
variables can share the same storage slot. The 2 variables of size 128
bits will fit into the same storage slot: slot 0 in this case (which is
256 bits in size). The third state variable of type \texttt{uint256}
would go into the second storage slot (or slot 1).

But if the declaration order is slightly changed (so for example putting
the 256 bit state variable in between the 128 bit variables), then the
new order would require 3 storage slots instead of 2: the first 128 bit
one would go into slot 0, the second one would not fit within slot 0, so
it would go to slot 1 and consume the whole slot 1. The third state
variable would then take up a slot.

This gives you an idea of how the state variable declaration order
impacts a number of storage slots, which has a big impact on the Gas
Cost used by that contract.

\subsubsection{Inheritance and Storage
Layout}\label{inheritance-and-storage-layout}

\textbf{How does inheritance affect the storage slot allocation?}

For contracts that use inheritance, the ordering of state variables is
determined by the \texttt{C3} linearization rule of the contract orders,
starting from the most base contract to the most derived contract. If
allowed by any of the rules discussed, state variables from the
different contracts (the different base and derived contracts) are
allowed to share the same storage slot with respect to the storage
packing concept we talked about.

\subsubsection{Storage Layout for Structs \&
Arrays}\label{storage-layout-for-structs-arrays}

State variables of type structs and arrays have specific rules with
respect to the storage slot allocation. Such state variables always
start a new storage slot as opposed to being packed into existing ones.
The state variables following them also start a new storage slot. The
elements of the structs and arrays themselves are stored contiguously
right after each other as if they were individual values, and depending
on their types the rules we just discussed earlier apply to these as
well.

\subsubsection{Mappings \& Dynamic
Arrays}\label{mappings-dynamic-arrays}

Storage slot allocation for mappings and dynamically sized arrays is a
bit more complex than their value type counterparts. These types are
unpredictable in their dynamic size by definition and because of that
reason the storage slots allocated for them can't be reserved in between
the slots for the state variables that surround them in the declaration
order within their contract.

Therefore these are always considered to occupy a single slot, that's 32
bytes in size with regard to the rules discussed so far and the elements
that they contain within, that that can change dynamically over the
duration of the contract, are stored in a totally different location:
the starting storage slot for those elements is computed using
\texttt{keccak-256} hash.

\begin{itemize}
\item
  \textbf{Dynamic Arrays}\\

  Let's say we have a state variable of type dynamic array and, based on
  the declaration order within that smart contract, let's say that it is
  assigned slot number \texttt{p}.\\

  This slot only stores the number of array elements within that state
  variable and is updated during the lifetime of the contract when this
  changes. The actual elements of the dynamic array itself are stored
  separately in different storage slots.\\

  The starting slot for those elements is determined by taking the
  \texttt{keccak-256} hash of the slot number \texttt{p}. The elements
  themselves starting from that storage slot that we just calculated are
  stored contiguously and can also share those storage slots if
  possible, depending on their types, on the size of those types and, if
  we have dynamic arrays that in turn have dynamic arrays within them,
  then the same set of rules apply recursively to determine their
  corresponding storage slots.
\item
  \textbf{Mappings}\\

  Again, depending on the declaration order, if there is a state
  variable of mapping type and it gets assigned a slot number
  \texttt{p}, then that particular slot stores nothing: it's an empty
  slot just assigned to that mapping. Compare this to the dynamic array
  that we just discussed where this slot stores the number of those
  array elements.\\

  The slots corresponding to the values for keys of this mapping are
  calculated as follows: for every key \texttt{k}, the slot that is
  allocated is determined by taking the \texttt{keccak-256} hash of
  \texttt{h(k).p}. The \texttt{.} is a concatenation of the two values
  of \texttt{h(k)} and \texttt{p}. We know that \texttt{p} is the slot
  number, which we mentioned earlier on. \texttt{h} is a function that
  is specific to the type of the key that we're talking about and, if
  this is a value type, then there is a padding that is done to make it
  up to 32 bytes. If it is a string or byte arrays, then \texttt{h()}
  computes the \texttt{keccak-256} hash of the unpadded data. The type
  specific rules that determine what \texttt{h()} is and those details
  are specified better at the reference provided.
\end{itemize}

\subsubsection{Bytes \& String}\label{bytes-string}

For this case, there is an interesting optimization. The storage layout
for these is very similar to arrays, so the actual storage slot
depending on the declaration order, stores the length of these types,
the elements themselves of the variable are stored separately in a
storage slot that is determined by taking the \texttt{keccak-256} hash
of the storage slot assigned to store the length.

However, if the values of these variables are short, then instead of
storing these elements separately they are stored along with the length
within the same storage slot. The way this is done is: if the data is at
most 31 bytes, then the first byte in the lowest order stores the value
\texttt{length*2} and all the other bytes (the higher order bytes) store
the elements that fit within the remaining 31 bytes.

If the length of the data is more than 31 (32 bytes or more), then the
lowest order byte stores the value \texttt{length*2\ +\ 1}, the elements
themselves don't fit within this storage slot that stores the length, so
they are stored separately using the \texttt{keccak-256} hash of this
slot's position.

The distribution of whether the data values values are stored within the
same storage slot as the length or if they're stored separately, is made
by looking at the lowest order bit. If that is set (\texttt{1}) it means
that they are stored separately and, if they're stored within the same
slot as a length, then this bit will not be set (\texttt{0}). This is
because of the length being stored as \texttt{length*2\ +\ 1} or just
\texttt{length*2}.
