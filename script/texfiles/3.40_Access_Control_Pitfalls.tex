\section{Access Control Pitfalls}\label{access-control-pitfalls}

Access control deals with assets, actors and actions, or in other words
which actors have access to which assets and how much of those assets
and what actions can the actors use to access those assets.

\subsection{Access Control
Specification}\label{access-control-specification}

The access control specification should detail who can access what and
why should they have that access, when can that access happen and how
much of those assets can the actors access. All these aspects should be
very accurately specified in great detail, so that they can be correctly
implemented and enforced across the different contracts and functions,
and across all the system transitions and flows that happen within those
contracts and functions.

This should help determine the trust, the threat models and any
assumptions that are being made from this model. Without such an access
control specification it will be very hard or even impossible to
evaluate if the implementation actually enforces all these aspects.

\subsection{Access Control
Implementation}\label{access-control-implementation}

The implementation of access control should make sure that every aspect
of the access control that was specified in the specification is
implemented uniformly and accurately across all the actors on all the
assets via all the actions possible. The implementation should make sure
that none of the actors, assets and flow conditions within actions are
missing or may be sidestepped. Such an implementation should help us
evaluate if the access control enforcement has been done correctly
according to the specification.

\subsection{Access Control Modifiers}\label{access-control-modifiers}

Access control is typically enforced in \texttt{Solidity} smart contacts
by means of modifiers. Instead of implementing access control checks
that are required for different functions multiple times in each of
those functions, modifiers allow us to encapsulate those checks in one
place and then these modifiers can be applied on any of those functions
that require the access control checks implemented within them. While
this encapsulation brings in the desired aspect of modularity, modifiers
also impact auditability.

There's a school of thought that believes that modifiers are good for
auditability: they make it easier because they implement all the checks
in one place, so instead of reviewing the same checks multiple times in
multiple functions these checks can be reviewed once within the
modifier, then check if these modifiers are applied correctly to all the
functions that require those checks, so that makes auditability easier.

On the flip side, there's another school of thought that believes that
modifiers are not as good for auditability as thought. The reason is
that if there is a contract that has multiple modifiers and many
functions that use those modifiers, then remember that the programming
style guidelines recommend modifiers to be declared and defined at the
beginning of the contract, and all the functions come thereafter so, if
an auditor is reviewing functions deep down within the contract and it
uses multiple modifiers, then they have to scroll up to the modifiers at
the beginning of the contracts to check if the desired checks were
implemented and if they were implemented correctly. This switching of
context in the process of scrolling is believed to not lead to good
auditability.

Nevertheless, modifiers are used extensively and reviewing these
modifiers should make sure that they are indeed present on the functions
that require the checks implemented by them, that modifiers implement
valid checks in a correct manner and their order is also correctly
specified for functions that use multiple modifiers.

\subsubsection{Modifiers Implementation}\label{modifiers-implementation}

Given the critical role of modifiers in access control, modifiers need
to be implemented correctly. But what does that mean? Access control in
smart contracts is enforced on different addresses that may be
classified into different roles with differing privileges.

Like we discussed in earlier modules, contracts may have a simple
ownership based access control or a more flexible one based on RBAC. In
such RBAC scenarios we need to check that modifiers are enforcing the
correct checks on the correct roles, that such checks are composed
correctly. Such a correct implementation is critical to access control
which is the fundamental aspect of smart contract security and therefore
needs to be reviewed very carefully.

\subsubsection{Modifiers Usage}\label{modifiers-usage}

It is not sufficient to have the modifiers implemented correctly, but
they should be used or applied correctly as well: the questions of which
modifiers are used, why are they used, the how/what aspects, what are
the parameters passed to them and what should they do with them, the
order of modifiers when more than one is present, the when aspect (under
what state transitions should they be applied), finally the where aspect
(the functions where they're applied to). All such aspects of modifiers
their functions and any parameters should have been considered
correctly.

\subsection{Access Control Changes}\label{access-control-changes}

The access control implemented may need to be changed in some scenarios.
In such cases, it is critical that the change is done correctly with
respect to the assets actors or actions that are impacted. Using the
wrong addresses for assets or actors, or allowing the changes to happen
at the wrong times in the context of the application logic may lead to
loss or locking of funds. Therefore, access control changes should be
validated for correctness, use a two-step process to allow recovery from
mistakes and also log changes for transparency and off-chain monitoring.
