\section{Hash Collisions and Byte Level
Issues}\label{hash-collisions-and-byte-level-issues}

\subsection{Hash Collisions}\label{hash-collisions}

Hash collisions are possible in certain scenarios where the
\texttt{abi.encodePacked()} primitive, is used with multiple variable
length arguments.

This happens because this primitive does not zero pad the arguments, and
it also does not save any length information for those arguments. As a
result, this packed encoding could lead to collisions in certain
scenarios, which you can imagine can affect the security of the smart
contract.

The best practice here is to avoid the use of the
\texttt{abi.encodePacked()} primitive where possible and use the
\texttt{abi.encode()} primitive instead.

In scenarios where it can't be avoided, one should at least make sure
that only one variable length argument is used in this parameter and
certainly, users who can reach this primitive via function calls should
not be allowed to write to the parameters used and tainted to force
collisions from happening.

\subsection{Dirty Bits}\label{dirty-bits}

There is a security risk from Dirty High Order Bits in
\texttt{Solidity}. Remember that the EVM word size is 256 bits or 32
bytes, and there are multiple types in \texttt{Solidity} whose size is
less than 32 bytes. Using variables of such types may result in their
higher order bits containing dirty values.

What this means is that they may contain values from previous writes to
those bits, that have not been cleared or zeroed out. Such Dirty Order
Bits are not a concern for variable operations because the compiler is
aware of these Dirty Bits and takes care to make sure that they do not
affect the values of variables.

By the way, if those variables end up getting used or passed around as
message data, then that may result in them having the different values
and causing malleability or non-uniqueness. This is a risk that needs to
be kept in mind when looking at contracts that have variables of such
types.

\subsection{Incorrect Shifts}\label{incorrect-shifts}

There is a security pitfall related to the use of incorrect shifts in
\texttt{Solidity} assembly specifically.

\texttt{Solidity} assembly supports three different Shift operations:

\begin{enumerate}
\def\labelenumi{\arabic{enumi}.}
\tightlist
\item
  Shift left \texttt{shl()}.
\item
  Shift right \texttt{shr()}
\item
  Shift arithmetic right \texttt{sar()}
\end{enumerate}

all of which take two operands \texttt{x} and \texttt{y}. These
operations shift the \texttt{y} operand by \texttt{x} bits and not the
other way around.

This can be confusing, understandably the developer may have used these
two operands interchangeably in which case the shift operation do
something completely different from what the developer anticipated. This
is something that needs to be checked when looking at Shift operations
in \texttt{Solidity} assembly.

\subsection{Assembly}\label{assembly}

The use of assembly in \texttt{Solidity} itself is considered as a
security risk because assembly bypasses multiple security checks such as
type safety, that is enforced by \texttt{Solidity}.

Developers end up using \texttt{Solidity} Assembly to make the
operations more optimized and efficient from a Gas perspective, but on
the flip side this is very error-prone because the assembly language
\texttt{Yul}, is very different from \texttt{Solidity} itself and
requires much greater understanding of the syntax and semantics of that
assembly language.

So the use of \texttt{Solidity} assembly not only affects readability
and maintainability, but also the auditability, because the auditors
themselves might not be aware of the \texttt{Yul} language: the syntax
and semantics.

All these aspects result in the recommended best practice of trying to
avoid \texttt{Solidity} assembly as much as possible or, if absolutely
required, then the developers and the security review should
double-check to make sure that they have been used appropriately in the
context of the smart contracts.
