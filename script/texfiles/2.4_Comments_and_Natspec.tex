\section{Comments \& NatSpec}\label{comments-natspec}

\texttt{Solidity} supports single line comments and multiline comments
as shown here

\begin{lstlisting}[language=Solidity,numbers=none]
// This is a single line comment
/* This is a 
multiline
comment */
\end{lstlisting}

Comments are recommended to be used as inline documentation of what the
contracts are supposed to do, what the functions, variables,
expressions, various control and data flow expected to do as per the
specification and what is really implemented. They can also be used to
specify certain assumptions that the developer is making in the
implementation and they can also represent some of the invariants that
need to be maintained.

Comments become a critical part of documentation that is included or
encapsulated within the code itself, affect the readability of the code
to a great extent and maintainability. In fact, comments become critical
when we start talking about evaluating the security of smart contracts:
comments give a lot of vital clues as to what the developer intended to
implement or just information related to the various syntax or the
semantics itself.

\texttt{Solidity} also supports a special type of comment called
\textbf{NatSpec} which stands for \textbf{Ethereum Natural Language
Specification Format}. These are specialized comments that are specific
to \texttt{Solidity} and \texttt{Ethereum}. They are written as follows

\begin{lstlisting}[language=Solidity,numbers=none]
/// This is a single line NatSpec comment
/** This is a 
multi line NatSpec comment */
\end{lstlisting}

and are located directly above the function declaration or statements
that are relevant to the \textbf{NatSpec}. These NatSpec comments come
in many different types: there are many different tags such as

\begin{itemize}
\tightlist
\item
  \textbf{\texttt{@title}}: describes the contract or the interface.
\item
  \textbf{\texttt{@author}}: specifies the developer (i.e.~who is
  authoring the contract).
\item
  \textbf{\texttt{@notice}}: explains to an end user what the contract
  or function does.
\item
  \textbf{\texttt{@dev}}: directed towards the developer for any extra
  implementation related details.
\end{itemize}

There are also specific tags related to function parameters
(\texttt{@param}), the return variable (\texttt{@return}) and so
on\ldots{} These NatSpec comments are meant to automatically generate
\texttt{JSON} documentation for both developers as well as users and
provide a lot of valuable information that the developer intended for
all these various aspects of parameters, returns, contracts and so
on\ldots{} They also form an important piece of the toolset that helps
evaluate smart contract security.
