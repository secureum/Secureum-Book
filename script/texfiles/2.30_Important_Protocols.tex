\section{Important Protocols}\label{important-protocols}

There are many protocols currently living in the EVM. The protocols
presented here are few of the most important protocols that an auditor
\textbf{must} be familiar with.

\subsection{\texorpdfstring{\texttt{WETH}}{WETH}}\label{weth}

Protocols often work with one or many \texttt{ERC20} tokens, be it
either their own or of other protocols. They also work with the Ether
that is sent to their smart contracts via \texttt{msg.value}. Instead of
having two separate sets of logic and two separate sets of control flow
within their contracts (one to deal with Ether, the other to deal with
\texttt{ERC20} tokens), it would be very convenient if we could have a
single logic, a single control flow to deal with both Ether and
\texttt{ERC20} tokens.

The \texttt{WETH} concept provides this capability: it allows smart
contracts to convert Ether that's been sent to their contracts to its
\texttt{ERC20} equivalent which is known as WETH. This conversion is a
process called \textbf{wrapping}, while the other direction of
converting the \texttt{ERC20} equivalent of WETH back to Ether is called
\textbf{unwrapping}.

This is made possible by sending the Ether to a \texttt{WETH} contract
which converts it into its \texttt{ERC20} equivalent at a $1:1$ ratio.
There are multiple versions of WETH contracts the most popular right
now, is the
\href{https://etherscan.io/address/0xc02aaa39b223fe8d0a0e5c4f27ead9083c756cc2\#code}{\texttt{WETH9}
contract} which holds anywhere between 6.5 to 7 million Ether as of this
point.

There are also some improvements being done: there is \texttt{WETH10}
that is more Gas efficient than the version 9. This version also
supports flash loans as per the \texttt{EIP3156} standard. So this WETH
concept is something that we often come across in smart contact
applications.

\subsection{Uniswap V2}\label{uniswap-v2}

Uniswap is an automated market making protocol on Ethereum. That's
powered by what is known as a constant product formula:

\[xy=k\]

where $x$ and $y$ are token balances of two different tokens and
$k$ is their constant product.

Uniswap allows liquidity providers to create pools of token pairs, and
whenever anyone provides liquidity to either of the two tokens of the
token pair, new tokens known as LP tokens liquidity provided tokens are
minted and sent back to the liquidity provider. This represents their
share of the liquidity in the tokens.

Uniswap is the most popular protocol on Ethereum currently for swapping
between tokens belonging to a token pair, and a big part of that is
because of the simplicity of the constant product formula as determined
by the curve $xy=k$.

Uniswap also provides support for on-chain Oracles. A price Oracle is a
tool that allows smart contracts to determine the price information
about a given asset on the blockchain. In the case of Uniswap V2, every
token pair measures the price of further tokens at the beginning of each
block.

So, in effect this is measuring the price at the end of the previous
block that is maintained within a cumulative price variable that's
weighted by the amount of time this price has existed for the token
pair. This particular variable can be used by different contracts on the
Ethereum blockchain to track what is known as ``\emph{time weighted
average prices}'' (TWAPs) across any particular time interval.

\subsection{Uniswap V3}\label{uniswap-v3}

Uniswap recently introduced their version 3 of the protocol, which is
considered as a big improvement over their version 2. This improvement
is specifically around the concept of concentrated liquidity. What this
means is it allows liquidity providers to provide liquidity for the
token pair, across custom price ranges instead of across the entire
constant product curve.

This brings about a big improvement to their \textbf{capital
efficiency}. This version of the protocol also introduces flexible fees
across different values as shown here.

Finally, for Oracle support, version V3 introduces advanced TWAP support
where the cumulative sum instead of being maintained and trapped in one
variable is now done so in an array. This allows smart contracts to
query the TWAP on demand for any period within the last 9 days.

\subsection{Chainlink}\label{chainlink}

Chainlink is perhaps the most widely used Oracle and source of price
feeds for smart contracts on Ethereum. Price data and even other kinds
of data are taken from multiple off-chain data providers and they are
put on-chain to create these feeds by the decentralized Oracles on the
chainlink network.

Chainlink has mechanisms for aggregating this data across the various
data providers and itself provides an extensive set of APIs for working
with these Oracles and price feeds.
