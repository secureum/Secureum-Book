\section{web2 vs.~web3: The Paradigm
Shift}\label{web2-vs.-web3-the-paradigm-shift}

In this section, we are going to assume that you are familiar to some
extent with web2 and most of the content will be focused on web3 and the
differences with web2.

\subsection{Objectives of web3}\label{objectives-of-web3}

The idea of web3 is for it to be a permissionless, trust minimized and
censorship resistant network for transfer of not only information but
also value.

Privacy and anonymity are again very big motivating factors in web3 and
have a huge implication on how we think about security in this space and
how we can actually conceive implementing various security measures.
However it isn't a completely fresh start: there are a lot of web2
security principles, best practices and software engineering best
practices that have been researched, experimented, developed and refined
over the last 3 or 4 decades that still apply very much to web3.

In the web3 world the aspirational goal is that of borderless,
permissionless innovation and censorship resistance. This inspires all
the applications, smart contracts or any other off-chain components to
be open and composable by design.

Composability means designing components and applications in a modular
way, so that other modules (or other applications) can interface with
them to increase the utility that is got from either of the two
components. This has to be done in a way that is very easy,
permissionless and effortless. Again, going back to the web2 space, a
lot of the work that has happened, a lot of the applications that are
built by the various enterprises, for various reasons such as protecting
their business interests, are designed in such a way that they work well
within their ecosystem or their suite of products, and they're not
really meant to be very composable or very interactive with other
applications or other components from other entities or other vendors
(potentially their competitors).

In the web3 world the aspirational goal is that of borderless,
permissionless innovation and censorship resistance. This inspires all
the applications, smart contracts or any other off-chain components to
be open and composable by design. This means that anybody (any user, any
part of the world) can deploy applications and interact with it (be it
contracts, be it any other component). This drives innovation in the
web3 space, which is great and has resulted in a very accelerated
innovation and compressed time. Again this has implications to to how
security is thought of and what is practically possible with security in
this space. When you have this unconstrained composability, where any
smart contract, any application that is deployed can be acted upon and
be combined or connected with any other smart contract on the chain,
then it leads to sort of an explosion of the dependencies and
configurations that are possible. This makes characterizing web3
vulnerabilities and exploit scenarios very challenging, because it
requires really a very deep knowledge of all these interacting
composable components along with their very different constraints that
could change because of composability itself, and their configurations
could be affected by interactions with other components.

Having said that, there are still many aspects of web3 security which
are really a paradigm shift.

\subsection{Open-source \& Transparency}\label{open-source-transparency}

Due to web3's ethos (or the design approach) towards everything here
being open source and transparent, the way it is thought about the
security of the ecosystem has to be changed.

We have known open source in the web2 world for several decades, so
that's not very new. But from an approach perspective, in the web2 space
we do see most of the products, or many of the products, being
proprietary from a licensing and from a source code availability
perspective. The web3 ethos stems from the permissionless aspect, the
trust minimization aspect and the censorship resistance aspect.

In the context of smart contracts, they're again expected to be open
sourced and also expected to be verified contracts. In this case it
means that the bytecode (machine instructions) of the contracts deployed
on the blockchain are expected to be source code verified using one of
the services available (such as
\href{https://etherscan.io/}{Etherscan}). This means that the source
code of the contract is the same one that was compiled, deployed and
that users interact with. That verification is something critical and
expected by default in this space.

Remember that everything that happens in the blockchain is transparent;
anyone can actually run an Ethereum node so that you don't even have to
rely on a block explorer or any service like that (more on that in
upcoming secitons). You can look at transactions as they happen in real
time on the blockchain, you can also look at transactions that are
waiting to get into the blockchain (through mempool explorers). One can
even do that by running an Ethereum node themselves. \textbf{There is no
security by obscurity}.

\subsection{Code Immutability}\label{code-immutability}

In the context of a blockchain being immutable (because of all the
blocks being linked with hashes), PoW and what it requires for somebody
to go back in time and change one of the blocks or any of its contents,
which contributes to the economic security and the immutability aspect
of the blockchain.

When it comes to code, the contracts that are deployed on the Ethereum
blockchain are designed in a way to be immutable as well. This means
that once a contract code is deployed, it is technically considered
immutable: it cannot be changed (at least in theory). There are some
exceptions, so theoretically it can done, but from a design perspective
(from an ethos perspective) this is not desirable. This code
immutability affects the way in which security is handled.

As it is already widely known, software will have bugs: one cannot prove
the absence of bugs, or the absence of vulnerabilities or errors in a
piece of software. Immutability affects this to a great extent: we know
that bugs will exist and, if you cannot change the contracts, then how
do we fix the code and redeploy the fixed code as we've been used to in
the web2 world (where we keep getting updates to the operating system or
to the different apps continuously to fix bugs, optimizations and so
on)?

This is something very fundamental to Ethereum or the web3 space that we
have to keep in mind.

There are practical exceptions: the deployed contract can be modified:
the functionality can be modified. This can be done in three ways:

\begin{enumerate}
\def\labelenumi{\arabic{enumi}.}
\item
  The contract can be modified and redeployed at a new address, but then
  you would have to carry over all the state. And all the users
  interacting would have to migrate to interacting with the new
  address.\\

  This is typically considered impractical but it can theoretically be
  done.
\item
  The modified contract (after bugfix or a version 2) can also be
  redeployed as a new implementation of what is known as a proxy
  pattern: the proxy contract points to an implementation contract, and
  that implementation can be modified.\\

  This is the most commonly used approach to contract upgrading, and
  again this has huge security applications if it is not done right, if
  it is not done in a trustworthy manner or if there are certain best
  practices that are not followed.
\item
  Using the \texttt{CREATE2} opcode, it allows updating a contract in
  place at the same address using the same unit code that was used to
  initialize the contract.
\end{enumerate}

We are going to further elaborate on the concepts briefly mentioned here
on upcoming sections.

\subsection{Client-Server
vs.~Peer-to-Peer}\label{client-server-vs.-peer-to-peer}

The pivotal difference between web2 and web3 is their underlying
paradigm.

\textbf{web2} relies on the \textbf{client-server paradigm}: we are used
to employ cloud services that have servers running, which we interact
with using our clients (laptops, phones, smart watches\dots) that are
\textbf{centrally managed}, i.e.~you have all these corporate entities
that determine what the infrastructure is, what the products are and
what the next versions of the products will be.

Applications are centrally rolled out and the consumers use them, but
they do not have a say in how the infrastructure is managed or how the
applications evolve. Sometimes, users don't even have a say as to what
kinds of applications can be deployed on the infrastructure: there is a
latent \textbf{censorship on web2}.

The former scenario is something that web3 is trying to get away from:
it's trying to go back to the original vision of the web which was for
it to be completely peer-to-peer without centralized entities that can
dictate what can be done on the platform, what can be deployed or who
can use it.

The idea is employing peer-to-peer communication not only for computing
power but also for storage and network, which are the building blocks of
web3. In the case of Ethereum, the peer-to-peer infrastructure that
supports these components is known as the \textbf{Ethereum triad}:

\begin{itemize}
\tightlist
\item
  \textbf{Coputing power}: Ethereum itself; Ethereum as a blockchain is
  used for decentralized compute.
\item
  \textbf{Storage}: Swarm.
\item
  \textbf{Network}: Whisper (now known as Waku).
\end{itemize}

\subsection{Business models}\label{business-models}

web2 business models are built around \textbf{freemium models} where the
basic application is free but if you want to upgrade, you have to pay
up. A lot of the business models (Google, Facebook especially) are built
around advertising, where the product is free.

In some sense, the user becomes the product and the interactions (the
data that the user shares with those platforms) is monetized in the form
of advertisements that are being delivered to the user. This is
something that we've just got used to and we don't even pay attention to
anymore.

In web3, since everything is decentralized, there has to be some
incentive for users to deploy the nodes and to contribute to the
development of the code, clients, smart contracts and applications. This
is driven by what is known as ``\emph{incentivized participation}'',
which goes back to the concept of crypto economics. This has big
implications to how security works because there is no real centralized
entity that can deal, manage and do instant response.

\subsection{Programming languages}\label{programming-languages}

Programming languages are critical because they are the means of how
projects implement their ideas, deploy them and let users interface with
them.

Again, programming languages are fundamental to the security of any
product, any architecture if you will. In the case of web2 there have
been numerous programming languages, being some of them way more popular
than the rest: it started with \texttt{C} and \texttt{C++} during the
Unix days 30 or 40 years ago, and since then we have seen different
declarative languages, subject-oriented languages and scripting
languages. Some of the most popular ones have been \texttt{Javascript}
and more recently \texttt{Go}, \texttt{Rust} and even some unique
languages such as \texttt{Nim}, that have been used to implement
Ethereum 2.0 clients recently.

All those languages are still applicable to the web3 space, because web3
is a combination of smart contracts that run on the blockchain and web
interface component (which is how users interact with the contracts on
the blockchain).

When it comes to the web component, all the web2 languages are relevant.
A lot of them are popular even here in terms of the interface, but also
in terms of the tooling that is used for the development of smart
contracts themselves.

Smart contracts themselves have a special language. In the case of
Ethereum, that's \texttt{Solidity}: it is the most widely used language
for writing smart contracts on Ethereum. There are others, like
\texttt{Vyper}, that are gaining traction, but for the majority of the
smart contracts that we see deployed, \texttt{Solidity} is that
language.

The smart contract languages were created specifically for web3, and
specifically for Ethereum in this case. So the security features of
those languages have obviously huge implications to the security of the
smart contracts themselves and the applications that are built on top of
them.

\subsection{Applications in web3:
ÐApps}\label{applications-in-web3-uxf0apps}

Building applications on a decentralized infrastructure (i.e.~a
blockchain) will cause such applications to be fundamentally different
from the mobile applications or the desktop applications that we are
used to today (as we can already guess by looking at the programming
languages and code immutability in web3). These applications are
referred to popularly as decentralized application or ÐApp.

These applications rely on a concept that is very unique to web3:
\textbf{on-chain} and \textbf{off-chain}.

\begin{itemize}
\tightlist
\item
  \textbf{On-chain} means something that is running or executing on the
  blockchain or within the blockchain's execution environment.
\item
  \textbf{Off-chain} is something that is running outside the
  blockchain.
\end{itemize}

In the web2 space everything is off-chain because, obviously, there is
no concept of blockchain. However, in order to function, the web3 space
makes use of an off-chain component combined with an on-chain component.

The off-chain component is all the web2 or ``\emph{the glue}'' that
binds the web interface to the smart contracts that are running
on-chain. This distinction is critical when thinking about security.

To put it simple, a ÐApp combines a web app/web front-end or a mobile
app (the off-chain component) that interacts with a smart contract on
the blockchain (the peer-to-peer infrastructure which is a combination
of compute, storage and network; the on-chain component). In many cases
we have one or more of any of the mentioned.

In the case of Ethereum, what is peer-to-peer is the compute part and
the peer-to-peer storage aspects (such as IPFs and some of the other
protocols that we won't talk about).

The security of web3 has to think about the security implications of
anything running or interacting off-chain with that running on-chain as
well. It's not just smart contract security but also the security of the
off-chain web or mobile app components that interface with the on-chain
components (the smart contracts).

The main difference however between web2 and web3 security is the
on-chain component of course. We will need to think about at the
pitfalls that are unique to smart contracts and look at the best
practices.

\subsection{Unstoppability \&
Immutability}\label{unstoppability-immutability}

Another difference between web2 and web3 is that the web3 applications
and infrastructure are unstoppable and immutable.

We talked about ÐApps and how they run on a decentralized
infrastructure. The goal is even for the governance of these protocols
and infrastructure to be decentralized, this means that there is no
single entity that can unilaterally decide to make changes, to start
something, to stop the application (be it adapt be the protocol, be the
surrounding infrastructure or the governance itself). Specifically, in
the case of smart contracts, there shouldn't be a single entity.

This could be the the project team itself that has built out this
application or the contracts: they shouldn't be unilaterally allowed to
change, stop and withdraw funds. This stems from the trust minimization
motivation and the censorship resistance aspect.

They are all very interconnected aspects and as you can imagine they
have huge implications to when it comes to upgrading the contracts,
fixing the bugs or doing anything with the smart contracts once they're
deployed; because they are expected in theory, by motivation and by
design (codewise; remember the code immutability we talked about
earlier) to be unstoppable and immutable.

From a security perspective, this makes it hard to deploy the software
because once you do it, if there are any vulnerabilities found, it's
very hard to do an instant response (just what we talked about on code
immutability). The latter is something we take for granted in the web2
world, where we get software updates even without our knowledge.

In the case of web3, and particularly Ethereum smart contracts, this has
huge implications to how we think about design and operationalize
security. This is the ultimate goal, in practice, as we go through
different stages of progressive decentralization.

These things are evolving, so the best practice right now is to do
something known as a ``\emph{guarded launch}'' (basically initially
limiting the functionality of the ÐApp for monitorization purposes). For
now the best practice is to have the ability to make changes, to upgrade
the contracts, to have an emergency withdrawal function, to remove the
tokens in case there is an emergency. It's sort of the stop cap measure
as we progress towards complete decentralization.

Once we have enough confidence that the contracts are running and there
are no bugs, vulnerabilities or exploits, then a lot of these things are
disabled in the code: the kill switches or the credibility aspects the
governance\ldots{}

All these things are in a progressive manner made more decentralized
over time.
