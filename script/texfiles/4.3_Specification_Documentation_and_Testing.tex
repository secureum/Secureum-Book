\section{Specification, Documentation \&
Testing}\label{specification-documentation-testing}

\subsection{Specification}\label{specification}

Specification as we have discussed earlier describes in detail the
\emph{what} and \emph{why} aspects of the project and its components. In
other words, what is the project supposed to do functionally as part of
its design and architecture as stemming from the requirements.

From a security perspective, it specifies what the assets are, where
they are held, who are the actors in this context, privileges of the
actors (who is allowed to access what and when), trust relationships,
threat model, potential attack vectors, scenarios and mitigations.

Analyzing the specification of a project provides auditors with the
above details, lets them evaluate any assumptions made and identify any
shortcomings. Few smart contract projects have detailed specifications
at their audit stage. At best they have some documentation about what is
implemented and auditors end up spending a lot of time inferring
specification from the documentation or implementation itself, which
leaves them with less time for deeper vulnerability assessment.

\subsection{Documentation}\label{documentation}

Documentation is a description of what has been implemented based on the
design and architectural requirements. It should describe in detail how
something has been designed, architected and implemented without
necessarily addressing the \emph{why} aspect (he design requirement
goals).

Documentation in smart contract projects is typically in the form of
``\emph{README}'' files in the GitHub repository describing individual
contract functionality combined with the functional NatSpec and
individual code comments.

As discussed earlier, documentation in many cases serves as a substitute
for missing specification and provides critical insights into the
assumptions, requirements and goals of the project team. Understanding
the documentation before looking at the code helps auditors save a lot
of time in inferring the architecture of the project, contract
interactions, program constraints, asset flow, actors, threat model and
risk mitigation measures mismatches between the documentation abd the
code.

The code could indicate either stale or poor documentation and software
defects or security vulnerabilities. Therefore, given this critical role
of documentation, the project team is highly encouraged to document
thoroughly so that auditors do not need to waste their time inferring
all of the aspects by reading code instead.

\subsection{Testing}\label{testing}

Software testing or validation is a well known fundamental software
engineering technique to determine if software produces expected outputs
when executed with different chosen inputs. Smart contract testing has a
similar motivation but is arguably more complicated despite their
smaller sizes in code.

Smart contract development platforms are relatively new, with different
levels of support for testing. Projects in general have very little
testing before arriving to the audit stage.

Test coverage and cases give a good indication of project maturity and
also provide valuable insights to auditors regarding assumptions and
edge cases for vulnerability assessments. Threfore auditors should
expect a very high level of testing and test coverage because it is a
must-have software engineering discipline, especially with smart
cotnracts that are by design exposed to everyone on the blockchain and
end up holding assets worth tens or hundreds of millions of dollars.

This famous quote from Dijkstra captures the role of software testing:
``\emph{program testing can be used to show the presence of bugs, but
never to show their absence}''. This is similar to what concerns
security audits.
