\section{Addresses}\label{addresses}

\subsection{Zero Addresses}\label{zero-addresses}

The zero address in Ethereum and \texttt{Solidity} has a special
consideration. Remember that Ethereum addresses are 20 bytes in length
and, if all these bytes are zeros, then it's referred to as the zero
address.

The zero address becomes significant, because state variables or local
variables of \texttt{address} type have a default value of zero in
\texttt{Solidity}. The zero address is also used as burn address because
the private key corresponding to this zero address is not known, so any
Ether or tokens that are sent to the zero address gets burnt or is
inaccessible forever.

If addresses used for access control within the smart contracts end up
being the zero address, such functions can't be invoked again because of
the lack of knowledge of the private key, which might in the worst case
ends up in such contract getting locked.

So the best practice is to perform input validation on all address
parameters that are of \texttt{address} type to check that they are not
the zero address.

This is a very commonly encountered security pitfall where address
parameters of constructor setters or public \texttt{external} functions
are not input validated to not be the zero address.

In the best case such scenarios only result in exceptional behavior at
runtime, but in the worst case they could result in tokens getting burnt
or the contract being locked.

\subsection{Critical Addresses}\label{critical-addresses}

Another security pitfall related to addresses is the aspect of changing
values of critical addresses. Certain addresses within the context of
the smart contract may be considered as critical. These may be special
privileged roles such as the owner address, which has special access to
certain functions for updating critical parameters or doing other
administrative aspects related to the smart contract.

Or these could also be addresses of other smart contracts that are used
within the context of the application. As you can imagine, there may be
scenarios where such addresses would need to be changed, so for example
the default owner of a contract could be the deployer of the contract
and we may want to change this to another address later on.

Or, if the addresses correspond to other smart contracts, then we may
want to change the value to another smart contract once we have updated
it.

In such scenarios, the security pitfall is when this change is done in a
single step, this may be using the \texttt{Owned} library from
\texttt{OpenZeppelin} where there is a transfer ownership function
provided that transfers the ownership from the existing owner to a new
owner that is provided as a parameter.

This happens in a single step where the \texttt{owner} variable is
updated to the new address provided. This single step change is prone to
errors. If an incorrect address is used as a new address, then it may
result in that contract getting locked forever. The reason for this is
the address used may be an address for which we do not have the private
key, so we can't sign any transactions from that address, which results
in all the administrative functions or any of the address change
functions being inaccessible.

Thereafter the mitigation here or the best practice is to move away from
a single step change and to move to what is known as a two-step change.

Where the first step grants or approves a new address as being the owner
or as being that changed address, the second step is a transaction from
the new address that claims itself as being the new owner or as being
the new address.

So this two-step change allows any errors that happen in the first step,
where we grant or approve it to an incorrect address for which we do not
have the key it allows us to recover from this mistake because the
second transaction which claims itself as a new address can never be
done if an incorrect address was used in the first step.

So this aspect of critical addresses being changed and allowing errors
to be recovered by moving away from a single step to a two-step change
is a critical aspect of mitigating the risk from incorrect address
changes.
