\section{Transaction Checks}\label{transaction-checks}

\subsection{\texorpdfstring{\texttt{tx.origin}}{tx.origin}}\label{tx.origin}

The use of \texttt{tx.origin} is considered dangerous in certain
situations within smart contracts. Remember that in the context of
Ethereum, \texttt{tx.origin} gives the address of the externally owned
account that originated the transaction.

If the \texttt{tx.origin} address is used for authorization, then it can
be abused by attackers to launch replay attacks by coming in between the
user and the smart contract of concern.

This is sometimes known as ``\emph{man in the middle replay attack}''
(or MITM as an abbreviation) because the attacker comes in between the
user and the contract, captures the transaction and later replaces it.
Because the smart contract uses \texttt{tx.origin}, it fails to
recognize that this transaction actually was originated from the
attacker in the middle.

So in this case the recommended best practice for smart contracts using
authorization is to use \texttt{msg.sender} instead of
\texttt{tx.origin}, because \texttt{msg.sender} would give the address
of the most recent or the closest entity. So in this case, if there is a
man in the middle attacker, then \texttt{msg.sender} would give the
address of the attacker and not that of the authorized user pointed to
by \texttt{tx.origin}.

\subsection{Contract Check}\label{contract-check}

There may be situations where a particular smart contract may want to
know, if the transaction or the call made to it is coming from a
contract account or an externally owned account.

There are two popular ways for determining that:

\begin{enumerate}
\def\labelenumi{\arabic{enumi}.}
\tightlist
\item
  Checking if the code size of the account of the originating
  transaction is greater than zero and if this is not zero, it means
  that that account has code and therefore is a contract account.
\item
  Checking if the \texttt{msg.sender} is the same as the
  \texttt{tx.origin} and, if it is, then it means that the
  \texttt{msg.sender} is an externally owned account.
\end{enumerate}

Remember that \texttt{tx.origin} can only be an externally owned account
in Ethereum as of now.

So these two techniques have pros and cons and depending on the specific
application it may make more sense to use one over the other.

There are risks associated and implications thereof of either of these
two approaches particularly with the code size approach the risk is
that, if this check is made while a contract is still being constructed
within the constructor the code size will still be zero for that account
so, if we determine based on that aspect that this is an externally
owned account, then it would be a wrong assumption.
