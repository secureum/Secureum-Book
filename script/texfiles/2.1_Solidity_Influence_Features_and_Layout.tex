\section{Solidity: Influence, Features \&
Layout}\label{solidity-influence-features-layout}

\texttt{Solidity} is a \textbf{high level language} specifically
designed for writing smart contracts on Ethereum. It was proposed in
2014 by Gavin Wood and was later developed (and continues to be
developed) by the Ethereum Foundation team led by Dr.~Christian
Reitwiessner, Alex Beregszsaszi and others.

It targets the underlying EVM and is mainly influenced by \texttt{C++}
(a lot of the syntax and object oriented programming), a bit from
\texttt{Python} (the use of modifiers, multiple inheritance, C3
linearization and the use of the \texttt{super} keyword) and some of the
early motivation was also from \texttt{Javascript} (things like function
level scoping or the use of \texttt{var} keyword, although those
influences have significantly been reduced since version
\texttt{0.4.0}).

One of the few alternatives to \texttt{Solidity} is \texttt{Vyper}: it's
a language that is mostly based on \texttt{Python} and has just started
to catch up with some of the high profile projects on Ethereum. However,
to a great extent, due to the maturity of the language and the tool
chains built around it, \texttt{Solidity} is by far the most widely
used, so it becomes critical that in order to evaluate security of smart
contracts we understand the syntax semantics, the pitfalls and various
other aspects related to it.

\texttt{Solidity} is known as a ``\emph{curly bracket language}'' (it
means that curly brackets are used to group together statements within a
particular scope), it is also an object oriented language (so there
exitsts the use of inheritance), statically typed (which means that the
types of variables defined are static and defined at compile time),
there is code modularity in the form of libraries and there are also
user defined types.

All these characteristics make \texttt{Solidity} a fully featured high
level language that allows the definition of complex logic in smart
contracts to leverage all the underlying features of the EVM.

\begin{itemize}
\item
  \textbf{So, how does the physical layout of a smart contract written
  in \texttt{Solidity} look like?}

  This is important to the readability aspect of the file and the
  maintainability aspect of the smart contract in the context of the the
  project itself.\\

  A \texttt{Solidity} source file can contain an arbitrary number of
  various directives and primitives. These include the \texttt{pragma}
  and the \texttt{import} directives, the declarations of structs, enums
  and contract definitions. Every contract can itself contain
  structures, enums, state variables, events, errors, modifiers,
  constructor and various functions that define the various
  functionalities that are implemented by the smart contract.
\end{itemize}

This physical layout is something that is specific to the syntax of
\texttt{Solidity}. When it comes to helping with the readability or the
maintainability, it is prime to layout all the components in the order
mentioned.

This is something that you will commonly see when you evaluate smart
contracts in \texttt{Solidity}. There might be cases where some of these
are out of order from what is considered as best practice, but it's
still something to keep in mind.
