\section{Solidity versions}\label{solidity-versions}

The \texttt{Solidity} language has evolved considerably in the last
several years. There have been many features added, some of them
removed. Security has been improved in several cases, optimizations have
been made.

As a result, there are many versions of \texttt{Solidity} that are
available for projects and developers to choose from. At least one
version is released every few months that make some optimizations and
fixes some bugs a couple of breaking changes are introduced every year
or so. As a result, the question is always about which version of the
\texttt{Solidity} compiler to use for a particular project, so that the
best combination of features and security aspects are considered.

The older compiler versions are time tested, but they have bugs. The
newer versions have the bug fixes which is good, but they may also have
new bugs which have been undetected so far.

The older versions have lesser features compared to the newer versions (
that usually have more features). Some of these are language level
features that are visible syntactically, others are semantic changes,
others are security features and some others are optimizations that are
not very visible.

As a result, the choice of an optimal version of the compiler for a
particular project is always a tricky thing. This has to take account
not just the functionality, but also the security aspect. As a result,
there is a trade-off to be made, there are risks as well as rewards. As
of this point many of the projects are transitioning to the
\texttt{Solidity} version \texttt{0.8.0} and beyond, because among other
things, this version has introduced default arithmetic checks for
underflow and overflows.

These aspects of security and functionality, the range of choices
available across the various \texttt{Solidity} compiler versions, have
to be kept in mind when determining which version to use for a
particular project.

\subsection{Unlocked Pragma}\label{unlocked-pragma}

Remember that \texttt{Solidity} supports the concept of \texttt{pragma}
directives and one of them is related to the \texttt{Solidity} compiler
version, that can be used with this smart contract.

There are many aspects related to that fragment directive, but the one
that is relevant from a security perspective, is the concept of that
\texttt{pragma} being unlocked or floating and what this means is that
in the \texttt{pragma} directive that specifies the compiler version, if
the caret (\texttt{\^{}}) symbol is used, then it is referred to as
being unlocked.

What this means, is that the use of this caret symbol, specifies that
any compiler version starting from the one specified in that
\texttt{pragma} directive all the way to the end of that breaking
version can be used to compile this smart contract. As an example, if
the \texttt{pragma} directive is \texttt{\^{}0.8.0} it means that any
compiler version from \texttt{0.8.0} all the way to the last version in
the \texttt{0.8.z} range can be used according to this \texttt{pragma}
for compiling this smart contract.

This becomes interesting from a security perspective. The use of such an
unlocked or floating \texttt{pragma} allows one \texttt{Solidity}
compiler version to be used for testing, but potentially, a different
one that is used for compiling the contracts while being deployed.

This aspect of using a different version for testing and deployment is
risky from a security perspective. That's because one could test with a
totally different set of compiler features and security checks, the
newer version or a different version that is used for deployment may
support a different set of features and a different set of security
checks, so this mismatch between testing and deployment is allowed by
the use of this unlocked \texttt{pragma} and hence is not recommended to
be used.

So what is recommended is to \texttt{lock} the \texttt{pragma} by not
using the caret symbol in that \texttt{pragma} directive, this will
enforce that the same compiler version is used for testing as well as
for deployment.

\subsection{Multiple Pragma}\label{multiple-pragma}

Another security aspect related to the use of the solution compiler
\texttt{pragma} in contracts is the use of different pragmas across
different contracts within a single project.

Remember that the \texttt{pragma} applies only to the contract where it
is used so. If there are different multiple contracts that are used
within a single project, then each one of them could have a different
\texttt{pragma} specifying a different compiler version.

The reason why this is not recommended is because these different
compiler versions like we just discussed can have different bugs,
different bug fixes, different features and even different security
checks across the versions. This will result in different components of
the application having different security properties which is not
desirable.

So from a security perspective, what is recommended is to use the same
\texttt{pragma} across all the different contracts that form that smart
contract application. This will result in all of them having the same
set of bugs, features and security checks which can be accounted for
while one is testing that smart contact application.
