\section{Math and Logic}\label{math-and-logic}

\subsection{Overflow/Underflow}\label{overflowunderflow}

This security pitfall is related to the notion of overflows and
underflows in \texttt{Solidity} smart contracts. This is applicable to
any integer arithmetic that is used within the contracts which is very
often encountered.

When such arithmetic is used in a way where the increments or decrements
to those integer variables are done without checking for the bounds,
then they could result in wrapped values where the value exceeds the
maximum storage for that integer type and hence overflows or wraps to
the lower end of that type or, If it's being decremented it could be
decremented below zero in which case it results in wrapping to the
maximum value of that integer type.

If those extremely high or extremely low data values resulting because
of wrapping are invalid in the applications logic, then it is okay. But
if it is not, if it's valid in the applications logic, then this could
result in unexpected behavior in the best case or in the worst case it
could result in some very serious vulnerabilities that can be exploited.
We have seen multiple vulnerabilities and exploits led to overflow and
underflow historically.

So the recommended best practice is to use the SafeMath libraries from
OpenZeppelin that enforce the overflow and underflow checks during
integer arithmetic or to use the latest \texttt{Solidity} versions
greater than or equal to \texttt{0.8.0} that introduce check arithmetic
by default.

\subsection{Dividing before
Multiplying}\label{dividing-before-multiplying}

Another security pitfall or best practice related to integer arithmetic
is dividing before multiplying. \texttt{Solidity} integer division might
truncate the value of results therefore, if division is done before
multiplication, then this may result in the loss of precision of the
values being computed.

So the recommended best practice is to always do the multiplication
operations first followed by any division that is required.

\subsection{Strict Equalities}\label{strict-equalities}

From a security perspective strict equalities are considered as
dangerous in specific contexts of the smart content applications.

Strict equality is referred to the ``\texttt{==}'' operator or the
``\texttt{!=}'' operator as compared to the less stricter
``\texttt{\textless{}=}'' or ``\texttt{\textgreater{}=}'' operators.

When these strict equalities are applied to Ether or token values, then
such checks could fail because the transferred Ether or tokens could be
slightly less or greater than what the strict equalities expect or the
balances computed could be different because of the different number of
decimals expected or the precision of the operations being slightly
different from the assumptions being made. Hence the use of strict
equalities with such operands and operations is considered dangerous
because they could lead to failed checks.

\textbf{So the security best practice is to default to less stricter
equalities and make sure that those constraints are satisfied as per the
assumptions.}

\subsection{Tautologies \&
Contradictions}\label{tautologies-contradictions}

An interesting security consideration is that of tautologies and
contradictions. A tautology is something that is always true whereas a
contradiction is something that is always false.

Within smart contracts this can be found in certain primitives used,
such as an unsigned integer variable \texttt{x} and then there is a
predicate that checks, if \texttt{x} is greater than or equal to 0. This
predicate because of \texttt{x} being an unsigned integer is a tautology
it's always going to be true because \texttt{x} can't take a negative
value.

The presence of such tautologies or contradictions in smart contracts
indicates either flawed logic or mistaken assumptions made by the
developer or these may just be redundant checks.

In either scenario these may be interesting from a security perspective,
so it is something to be paid attention to and flagged as potential
concerns.

\subsection{Boolean Constant}\label{boolean-constant}

The use of boolean constants \texttt{true} or \texttt{false}, directly
in conditionals is unnecessary.

The reason for this is that if there's a conditional whose predicate is
true, then that can be removed because that code block would get
executed nevertheless and similarly, if the predicate is the boolean
constant \texttt{false}, then that could be removed as well and along
with the code in that associated block because that code would never
execute because the conditional is always going to be \texttt{false}.

So these usages of boolean constants specifically within conditionals is
indicative of flawed logic or assumptions or they could just be used in
a redundant manner. The recommendation upon identifying such usage, it
is removing those constants and any code blocks associated with them, so
that it becomes simpler to read and to maintain.

\subsection{Boolean Equality}\label{boolean-equality}

An aspect related to boolean constants is that of boolean equality, this
is where the boolean constants \texttt{true} or \texttt{false} are used
within conditionals for an equality check, so the \texttt{x} variable is
checked against the \texttt{true} constant.

This usage is redundant because the variable \texttt{x} can be used
directly within the conditionals predicate without actually comparing it
to \texttt{true} and both of them are equivalent.

So the use of the boolean constant \texttt{true} within the predicate is
actually unnecessary, so while this may not be a big security
consideration and perhaps indicative of the developer not fully
understanding how \texttt{Solidity} booleans work.

It is interesting from an optimization perspective and certainly
improves the readability aspect of the code.
