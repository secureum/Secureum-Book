\section{Transaction Order
Dependence}\label{transaction-order-dependence}

This security pitfall is related to \textbf{transaction order
dependence} (TOD for short). Remember that in Ethereum transactions
submitted by users sit in a data structure known as the \texttt{mempool}
and get picked by the different miners for inclusion within blocks.

The specific transactions that are picked, the specific order of those
transactions included within the blocks depends on multiple factors and
specifically the Gas Price of those transactions itself.

So from an attacker's perspective one can monitor the \texttt{mempool}
for interesting transactions that may be exploited by submitting
transactions with a Gas Price appropriately chosen, so that the
attackers transaction either executes right before or right after the
interesting transaction. This is typically known as
\textbf{Front-running} and \textbf{Back-running} and may lead to what
are known as \textbf{Sandwich Attacks}.

All these aspects are related to assumptions being made on the
transaction being included in a specific order by the minor within a
block.

So from a security perspective logic within smart contracts should be
evaluated to check, if transactions triggering that logic can be front
run or background to exploit any aspect of it.

\subsection{\texorpdfstring{ERC20 \texttt{approve()} Race
Condition}{ERC20 approve() Race Condition}}\label{erc20-approve-race-condition}

A classic example of transaction order dependence is the
\texttt{approve()} functionality in the popular \texttt{ERC20} token
standard. Remember that the \texttt{ERC20} token standard has the notion
of an owner of a certain balance of those tokens and there's also the
notion of a spender which is a different address that the owner of
tokens can approve for a certain allowance amount which the spender is,
then allowed to transfer.

Let's take an example to see how the race condition works. Let's say
that I am the owner of a certain number of tokens of an \texttt{ERC20}
contract and I want to approve a particular spender with 100 tokens of
allowance, so I go ahead and do that with an \texttt{approve(100)}
transaction and later I change my mind and I want to reduce the
allowance of the spender from 100 to 50.

So I submit a second approve 50 transaction and, if that spender happens
to be malicious or untrustworthy and monitors the \texttt{mempool} for
this approval transaction, they would see that I'm reducing their
approval to 50 by noticing the \texttt{approve(50)} transaction.

In that case they can front run the reduction of approval transaction
with a transaction that they send that spends the earlier approved
hundred tokens. So that goes through first because of Front-running and
when my \texttt{approve(50)} transaction goes through that, would give
the spender an allowance of 50.

Now the spender would further go ahead and spend those 50 tokens as
well, so effectively instead of allowing the spender to spend only 50
tokens I have let them spend 150 tokens of mine, this is made possible
because of transaction order dependence or Front-running.

The mitigation to this the best practice recommended is to not use the
\texttt{ERC20} \texttt{approve()} that is susceptible to this
race-condition, but to instead use the \texttt{increaseAllowance()}, the
\texttt{decreaseAllowance()} functions that are supported by such
contracts.
