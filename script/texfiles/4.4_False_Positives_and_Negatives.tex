\section{False Positives \& Negatives}\label{false-positives-negatives}

Let's now talk about the concept of false positives and false negatives,
which are critical to understand in the context of smart contract audits
or security.

\subsection{False Positives}\label{false-positives}

False positives are findings which flag the presence of vulnerabilities,
but which in fact are not vulnerabilities. They could arise due to
incorrect assumptions or simplifications in analysis which do not
correctly consider all the factors required for the actual presence of
vulnerabilities.

False positives require further manual analysis on findings to
investigate if they are indeed false positives or if they are true
positives. A high number of false positives increases the manual effort
required in verification and also lowers the confidence in the accuracy
of findings from the earlier automated analysis.

On the flip side, true positives might sometimes be incorrectly
classified as false positives, which leads to the vulnerabilities behind
those findings being ignored and left behind in the code instead of
being fixed, and may end up getting exploited later.

\subsection{False Negatives}\label{false-negatives}

On the other hand false negatives are missed findings that should have
indicated the presence of vulnerabilities, but which are in fact not
reported at all. Such false negatives again could be due to incorrect
assumptions or inaccuracies in analysis which did not correctly consider
the minimum factors required for the actual presence of vulnerabilities.

False negatives, per definition, are not reported or even realized
unless a different analysis reveals their presence, or the
vulnerabilities are realized when they're exploited. A high number of
false negatives lowers the confidence in the effectiveness of the
earlier manual or automated analysis. In contrast, true negatives are
findings that are analyzed and dismissed which are in fact not
vulnerabilities

So these concepts of true positives, false positives, true negatives and
false negatives come up often in smart contract auditing and in security
in general, and therefore this terminology (the distinction between
these types) should be well understood.
