\section{Test-in-Prod. SSLDC
vs.~Audits}\label{test-in-prod.-ssldc-vs.-audits}

\textbf{Test-in-prod} is a concept that, although it may have started as
a meme, has certainly an element of truth to it in the web3 space and
Ethereum. If you go back to the concepts of compressed time scales,
unrestricted composability of contracts and applications in this space,
the byzantine threat model and the challenges of replicating the full
state of a live blockchain in a test setting; all these are really what
make testing in a testing environment very hard.

Again, this contasts with the web2 world where there are very clear
distinctions between a test environment and a production setting for
various reasons of owning the complete stack, the maturity of the tools,
the lack of sort of unconstrained composability with arbitrary
components outside of the stack for that particular product.

All those aspects that are very well defined in the web2 space from a
testing perspective, are very challenging to set in the web3 space. This
is further complicated by the maturity of the tools that are still
experimental in some sense in the web3 space, and also the mechanism
design aspect of it: the attackers and the users potentially being the
abusers.

All these things come together to make testing, which is really
fundamental when thinking about security and making something more
secure or getting a better level of assurance from the product, very
hard to do because the real world failure models cannot be replicated
very easily in a test environment.

This implies that it forces ``\emph{realistic}'' testing to happen only
in production. In the case of web3, in the case of Ethereum, on mainnet.
So none of the testnets we talked about can match to some of the
assumptions and the constraints that their software contracts will be
subjected to.

So, the complex technical exploits (i.e.~crypto economic exploits), can
only be discoverable upon production deployment on the mainnet. This is
again a hypothesis, but it is worth thinking about.

\subsection{SSDLC}\label{ssdlc}

An interesting concept to go through, is web2's concept of
\textbf{SSDLC} which stands for \textbf{Secure Software Development Life
Cycle}. There are many approaches to this, but in general, any web2
product software, product hardware or product service has a version of
SSDLC which is used during the development life cycle.

This version guarantees that some minimum requirements have been met in
a combination of testing, internal validation and some sort of external
assessment depending on the product. It could be a product audit, a
process audit, maybe even penetration testing if it is applicable to
that product.

Also depending on the nature of assets that are managed, the risk that
is faced, the threat model that is anticipated and even the specific
sector or domain that the products are introduced in (such as he
financial sector) there exist certifications assuring that the product
application or service has to met to be succesfully deployedright. This
is prevalent in the web2 space and has evolved again over the last
several decades.

\subsection{Audits}\label{audits}

When it comes to the \textbf{web3} space, however, we do not see a
mature SSDLC yet. What we see is this concept of \textbf{audits}, and
unfortunately the life cycle of development has boiled down to building
the product (be it a smart contract or a web3 application), getting an
audit done from an external company (a security firm/individual that
specializes in, let's say, smart contract security) and then going ahead
and launching it.

There is an expectation both from the development team as well as the
users (the market in general) to perceive this audit as a silver bullet:
something that detects all the security issues in the smart contract,
fixes everything and then sort of guarantees that the product is free of
bugs and vulnerabilities when it is launched.

\textbf{Audits are not a ``\emph{stamp of security approval}''}. There
are some fundamental aspects that contribute to audits being perceived
in this fashion (at least this is a hypothesis). The big one is in
general the lack of in-house security expertise: given the rapid
innovation time scales in the space, the developers are few and there's
a huge demand for developers.

There is even a bigger demand for people who not only understand how to
develop in Ethereum and the web3 space, but to understand the security
pitfalls, which require a greater level of effort and expertise. This
lack of in-house security expertise and the challenge of wanting and
having to launch some of these protocols as fast as the team can, forces
such teams to seek out external audit firms and get these audits,
leading to think of them, market them and brand them as stamps for
security approval.

So there's this very unrealistic expectations from audits to be
``\emph{catch-all}'' for all the security vulnerabilities and bugs that
are anticipated in a smart contract or in a web3 application. For
reasons of great demand and very low supply of this expertise, these
audits are also very expensive: there are very few audit firms compared
to the demand, which leads to a vicious loop where projects want audits
but all the audit firms are really booked 6 or 9 months (depending on
the market conditions). It is a core problem in the current space and
state-of-art.
