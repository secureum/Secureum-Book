\section{Token Pitfalls}\label{token-pitfalls}

Contracts that accept manage or transfer ERC tokens should ensure
several things:

\begin{itemize}
\tightlist
\item
  They should ensure that functions handling tokens, account for
  different types of ERC tokens such as ERC20, ERC777, ERC721, ERC1155,
  \ldots{}
\item
  They should account for any deflationary or inflationary aspects of
  such tokens.
\item
  Whether they are rebasing or not.
\item
  Differentiate between trusted internal tokens and untrusted external
  tokens.
\end{itemize}

Different tokens could come with different peculiarities in terms of
their decimals of precision, their return values, reverting behavior,
support for hooks, fungibility, supporting multiple token types or
deviations from specifications. All of which could again result in
susceptibility to reentrances, locking or even loss of funds.

Therefore, functions handling tokens should be checked extra carefully
for access control input validation and error handling to ensure that
these aspects are handled correctly.

Next, we are going to go through several common token related pitfalls.

\subsection{ERC20 Transfer}\label{erc20-transfer}

This pitfall is specifically related to the \texttt{transfer()} and
\texttt{transferFrom()} functions that allow transferring of ERC20
tokens between addresses. According to the specification, these should
return \texttt{bool} values, however not all token contracts adhere to
the specification, so they may not return \texttt{bool} values: they may
not return any value at all or they may return a different value of a
different type.

In such cases, callers that assume the \texttt{bool} values to be
returned may fail, so the best practice here is for ERC20 token
contracts to make sure that they are returning \texttt{bool} values, and
for call sites to not make such assumptions: preferably using
\texttt{safeERC20} wrappers from \texttt{OpenZeppelin} that handle all
the possible scenarios where \texttt{bool}s, non-booleans or no values
are being returned (and the contract simply revoked).

\subsection{ERC20 Optional}\label{erc20-optional}

The ERC20 specification makes it optional for token contracts to
implement \texttt{name}, \texttt{symbol} and \texttt{decimals}
primitives. As a result, any contract that is interacting with ERC20
contracts should make sure that these primitives are indeed present and
implemented by those contracts. If they want to use them, the best
practice is not to make an assumption that these perimeters will always
be implemented by the ERC20 contract because they are optional.

\subsection{ERC20 Decimals}\label{erc20-decimals}

ERC20 contracts have a notion of decimals which typically are 18 digits
in precision, and therefore the token standard specifies using an
\texttt{uint8} type to represent decimals, because that is sufficient to
represent a value of 18. However, token contracts that do not adhere to
the standard sometimes incorrectly use a \texttt{uint256} type for
decimals.

The best practice here is to check which type is being used by the ERC20
contract and, if it is a \texttt{uint256} type, then have a further
check to make sure that the decimal value is less than or equal to 255,
because that is the maximum value that can fit within a \texttt{uint256}
as required by the token standard.

\subsection{\texorpdfstring{ERC20
\texttt{approve()}}{ERC20 approve()}}\label{erc20-approve}

We have talked about the race-condition risk from the \texttt{approve()}
function of ERC20. To summarize it again let's take a look at the same
example that we discussed: we have a token owner who has given an
allowance of 100 tokens (\texttt{approve(100)}) to a spender, then wants
to later decrease that allowance to 50 tokens (\texttt{approve(50)}).
The spender may be able to observe this decrease operation and before
that happens, it can frontrun in order to first spend the 100 tokens for
which they already had the allowance from earlier. Then once
\texttt{approve(50)} operation succeeds, they further spend those 50
tokens as well.

So effectively they have ended up spending 150 tokens while the owner
intended for them to only be able to spend 50 tokens. This is possible
because of frontrunning (because of the Race-condition opportunity). The
best practice here is to not use the \texttt{approve()} function, but
instead use the \texttt{increaseAllowance()} or
\texttt{decreaseAllowance()} functions that do not have this risk.

\subsection{ERC777 Hooks}\label{erc777-hooks}

We have discussed the ERC777 token standard which aims to improve some
of what are considered as shortcomings of the ERC20 standard, and one of
these improvements is the concept of hooks. These hooks get called
before \texttt{send()}, \texttt{transfer()}, \texttt{mint()},
\texttt{burn()} and some other operations in these tokens. While they
may enable a lot of interesting use cases, special care should be taken
to make sure that these hooks do not make any external calls because
such calls can result in reentrancy vulnerabilities. The best practice
with ERC777 tokens is to check for their hooks and make sure that
external calls are not being made.

\subsection{Token Deflation}\label{token-deflation}

There's a concept of token deflation that may happen in ERC20 Token
contracts. Some of these token contracts may take a fee when tokens are
being transferred from one address to another. Because of this fee, the
number of tokens across all the user addresses will reduce over time
when they are transferred between those, so the number of tokens
received by the target address may not be the same as the number of
tokens sent by the sender. This depends on the amount of fee and if the
fee is being charged at all.

The best practices here with respect to token deflation is for token
contracts to generally avoid the notion of a fee that causes deflation
because that could break assumptions with the contracts that interact
with this token contract. For smart contract applications that work with
ERC20 contracts, they should be aware if those ERC20 contracts have this
notion of deflation or not, and if so, make sure that their accounting
logic takes care of this deflation. This is more of a concern in smart
contact applications that allow their users to interact with them using
arbitrary ERC20 token contracts, and in such cases consider a guarded
launch approach where the initial set of ERC20 tokens that can be used
with this contract does not have this notion of deflation.

\subsection{Token Inflation}\label{token-inflation}

ERC20 contracts could also have the opposite effect: token inflation. In
this case, contracts generate interest for their token holders. This
interest is distributed to holders while they make transfers. This
effectively increases the number of tokens that are held by the user
addresses over time, effectively meaning that when a token transfer
happens, the recipient may receive more tokens than the amount
originally sent that (reflecting the interest being distributed).

If the smart contract application is not aware of the ERC20 contract
generating interest, then those interest tokens may end up getting
trapped in the ERC20 contract without being realized. The best practice
is again to avoid this notion of interest that causes inflation because
their interacting contracts may make an assumption that no such thing is
happening that could break a lot of the critical assumptions leading to
vulnerabilities, or again such smart contact applications could consider
a guarded launch approach where the ERC20 tokens that they work with are
known not to have this notion of interest and inflation.

\subsection{Token Complexity}\label{token-complexity}

High token complexity is considered as a security risk. We have long
known that complexity in general is very detrimental to security. The
same aspect holds good for ERC20 token contracts. These contracts should
have a well-defined specification, they should be implementing a very
simple contract because any unnecessary complexity could result in bugs:
developers could make errors while developing these complex features. It
is also much harder to reason about these complex features and
definitely harder to find and fix bugs in such features, so the best
practice is at a high level to avoid any unnecessary complexity when it
comes to implementing token contracts.

\subsection{Token Functions}\label{token-functions}

In computer science, there is a notion of ``\emph{separation of
concerns}'' which says that in a computer application there should be
different sections, each of which addresses a very specific concern.
This applies to smart contact applications that work with ERC20 token
contracts as well. In this case what we mean is that a ERC20 contract
should only or mostly have functions that are relevant to ERC20 tokens.
They should not include any non-token related functions in them because
that could introduce additional complexity, and like we just discussed
complexity is detrimental to security because it could introduce bugs.
At a high level one should avoid unnecessary complexity by bundling
non-token related functions within a ERC20 token contract because that
increases likelihood of issues in general or in the worst case security
vulnerabilities.

\subsection{Token Address}\label{token-address}

ERC20 contracts should be working with a single token address. What this
means is that there should be a single address that maintains the
balances of different users interacting with that contract, thus there
is a single entry point for checking the balances of users. This is
because multiple addresses within a contract can result in multiple
entry points for the different balances that are held or maintained by
those addresses, and not being aware of these multiple addresses and
their balances can result in accounting bugs. The best practice is to
make sure that an ERC20 contract works with a single address.

\subsection{Token Upgradeable}\label{token-upgradeable}

We have talked about upgradability in the context of the Proxy contact
pattern. This upgradability is interesting in smart contract
applications where the implementation part of the Proxy can be changed
to a newer version to introduce new features or to fix any bugs in the
previous versions. When it comes to ERC20 token contracts, upgradability
is a concern. The reason is because any change in functionality that is
introduced by this upgreadeability is detrimental to the trust that the
users place in these contracts. The rationale is that token functions in
the contract are meant to be very simple: the \texttt{mint()},
\texttt{burn()} and \texttt{transfer()} functions are required to adhere
to the specifications so that all the contracts or all the users
interacting with this token contract are assured of the functionality
implemented by these functions. The best practice here is to make sure
that these token contracts are not upgradable and, if an application is
interacting with token contracts to check and verify that it is not
upgradable.

\subsection{Token Mint}\label{token-mint}

Remember that in the context of token contracts, minting refers to the
act of incrementing the account balances of addresses to which those new
tokens are credited, and in this context of token minting one should be
aware of the contract owner having any extra capabilities over this
functionality. If this is the case, then a malicious owner could
effectively mint an arbitrary number of tokens to any address of their
choice, which as you can imagine is very detrimental to the security of
the token contract because all the other users using this contract and
maintaining balances of these tokens in that contract will be affected
because their relative share of tokens will be much smaller. The best
practice here is to be aware of any such extra capabilities over this
printing functionality by the contact owner because that could be
abused.

\subsection{Token Pause}\label{token-pause}

Remember from the previous module that the ability to pause certain
contract functionality is part of what is known as a guarded launch.
However, when this guarded launch approach is applied to ERC20 tokens,
pausing some of their functionalities (like minting, burning or
transferring) could be a concern because the authorized owners (who are
allowed to pause and unpause such functionalities) or their
addresses/accounts could be compromised (or they may even be malicious),
resulting in pausing the contract functionality and trapping the funds
of all the users interacting with that contract. The best practice here
is to be aware of this risk and when interacting with token contracts to
check and verify if those contracts are possible or not by certain
authorized owners.

\subsection{Token Blacklist}\label{token-blacklist}

While the concept of blacklisting is commonly used in security for a
long time to prevent malicious actors or actions from abusing the system
this notion when applied to ERC20 contracts is of concern the reason
again is because authorized users who are allowed to create and maintain
this blacklist by adding actors or actions into that list or by taking
them out of that list. Those owners could be malicious or they could be
compromised and in such scenarios where a token contract has this notion
of a blacklist, then because of such malicious or compromised owners the
users funds could get trapped, if their addresses are blacklisted, so
the best practice is again to be aware of this risk and check and verify
contracts to make sure that they do not have this notion of a blacklist
and, if they do be aware of what can go wrong.

\subsection{Token Team}\label{token-team}

Let's now talk about the deal behind the ERC20 project and its
implications on any security aspects. The team behind the ERC20 project
may be publicly known (we know who the project members are, what their
past projects have been and how they are connected within the community)
or this team could be anonymous (the project members are only known by
their handles on github, twitter, telegram or discord\ldots{} and we
have very little information about what they have done in the past or
about their real world identities and how they are connected within the
social circles of the community).

In the latter context, there are two schools of thought:

\begin{itemize}
\item
  One school of thought thinks that an anonymous team is riskier from a
  security perspective of the project because we do not have a good ways
  to evaluate what their reputation is within the social circles, with
  the community based on their past projects and so on\ldots{}

  \begin{center}\rule{0.5\linewidth}{0.5pt}\end{center}

  In this case the assumption is that there's a greater risk a security
  risk to the project because these anonymous teams could not be as
  concerned about security, because any security implications or
  exploits might not hurt the reputation from this project, or the team
  members could also be imagined to have left behind bugs or back doors
  within the project so that later they themselves could exploit the
  project (what is known as ``\emph{rugging}'' the project).

  \begin{center}\rule{0.5\linewidth}{0.5pt}\end{center}

  This school of thought believes that such anonymous teams should meet
  a higher bar when it comes to security (or security reviews on the
  flip side).

  \begin{center}\rule{0.5\linewidth}{0.5pt}\end{center}
\item
  The other school of thought believes that anonymity (or
  pseudo-anonymity) should not matter to the security of the project.
  Any of your past projects based on who you are, what you have done and
  what your connections are within the community should not impact the
  security of a project. The project should be evaluated independently
  of who the project team members are.

  \begin{center}\rule{0.5\linewidth}{0.5pt}\end{center}

  Irrespective of which school of thought you may subscribe to, this is
  something to be kept in mind because privacy and anonymity are strong
  aspirational goals of web3 and any such team risk could potentially
  translate into a legal risk for someone who may review such projects
  or interact with it (as users).
\end{itemize}

\subsection{Token ownership}\label{token-ownership}

Token ownership refers to who owns the tokens and how many tokens they
own. In scenarios where there are very few users who own a lot of those
tokens, then such ownership situation will allow those owners to
influence the price of those tokens, the liquidity of those tokens and
any potential governance actions around those tokens, because those
actions will be controlled by the token ownership. This is an aspect of
risk from centralization because there are very few owners holding a lot
of tokens. This risk could manifest itself into a security risk as well.

\subsection{Token Supply}\label{token-supply}

It refers to the number of ERC20 tokens that is supported by the token
contract. This supply depends on what has been implemented for that
particular contract and application. It could be either low or high. The
concerning situation is when a particular ERC20 contract has a very low
supply of its tokens as this by implication means that the ownership may
end up being concentrated within a few owners who own a significant part
of the supply, in which case they have a significant influence over the
price of those tokens their liquidity and therefore their volatility, so
this scenario brings in an increased manipulation risk for such tokens
with limited supply.

\subsection{Token Listing}\label{token-listing}

ERC20 tokens get listed in various places to allow trading between
users. These tokens may get listed on centralized exchanges or
decentralized exchanges.

\begin{itemize}
\item
  Decentralized exchanges are expected to be more resilient to failures
  and therefore are expected to be up and accessible all the time.
\item
  However, if token is listed on very few centralized exchanges and
  those exchanges happen to be inaccessible because they are down for
  maintenance or maybe in extreme situations where they are hacked, then
  a concern arises because majority of the tokens will now be
  inaccessible.

  \begin{center}\rule{0.5\linewidth}{0.5pt}\end{center}

  This new low liquidity increase the price volatility of such tokens.
\end{itemize}

This is another aspect of centralization risk that one should be aware
of when looking at tokens that are listed in very few exchanges.

\subsection{Token Balance}\label{token-balance}

Assumptions on token balances pose a security risk smart contract
applications where the logic assumes that the balance of tokens that it
is working with is always below a certain threshold. These applications
stand the risk of those assumptions breaking if the balance exceeds
those thresholds. This may be triggered by users who own a large number
of tokens (typically known as whales), or it may also be triggered by
what are known as flash loans.

A \textbf{flash loan} is a capability where a user is allowed to borrow
a significant number of tokens without providing any collateral, but
this loan has to be repaid or is forced to be repaid within the
transaction itself. So by the end of the transaction, the flash loan
capability makes sure that the tokens that were lent to the user are
paid back to that contract, but within that context of the transaction
the user has access to a significant number of tokens as provided by
that flash loan contract.

Such a use of large funds or flash loans may be used by users or
attackers to amplify arbitrage opportunities or exploit vulnerabilities
where the logic incorrectly depends on load token balances. This risk
from large funds or flash loans needs to be kept in mind because it
could be manipulated.

\subsection{Token Flash Minting}\label{token-flash-minting}

Similar to flash loans, there is the concept of flash minting that has
similar concerns. Unlike flash loans, where the total amount of tokens
that can be borrowed by a user is limited by the liquidity of tokens in
that particular protocol, flash minting simply mints the new tokens that
are handed to the user. These again are only available within the
context of a transaction because at the end of the transaction, the
flash minting mechanism will destroy all the tokens that were just
minted and handed to the user.

Similar to flash loans, if smart contracts that are working with such
ERC20 tokens make assumptions about the balances of those tokens that
are available for a user, then they could lead to overflows or other
serious security vulnerabilities, these again can be manipulated and
there's a risk that needs to be kept aware of when dealing with external
ERC20 tokens.
