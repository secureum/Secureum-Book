\section{Constructor}\label{constructor}

\subsection{Constructor Names}\label{constructor-names}

Constructor names in \texttt{Solidity} have had security implications
historically. If you go back all the way to \texttt{Solidity} compiler
version \texttt{0.4.22}, the versions prior to that one required the use
of the contract name as the name of the constructor. And between that
version and \texttt{0.5.0} one could either use the contract name as a
constructor or use the \texttt{constructor} keyword itself. It was only
after \texttt{0.5.0} that \texttt{Solidity} forced the use of the
\texttt{constructor} keyword for constructors.

So this flexibility, the use of the contract name as the constructor
name, has historically caused bugs, where the contact name was
misspelled which led to that function not being the constructor, but a
regular function.

Also the flexibility between allowing both the old style and the new
style constructor names caused security issues, because there was a
precedence that was followed, if both of them existed. So this
constructor naming confusion has been a historical source of bugs and
\texttt{Solidity} smart contracts, although \textbf{it is not a concern
anymore}.

\subsection{Void Constructor}\label{void-constructor}

There's a security concern related to ``\emph{void constructors}''. What
this means is that if a contract derives from other contracts, then it
makes calls to the constructors of base contracts assuming they're
implemented, but if in fact they are not, then this assumption leads to
security implications.

So the best practice for derived contracts is to check if the base
constructor is actually implemented and remove the call to that
constructor, if it is not implemented at all.

\subsection{Constructor Call Value}\label{constructor-call-value}

This security pitfall is related to the checks for any value sent in
contact creation transactions triggering the constructor.

Typically, if a constructor is not explicitly payable and there is an
Ether value that is sent in a contract creation transaction that
triggers such a constructor, then the constructor reverts that
transaction.

However, because of a compiler bug, if contract did not have an explicit
constructor but had a base contract that did define a constructor, then
in those cases, it was possible to send Ether value in a contract
creation transaction, that would not cause that revert to happen. This
compiler bug was present all the way from version \texttt{0.4.5} to
version \texttt{0.6.8}, and thus \textbf{is not an issue anymore}.
