\section{DAppSys libraries}\label{dappsys-libraries}

We now move on to a different set of libraries provided by the DAppSys
teams at DappHub. These are used commonly in smart contracts as an
alternative to the OpenZeppelin libraries that we have discussed.

\subsection{\texorpdfstring{\texttt{DSProxy}}{DSProxy}}\label{dsproxy}

The first one is the DAppSys \texttt{DSProxy}. This implements a simple
Proxy that is deployed as a standalone contract and can be used by the
Owner to execute the code the logic that is implemented in the
implementation contract.

The user would pass in the contract byte code along with the function
call data, the call data remember that it specifies the function
selector of the function to be called along with the arguments for that
function. This library provides a way for the user to both create the
implementation contract using the bytecode provided, then delegating the
call, to that contract, the specific function, the arguments as
specified in the call data. There are associated libraries related to
DSProxy that help implement a factory contract as well as some caching
mechanism.

\subsection{\texorpdfstring{\texttt{DSMath}}{DSMath}}\label{dsmath}

DAppSys provides a \texttt{DSMath} library that provides math parameters
for arithmetic functions. The first set of primitives are arithmetic
functions that can be safely used without the risk of underflow and
overflow. These are equivalent of the SafeMath library from
OpenZeppelin. Here we can find the \texttt{add}, \texttt{sub},
\texttt{mul} functions. There is no \texttt{div} function because the
\texttt{Solidity} compiler has built-in divide by zero checking.
\texttt{DSMath} also provides support for fixed-point math.

It introduces two new types:

\begin{itemize}
\tightlist
\item
  The \texttt{Wad} type: for decimal numbers with 18 digits of
  precision.
\item
  The \texttt{Ray} type: for decimal numbers with 27 digits of
  precision.
\end{itemize}

There are different functions that help one operate on the \texttt{Wad}
and \texttt{Ray} types.

\subsection{\texorpdfstring{\texttt{DSAuth}}{DSAuth}}\label{dsauth}

The \texttt{DSAuth} library provides support for developers to implement
an authorization pattern that is completely separate from the
application logic.

It does so by providing an \texttt{auth} modifier that can be applied to
different functions and internally this modifier calls the
\texttt{isAuthorized()} function that checks, if the \texttt{msg.sender}
is either the owner of this contract or the contract itself. This is the
default functionality.

This can also be specified to check, if the \texttt{msg.sender} has been
granted permission by a specified authority.

\subsection{\texorpdfstring{\texttt{DSGuard}}{DSGuard}}\label{dsguard}

The \texttt{DSGuard} library helps implementing an access control list
(ACL). This is a combination of a source address destination address and
a function signature.

This library can be used as the authority that we just discussed in the
context of the \texttt{DSAuth} library. This implements a function
\texttt{canCall()} that looks up the access control list and determines
if the source address can call the function specified by the function
signature at the destination address.

So it's a combination of the source, destination and the signature that
determines the value of the \texttt{bool} that's either \texttt{true} or
\texttt{false}:
\texttt{{[}src{]}{[}dst{]}{[}sig{]}\ =\textgreater{}\ boolean}.

When used as an authority by \texttt{DSAuth}, the source refers to the
\texttt{msg.sender}, the destination is the contract that includes this
library, the signature refers to the function signature.

\subsection{\texorpdfstring{\texttt{DSRoles}}{DSRoles}}\label{dsroles}

The \texttt{DSRoles} library provides support for implementing
role-based access control (this is something we discussed in the context
of OpenZeppelin's \texttt{AccessControl} library as well). In this case
it implements different access control lists, that specify roles and
associated capabilities. It provides a \texttt{canCall()} function that
determines, if a user is allowed to call a function at a particular
address by looking up the roles and capabilities defined in the access
control list.

RBAC is implemented via mechanisms, there is a concept of root users,
who are users allowed to call any function regardless of what roles and
capabilities are defined for that function. There's a concept of public
capabilities that are global capabilities that apply to all users.
Finally, there are role specific capabilities that are applied when the
user is not the root user and the capability is not a public capability.
