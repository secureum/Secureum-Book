\section{Legal Aspects in web3: Pseudonymity \&
DAOs}\label{legal-aspects-in-web3-pseudonymity-daos}

When it comes to legal and regulatury aspects of \textbf{who is
responsible}, \textbf{what} are they responsible for \textbf{if
something goes wrong}, everything changes dramatically in the web3 space
compared to web2.

One of the things is the \textbf{pseudonymity} or \textbf{who} is the
team behind a particular project. There is an increasing trend towards
some of the people involved in the projects being pseudonymous. This
could be because of the regulatory uncertainty regarding
cryptocurrencies (or crypto space in general), or also be because of the
legal implications thereof.

This changes the way we think about reputation and trustworthiness when
it comes to applications, projects or products. It also affects the
legal or social accountability when it comes to projects: who is
responsible, who is accountable if the team is pseudonymous, how do you
even know what what they're doing with the project, with the governance
and so on\ldots{} There's this concept of trusting software and not
wetware, which is great but there are still social processes where
people are involved to a great extent around building the project,
rolling it out and the governance of the projects that has a huge
implication towards the security posture.

\subsection{DAOs}\label{daos}

\textbf{DAO}s (\textbf{Decentralized Autonomous Organizations}) stem
from the trust minimization and censorship resistance aspects of web3.
Their objective is to minimize the role and the influence of centralized
parties, or a few privileged individuals, in the life cycle of the
projects. This means that the project ultimately evolves or aspires to
be governed by a DAO, which can be comprised of a \textbf{community of
token holders for that particular project}. They make voting based
decisions on how the project treasury should be spent, what the protocol
changes should be and, in some of the cases, all these are decided
on-chain and affected on-chain as well.

While this reduces the centralized points of wetware failure, as we call
it, it also slows down decision making on a lot of the security critical
aspects: imagine if there were vulnerabilities to be found in a deployed
contract, and somebody had to create a fix and deploy the fix. If that
had to go through a DAO for the decision making, you would have to give
a certain amount of time for the token holders to vote for that
decision.

A centralized party entity in the web2 space can make this decision
immediately, unilaterally and deploy that fix in a few hours, if not
less. In web3 (i.e.~DAOs), the decision making is decentralized and has
that downside.
