\section{\texorpdfstring{\texttt{Solidity}
Typing}{Solidity Typing}}\label{solidity-typing}

\texttt{Solidity} is a \textbf{statically typed} language, which means
that the type of the variables used within the contracts written in
\texttt{Solidity} needs to be specified in the code explicitly at
compile time. This applies both to state variables and local variables.

Statically typed languages perform what is known as \textbf{compile time
type checking} according to the language rules. So when variables of
different types are assigned to each other at compile time, the language
can enforce that the types are used correctly across all these
assignments and usages. Many of the programming languages that you may
be familiar with such as \texttt{C}, \texttt{C++}, \texttt{Java},
\texttt{Rust}, \texttt{Go} or \texttt{Scala} are statically typed
languages. From a security perspective, the type checking is a critical
part and helps in improving the security of the contracts.

\subsection{Types}\label{types}

\texttt{Solidity} has two categories of types:

\begin{itemize}
\tightlist
\item
  \textbf{value}: they're always passed by value, which means that
  whenever they are used as function arguments or in assignments of
  expressions, they are always copied from one location to the other.
\item
  \textbf{reference}: they can be modified via multiple names all of
  which point to or reference the same underlying variable (i.e.~the
  same memory address; this is easier to understand when it is thought
  like the concept of pointers).
\end{itemize}

From a security perspective you can imagine that this becomes important
because it affects which state is being updated and what those
transitions are in the states as affected by the transactions.

\subsubsection{Value Types}\label{value-types}

As discussed value type is one of the two types in \texttt{Solidity}
where variables of these value types are passed by value (which means
they are copied when used as function arguments or in assignments of
expressions).

There are different value types in \texttt{Solidity}: booleans,
integers, fixed point numbers, address, contract, fixed size byte
arrays, literals, enums and functions themselves.

From a security perspective, value types can be thought of as being
somewhat safer because a copy of that variable is made so that the
original value of the original state itself is not modified
accidentally. But then one should also check that any assumptions around
the persistence of the values is being considered properly, so this will
become clearer once we talk about the reference types and once we look
at some of the security aspects.

\subsubsection{Reference Types}\label{reference-types}

In contrast to value types, reference types are passed by reference:
there can be multiple names for the variable, all pointing to the same
underlying variable state. There are 3 reference types in
\texttt{Solidity}: \textbf{arrays}, \textbf{structs} and
\textbf{mappings}.

From a security perspective, reference types can perhaps be considered a
little more riskier than value types because now you have multiple names
pointing to the same underlying variable, which could, in some
situations, lead to unintentional modification of the underlying state.
