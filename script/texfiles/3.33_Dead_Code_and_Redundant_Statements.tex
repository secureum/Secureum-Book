\section{Dead Code \& Redundant
Statements}\label{dead-code-redundant-statements}

\subsection{Dead Code}\label{dead-code}

Dead Code is any contract code that is unused from the contract's
perspective or even unreachable from a control flow perspective.

This could be indicative of programmer error or missing logic that leads
to the developer adding this code to the contract, but not adding the
logic that actually makes use of this code. This is certainly an
opportunity for optimization because dead code increases the code size
of the contract which, during deployment, leads to increased Gas costs.

However, this also impacts readability, maintainability and auditability
of the code, all of which affect security indirectly. Let's consider
three scenarios in which dead code affects the security of smart
contracts:

\begin{enumerate}
\def\labelenumi{\arabic{enumi}.}
\item
  There is code in the contract that is in fact dead, but the developer
  or the smart contract auditor does not realize that this is dead code.

  \begin{center}\rule{0.5\linewidth}{0.5pt}\end{center}

  If such code implements security checks, then we may assume that those
  checks are being enforced and improving the security, but in fact they
  are not effective because they are in dead code, so they reduce their
  security of the smart contracts again.

  \begin{center}\rule{0.5\linewidth}{0.5pt}\end{center}
\item
  There is dead code in the smart contract and the developers are aware
  that this code is dead, but decide to leave it (without removing it).

  \begin{center}\rule{0.5\linewidth}{0.5pt}\end{center}

  In such cases, such code may not be tested because the developers know
  that this is dead code and, because of this, they may end up with
  security vulnerabilities contained in them or they may contribute to
  such vulnerabilities.

  \begin{center}\rule{0.5\linewidth}{0.5pt}\end{center}

  Later on, if someone else decides to use this dead code, the
  vulnerabilities contained by it (or affected by it) get manifested in
  the contract and affects the security negatively.

  \begin{center}\rule{0.5\linewidth}{0.5pt}\end{center}
\item
  There is code that is actually used within the smart contracts, but
  the developers incorrectly determine that this is dead code (mistaken
  identity) and remove it.

  \begin{center}\rule{0.5\linewidth}{0.5pt}\end{center}

  In such scenarios, if that code implemented security checks are
  actually improved security because of their logic, then removing it
  reduces the security of the code
\end{enumerate}

Effectively, dead code contributes to the security of smart contracts
indirectly in potentially significant ways. The best practice is for the
developers to determine if a particular piece of code is used or dead
and, if it is dead, determine if it actually needs to be used. If it is
not, then remove it from the contracts. If it needs to be used, then add
logic that uses that code in the correct manner.

\subsection{Redundant Statements}\label{redundant-statements}

Redundant statements are statements that either have no side effects or
that do have side effects, but are made redundant because there are
other statements that have the same side effect.

In either scenario these are indicative of programmer error or missing
logic that needs to exist to make these statements not redundant, or
they may just present an opportunity for optimization where these
redundant statements need to be removed.

Removal reduces the size of the contract and therefore makes it more Gas
efficient at deploy or execution time.

The best practice here is to evaluate if statements are redundant and,
if so, determine if they should indeed be having any side effects. If
that's the case, add such side effects. If contrarily they are indeed
redundant and do not affect the security in any way, then remove them.

The impact of such redundant statements could be indirect to security
because of the errors (the logic that we talked about) or they could be
direct, where such redundant statements are actually meant to enforce
certain security checks. Because they are redundant those checks never
get executed and directly impact the security of the contract in a
negative way.
