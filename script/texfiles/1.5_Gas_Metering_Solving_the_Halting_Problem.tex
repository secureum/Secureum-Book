\section{Gas Metering: Solving the Halting
Problem}\label{gas-metering-solving-the-halting-problem}

\subsection{The Halting Problem}\label{the-halting-problem}

We talked earlier about Turing completeness and how Ethereum supports a
Turing complete programming language.

Turing completeness lends itself to a fundamental problem in computer
science: the halting problem. Imagine you have a Turing complete
programming language, then the halting problem states that: it cannot be
predicted if an arbitrary program in that language with an arbitrary
input will ever stop execution for that input.

This problem is taken for granted on our personal laptops or on our
phones: if there are programs that run into an infinite loop that hangs
them, then we just manually kill the execution.

In the context of a blockchain however, we are talking about being able
to deploy a smart contract that is not going to run on my Ethereum node
exclusively, but it's going to run on all the Ethereum nodes in the
blockchain. With that in mind, if any of those smart contracts ran into
an infinite loop, then the whole infrastructure would come to a grinding
heart, which is obviously undesirable. We would not make further
progress and everything would come to a standstill.

The way in which Ethereum deals with this issue is constraining the
resources that are given to the smart contract on the Ethereum node by
metering them through the concept of Gas.

\subsection{Gas Metering}\label{gas-metering}

The EVM runs smart contracts, which have machine code instructions.
Every single one of these instructions has a predetermined execution
cost in Gas units. So when a transaction triggers the execution of that
smart contract, and the smart contracts starts executing those
instructions, then the Gas units for those instructions are ``consumed''
by the smart contract. This is the concept of Gas metering within a
smart contract, where it accounts for every instruction (it could be a
computational instruction, a data access, \ldots{} all those
instructions consume Gas units from the transaction).

This implies that a transaction that triggers a smart contract has to
include a specific amount of Gas as required by the logic of the smart
contract. Depending on what logic needs to be executed by said smart
contract, there is a limit to the gas: a certain amount of Gas units are
required for a particular flow, so if a transaction is triggering that
it must include so many Gas units. In fact, even if the transaction is
just triggering a transfer of value, it has to supply the amount of Gas
required for what it is triggering.

If the Gas that is required during the execution of the smart contract
exceeds what is supplied, or if it exceeds the limit, then the EVM will
terminate execution and the transaction will fail. This is fundamental
to how Ethereum works, since there is a need to bound the use of
resources by anyone interacting with a smart contract that runs on all
the blockchain nodes all over the world.

Up until this point, all we have talked about is the units of Gas that
have to be provided. However, Gas itself has a price, which is measured
in Ether. So this Gas price is not fixed as it depends on the supply and
demand of Ether.

In this context it depends on the demand for the block space within
Ethereum, and if there are many applications (many smart contracts)
and/or users competing for that block space, then the Gas price can
increase vastly. So just like the automobile analogy (which requires Gas
or petrol, and with a fixed amount of petrol one can drive a certain
amount of kilometers), the petrol is the equivalent of the Gas units in
Ethereum (which allows to run a transaction in a certain number of
instructions within the smart contract based on the amount of Gas you
supplied). But the Gas price, similar to the price that you pay in Gas
stations (petrol bunks) is different and depends on the supply and
demand.

This Gas mechanism has recently been updated; take a look at EIP1559
(Ethereum Improvement Proposal 1559).

The take home message for this section is: one needs to obtain gas,
which is obtained by purchasing Ether. A transaction requires gas, and
if it exceeds the amount of Gas that is supplied, then the transaction
fails. But if there is more Gas than what is required, that is supplied
as part of the transaction, then the remaining Gas is sent back to the
sender who executed the transaction as part of the protocol.
