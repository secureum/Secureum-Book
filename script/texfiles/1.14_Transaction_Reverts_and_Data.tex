\section{Transaction Reverts \& Data}\label{transaction-reverts-data}

\subsection{Reverts}\label{reverts}

A transaction can revert for different exceptional conditions:

\begin{itemize}
\tightlist
\item
  The transaction could run out of Gas depending on how much was
  supplied as part of it and what that transaction actually needs when
  it is executing.
\item
  The transaction could also revert because of invalid instructions that
  are encountered as part of executing the smart contract.
\end{itemize}

When the transaction gets reverted, all the state changes made in the
context of the EVM so far from all the previous instructions in the
contract are discarded, and the original state before the transaction
started executing is restored. It is as if the transaction never
executed from the perspective of the EVM state.

\subsection{Data}\label{data}

Recall that the data field within a transaction is relevant when the
recipient of said transaction is a contract account. In that case, the
transaction data contains two components:

\begin{itemize}
\tightlist
\item
  It has to specify the function of that contract that is being invoked
  and\ldots{}
\item
  If that function requires any arguments, then it needs to specify
  those as well
\end{itemize}

All this is encoded according to the \textbf{Application Binary
Interface (ABI)}: it's the contract's interface that's specified in a
standard way, so that contracts can interact with each other. This is
critical for contracts to interact both from outside the blockchain
(when a transaction is triggered targeting a destination contract) but
also for messages that are sent between two or more contracts within the
EVM context.

These interface functions that are specified as part of the ABI are
strongly typed, are known at compilation time and they are static: the
types of the function parameters are well known at compile time and they
cannot change, because if they did, then what is specified as part of
the contract call during execution will not reflect to what the
destination contract requires in terms of the function encoding, or in
terms of the arguments that are supplied.

\begin{itemize}
\item
  So, \textbf{how does a callee contract specify the function to be
  invoked on the destination contract}?\\

  It does that through the \textbf{function selector}. The way that it
  is specified is by taking the function signature (of the function that
  needs to be invoked), running that through a Keccak-256 hash and
  taking the first four bytes of the output hash.
\item
  How is this function signature calculated from the function
  declaration?\\

  The function name is taken and appended with the parenthesized list of
  the parameter types that it accepts. These parameter types are
  specified one after another, with the comma being the delimiter.\\

  Note that there are no spaces used (this is something that is enforced
  as part of the ABI and it's a standard, because if different contracts
  use different notations for function signatures, then you can imagine
  that when a transaction triggers a contract and sends the function
  selector, the receiving contract will not know which function to
  execute).\\

  So everyone has to know what the format is. This allows the EVM to
  function in a very deterministic manner.\\

  Besides the function selector we have the function arguments that are
  also part of the transaction data (like we just discussed). These are
  encoded as well immediately following the four bytes of function
  selector: they span from the fifth byte onwards and go on depending on
  the number of arguments that the particular function needs.
\end{itemize}
