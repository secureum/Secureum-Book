\section{Access Control}\label{access-control}

Access control is perhaps the most significant and fundamental aspect of
security. When it comes to smart contracts, what it means is access to
functions. Incorrect or insufficient access control or authorization
related to system actors, rules, assets and permissions, may certainly
lead to security issues. Indeed, the notion of assets, actors and
actions in the context of trust and threat models should be reviewed
with the utmost care to avoid such security issues.

Remember that functions can have different visibility. \texttt{public}
and \texttt{external} functions are those that can be called by any user
interacting with the smart contract.

So from an access control perspective, we need to make sure that the
right set of addresses can call these functions. We need to ask know if
it might be okay for anyone to access these functions, any address to
access this function, or it might be required only for the Owner to
access this or there could be an extensive role based access control
that is desirable as well.

This means that when we are reviewing smart contracts for security, we
need to make sure that the right access control is enforced by the use
of the correct modifiers. That make sure that the correct checks are
enforced on the different sets of addresses used with the smart
contract. Any of these missing checks either missing modifiers or the
use of incorrect addresses or even the access control specification
might allow attackers to control critical logic that is executed within
some of these critical functions.

\subsection{Withdrawal of Funds}\label{withdrawal-of-funds}

Smart contracts typically manage a significant amount of funds related
to the amount of Ether or the \texttt{ERC20} tokens that they hold and
manage it in different ways for different users. So they have different
functionality for users to deposit these funds and similarly they have
different mechanisms for users to withdraw their funds.

These withdrawal functions need to be protected, from an access control
perspective. What this means is that, if these withdrawal functions are
unprotected, that's if they are public and external and they do not have
the right access control enforced on the different addresses via checks
implemented within the modifiers applied on these functions, then it may
let attackers call these unprotected withdrawal functions and withdraw
Ether or \texttt{ERC20} tokens that belong to other users. This
unauthorized withdrawal leads to loss of funds for the users and loss of
funds for the protocol itself.

So in this context of withdrawal of funds access control again becomes
important, the security checks have to make sure that the right access
control is applied with respect to the different addresses or different
modifiers on these withdrawal.

\subsection{\texorpdfstring{\texttt{selfdestruct}}{selfdestruct}}\label{selfdestruct}

The use of the \texttt{selfdestruct} primitive is critical and dangerous
from a security perspective. Remember that \texttt{SELFDESTRUCT} is an
EVM instruction that is further supported by a \texttt{Solidity}
primitive, which when used within the smart contract, destroys or kills
that contract and transfers all its Ether balance to the specified
recipient address.

So from a security perspective, any smart contract that uses
\texttt{selfdestruct} within a particular function, needs to protect
access to that function because, if not, an user can mistakenly call
that function or an attacker can intentionally call that function to
kill that contract and remove its existence thereafter.

This means that from a security perspective, unauthorized calls to
functions within smart contracts that may use the self-destruct
primitive should be prevented, so that the contract does not get killed
intentionally or mistakenly.

Access control to such functions again becomes critical to make sure
that only authorized users may call such functions. At a high level,
even the use of self-destruct is considered as being very risky and
dangerous from a security perspective.
