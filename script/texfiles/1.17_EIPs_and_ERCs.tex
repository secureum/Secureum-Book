\section{EIPs \& ERCs}\label{eips-ercs}

\subsection{EIPs}\label{eips}

EIP stands for \textbf{Ethereum Improvement Proposal}: proposals put
forward by researchers, developers and/or community members in the
Ethereum ecosystem to make changes to different aspects of the Ethereum
protocol.

There's a very well defined specific process for EIP from the time
somebody proposes one to the way it is discussed, debated, voted upon
and finally made it into a standard or a specification.

Depending on the different layers of the Ethereum protocol, the proposal
targetting these could be either addressing the core aspects of the
protocol, the networking aspects, the interface or some of the token
standards.

\subsection{ERCs}\label{ercs}

ERC stands for \textbf{Ethereum Request for Comments}. It has (sort of)
become the used term for token standards. For example you have probably
heard about ERC20 token standard or ERC721 token standard and so
on\ldots{} These are being created as part of the EIP process.

There are also some meta and informational EIPs that don't address the
protocol as such, but that address some of the governance aspects of
this whole ecosystem, the process and so on\ldots{} They also address
some of the informational aspects of how these standards and
specifications are written and distributed within the community.
