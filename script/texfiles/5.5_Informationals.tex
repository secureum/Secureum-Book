\section{Informationals}\label{informationals}

\subsection{ConsenSys Diligence}\label{consensys-diligence}

\subsubsection{Umbra}\label{umbra}

This finding was a ConsenSys Diligence audit of Umbra where it was an
informational issue related to specification and documentation of token
behavior restrictions.

For context, as with any protocol that interacts with arbitrary ERC20,
tokens it is important to clearly document which tokens are supported.
This is best done by providing a specification for the behavior of the
expected ERC20 tokens and only relaxing the specification after careful
review of a particular class of tokens and their interactions with the
protocol.

The recommendation was that node deviations from normal ERC20 behavior
should be explicitly noted as not supported by Umbra protocol, such as
deflationary tokens or fee on transferred tokens. These are tokens in
which the balance of the recipient of a transfer may not be increased by
the amount of the transfer there may also be some alternative mechanism
by which balances are unexpectedly decreased.

While these tokens can be successfully sent, the internal accounting of
Umbra contract will be out of sync with the balance as recorded in the
token contract resulting in loss of funds. For inflationary tokens, the
Umbra contract provided no mechanism for claiming positive balance
adjustments. For rebasing tokens (a combination of deflationary and
inflationary tokens; rebasing tokens are tokens in which an account's
balance increases or decreases along with expansions or contractions in
their supplies) this applies too.

This is related to token deflation via fees in 107, total inflation via
interest in 108, garden launch via asset types and 129, and broader
aspects of token handling in 159, system specification and documentation
in 136, 137 and accounting issues in 171, that we discussed in security
pitfalls and best practices 201 model.

\subsubsection{DeFi Saver}\label{defi-saver}

\begin{itemize}
\item
  This finding was a ConsenSys Diligence audit of DeFi where it was an
  informational issue related to testing: the test suite was not
  complete and many of the tests failed to execute.\\

  For complicated systems such as DeFi Saver, which uses many different
  modules and interacts with different DeFi protocols, it is crucial to
  have a full test coverage that includes edge cases and fail scenarios,
  which is critical for safer development and future upgrades.\\

  As seen in some smart contact incidents, a complete test suite could
  have prevented issues that might be hard to find with manual reviews.
  So the recommendation was to add a full coverage test suite.\\

  This is related to the broad aspect of testing in 155 that we
  discussed in security pitfalls and best practices 201 modules.
\item
  Another informational finding from ConsenSys audit of DeFi Saver was
  related to naming, documentation and refactoring, where
  \texttt{hyperGetRatesCode} was unclear because function names did not
  reflect their true functionalities. Additionally, the code used some
  undocumented assumptions as well.\\

  The recommendation was to refactor the code to separate
  \texttt{getRateFunctionality} with \texttt{getSellRate} and
  \texttt{getPiRate}, and also to explicitly document any assumptions in
  the code.\\

  This is related to broad aspects of system documentation in 137 and
  clarity in 188 that we discussed in security pitfalls and best
  practices to one module.
\end{itemize}

\subsubsection{DAOfi}\label{daofi}

\begin{itemize}
\item
  This finding was a ConsenSys Diligence audit of DAOfi where it was an
  informational issue related to documentation, where stale comments
  about storage slots were present in the codebase. The recommendation
  was to remove such stale comments.\\

  This is related to comments in 154 redundant constructs in 157 and
  broad aspects of system documentation in 137 that we discussed in
  security pitfalls and best practices to one module.
\item
  Another informational finding from ConsenSys Diligence audit of DAOfi
  was related to unnecessary code, or logic, where there was an
  unnecessary call to \texttt{DAOfiV1Factory} \texttt{formula()}
  function. For context, few \texttt{DAOfiV1} pair functions used a
  external function, which made a call to the factory to retrieve the
  immutable formula address set in the constructor. Such calls could
  simply be replaced with that immutable value instead. The
  recommendation was therefore to remove such unnecessary calls and
  replace them with variable reads.\\

  This is related to broad aspects of redundant constructs in 157, the
  principle of economy of mechanism in 197 that we discussed in security
  pitfalls and best practices 201 module.
\item
  Another informational finding from ConsenSys Diligence audit of DAOfi
  was related to testing, where increased testing of edge cases in
  complex mathematical operations, could have identified at least one
  issue raised in the report.\\

  The recommendation was adding additional unit tests as well as Fuzzing
  or property based testing of curve related operations for more
  validation of mathematical operations.\\

  This is related to the broad aspect of testing in 155 and numerical
  issues in 170 that we discussed in security pitfalls and best
  practices to one module.
\end{itemize}

\subsubsection{MStable}\label{mstable}

This finding was a ConsenSys Diligence audit of DAOfi where it was an
informational issue related to documentation, where there was a mismatch
between what the code implemented and what the corresponding comment
described. The recommendation was to update the code or the comment to
be consistent.

This is related to comments in 154 and broad aspects of system
documentation in 137 and clarity issues in 188 that we discussed in
security pitfalls and best practices 201 module.

\subsubsection{1inch}\label{inch}

\begin{itemize}
\item
  This finding was a ConsenSys Diligence audit of 1inch where it was an
  informational issue related to documentation and testing, where the
  source code hardly contained any inline documentation which made it
  hard to reason about functions and how they were supposed to be
  used.\\

  Additionally, the test coverage seemed to be limited, whereas
  especially for a public facing exchange contract system test coverage
  should have been extensive covering all functions especially those
  that could be directly accessed including potentially security
  relevant and edge cases. This would have helped in detecting some of
  the findings raised with the report.\\

  The recommendations were to consider adding NatSpec format compliant
  inline code documentation, describing what were functions used for and
  who was supposed to interact with them. i.e.~documenting specific
  assumptions and and increasing test coverage.\\

  This is related to system documentation in 137 comments in 154 and
  broad aspects of testing in 155 and clarity issues in 188 that we
  discussed in security pitfalls and best practices 201 modules.
\item
  Another informational finding from ConsenSys Diligence audit of DAOfi
  was related to configuration, where the compiler version
  \texttt{pragma} was unspecific in that it was floating or unlocked
  with caret \texttt{\^{}0.6.0}.\\

  That often makes sense for libraries to allow them to be included with
  multiple different versions of an application. It may be a security
  risk for the application implementation itself: a known vulnerable
  compiler version may accidentally be selected for deployment or
  security tools might fall back to an older compiler version, ending up
  actually checking a different version from what is ultimately deployed
  in the blockchain.\\

  The recommendation was to avoid floating parameters and pin a concrete
  compiler version, the latest without known security issues in at least
  the top level deployed contracts to make it unambiguous as to which
  compiler version was being used. The suggested rule of thumb was that
  a flattened source unit should have at least one non-floating concrete
  \texttt{Solidity} compiler version.\\

  This is related to unlocked \texttt{pragma} in number two of security
  pitfalls and best practices 101 module and broader aspects of tests in
  155 of security pitfalls and best practices 201 model.
\end{itemize}

\subsubsection{Growth DeFi}\label{growth-defi}

\begin{itemize}
\item
  This finding was a ConsenSys Diligence audit of Growth DeFi where it
  was an informational issue related to specification and documentation,
  where the only documentation for Growth DeFi was a single README file,
  as well as some code comments.\\

  A system's design specification and supporting documentation should be
  as important as the system's implementation itself because users rely
  on high level documentation to understand the big picture of how a
  system works.\\

  Without spending time and effort to create such documentation, a
  user's only resource is the code itself, something the vast majority
  of users can't understand.\\

  Security assessments depend on a complete technical specification to
  understand the specifics of how a system works when a behavior is not
  specified or is specified incorrectly security assessments must base
  their knowledge in assumptions leading to less effective review.\\

  Also maintaining and updating code relies on supporting documentation
  to know why the system is implemented in a specific way, if code
  maintainers can't reference documentation they must rely on memory or
  assistance to make high quality changes. The recommendation therefore
  was to improve system documentation and create a complete technical
  specification.\\

  This is related to broad aspects of system specification and
  documentation in 136 and 137 undefined behavior issues in 179 and
  clarity issues in 188 that we discussed in security pitfalls and best
  practices 201 modules.
\item
  Another informational finding from ConsenSys Diligence audit of Growth
  DeFi was related to naming convention and readability, in which the
  codebase made use of many different contracts, abstract contracts,
  interfaces and libraries for inheritance and code reuse. This is in
  principle a good practice to avoid repeated use of similar code, but
  without descriptive naming conventions to indicate which files would
  contain meaningful logic, the code pairs became difficult to navigate.
  The recommendation was to use descriptive names for contracts and
  libraries.\\

  This is related to broad aspects of programming style code layout and
  any convention in number 97 to 101 of \texttt{Solidity} 101 module and
  clarity issues in 188 principle of economy of mechanism at 197 and
  principle of psychological acceptability in 199 that we discussed in
  security pitfalls and best practices 201.
\end{itemize}

\subsubsection{Aave CPM}\label{aave-cpm}

This finding was a ConsenSys Diligence audit of Aave CPM where it was an
informational issue related to testing of ChainLink's performance at
times of price volatility. The recommendation was that, in order to
understand the risk of the Oracle deviating significantly from the
price, it would help to compare historical prices on Chainlink, when
prices were moving rapidly and see what the largest historical delta was
compared to the live price on a large exchange.

This is related to the broad aspect of testing in 155 external
interaction in 180 and freshness issues in 185 that we discussed in
security pitfalls and best practices 201 module.

\subsubsection{Lien Protocol}\label{lien-protocol}

This finding was a ConsenSys Diligence audit of Lien Protocol where it
was an informational issue related to configuration, where the concern
was that the system had many components with complex functionality
leading to a high attack surface, but no operand upgrade path if any
vulnerabilities were to be discovered after launch.

The recommendation was to identify which components were crucial for a
minimum viable system, to focus efforts on ensuring the security of
those components first, then moving on to others. Also to have a method
for pausing and upgrading the system at least at the early phases of the
project.

This is related to the various guarded launch approaches in 128 to 135,
the broader principles of economy of mechanism and work factor in 197
and 200 of security pitfalls and best practices 201.

\subsubsection{Balancer}\label{balancer}

\begin{itemize}
\item
  This finding was a ConsenSys Diligence audit of Balancer where it was
  an informational issue related to code factoring, where it is
  generally considered error-prone to have repeated checks across the
  codebase, and therefore it was recommended to use modifiers for common
  checks within different functions, because that would result in less
  code duplication and increased readability.\\

  This is related to function modifiers in 141 and broader aspects of
  clarity in 188 and cloning issues in 190 of security pitfalls and best
  practices 201 module.
\item
  Another informational finding from ConsenSys Diligence audit of
  Balancer was related to ordering, where \texttt{BPool} functions used
  modifiers \texttt{\_logs} and \texttt{\_lock} in that order. Because
  \texttt{\_lock} is a reentrancy guard, it should have taken precedence
  over \texttt{\_logs} in order to prevent \texttt{\_logs} from
  executing first before checking for re-entrancing.\\

  The recommendation was to place \texttt{\_lock} before other modifiers
  to ensure that it was the very first and very last thing to run when a
  function was called because we call that the order of execution is
  from left to right for modifiers.\\

  This is related to function modifiers and 141 incorrectly used
  modifiers in 152 and broader aspects of ordering in 178 and business
  logic issues in 191 of security pitfalls and best practices 201
  module.
\end{itemize}

\subsubsection{MCDEX V2}\label{mcdex-v2}

This finding was a ConsenSys Diligence audit of MCDEX V2 where it was an
informational issue related to codebase fragility. Software is
considered fragile when issues or changes in one part of the system can
have side-effects in conceptually unrelated parts of the codebase.
Fragile software tends to break easily and may be challenging to
maintain.

The recommendation was to prioritize two concepts:

\begin{enumerate}
\def\labelenumi{\arabic{enumi}.}
\tightlist
\item
  follow the single responsibility principle for functions where one
  function does exactly one thing and nothing else
\item
  reduce reliance on external systems.
\end{enumerate}

This is related to broad aspects of external interactions in 180
dependency 183 clarity in 188 and principle of economy of mechanism in
197 of security pitfalls and best practices 201 module.

\subsection{Trail of Bits}\label{trail-of-bits}

\subsubsection{Origin Dollar}\label{origin-dollar}

\begin{itemize}
\item
  This finding was a Trail of Bits audit of Origin Dollar where it was
  an informational issue related to error handling. For context,
  \texttt{worldCoreRebase} functions were declared to return a
  \texttt{uint}, but lacked a return statement. As a result, these
  functions would always return the default value of 0 and were likely
  to cause issues for their callers. Third party code relying on the
  return values might therefore not have worked as intended.\\

  The recommendation was therefore to add the missing return statements
  or remove the return type in those functions, then adjust the
  documentation as necessary.\\

  This is related to function \texttt{return} values in 142 and error
  reporting issues in 175 of security pitfalls and best practices to one
  module.
\item
  Another informational finding from Trail of Bits audit of Origin
  Dollar was related to inheritance, where the concern was about
  multiple contracts missing inheritances. Multiple contracts where the
  implementations of their interfaces inferred based on their names and
  implemented functions, but did not inherit from them. This behavior is
  error-prone and might lead the implementation to not follow the
  interface if the code were to be updated. The recommendation was to
  ensure that contracts inherit from their interfaces.\\

  This is related to unused constructs in 156 and undefined behavior
  issues in 179 of security pitfalls and best practices to one body.
\end{itemize}

\subsubsection{Yield Protocol}\label{yield-protocol}

This finding was a Trail of Bits audit of Yield Protocol where it was an
informational issue related to \texttt{Solidity} compiler optimizations,
where the concern was that such compiler optimizations could be
dangerous. Yield protocol had enabled optional compiler optimizations in
\texttt{Solidity}, but there have been bugs with security implications
related to such optimizations. \texttt{Solidity} compiler optimizations
are disabled by default therefore it was unclear how many contracts in
the wild actually used them and how well they were being tested and
exercised.

The short-term recommendation was to measure Gas savings from
optimizations and evaluate the trade-offs against the possibility of an
optimization related bug, and in the long term monitor the development
and adoption of \texttt{Solidity} compiler optimizations to assess their
maturity.

This is generally related to \texttt{Solidity} versions in number one
and compiler bugs in 77 to 94 of \texttt{Solidity} 101 module and
dependency issues in 183 of security pitfalls and best practices 201
module.

\subsubsection{DFX Finance}\label{dfx-finance}

\begin{itemize}
\item
  This finding was a Trail of Bits audit of DFX Finance where it was an
  informational issue related to specification, in that the \texttt{min}
  and \texttt{max} family of functions had unorthodox semantics.
  Throughout the curve contract, \texttt{minTargetAmount} and
  \texttt{maxOriginAmount} were used as open ranges, that is ranges that
  exclude the value itself. This is unlike the conventional
  interpretation of the terms minimum and maximum which are generally
  used to describe closed ranges.\\

  The recommendation was to make the inequalities in the required
  statements non-strict unless they are intended to be strict or
  alternatively document to convey that they are meant to be exclusive
  bonds. And in the long term ensure that mathematical terms such as
  minimum at least and at most are used in the typical way to describe
  values inclusive of minimums or maximums.\\

  This is related to dangerous equalities in 28 now security pitfalls
  and best practices 101 module and broad aspects of system
  specification and documentation in 136 and 137 and numerical issues in
  170 that we discussed in security pitfalls and best practices to one
  body.
\item
  Another informational finding from Trail of Bits audit of DFX Finance
  was related to configuration, in that curve being an ERC20 token
  implemented all six required ERC20 methods: \texttt{balanceOf},
  \texttt{totalSupply}, \texttt{allowance}, \texttt{transfer},
  \texttt{approve} and \texttt{transferFrom}. However, it did not
  implement the optional but extremely common \texttt{view} methods for
  \texttt{symbol}, \texttt{name} and \texttt{decimals}.\\

  The recommendation was to implement \texttt{symbol},
  \texttt{name\ and}
  decimals\texttt{on\ curve\ contracts\ to\ ensure\ that\ contacts\ confirm\ to\ all\ required\ and\ recommended\ aspects\ of\ the}ERC20`
  specification.\\

  This is related to \texttt{ERC20} name decimals and simple functions
  in 103 and configuration issues in 169 of security pitfalls and best
  practices 201 modules.
\end{itemize}

\subsubsection{Hermez}\label{hermez}

\begin{itemize}
\item
  This finding was a Trail of Bits audit of Hermez Network where it was
  an informational issue related to patching, in that contracts used as
  dependencies did not track upstream changes. For context, third-party
  contracts like \texttt{concatStorage} were copy-pasted into the Hermez
  repository, the code documentation did not specify the exact version
  used or if it was modified.\\

  This would make updates and security fixes on such dependencies
  unreliable since they would have to be updated manually. Specifically,
  \texttt{concatStorage} was borrowed from the \texttt{Solidity}
  \texttt{BytesUtils} library, which provided helper functions for
  bite-related operations and a critical vulnerability was discovered in
  that library's \texttt{slice} function, that allowed arbitrary writes
  for user supplied inputs.\\

  The recommendation was to review the codebase and document each
  dependency (source and version) and also include the third party
  sources as git submodules in the repository, so that internal path
  consistency could be maintained and dependencies could be updated
  periodically.\\

  This is related to the broad aspect of configuration issues in 165
  external interaction of 180 dependency of 183 and cloning issues in
  190 that we discussed in security pitfalls at best practices 201
  modules.
\item
  Another informational finding from Trail of Bits audit of Hermez
  Network was related to access control, in that the expected behavior
  regarding authorization for adding new tokens was unclear. For context
  \texttt{addToken} allowed anyone to list a new token on Hermez, which
  contradicted the online documentation that implied that only the
  governance should have had this authorization. It was therefore
  unclear whether the implementation or the documentation was correct.\\

  The recommendation was to update either the implementation or the
  documentation to standardize the authorization specification for
  adding new tokens.\\

  This is related to the broad aspects of guarded launch via asset types
  and 129 system specification in 136 access control in 172 and clarity
  issues of 180a we discussed in security pitfalls and best practices to
  one module.
\item
  Another informational finding from Trail of Bits audit of Hermez
  Network was related to undefined behavior, in that contract name
  duplication left the code page error-prone. The codebase had multiple
  contracts that shared the same name which allowed Builder Waffle to
  generate incorrect JSON artifacts preventing third parties from using
  their tools. Builder Waffle did not currently support a code base with
  duplicate contact names, the compilation overwrote its artifacts and
  prevented the use of third-party tools such as Slither.\\

  The recommendation was to avoid duplicate contact names change the
  compilation framework or use Slither which helps detect duplicate
  contract names.\\

  This is related to broad aspects of programming style code layout and
  aiming convention in 97 to 101 percentage 101 module and clarity
  issues in 188 principle of economy of mechanism in 197 and principle
  of psychological acceptability in 199 that we discussed in security
  pitfalls and best practices to one module.
\end{itemize}

\subsubsection{Advanced Blockchain}\label{advanced-blockchain}

\begin{itemize}
\item
  This finding was a Trail of Bits audit of Advanced Blockchain where it
  was an informational issue related to patching, in that there was use
  of hard-coded addresses which may have caused errors. For context,
  each contract needed contract addresses in order to be integrated into
  other protocols and systems. These addresses were hardcoded, which
  could have cast errors and resulted in the code basis deployment with
  a correct asset. Using hardcoded values instead of deploying provided
  values would have made these contracts difficult to test.\\

  The recommendation was to set addresses when contacts were created
  rather than using hardcoded values which would also facilitate
  testing.\\

  This is related to tests in 155 configuration and initialization
  issues in 165 and 166 that we discussed in security pitfalls and best
  practices 201.
\item
  Another informational finding from Trail of Bits audit of Advanced
  Blockchain was related to patching, in that the logic in the
  repositories contained a significant amount of duplicated code, which
  increased the risk that new bugs would be introduced into the system
  as bug fixes must be copied and pasted into files across the system.\\

  The recommendation was to use inheritance to allow code to be used
  across contracts and to minimize the amount of manual copying and
  pasting required to apply changes made in one file to other files.\\

  This is related to programming style code layout and naming convention
  in 97 to 101 of \texttt{Solidity} 101 module and broad aspects of
  configuration in 165 clarity 188 cloning in 190 principle of economy
  of mechanism in 197 and principle of psychological acceptability in
  199 that we discussed in security pitfalls and best practices 201
  modules.
\item
  Another informational finding from Trail of Bits audit of Advanced
  Blockchain was related to insufficient testing. The repositories under
  review lacked appropriate testing which increased the likelihood of
  errors in the development process and made code more difficult to
  review.\\

  The recommendation was to ensure that unit tests cover all public
  functions at least once as well as all known corner cases, and also to
  integrate coverage analysis tools into the development process and
  regularly review the coverage. This is related to broad aspect of
  testing in 155 that we discussed in security pitfalls and best
  practices 201.
\item
  Another informational finding from Trail of Bits audit of Advanced
  Blockchain was related to project dependencies containing
  vulnerabilities. Yarn audit identified off-chain dependencies with no
  vulnerabilities and due to the sensitivity of the deployment code and
  its environment it was important to ensure that dependencies were not
  malicious.\\

  The recommendation was to ensure that dependencies were tracked
  verified patched and audited. This is related to the broad aspects of
  configuration in 165 external interaction in 180 and dependency of 183
  that we discussed in security pitfalls and best practices to one
  module.
\item
  Another informational finding from Trail of Bits audit of Advanced
  Blockchain was related to documentation, where the codebase lackied
  documentation, high level descriptions and examples, making the
  contracts difficult to review and increasing the likelihood of user
  mistakes.\\

  The recommendation was to review and properly document the code base
  and also consider writing a formal specification of the protocol. This
  is related to broader aspects of system specification and
  documentation in 136 and 137, the principle of psychological
  acceptability in 199 that we discussed in security pitfalls and best
  practices 201.
\end{itemize}

\subsubsection{dForce}\label{dforce}

This finding was a Trail of Bits audit of dForce where it was an
informational issue related to poor error handling practices in the test
suite. For context, the test suite did not properly test expected
behavior and certain components lacked error handling methods, which
would cause failed tests to be overlooked. For example, errors were
silenced with a \texttt{try}/\texttt{catch} statement, which meant that
there was no guarantee that a call had reverted for the right reason. As
a result, if the test suite passed, it would have provided no guarantee
that the transaction call had reverted correctly.

The recommendation was to test operations against a specific error
message and ensure that errors were never silenced to check that a
contact call had reverted for the right reason and overall follow
standard testing best practices for smart contracts to minimize the
number of issues during development. This is related to the broad aspect
of testing in 155 of security pitfalls and best practices 201 modules.

\subsection{Sigma Prime}\label{sigma-prime}

\subsubsection{Synthetix Ether Collateral
Protocol}\label{synthetix-ether-collateral-protocol}

\begin{itemize}
\item
  This finding was a Sigma Prime audit of Synthetix Ether Collateral
  Protocol where it was an informational issue related to redundant and
  unused code. For example, the \texttt{recordLoanClosure} function
  returned a \texttt{bool} which was never used by the calling function,
  and there were also some \texttt{if} statements that were redundant
  and unnecessary.\\

  The recommendation was to remove such redundant constructs or use them
  in meaningful ways. This is related to redundant construction 157 of
  security pitfalls and best practices 201 module.
\item
  Another informational finding from Trail of Bits audit of Synthetix
  Ether Collateral Protocol was related to unused event logs, in that
  log events were declared, but never emitted.\\

  The recommendation was to emit these events where required
  appropriately or remove them entirely. This is related to unused
  constructs in 156 and auditing and locking in 173 of security pitfalls
  and best practices 201 module.
\end{itemize}

\subsubsection{InfiniGold}\label{infinigold}

This finding was a Sigma Prime audit of InfiniGold where it was an
informational issue related to an unnecessary \texttt{require} statement
in \texttt{blacklistable} contract, which implemented a zero-address
check on the \texttt{to} address, when this check was also implemented
in the \texttt{transfer} function of ERC20 contract.

The recommendation was to consider removing the \texttt{require}
statement for Gas saving purposes. This is related to redundant
constructs in 157 now security pitfalls and best practices 201.

\subsection{OpenZeppelin}\label{openzeppelin}

\subsubsection{HoldeFi}\label{holdefi}

\begin{itemize}
\item
  This finding was a OpenZeppelin audit of HoldeFi where it was an
  informational issue related to business logic. The concern was about
  insufficient incentives to liquidators. For context, the liquidation
  process is a very important part of every DeFi project because it
  addresses the problem of a system being under collateralized when the
  market finds itself in critical conditions and therefore needs a
  design that incentivizes speed of liquidation execution, as per modify
  specification and implementation the liquidators would end up paying
  for the expensive liquidation process without receiving any benefit
  other than buying potentially discounted collateral assets.\\

  The recommendation was to consider improving the incentive design to
  give liquidators higher incentives to \texttt{execute} the liquidation
  process this is related to function invocation timeliness in 143 and
  incentive issues in 187 of security pitfalls and best practices 201
  module.
\item
  Another informational finding from OpenZeppelin audit of HoldeFi was
  related to patching. The concern was that the project re-implemented
  some of OpenZeppelin's libraries and copied them as is in some others,
  instead of importing the official ones. OpenZeppelin maintains a
  library of standard, audited, community reviewed and partly tested
  smart contracts. Re-implementing or copying them increases the amount
  of code that the HoldeFi team would have to maintain and missed all
  the improvements and bug fixes that the OpenZeppelin team was
  constantly implementing with the help of the community.\\

  The recommendation was to consider importing the open zipline
  libraries instead of re-implementing or copying them and further
  extend them where necessary to add extra functionalities this is
  specifically related to cloning issues in 190 of security pitfalls and
  best practices 201.
\item
  Another informational finding from OpenZeppelin audit of HoldeFi was
  related to event emission. The concern was that there was a lack of
  indexed parameters in events throughout the whole device codebase.\\

  The recommendation was to consider indexing event parameters to
  facilitate off-chain services searching and filtering for specific
  events because remember that indexed event parameters are put into the
  topic part of the event log, which is faster to look up than the data
  part. This is specifically related to unindexed event parameters and
  46 or security pitfalls and best practices 101 module and broadly
  related to auditing logging issues in 173 of security pitfalls and
  best practices 201 modules.
\item
  Another informational finding from OpenZeppelin audit of HoldeFi was
  related to naming conventions. The concern was that there was an
  inconsistent use of named return variables across the codebase that
  affected explicitness and readability.\\

  The recommendation was to consider removing all named return variables
  explicitly declaring them as local variables in the function body and
  adding the necessary explicit return statements where appropriate.
  This is related to function return values in 142 explicit over
  implicit in 164 and clarity issues in 188 of security pitfalls and
  best practices 201 module.
\end{itemize}

\subsubsection{BarnBridge}\label{barnbridge}

\begin{itemize}
\item
  This finding was a OpenZeppelin audit of BarnBridge where it was an
  informational issue related to conventions. The concern was about a
  \texttt{require} statement that made an assignment which deviates from
  standard usage and intention of \texttt{require} statements, and could
  lead to confusion.\\

  The recommendation was to consider moving the assignment to its own
  line before the \texttt{require} statement. Then, using the
  \texttt{require} statement only for condition checking. This is
  related to \texttt{assert}/\texttt{require} state change in 51 of
  security pitfalls and best practices 101 module and broader aspects of
  error reporting in 175 and clarity issues in 188 of security pitfalls
  and best practices 201 module.
\item
  Another informational finding from OpenZeppelin audit of BarnBridge
  was related to stale comments. The concern was that the codebase had
  lines of code that had been commented up. This could lead to confusion
  and affected code readability and auditability.\\

  The recommendation was to consider removing commented out lines of
  code that were no longer needed. This is related to comments in 154
  and clarity issues in 188 of security pitfalls and best practices 201.
\end{itemize}

\subsubsection{Compound}\label{compound}

This finding was a OpenZeppelin audit of Compound where it was an
informational issue related to misleading error messages. Error messages
are intended to notify users about failing conditions and should provide
enough information so that appropriate corrections needed to interact
with the system can be applied. Uninformative error messages affect user
experience.

The recommendation therefore was to consider reviewing the code pairs to
make sure error messages were informative and meaningful and also reuse
error messages for similar conditions. This is related to error
reporting issues in 175 clarity issues in 188 and principle of
psychological acceptability in 199 of security pitfalls and best
practices 201 modules.

\subsubsection{Fei}\label{fei}

\begin{itemize}
\item
  This finding was a OpenZeppelin audit of Fei where it was an
  informational issue related to \texttt{Solidity} versions. The concern
  was about multiple outdated \texttt{Solidity} versions being used
  across contracts. The compiler options in the Truffle config file
  specified version \texttt{0.6.6} which was released in April 2020, and
  throughout the codebase there were also different versions of
  \texttt{Solidity} being used.\\

  The recommendation was that given \texttt{Solidity}'s fast release
  cycle to consider using a more recent version of the compiler and
  specifying explicit \texttt{Solidity} versions in \texttt{pragma}
  statements to avoid unexpected behavior. This is related to
  \texttt{Solidity} versions unlocked \texttt{pragma} and multiple
  \texttt{Solidity} \texttt{pragma}s in 1, 2 and 3 of security pitfalls
  and best practices 101 module and explicit over implicit in
  configuration in 165 and dependency issues in 183 of security pitfalls
  and best practices 201.
\item
  Another informational finding from OpenZeppelin audit of Fei was
  related to test and production constants being in the same codebase.
  For example, the \texttt{coreOrchestrator} contract defined the
  \texttt{testMode} boolean variable which was then used to define
  several other test constants in the system. This decreased the
  legibility of production code and made the system's integral values
  more available.\\

  The recommendation was to consider having different environments for
  production and testing with different contracts. This is related to
  tests in 155 and configuration issues in 165 of security pitfalls and
  best practices 201 module.
\item
  Another informational finding from OpenZeppelin audit of Fei was
  related to the use of unnecessarily smaller sized integer variables.
  In \texttt{Solidity}, using integers smaller than 256 bits (which is
  the EVM word size) tends to increase Gas Cost because the EVM must
  then perform additional operations to zero or mask out the unused bits
  in remaining parts of storage slots for such integers. This can be
  justified by savings and storage costs in some scenarios. However that
  was not the case for this code base.\\

  The recommendation was to consider using integers of size 256 bits to
  improve Gas efficiency. This is related to system specification in 136
  and principle of economy of mechanism in 197 of security pitfalls and
  best practices 201 module.
\item
  Another informational finding from OpenZeppelin audit of Fei was
  related to the use of \texttt{uint} instead of \texttt{uint256} across
  the codebase.\\

  The recommendation was to consider replacing all instances of
  \texttt{uint} with \texttt{uint256} in favor of explicitness. This is
  related to explicit over implicit in 164 and clarity 188 of security
  pitfalls at best practices 201.
\end{itemize}

\subsubsection{UMA Protocol}\label{uma-protocol}

This finding was a OpenZeppelin audit of Fei where it was an
informational issue related to functions with unexpected Side-effects.
For example, the \texttt{getLatestFundingRate} function of the
\texttt{fundingRateApplier} contract might also update the funding rate
and send rewards. The \texttt{getPrice} function of the
\texttt{optimisticOracle} might also settle a price request. These
setter-like side-effect actions were not clear in the getter-like names
of functions and were thus unexpected and could lead to mistakes if the
code were to be modified by new developers not experienced in all the
implementation details of the project.

The recommendation was to consider splitting such functions into
separate getters and setters, or alternatively consider renaming the
functions to describe all the actions that they performed. This is
related to broad aspects of programming style code layout and naming
convention in 97 to 101 of \texttt{Solidity} 101 module and clarity in
188 principle of economy of mechanism in 197 and principle of
psychological acceptability in 199 that we discussed in security
pitfalls and best practices 201 module.

\subsubsection{GEB Protocol}\label{geb-protocol}

\begin{itemize}
\item
  This finding was a OpenZeppelin audit of GEB Protocol where it was an
  informational issue related to missing error messages in
  \texttt{require} statements. There were many places where
  \texttt{require} statements were correctly followed by their error
  messages, clarifying what the triggered exception was. However, there
  were also places where \texttt{require} statements were not followed
  by corresponding error messages. If any of those required statements
  were to fail the check condition, the transaction would revert
  silently without an informative error message.\\

  The recommendation was to consider including specific and informative
  error messages in all \texttt{require} statements. This is related to
  error reporting issues in 175 clarity issues in 188 and principle of
  psychological acceptability in 199 of security pitfalls and best
  practices to one module.
\item
  Another informational finding from OpenZeppelin audit of GEB Protocol
  was related to the use of Assembly in multiple contracts. While this
  did not pose a security risk per se, it is a complicated and critical
  part of the system. Remember that the use of Assembly discards several
  important safety features of \texttt{Solidity} which may render the
  code unsafe and more error-prone.\\

  The recommendation therefore was to consider implementing thorough
  tests to cover all potential use cases of these functions to ensure
  that they behaved as expected and to consider including extensive
  documentation regarding the rationale behind its use, clearly
  explaining what every single Assembly instruction does, which would
  make it easier for users to trust the code, for reviewers to verify it
  and for developers to build on top of it or update it.\\

  This is related to Assembly usage in 63 of security pitfalls and best
  practices 101 module and broader aspects of system documentation in
  137 tests in 155 clarity 188 and principle of psychological
  acceptability in 199 security pitfalls and best practices 201.
\item
  Another informational finding from OpenZeppelin audit of GEB Protocol
  was related to the \texttt{try}/\texttt{catch} statements. The concern
  was about the \texttt{catch} clause not being handled appropriately in
  a couple of functions. The \texttt{catch} clause of
  \texttt{Solidity}'s \texttt{try}/\texttt{catch} exception handling
  primitive was neither emitting events nor handling the error, but
  simply continuing the execution.\\

  The recommendation was that, even if continuing execution after a
  possible fail was something explicitly wanted, to consider handling
  the \texttt{catch} clause by either emitting an appropriate event or
  registering the failed (tricol?) in the spirit of the failed early and
  loudly principle. This is related to error reporting in 175 clarity in
  188 and principle of psychological acceptability in 199 of security
  pitfalls and best practices 201 modules.
\item
  Another informational finding from OpenZeppelin audit of GEB Protocol
  was related to unnecessary event emission. For example, the
  \texttt{popDebtFromQueue} function of the \texttt{accountingEngine}
  contract was emitting an unnecessary event whenever it was called with
  a debt \texttt{block.timestamp} that had not been saved before.\\

  The recommendation was to remove such event emits and prevent Gas
  wastage. This is related to redundant constructs in 157 of security
  pitfalls and best practices 201.
\end{itemize}

\subsubsection{Opyn Gamma}\label{opyn-gamma}

This finding was a OpenZeppelin audit of Opyn Gamma Protocol where it
was an informational issue related to mismatches between contracts and
interfaces. Interfaces define the exposed functionality of implemented
contracts. However, in several interfaces there were functions from the
counterpart contracts that were not defined.

The recommendation was to consider applying necessary changes in the
mentioned interfaces and contracts, so that definitions and
implementations fully match. This is related to system specification in
136 undefined behavior in 179 and clarity issues in 188 of security
pitfalls and best practices 201.

\subsubsection{Set Protocol}\label{set-protocol}

This finding was a OpenZeppelin audit of Set Protocol where it was an
informational issue related to clearing address variables by setting
them to zero-addresses instead of using \texttt{delete}.

The recommendation was to consider replacing assignments of zero with
\texttt{delete} statements because \texttt{delete} better conveyed the
intention and was also more idiomatic. This is related to explicit over
implicit in 164 cleanup in 167 and clarity issues in 188 of security
pitfalls and best practices 201 module.
