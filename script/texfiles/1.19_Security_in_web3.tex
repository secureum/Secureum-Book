\section{Security in web3}\label{security-in-web3}

\subsection{Architectures, Languages \& Tools: from web1 to
web3}\label{architectures-languages-tools-from-web1-to-web3}

Going all the way back to the advent of internet 40 or 50 years ago, the
various protocols that were developed as part of the TCP stack, some of
the competing ones, the way they were standardized, then the advent of
the world wide web that really launched the web1 to the world, the
concept of browsers, the concept of web applications, the client-server
paradigm\ldots{}

Then came the web2: this is where the enterprises (be it IBM, Microsoft
entered the picture), the introduction of Linux to the world, various
hardware architectures, various operating systems, the dominance of
Microsoft, Apple and more lately Facebook, Amazon and the likes\ldots{}
All these have contributed immensely to the development and the maturity
of the web2 ecosystem over the last 40 years.

This has huge implications to the security in that ecosystem as well,
which has been developed in tandem with all those technologies over all
those years: the firewalls, anti-viruses, intrusion detection systems,
intrusion prevention systems, various kinds of security systems for
email, for the world wide web, for your personal laptops\ldots{}

They've evolved with the technology stack, with the various languages,
with the various systems, the new use cases and so on\ldots{} More
lately, if you think about the entire ecosystem of smartphones, the apps
around it, the advent of the iPhone, Android\ldots{} They didn't exist
15 years ago and they entirely changed the way applications were built,
deployed, distributed, the containerization withing those mobile devices
and the security of those apps\ldots.

Now contrast that with Ethereum; with web3 in general. Ethereum itself
is not more than 6 to 8 years old protocol that got inspired from
Bitcoin. Bitcoin itself is not more than 12 or 13 years old. This entire
ecosystem, and specifically the technical stack of Ethereum (starting
from the protocol and going to the EVM) has again taken a lot of
inspiration from some simple architectures from the web2 space, but has
some very unique properties: like that of 256 bit words or more uniquely
or the associated Gas semantics (which has no parallel in the qeb2
world).

The same happens if you look at the languages that are used to write
smart contracts, the developer tool chain that is critical to building
deploying monitoring applications on Ethereum (Foundry, Hardhat,
Truffle, Ape, Brownie, OpenZeppelin libraries\ldots), they are barely 3
to 4 years old. There's an order of magnitude of difference with the
web2 world.

If you look at the security tools like \texttt{Slither} from Trail of
Bits, \texttt{MythX} from ConsenSys Diligence and some of the others
from OpenZppelin and other companies in the space; although they're
fantastic tools, they've been around for not more than 4 or 5 years. The
test of time, evolution and adaptation of these tools to differing use
cases, protocols and needs is very critical when you start thinking
about implications to security, and all these are not happening in a
very coordinated manner. They're all happening in different timelines by
different teams around the world, often not very coordinated.

\subsubsection{The Byzantine Threat
Model}\label{the-byzantine-threat-model}

This is central to how security is thought about and critical to how
security is designed. \textbf{web3} is all about what is known as the
\textbf{Byzantine Threat Model}, which is based around the byzantine
generals problem.

\textbf{web2} has very defined concepts of \textbf{trusted insiders} and
\textbf{untrusted outsiders}. Some of this has changed over the years
because there is obviously a huge aspect of insider threat that has been
recognized in the web2 system as well. But if you look at the products
and their evolution of in the web2 security space, be it anti-viruses,
firewalls or any of the network security (perimeter security devices and
applications), there is still an aspect of insiders and outsiders.

This goes away to a great extent (if not completely) with \textbf{web3}
because in this case the threat model is really all about byzantine
fault tolerance. This means that \textbf{anyone} (including the users)
could become the \textbf{abusers of that system}. This is can be done in
a very arbitrary malicious way, which is governed by the crypto
economics (or what is known as mechanism design).

It has obviously big implications to how security is designed and
deployed because you have arbitrarily malicious adversaries that are
motivated by mechanism design, and these adversaries could be users,
intermediaries or people who are thought of as being critical to the
ecosystem. They could include anyone: developers, miners, validators,
infrastructure providers and users.

This is the main reason why in web3, security aspects are challenging
and it's the underpinning of web3 being untrusted by default, where the
users could become the abusers. web3 is the ultimate zero trust
scenario.

\subsection{Keys vs.~Passwords}\label{keys-vs.-passwords}

Keys and tokens are very commonly used in terminology as well as the
implementations of various protocols in the web3.

For example we have the private keys that control the EOAs in Ethereum,
which is all about the public key cryptography that is used in web3.
More specifically, in Ethereum, cryptographic keys are first class
members of the web3 world, and as much as we unknowingly use
cryptography in the web2 world, web3 is taking this to everyone because
the whole point is for the end users to take control of their assets
(their tokens) with keys that are in their control, as there is no
centralized entity that is responsible for them.

At least aspirationally, the goal is for there to be no centralized
intermediaries that can sit between you and your access credentials
(your keys) or your assets (your tokens).

Let's contrast keys with passwords (that have become synonymous with the
security) or what is wrong with security in the web2 world. For several
decades now, all of us have tens or hundreds of passwords. Most of them
very simple and reused, and very few of the users really use password
managers. But they rely on passwords being reset or changing them when
they are lost by the entity that actually controls access to the website
or to any service that is using these passwords.

That ideology of passwords is intentionally by design absent in the web3
world, at least aspirationally. The goal is that in the future web3
applications are headed towards this. The pathway to enable this is by
the use of keys that are expected to be always under the control of the
end user. So, loss of keys (or loss of the seed/secret phrases that
generate those keys) is irreversible and there is no recourse or entity
that you can go to and have them restored.

This is a significant shift in the security mindset coming from the web2
world, where passwords again are ubiquitous and we see the problems with
passwords being reused despite the use of QFAs, password databases being
dumped and the various password replacing technologies such as
biometrics still very slowly picking up adoption.

\subsection{web3 Tokens vs.~web2 Financial
Data}\label{web3-tokens-vs.-web2-financial-data}

A similar situation exists with tokens and their data equivalent in the
web2 world: the data that we have on the various services, websites or
even the financial assets (the financial data), if something happens to
them (i.e.~if they're stolen in some fashion; the worst thing that can
happen is that the private personal data is maybe sold in the dark web
and used to create accounts on your behalf or take loans for some
monetary gain, which takes a certain bit of effort on the attacker's
side because of the various checks and measures) there are technical and
regulatory measures put in place for security.

In web2 the implications of any data loss is indirect, takes time and
effort from the attacker's perspective and in some cases, because of
regulations or because of centralized entities, it can also be reversed.
With tokens that used in the web3 space used by protocols (let's say the
example of Ether or any of the cryptocurrencies), if they are taken away
from the account that you control with your private key, then there's
really no recourse unless these tokens happen to be in a centralized
crypto exchange, or in the control of some other centralized parties
that take the responsibility for any loss of such tokens.

The end user typically ends up losing those tokens irreversibly. These
are again interrelated to the immutability aspects and trust
minimization aspects of this whole space, which again contrasts between
the fines, regulations and the possible reversals on the web2 world.
