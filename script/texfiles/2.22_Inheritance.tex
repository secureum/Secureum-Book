\section{Inheritance}\label{inheritance}

Remember that \texttt{Solidity} is an object-oriented programming
language, so it supports various aspects of inheritance: multiple
inheritance and polymorphism. If you have studied other object-oriented
programming languages, a lot of these concepts must be familiar to you,
and are very similar in \texttt{Solidity}.

Languages that allow multiple inheritance have to solve some problems.
One of them is known as the \textbf{diamond problem}: this is solved in
\texttt{Solidity} in a very similar way to how it is solved in
\texttt{Python}, using what is known as \texttt{C3} linearization, that
forces a specific order in the directed acyclic graph constructed from
the base classes. At a high level, when a function is called that is
defined multiple times in different contracts (in the base and derived
classes) the given bases are searched in a specific order from right to
left, in a depth first manner and stopping at the first match that is
found. The difference between how \texttt{Solidity} implements this
versus \texttt{Python} is \texttt{Solidity} searches these classes from
right to left in the specified order as opposed to left to right in
\texttt{Python}.

\subsection{Polymorphism}\label{polymorphism}

\textbf{Polymorphism} means that a function call executes the function
of the specified name and parameter types in the most derived contract
in the inheritance hierarchy. When a contract inherits from multiple
other contracts, only a single contract is created on the blockchain
with the code from all the base contracts compiled into the created
contract.

\subsection{Function Overriding}\label{function-overriding}

\textbf{Function overriding} means that functions in the base classes
can be overridden by those in the derived classes which can change their
behavior. If they are marked as virtual using the \texttt{virtual}
keyword the overriding function must then use the \texttt{override}
keyword to specify that it's overriding the virtual function in the base
classes.

Note that virtual functions are functions without implementation. It is
mandatory for them to be marked as \texttt{virtual} outside of
interfaces. In interfaces all functions are automatically considered
\texttt{virtual}, so they don't need to use the \texttt{virtual}
keyword. However in \texttt{abstract} contracts for example, if a
function has to be considered as \texttt{virtual}, that is without
specifying an implementation, then it should specifically use the
\texttt{virtual} keyword to indicate as such. Functions with
\texttt{private} visibility can't be made \texttt{virtual}.

An interesting feature is that the overriding functions may also change
the visibility of the overridden function, but this can only be done
from changing them from \texttt{external} to \texttt{public}. The
mutability of these functions may also be changed, but only to a more
stricter one following this order:

\begin{itemize}
\tightlist
\item
  non-\texttt{payable} mutability can be changed to either \texttt{view}
  or \texttt{pure}.
\item
  \texttt{view} mutability may be changed to \texttt{pure}.
\item
  \texttt{payable} mutability is an exception: it can't be changed to
  any other mutability.
\end{itemize}

\subsection{Function Modifiers
Overriding}\label{function-modifiers-overriding}

Function modifiers can also override each other. This is very similar to
how function overriding works except that there is no concept of
overloading for modifiers. The \texttt{virtual} keyword again must be
used on the overridden modifier and the \texttt{override} keyword must
be used in the overriding modifier. Again very similar to the concept of
\texttt{virtual} and \texttt{override} functions.

\subsection{Base Class Functions}\label{base-class-functions}

When considering the inheritance hierarchy, there are base classes and
then derived classes. It is possible to call functions further up in the
inheritance hierarchy (e.g.~the base classes) from the derived classes.
If we specifically know the contract that has the function that we would
like to call, then we could specify that as shown here

\begin{lstlisting}[language=Solidity,numbers=none]
Contract.function();
\end{lstlisting}

If we wanted to call the function exactly one level higher up in the
flattened inheritance hierarchy, this can be done by using the
\texttt{super} keyword as shown here

\begin{lstlisting}[language=Solidity,numbers=none]
super.function();
\end{lstlisting}

\subsection{Shadowing}\label{shadowing}

It was supported in \texttt{Solidity} for the state variables until
version \texttt{0.6.0}. This effectively allowed state variables of the
same name to be used in the derived classes as they were declared in the
base classes. These shadowed variables could effectively be used for
purposes other than those declared in the base classes.

This was removed from version \texttt{0.6.0} onward because it caused
quite a bit of confusion and potentially could lead to serious errors
from a security perspective. As of the latest versions, state variable
shadowing is not allowed in \texttt{Solidity}. This means that
\textbf{state variables in the derived classes can only be declared if
there is no visible state variable with the same name in any of its base
classes}.

\subsection{Base Constructor}\label{base-constructor}

When you have classes deriving from other base classes, then the base
and the derived classes could have constructors. The constructors of all
the base contracts will be called following the linearization rules
(which we touched upon earlier in the context of \texttt{Solidity}). If
the base constructors have arguments, then the derived contracts need to
specify those arguments. This can be done either in the inheritance list
of the derived contract or it can be explicitly done, so within the
derived constructor itself.

\subsection{Name Collision}\label{name-collision}

Name collision is always an error in \texttt{Solidity}. It is an error
when any one of the following pairs in a contract have the same name due
to inheritance. A function and a modifier can't have the same names in
the base and derived classes. A function and an event can't have the
same name either. Finally, an event and a modifier also can't have a
same name, if this happens, then this is a compile time error.

\subsection{Contract Types}\label{contract-types}

Besides the typical contracts supported by \texttt{Solidity}, it also
supports three other contract types that are relevant when it comes to
inheritance. Those are abstract contracts, interfaces and libraries.

\subsubsection{Abstract Contracts}\label{abstract-contracts}

Abstract contracts are contracts where at least one of the functions in
the contract is not implemented. These are specified using the
\texttt{abstract} keyword.

\subsubsection{Interfaces}\label{interfaces}

Interfaces, in contrast to abstract contracts, \textbf{can't have any of
the functions implemented within them}, they can't inherit from other
contracts, all the declared functions must be external, they can't
declare a constructor and they can't have any state variables. These are
specified using the \texttt{interface} keyword.

\subsubsection{\texorpdfstring{Libraries. The \texttt{using\ for}
directive.}{Libraries. The using for directive.}}\label{libraries.-the-using-for-directive.}

Libraries are meant to be deployed only once at a specific address. The
callers call the libraries using the \texttt{DELEGATECALL} opcode. This
means that if library functions are called, their code is executed in
the context of the calling contract. Libraries are specified using the
\texttt{library} keyword.

Libraries in particular have several restrictions compared to typical
contracts: they can't have state variables, they can't inherit from
other classes or be inherited themselves, they can't receive Ether, they
can't also be destroyed, they have access to state variables of the
calling contract only, if they are explicitly supplied.

Library functions can only be called directly without the use of
\texttt{delegatecall}, if they do not modify the state, that is, if they
are \texttt{view} or \texttt{pure} functions. This is because libraries
are assumed to be stateless by default.

Additionally, \texttt{Solidity} supports \texttt{using\ for} directive,
which is used for attaching library functions to specific types in the
context of a contract. So for the directive \texttt{using\ A\ for\ B;},
\texttt{A} specifies the library and \texttt{B} specifies a particular
type.

This means that the library functions in \texttt{A} will receive objects
of type \texttt{B} as their first parameter when they are called on such
types. This directive is applicable only within the current contract,
including within all its functions.

It has no effect outside of the contract in which it is used, so for
example, if this directive is used as shown here saying
\texttt{using\ safeMath\ for\ uint256;}, it means that variables of type
\texttt{uint256} within that contract where this directive is used can
be attached functions from the \texttt{SafeMath} library.

\subsection{State Variables}\label{state-variables}

Remember that state variables in \texttt{Solidity} can have different
visibilities. One of them is \texttt{public}. \texttt{public} state
variables have automatic getter functions generated by the
\texttt{Solidity} compiler. These getters are just functions that are
generated to allow accessing the value of the \texttt{public} state
variable, so they return the value of those state variables. Such
\texttt{public} state variables can override external functions in their
base classes that have the same name as the \texttt{public} state
variables, parameter and return types of those \texttt{external}
functions match the getter function of these variables. so while public
state variables in \texttt{Solidity} can override \texttt{external}
functions according to thpse, they themselves can't be overridden.
