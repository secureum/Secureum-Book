\section{Transactions, Messages \&
Blockchain}\label{transactions-messages-blockchain}

\subsection{Distinction between Transactions and
Messages}\label{distinction-between-transactions-and-messages}

So far, we have used both the terms Transaction and Message
interchangeably, but in the context of the protocol they are actually
very different:

\begin{itemize}
\item
  \textbf{Transactions}: originate off chain (by an EOA when an external
  actor, that is, external to the blockchain, sends a signed data
  package onto the blockchain) and target an entity on the blockchain.\\

  This transaction can trigger a message that can do one of two things:

  \begin{itemize}
  \tightlist
  \item
    It can trigger a message to another EOA, in which case it leads to a
    transfer of value (transfer of Ether).
  \item
    It can trigger a message to a contract account, in which case it
    leads to the recipient contract account running its contract code
    and doing whatever the code is intended to do.
  \end{itemize}
\item
  \textbf{Messages}: the origination and the destination are both
  onchain\\

  Messages can be triggered in two ways:

  \begin{itemize}
  \tightlist
  \item
    externally by a transaction. The destination of that message could
    be another EOA or another contract account.
  \item
    internally within the EVM. This happens when a smart contract
    executes the call family of opcodes and that leads to the recipient
    contract account running its code, or value transfer to the
    recipient.
  \end{itemize}
\end{itemize}

\subsection{How to build a Blockchain}\label{how-to-build-a-blockchain}

Blocks are batches of transactions that are grouped together plus the
hash of the previous block (a cryptographic hash that is derived from
the previous flux data), creating thus a ``chain'' among the blocks.
This is how at high level a blockchain is constructed, and it is also
the fundamental reason why a blockchain is considered immutable, which
lends itself to the blockchain's integrity.

The reason for that is because if someone were to change any component
of a historical block (any of the transaction data: the destination or
any other aspect) then that change would affect all of the following
blocks: all the hashes that are included in the following blocks would
be different from the hash of the modified block, and this is something
that anybody running the blockchain or looking at the blockchain would
notice. That would break the the immutability of the blockchain.

To preserve the transaction history, blocks are ordered. Therefore,
every new block created contains a reference to its parent block and
similarly, transactions within the blocks are also strictly ordered. All
these critical aspects are the reason why the integrity of a blockchain
is maintained and prevents any fraud from happening.

\subsection{Block Header}\label{block-header}

So far we have said that the blocks in the Ethereum blockchain contain
transactions, but there's more to it: every block contains a block
header along with the transactions. The headers of the Ommer's blocks.

Each block header itself has several components to it that are critical
to how the Ethereum blockchain functions. They contain several things
such as:

\begin{itemize}
\item
  \textbf{The parent hash}: the hash of the parent block's header.\\

  This is what chains the blocks and the Ethereum blockchain together to
  make it immutable and provides the fantastic integrity property of the
  blockchain.
\item
  \textbf{The Ommer's hash}
\item
  \textbf{Beneficiary address}: the address of the Ethereum account to
  which the block reward for mining this block and all the transaction
  fees collected from the mining of the transactions included in this
  block are transferred to.\\

  This address is typically controlled by the miner who has mined this
  block.
\item
  \textbf{\texttt{stateRoot}}: one of the three root hashes of the
  modified Merkle-Patricia tree. These root hashes are 256 bit in
  length.\\

  The manner in which the state root is derived is critical to how the
  Ethereum state is captured within the blockchain: the leaves of the
  state root are \texttt{(key,\ value)} pairs of all the Ethereum
  address accounts.\\

  The keys are the Ethereum addresses of the accounts and the values
  represent the Ethereum state within that account.\\

  Recall that every Ethereum account has four fields: a \texttt{nonce},
  a \texttt{balance}, a \texttt{codeHash} and a \texttt{storageRoot}. If
  that account happens to be an EOA, then the \texttt{codeHash} and
  \texttt{storageRoot} don't really matter (they don't contain anything
  in them).\\

  But if that account happens to be a contact account, then the
  \texttt{codeHash} has the Keccak-256 hash of the code that is
  contained within that contract account, and the \texttt{storageRoot}
  of that contract account has the rootHash of another Merkle-Patricia
  tree where the leaves represent the storage that is associated with
  that contract.
\item
  \textbf{\texttt{transactionsRoot}}: one of the three root hashes of
  the modified Merkle-Patricia tree, where the leaves represent the
  transactions. Also 256 bit in length.
\item
  \textbf{\texttt{receiptsRoot}}: one of the three root hashes of the
  modified Merkle-Patricia tree, where the leaves represent the
  transaction receipts. Also 256 bit in length.\\

  \textbf{But what is a transaction receipt?}\\

  A transaction receipt can be thought of as the side effects of a
  particular transaction that are captured on the blockchain. Besides
  any changes to the account state that transactions might make, there
  are other side effects that are captured on the blockchain for this
  particular transaction. It is a tuple that contains four items:

  \begin{itemize}
  \tightlist
  \item
    \textbf{The cumulative Gas used}: the total Gas used in the block up
    until right after this particular transaction has happened, so in
    some sense it captures the ordering of the transactions within the
    block.
  \item
    A \textbf{set of logs}: related to the concept of events in
    \texttt{Solidity} (which we will study in the \texttt{Solidity}
    chapter). These are events that can be generated by any transaction
    that is captured on the blockchain.\\
    \strut \\
    They're really critical to how off-chan components, user interfaces
    and other components monitor what's happening with a smart contract.
  \item
    The \textbf{Bloom filters} specifically associated with those logs:
    they capture the indexed parameters for every event, so that one can
    query particular parameters of that event in a faster manner.
  \item
    A \emph{status quo}: what really happened with the transaction.
  \end{itemize}
\item
  \textbf{\texttt{logsBloom}}
\item
  \textbf{\texttt{dificulty}}: difficulty of the block in the context of
  PoW.
\item
  \textbf{\texttt{blockNumber}}: the number of blocks that have been
  mined so far right. This number sort of indicated the position of the
  block within the blockchain.
\item
  \textbf{\texttt{gasLimit}} (called Block Gas Limit under more formal
  contexts): the Gas limit that's specific to this block.\\

  This concept is essential to Ethereum as it dictates \textbf{the
  number of transactions that are added in this block}.\\

  This concept is different from the Gas limit which we talked about
  earlier that was specific to the transaction. This Block Gas Limit
  refers to the total Gas that is spent by all the transactions in that
  block and this effectively caps the number of transactions that can be
  included within that block.\\

  So the block size is in fact not fixed in terms of the number of
  transactions, but it's fixed in terms of the Gas used by all the
  transactions. The reason for that is that every transaction can
  consume a different amount of Gas.\\

  The Block Gas Limit is set by the Ethereum miners in a very
  interesting way: by voting on the blockchain. This is currently set to
  15 million and it has also changed over time, depending on the miners'
  voting. It also represents the level of demand there is for the block
  space on Ethereum.
\item
  \textbf{\texttt{gasUsed}}: the total Gas used by all the transactions
  in this block.
\item
  \textbf{\texttt{extraData}}
\item
  \textbf{\texttt{timestamp}}: (derived from the unix time) indicates at
  what point in time was the block was mined.
\item
  \textbf{mix-hash}: critical component of the PoW. See
  \textbf{subsection}
  \href{1.4_Ethereum_core_components.md\#Ethereum-PoW:-Present-and-Future}{\emph{\textbf{Ethereum
  PoW: Present and Future}}}.
\item
  \textbf{\texttt{nonce}}: critical component of the PoW. See
  \textbf{subsection}
  \href{1.4_Ethereum_core_components.md\#Ethereum-PoW:-Present-and-Future}{\emph{\textbf{Ethereum
  PoW: Present and Future}}}.
\end{itemize}
