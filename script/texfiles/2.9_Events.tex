\section{Events}\label{events}

Events are an abstraction that are built on top of the EVM's logging
functionality. Emitting events cause the arguments that are supplied to
them to be stored in what is known as the transactions log. This log is
a special data structure in the blockchain associated with the address
of the specific contract that created the event. This log stays there as
long as that block is accessible.

The log and its event data are not accessible from within the contracts,
not even from the contract that created them. This is an interesting
fact of logs in EVM: they're meant to be accessed off-chain and this is
allowed using \textbf{RPC}s (\textbf{Remote Procedure Calls}). So
applications, off-chain interfaces or monitoring tools can subscribe and
listen to these events through the RPC interface of an Ethereum client.

From a security perspective, these events play a very significant role
when it comes to auditing and logging for off-chain tools to know what
the state of a contract is and monitor the state along with all the
transitions that happen due to the transactions.

\subsection{Indexed Parameters in
Events}\label{indexed-parameters-in-events}

Up to three parameters of every event can be specified as being indexed
by using the \texttt{indexed} keyword. This causes those parameters to
be stored in a special data structure known as topics instead of the
data part of the log. Putting parameters into the topics part allows one
to search and filter those topics in a very optimal manner.

Parameters are commonly part of some of the specifications such as the
\texttt{ERC20} token standard, and the events in that standard. These
indexed parameters use a little more gas than the non-indexed one but
they allow for faster search and query.

\subsection{Event Emission}\label{event-emission}

Events are triggered by using the \texttt{emit} keyword. Every contract
would declare a certain set of events as relevant, and within the
contract functions, wherever these events need to be created and stored
in the log, they would be done so by using the \texttt{emit} keyword. An
example would look like this

\begin{lstlisting}[language=Solidity,numbers=none]
emit Deposit(msg.sender, _id, msg.value);
\end{lstlisting}

for the above example, let's say that we have a deposit event as part of
a particular contract, and we have specific parts of functions where we
would want to create this event and store them in the log.

So, following the example above, we specify the event plus the arguments
that are required according to the parameters of the event. These look
in some way very similar to a function call, where the event corresponds
to the function and the arguments that are supplied to it correspond to
the event parameters.

From a security perspective, it's critical for the contract and for the
developers to emit the correct event and to use the correct parameters
that are required by that event.

This is something that is sometimes missed or not paid attention to
because it's harder to be tested perhaps, and not critical to the
control flow of the contract.

But the only way for off-chain entities, any kind of user interfaces or
monitoring tools to keep track of the contract state and the transitions
is by looking at these event parameters stored in the logs.
