\section{Decentralization}\label{decentralization}

\begin{itemize}
\tightlist
\item
  \textbf{What does decentralization really mean?}
\end{itemize}

This term is used very casually although it has huge implications to how
we think about security, even for smart contracts. There is a definition
put forth by Vitalik in his article on decentralization. There are three
types of decentralization:

\begin{itemize}
\item
  \textbf{Architectural decentralization}: it refers to the hardware
  (the physical computers); who runs them, owns them, who is managing
  them, who can start them and stop them. This can be done in a
  decentralized fashion or not.
\item
  \textbf{Political decentralization}: it refers to the people behind
  the hardware or what is commonly referred to as ``\emph{wet ware}''.
  Who are the individuals or the organizations who control the hardware
  or the infrastructure? Is it just one individual or is it a group of
  individuals? Are the colluding or are they independent and
  decentralized?
\item
  \textbf{Logical decentralization}: it refers to the software: used to
  build out the applications (the framework itself, the Ethereum code,
  the protocol itself, the data structures in it, the smart contracts or
  any other software that runs in that stack). Is that decentralized? Is
  it a monolithic entity that cannot be split apart and used in a
  decentralized fashion?
\end{itemize}

All of these have security implications.
