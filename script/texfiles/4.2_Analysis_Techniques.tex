\section{Analysis Techniques}\label{analysis-techniques}

The analysis techniques used in audits involve a combination of
different methods that are applied to the project codebase along with
any accompanying specification and documentation. Many are automated
analysis performed with tools with some level of manual assistance and
there are generally eight broad categories:

\begin{itemize}
\tightlist
\item
  There's specification analysis that is completely manual.
\item
  There's documentation analysis that's also manual.
\item
  There's software testing which is automated.
\item
  Static analysis again automated.
\item
  Fuzzing.
\item
  Combination.
\item
  Automated techniques.

  \begin{itemize}
  \tightlist
  \item
    Symbolic checking.
  \item
    Formal verification s automated with some level of manual
    assistance.
  \end{itemize}
\item
  Manual analysis (that is entirely manual).
\end{itemize}

Let's discuss each of these categories in some detail.

\subsection{Static Analysis}\label{static-analysis}

Static analysis is a technique for analyzing program properties without
actually executing the program. This contrasts to software testing,
where programs are actually executed or run with different inputs to
examine their behavior.

With smart contracts, static analysis can be performed on the
\texttt{Solidity} code directly or on the EVM bytecode, and it is
usually a combination of control flow and Data Flow analysis.

Some of the most widely used static analysis tools with smart contracts
are Slither (which is a static analysis tool from Trail of Bits) and
Maru (which is a static analysis tool from ConsenSys Diligence), both of
which analyze intermediate representations derived from
\texttt{Solidity} code of smart contracts.

\subsection{Fuzzing}\label{fuzzing}

Fuzzing (or fuzz testing) is an automated software testing technique
that involves providing invalid, unexpected or random data as inputs to
software. This contrasts again with software testing in general where
chosen and valid data is used for testing.

So, firstly these invalid, unexpected or random data are provided as
inputs, then the program is monitored for exceptions such as crashes,
failing built-in code assertions or potential memory leaks.

Fuzzing is especially relevant to smart contracts because anyone can
interact with them on the blockchain by providing random inputs without
necessarily having a valid reason to do so, or any expectation from such
an interaction. This is in the context of arbitrary Byzantine fault
behavior that we have discussed multiple times earlier. The widely used
Fuzzing tools for smart contracts are Echidna from Trail of Bits, Harvey
from ConsenSys Diligence and most recently, Foundry's Fuzz testing.

\subsection{Symbolic Checking}\label{symbolic-checking}

Symbolic checking is a technique of checking for program correctness by
using symbolic inputs to represent a set of states and transitions
instead of using real inputs and enumerating all the individual states
or transitions separately. The related concept of model checking (or
property checking) is a technique for checking whether a finite state
model of a system meets a given specification, and in order to solve
such a problem algorithmically both the model of the system and its
specification are formulated in some precise mathematical language.

The problem itself is formulated as a task in logic with the goal of
solving that formula. There is decades of research and development in
this domain and I would encourage anyone interested to explore the many
references available. Here, for smart contracts, Manticore from Trail of
Bits and Mythril from Consensys Diligence are two widely used symbolic
checkers which we will touch upon in later.

\subsection{Formal Verification}\label{formal-verification}

Formal verification is the act of proving or disproving the correctness
of algorithms underlying the system with respect to a certain formal
specification of a property using formal methods of mathematics.

Formal verification is effective at detecting complex bugs, which are
generally hard to detect manually or using simpler automated tools.
Formal verification needs a specification of the program being verified
and techniques to compare the specification with the actual
implementation. Some of the tools in this space are Certora's Prover and
Chain Security's VerX. kEVM from Runtime Verification is a formal
verification framework that models EVM semantics.

\subsection{Manual Analysis}\label{manual-analysis}

Manual analysis is complementary to automated analysis using tools. It
serves a critical need in smart contact audits. Today, automated
analysis using tools is cheap because it typically uses open source
software that is free to use. Automated analysis is also fast
deterministic and scalable, however it's only as good as the properties
it is made aware of, which is typically limited to those concerning
\texttt{Solidity} and EVM related constraints.

Manual analysis on the other hand is expensive, slow, non-deterministic
and not scalable because human expertise in smart contract security is a
rare and expensive skillset today, and we are slower, more prone to
error and also inconsistent. Manual analysis however is the only way
today to infer and evaluate business logic and application level
constraints which is where a majority of the serious vulnerabilities are
being found.
