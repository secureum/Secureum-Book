\section{Application Logic Pitfalls}\label{application-logic-pitfalls}

The following concepts we're about to discuss are generalizations and
higher level concepts related to application logic level issues that
can't be specifically codified in tools or generalized because they
differ across applications. These are perhaps much harder to reason and
detect because it requires deep understanding of the application logic
and hence there's mostly manual effort in security reviews. Such
business logic which is application specific should have been translated
from requirements to the specification, then implementation with all of
it validated and documented accurately especially the security relevant
aspects.

Without that security reviewers have to infer assumptions constraints
program and variants trust and threat models which is not very effective
or efficient. Application logic related vulnerabilities are perhaps the
hardest to detect and have resulted in serious exploits. This is
therefore of utmost importance to security.

\subsection{Actors and Privileged
Roles}\label{actors-and-privileged-roles}

\subsubsection{Actors}\label{actors}

The aspirational goal in web3 is for it to be a completely
permissionless system where, ideally, there are no centralized trusted
actors, such as admins, responsible for any aspect of smart contracts
related to either development or management.

Remember that web3 aspires to be a zero trust system where no one needs
to be trusted to use and not abuse the system, because everything is and
should be verified. However, in guarded launch scenarios, the goal is to
start with trusted actors/assets/actions and then progressively
decentralize towards automated governance by the community.

For the trusted phase, all the trusted actors (their roles and
capabilities) should be clearly specified in the trust and threat
models, implemented accordingly and documented for user information and
any evaluation. This is a critical consideration in web3's Byzantine
Threat Model.

\subsubsection{Privileged Roles}\label{privileged-roles}

Let's now talk about privileged roles. Trusted actors who have
privileged roles in the context of the smart contact application with
capabilities to deploy contracts modify critical parameters, pause and
pause the system, trigger emergency shutdown, withdraw, transfer, drain
funds and allow deny other actors, should ideally be addresses
controlled by multiple independent and mutually distrusting entities.

They should not be controlled by private keys of externally owned
accounts, but we are multiSig with the high pressure, say 5-7 or 9-11
depending on the criticality of the application, the value address, and
eventually they should be governed by a community or a DAO
(decentralized autonomous organization) of token holders. This is
because an EOA is a single point of failure, if its key is compromised
or the order is malicious multiSig on the other hand brings in the
security design principle of privileged separation, which is tolerant to
a few of the holders being malicious or compromised.

When such privileged roles within smart contracts are being changed it
is recommended not to use a single step change because it is
error-prone. For example in a single step change, if the current admin
accidentally changes the new admin to the zero address, or an incorrect
address that's where the private key is not available, the system is
left without an operational and the contract will have to be redeployed
which is not easy or even entirely feasible in some scenarios.

Instead one should follow a two-step approach that we have discussed
earlier, the current privileged role proposes a new address for the
change and in the second step the newly proposed address, then claims
the privileged role in a separate transaction. This two-step change
mitigates risk by allowing accidental proposals to be corrected instead
of leaving the system unoperational with no or malicious privileged.

\subsection{Critical Parameters}\label{critical-parameters}

When critical parameters of systems need to be changed it is recommended
to enforce the changes after a time delay that is coupled and locked
with that logic. This is to allow systems users to be aware of such
critical changes and give them an opportunity to exit from that system,
if they do not like the upcoming changes, or adjust their engagement in
any other way with the system accordingly.

For example reducing rewards increasing fees or changing trust models in
a system might not be acceptable to some users who may wish to withdraw
their funds before the change and exit. Such a time locked execution of
delayed change enforcement needs to be combined with event emission to
notify users of upcoming changes via off-chain interfaces or monitoring
tools. So the best practice is a time delay change for critical
parameters that is broadcasted using events to users monitoring via
off-chain interfaces the goal is to surprise less be more transparent
and fair.

\subsection{Explicitness
vs.~Implicitness}\label{explicitness-vs.-implicitness}

As a general principle everything in security is about being explicit.
Instead of being implicit (implicit assumptions, implicit trust or
threat models, implicit acceptance of assets, actors, actions) which
leads to security vulnerabilities whereas, if they are explicitly
specified implemented and documented they can be reasoned about and
evaluated from a security perspective.

Even with the \texttt{Solidity} language it has progressively adopted
explicit declarations of intent over the versions such as with function
visibility and variable storage. So it's recommended to do the same at
the application level where all requirements should be explicitly
specified, so they're accurately implemented and lend themselves to
validation. Implicit requirements specification and implementation
assumptions should be explicitly documented and validated for
correctness. Any latent implicit requirements and assumptions should be
flagged as being dangerous.

\subsection{Configuration}\label{configuration}

Security issues arise not only from implementation errors, but also from
this configuration of system components, such as contracts, parameters,
addresses and permissions all of which may lead to security issues. Such
configuration aspects should be documented and validated test
configurations should be clearly marked as such and separated
appropriately from production configurations.

This is critical because testing is typically done with lower thresholds
of different values to allow for faster or easier testing, they may also
use more acceptable trust models or lower levels of thread than what is
encountered in a production setting. So the best practice is to check
configuration settings and make sure that they are correct relevant and
validated for a production deployment.

\subsection{Initialization}\label{initialization}

Lack of initialization. Initializing with incorrect values or allowing
untrusted actors to initialize system parameters may lead to security
issues this is especially true for critical parameters, addresses,
permissions and rules within the system because the default or incorrect
values may be used to exploit the system. Either technically or
economically. The best practice therefore to avoid security pitfalls
from initialization is to check that it is done and done correctly using
the right values and done, so by only the authorized users.

\subsection{Cleanup}\label{cleanup}

Missing the cleaning up of old state or cleaning up incorrectly or
insufficiently will lead to reuse of stale state which may lead to
security issues. Cleaning could be in the context of using
\texttt{Solidity}'s delayed primitive or even simply re-initializing
variables to default values in the context of the application's logic.

For example this is applicable to contract state maintained in state
variables within storage or even local variables within contact
functions, where old scale values may lead to incorrect reads or rights
in the context of the contract's logic. Cleaning up storage state using
\texttt{delete} primitive provides Gas refunds with an EVM some of which
has changed in recent upgrades, London upgrade for example reduced Gas
refunds of s stores. Nevertheless, there are benefits besides security
to this aspect of cleaning up.

\subsection{Data Pitfalls}\label{data-pitfalls}

At a very high and perhaps abstract level data, processing issues may
lead to security issues in the application logic's context this could
arise from several reasons such as while processing critical data or
from processing of painted input data.

Processing could be missing or incorrectly implemented this could either
resolve from a faulty specification or implementation without being
caught during validation therefore all aspects of data processing should
be reviewed for potential security impact

\subsubsection{Data Validation}\label{data-validation}

A specific aspect of data processing that we just discussed is data
validation where contract functions check, if they receive data from
external users or other contracts is valid, based on aspects of variable
types, lower high thresholds or any other application logic specific
context.

Validation issues very frequently lead to security issues. Missing
validation of data or incorrectly insufficiently validating data
especially tainted data from untrusted users will cause untrustworthy
system behavior which may lead to security issues. Sanity and threshold
checks are therefore critical aspects of data validation.

\subsubsection{Numerical Issues}\label{numerical-issues}

Another specific type of data processing is numerical processing, where
the logic operates on numerical values incorrect numerical computation
will almost always cause unexpected behavior some of which may lead to
serious security issues. If not accounting miscalculations these may be
related to overflow/underflow, precision handling, type casting,
parameter return values, decimals, ordering of operations with
multiplication/division and loop indices among other things.

The recommended best practice is to adopt widely used libraries for
special mathematical support such as Fixed-point or floating point
numbers and combine this with extensive testing using fuzzing and other
tools meant to specifically test constraints and invariants for
numerical issues.

\subsubsection{Accounting Issues}\label{accounting-issues}

A specific type of data numerical processing is that related to
accounting incorrect or insufficient tracking or accounting of business
logic related aspects. Such as states phases permissions, rules,
deposits, withdrawals of funds, mints. births, transfers of tokens or
rewards penalties, fees within DeFi applications all these may lead to
serious security issues. We have seen numerous vulnerabilities
specifically related to this aspect.

Therefore accounting aspects related to application logic states or
transitions or numerical aspects as outlined earlier should be carefully
reviewed to make sure they are correct and complete.

\subsection{Audit Logging}\label{audit-logging}

Recording or accessing snapshots or logs of important events within a
system is known as audit logging. The recorded events are called audit
logs. Note that this auditing from a logging perspective is different
from the concept of external reviews, which is also called auditing.
Auditing and logging are important for monitoring the security of an
application.

In the context of smart contracts this applies to event emissions, the
ability to query values of public state variables, exposed getter
functions, and recording appropriate error strengths from requires,
asserts and rewards. Incorrect or insufficient implementation of these
aspects will impact off-chain monitoring and instant response
capabilities which may lead to security issues. Correct and sufficient
audit and logging is therefore something that also needs to be paid
attention to for reasons of monitoring detecting and recovery aspects of
security.

\subsection{Cryptographic Issues}\label{cryptographic-issues}

Incorrect or insufficient cryptography, especially related to on-chain
signature verification, or off-chain key management, will impact access
control and may lead to security issues. So, aspects of keys, accounts,
hashes signatures, and randomness need to be paid attention to along
with the fundamental concepts of ECDSA signatures and
\texttt{keccak-256} hashes.

There are also other deeper and dual cryptographic aspects one will
encounter in Ethereum applications or protocol upgrades with
abbreviations such as \texttt{BLS}, \texttt{RANDAO}, and \texttt{VRF}.
Also zero knowledge (ZK) aspects. At a high level, cryptography is
fundamental and critical to security and even a tiny mistake here can be
disastrous surely leading to security vulnerabilities.

\subsection{Error Reporting}\label{error-reporting}

Incorrect or insufficient detecting reporting and handling of error
conditions will cause exceptional behavior to go unnoticed which may
lead to security issues. At a high level security exploits almost always
focus on exceptional behavior that is normally not encountered or
validated or noticed.

Such exceptional behavior is what is anticipated caught and reported by
error conditions. Any deviations from the specification are errors that
should be detected reported and handled appropriately by the
implementation.

\subsection{DoS Attacks}\label{dos-attacks}

Denial of service (DoS) Attacks are also a security concern.
Traditionally security has been considered as a triad referred to as the
CIA triad which stands for \textbf{confidentiality, integrity and
availability}. DoS affects availability, and in this case that of the
smart contract application. Preventing other users from successfully
accessing system services by either modifying system parameters or
shared state causes denial of service issues which affects the
availability of the system.

The effects of this could cause users to have their funds locked reduce
profits prevent from having their transactions included and therefore
interactions with the contracts denied. Attackers may cause DoS without
any apparent or immediate economic benefits to themselves and do so by
spending Ether on the Gas or any other tokens required for such duress
causing interactions, which is typically referred to as griefing. So the
best practices here are to recognize and minimize any such attributes in
the smart contracts or application logic that could enable dos.

\subsection{Timing}\label{timing}

Timing issues can have a security impact. Incorrect assumptions on
timing of user actions which can't be controlled. Triggering of system
state transitions or dependencies on blockchain state blocks
transactions may all lead to security issues depending on the
application logic context. Therefore any timing attributes or logic
within smart contact applications should be analyzed to check for such
issues.

\subsubsection{Freshness}\label{freshness}

Freshness of an object is a concept related with timing and indicates if
it is the latest one in some relevant timeline or, if it is stale
indicating that there is an updated value or version in that
corresponding timeline. Using stale values and not the most recent
values leads to freshness issues that could manifest into security
issues.

Concrete examples are the use of nonsense in transactions to prevent
replay attacks by repeating older transactions or the asset prices
obtained from Oracles which, if stale can cause significant accounting
issues leading to price manipulations and resulting vulnerabilities.
Therefore increased assumptions about the status of or data from system
actors being fresh because of lack of updation or availability may lead
to security issues, if and when such factors have been updated and
result resultant stale values being used instantly.

\subsection{Ordering}\label{ordering}

Similar to timing issues incorrect assumptions on ordering of user
actions or system state transitions may also lead to security issues.
For example a user may accidentally or maliciously call a finalization
function or other contract functions even before the initialization
function has been called, if the system allows this to happen.

Attackers can front run or back run user interactions to force
assumptions or ordering to fail Front-running is when the attackers race
to finish their transaction or interaction before the user. Back-running
is when they raise to be behind or right after the user's transaction of
interaction.

Combining these two aspects can also be exploited in what are known as
sandwich attacks where the user's transaction is sandwiched between
those from the patent. So the best practice is to pay attention to the
related aspects of timing and ordering attributes and evaluate, if they
can be abused in any manner.

\subsection{Undefined Behavior}\label{undefined-behavior}

Undefined behavior that is triggered accidentally or maliciously may
lead to security issues. But what is undefined behavior? Any behavior
that is not defined in the specification, but is allowed either
explicitly or inadvertently in the implementation is undefined behavior.

Such behavior may never be triggered in normal operations but, if they
are triggered accidentally in exceptional conditions that may result in
rewards. However, if such behavior can also be exploited in some manner
that leads to security issues in some cases it may not be clear, if such
undefined behavior is a security concern or not, but nevertheless should
be treated as such. The best practice is to make sure all acceptable
behavior is detailed in the specification implemented accordingly and
documented thoroughly.

\subsection{Trust}\label{trust}

Trust is a fundamental concept in security. Thus minimization (or zero
trust in the extreme case) is often the aspirational goal because
trusted assets actors actions may be compromised or become malicious to
subvert security. Trust minimization is a foundational value upon which
web3 is being picked, and one of the key tenets of decentralization
where the notions of insiders and outsiders are blurred, and users may
misuse the system under assumptions of the Byzantine Threat model.

So incorrect or insufficient trust assumptions about or among system
actors and external entities may lead to privileged escalation or
misuse, which may further lead to security issues the best practice
therefore is to never trust, but always verify both the principle as
well as in practice.

\subsection{Interactions}\label{interactions}

Following up with the trust minimization ideas we commented earlier, it
is obvious that external interactions can have a security impact. Such
interactions could be with assets actors or actions that are outside the
adopted trust and threat models and hence external. Interacting with
such external components for example tokens contracts or Oracles forces
the system to trust or make assumptions about their correctness or
availability which requires validation of their existence before
interacting with them and any outputs from such interactions

Therefore such external interactions can have security implications and
need to be considered carefully. Increasing dependencies and
composability make this a significant challenge.

\subsection{Dependencies}\label{dependencies}

In a similar way, dependencies on external actors assets actions or
software such as contracts, libraries, tokens, Oracles or Relayers will
lead to trust correctness and availability assumptions which, if or when
broken may lead to security issues. Dependencies therefore should be
well documented and evaluated for such trust assumptions and threat
models.

\subsection{Clarity}\label{clarity}

Lack of clarity in assets actors or actions or system specification,
documentation, implementation, user interface or user experience will
lead to incorrect assumptions and unanticipated expectations or outcome
which may lead to security issues. Therefore increasing the clarity by
clearly thoroughly and accurately specifying implementing and
documenting all security relevant aspects will help in mitigating risks
from lack of clarity.

\subsection{Privacy}\label{privacy}

Privacy and security are very closely related. In this context there
could be privacy issues related to assets actors and their actions.
Remember that data and transactions on the Ethereum blockchain are not
private anyone can observe contract state and track transactions both
included in the block, those pending in the \texttt{mempool}. So
incorrect assumptions about such privacy aspects of data or transactions
that manifest in implementation or assumed trust and threat models can
be abused leading to security issues.

\subsection{Cloning}\label{cloning}

Cloning in this context refers to copy pasting code from other libraries
contracts different parts of the same contract or from entirely
different projects with minimal or no changes. The configurations
context assumptions bugs and bug fixes for the original code may be
ignored or used incorrectly in the context of the cloned code.

This may result in incorrect code semantics for the context being copied
to copy over any vulnerabilities or miss any security fixes applied to
the original code all of which may lead to security issues. There have
been security vulnerabilities because of cloning incorrectly some of
which have led to exploits as well. Cloning therefore is risky and has
serious security implications.

\subsection{Gas}\label{gas}

Remember that the notion of Gas and Ethereum stems from the need to
bound computation because of the Turing completeness of the underlying
EVM. Incorrect assumptions about Gas requirements especially for loops
or external calls will lead to Out-of-Gas exceptions which may further
lead to security issues such as failed transfers or locked funds. Gas
usage must therefore be considered while reviewing smart contracts to
evaluate any assumptions leading to security implications of denial of
service.

\subsection{Constants}\label{constants}

Issues may arise, if you assume certain aspects to be constant. That is
they do not change for the duration of a transaction or even the
contract's lifetime, but in fact they are not constant and change for
some reason. Hardcoded assumptions could manifest for example in
hardcoded contract configuration parameters. Example: Block times, block
Gas Limits, opcode Gas prices, addresses, roles or permissions. Any such
incorrect assumptions about system actors entities or parameters being
constant may lead to security issues, if and when such factors change
unexpectedly.

\subsection{Scarcity}\label{scarcity}

Scarcity refers to the notion that something is available in only few
numbers. This may refer to assets or actors in the context of an
application where assumptions may be made that, there are only a few
assets or actors interacting with the application. Incorrect assumptions
about such Scarcity say for example tokens funds available to any system
actor will lead to unexpected outcomes, if those assumptions are broken
which may further lead to security issues.

For example susceptibility to flash loads or flash mints, related
overflows is an example where the vulnerable contract makes a Scarcity
related assumption and applies that to the size or type of variables
used to maintain token balances. Which, if broken because of flash loans
or mints can lead to overflows, if not mitigated appropriately. This is
also related to civil attacks where an attacker subverts a system by
creating a large number of identities and uses them to gain a
disproportionately large influence where the system assumption on fewer
unique identities is broken in some sense. Therefore one needs to
evaluate if there are any Scarcity or abundance assumptions in an
application that could cause security issues.

\subsection{Incentives}\label{incentives}

Incentives are another fundamental aspect of blockchains and web3.
Mechanism design or crypto economics dictates almost everything in this
space including infrastructure provisioning development and governance
of systems. What incentives are provided and how much incentives are
provided may be used or abused while interacting with smart contract
applications.

Incentives could be either rewards or penalties, so for example
incentives to liquidate positions in defile lending or applications of
incentives to cause denial of service or briefing of a system. Incorrect
assumptions about such incentives for system or external actors to
either perform or not perform certain actions will lead to expected
behavior not being triggered or unexpected behavior being triggered both
of which may lead to security issues.

\subsection{Clarity}\label{clarity-1}

Lack of clarity in assets actors or actions or system specification,
documentation, implementation, user interface or user experience will
lead to incorrect assumptions and unanticipated expectations or outcome
which may lead to security issues. Therefore increasing the clarity by
clearly thoroughly and accurately specifying implementing and
documenting all security relevant aspects will help in mitigating risks
from lack of clarity.

\subsection{Privacy}\label{privacy-1}

Privacy and security are very closely related. In this context there
could be privacy issues related to assets actors and their actions.
Remember that data and transactions on the Ethereum blockchain are not
private anyone can observe contract state and track transactions both
included in the block, those pending in the mempool. So incorrect
assumptions about such privacy aspects of data or transactions that
manifest in implementation or assumed trust and threat models can be
abused leading to security issues.

\subsection{Cloning Contracts}\label{cloning-contracts}

Cloning in this context refers to copy pasting code from other libraries
contracts different parts of the same contract or from entirely
different projects with minimal or no changes. The configurations
context assumptions bugs and bug fixes for the original code may be
ignored or used incorrectly in the context of the cloned code.

This may result in incorrect code semantics for the context being copied
to copy over any vulnerabilities or miss any security fixes applied to
the original code all of which may lead to security issues. There have
been security vulnerabilities because of cloning incorrectly some of
which have led to exploits as well. Cloning therefore is risky and has
serious security implications.
