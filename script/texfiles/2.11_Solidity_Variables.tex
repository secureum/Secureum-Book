\section{\texorpdfstring{\texttt{Solidity}
Variables}{Solidity Variables}}\label{solidity-variables}

\subsection{Scoping}\label{scoping}

This is fundamental to every programming language as it affects what is
known as variable visibility, or in other words ``\emph{where can
variables be used in relation to where they're declared}''. In the case
of \texttt{Solidity}, it uses the widely used scoping rules of
\texttt{C99} standard.

So variables are visible from the point right after the declaration
until the end of the smallest curly bracket block that contains that
declaration. As an exception to this rule, variables declared in the
initialization part of a \texttt{for} loop are only visible until the
end of the loop.

Variables that are parameters, like function parameters, modifier
parameters or catch parameters are visible inside the code block that
follows the body of the function (or modifier or catch).

Other items declared outside of a code block such as functions,
contracts, state variables or user defined types are visible even before
they are declared. This means that we can see the usage of state
variables even before they are declared within the context of a
contract. This is what allows functions to be called recursively.

From a security perspective, understanding the scoping rules of
\texttt{Solidity} becomes important when we are doing data flow
analysis. This could be in the context of a manual review, where you're
looking at the code yourself or when you're writing tools to do static
analysis on \texttt{Solidity} smart contracts.

\subsection{Default Values}\label{default-values}

Variables that are declared but not initialized have default values. In
the case of \texttt{Solidity}, the default values of variables are what
is known as a ``\emph{zero state}'' of that particular type. This means
is that in the case of a boolean, it has a value of zero as a default
which represents a value of \texttt{false} for the boolean. For unsigned
integers or integer types, this is \texttt{0} (as expected). For
statically sized arrays and \texttt{bytes1} to \texttt{bytes32}, each
individual element will be initialized to the default value
corresponding to its type. For dynamically sized arrays, \texttt{bytes}
and \texttt{string} the default value is an empty array or
\texttt{string}. For enum types, the default value is its first member.

From a security perspective this becomes important because variables
that are declared and not initialized end up with these default values.
In some cases, such as an address type, the zero address (which is a
default value) has a special meaning in Ethereum, and that affects some
of the security properties within the contract depending on how those
address variables are used.

\subsection{Literals}\label{literals}

This is something that you would have come across in other programming
languages, as well \texttt{Solidity} supports five types of literals:
\textbf{address types}, \textbf{rational}/\textbf{integers},
\textbf{strings}, \textbf{unicode} and \textbf{hexadecimals}.

The address literals are hexadecimal literals that pass the address
checksum test. Remember that Ethereum addresses are 20 bytes in length,
so in the case of the hexadecimal address representation, half a byte is
represented by a hexadecimal character. This results in the address
literal having 40 characters: 2 for every byte. These should pass the
checksum test.

The checksum is something that has been introduced in EIP55 to make sure
that there are no typographical errors when you're using addresses in
the context of Ethereum. This is a mixed case addressed exam.

Rational literals and integer literals are also supported. Integer
literals have a sequence of numbers in the 0 to 9 range. Decimal
fraction literals are formed by using a decimal point, with at least one
number in each side. Scientific notation is supported where the base can
have fractions and the exponent cannot. Underscores can be used to
separate these digits, which is used to help with readability and does
not have any semantic significance.

String literals are written with either double quotes (\texttt{""}) or
single quotes (\texttt{\textquotesingle{}\textquotesingle{}}). They can
only contain printable \texttt{ASCII} characters and a set of escape
characters. Unicode literals they have to be prefixed with the keyword
\texttt{unicode}. They can contain any \texttt{utf-8} sequence.

The hexadecimal literals are hexadecimal digits prefixed with the
keyword \texttt{"hex"} or
\texttt{\textquotesingle{}hex\textquotesingle{}}. The usage of all these
literals is in the context of constants.

\subsection{Booleans}\label{booleans}

\textbf{Boolean types} are declared using the \texttt{bool} keyword.
They can have only two possible values: \texttt{true} or \texttt{false}.

There are five operators that can operate on boolean types:

\begin{longtable}[]{@{}ll@{}}
\toprule\noalign{}
Name & Operator \\
\midrule\noalign{}
\endhead
\bottomrule\noalign{}
\endlastfoot
not operator & \texttt{!} \\
equality operator & \texttt{==} \\
inequality operator & \texttt{!=} \\
and operator & \texttt{\&} \\
or operator &
\texttt{\textbackslash{}\textbar{}\textbackslash{}\textbar{}} \\
\end{longtable}

The latter two operators are also known as logical conjunction and
logical disjunction operators.

\textbf{Operators apply the short circuiting rules}. For example, in an
expression that uses the logical disjunction or, operator if there are
two booleans let's say \texttt{x} or \texttt{y}, if x evaluates to
\texttt{true}, then the boolean \texttt{y} will not be evaluated at all
even if it may have side effects.

This is because the expression already evaluates to \texttt{true} and
there's no need for the second boolean to be evaluated at all and
similarly this applies to the \textbf{and} operator logical conjunction
as well. So if there are two booleans that have this operator, let's say
\texttt{x} and \texttt{y}, and if \texttt{x} happens to be
\texttt{false}, then we know that the expression finally will evaluate
to \texttt{false}, so there is no reason for the compiler to evaluate,
because the result is already known to be \texttt{false} from a security
perspective.

\textbf{Booleans are used significantly in smart contract functions} for
various conditionals and evaluations of expressions. \textbf{This
affects the control flow and specifically when it comes to certain
checks access control checks}.

\subsubsection{\texorpdfstring{\textbf{\emph{History}}}{History}}\label{history}

There have been cases where booleans have been used, and the wrong
operator has been used in those checks. So for example using the
\textbf{not} or logical disjunction instead of logical conjunction.

It can have big implications to how that particular expression evaluates
and that check, the access control check or whatever that might be,
might not be effective at all as intended by the specification. So this
is again something to pay attention to when you're looking at booleans
and the operators that evaluate that operate on the booleans in smart
contracts.

\subsection{Integers}\label{integers}

\textbf{Integer types} are very common in \texttt{Solidity} and any
programming language. There are unsigned and signed integers of various
sizes. In \texttt{Solidity} they use the \texttt{uint} or \texttt{int}
keywords. They come in \textbf{sizes from 8 bits all the way to the word
size of 256 bits}.

So you'll see declarations of unsigned integers or integers signed
intgers in the form of \texttt{uint8} all the way to 256.

\begin{lstlisting}[language=Solidity,numbers=none]
uint8, uint16, ..., uint256 
int8, int16, ..., int256
\end{lstlisting}

There are various operators for integer types. There are different
categories that we saw in the EVM instruction set: \textbf{arithmetic
operators}, \textbf{comparative operators}, \textbf{bit operators} and
\textbf{shift operators}.

From a security perspective, given that integer variables are vastly
used in \texttt{Solidity} contracts, they \textbf{affect the data flow
of the contract logic} and specifically there is an aspect of integers
that becomes \textbf{security critical} which is that of
\textbf{underflow and overflow}.

\subsubsection{Integer Arithmetic}\label{integer-arithmetic}

Integer arithmetic is arithmetic that operates on integer operands,
signed integer operands or unsigned integers operands \texttt{Solidity}.
Like in any other language, they are really restricted to a certain
range of values, so for example if you have \texttt{uint256}, then the
range of that variable is from a value of $0$ to $2^{256} - 1$.

If there is any operation on a variable of, let's say \texttt{uint256}
type that forces it to go \textbf{beyond this range}, then it leads to
what is known as an \textbf{overflow} or an \textbf{underflow}: this
causes wrapping.

In the case of \texttt{uint256} (let's say that the value of one of
those \texttt{uint256} variables was the maximum value), if the contract
logic incremented it by 1 more, then that integer value would overflow:
it would wrap to the other side of the range and would become 0.

Similarly an underflow, let's say in the case of the value was 0, if the
logic decremented it by one more, then it would again cause wrapping to
the other end of the range and the value of that variable would now be
$2^{256} - 1$. This can have significant unintended side effects when
it comes to the integer values used in that logic.

There have been numerous cases of certain integer values being
overflowed or underfloored, leading to huge exploits vulnerabilities
from a security perspective. This is something that is critical when it
comes to the security of integers, basically in the smart contract.

To address this specific aspect in versions of \texttt{Solidity} below
\texttt{0.8.0}, \textbf{the best practice was to use the
\texttt{SafeMath} libraries from OpenZeppelin} that made operating on
integer variables safe with respect to overflows and underflows.
\texttt{Solidity} itself as a language recognized this aspect of
security and introduced in version \texttt{0.8.0} default overflow and
underflow checks for integers.

In contracts that are written with the compiler version \texttt{0.8.0}
and above, one can actually switch between the default checked
arithmetic (that checks for underflows and overflows and causes
exceptions when that happens) versus unchecked arithmetic (where the
programmer or the developer asserts that for the expressions used in
that unchecked arithmetic there is no way or no cause for concern when
it comes to overflows and underflows), so all the default underlying
checks in the language in the compiler itself should be disabled.

This is something to be paid attention as it is a critical aspect of
smart contact security. When looking at smart contracts, pay attention
to the solution compiler version that was used: if it is below
\texttt{0.8.0}, then there should be the use of safe map from
OpenZeppelin, or some of the other equivalents that make sure that the
integers don't overflow and underflow and cause security
vulnerabilities. If the compiler version is \texttt{0.8.0} or beyond,
then one should pay attention to any expressions, integer expressions,
that are using \texttt{unchecked} blocks to make sure that those don't
have any overflows or underflows.

\subsubsection{Fixed point arithmetic}\label{fixed-point-arithmetic}

Conceptually you would have seen this in other languages, as well for
numbers that have an integer part and a fractional part, the location or
the position of the decimal point indicates if it is fixed or floating.
If that position or location of the decimal point can change for that
type then it is referred to as a floating point type.

But if that position is fixed for all variables of that type, then it is
known as fixed point arithmetic. In the case of \texttt{Solidity}, these
can be declared but cannot be assigned. There's no real support in
\texttt{Solidity}. \textbf{For any use of fixed point arithmetic}, one
has to \textbf{depend on some of the libraries such as} \texttt{DSMath},
\texttt{PRBMath}, \texttt{ABDKMath64x64} or others.

\subsubsection{Integer Members}\label{integer-members}

Integers have some members accessible with the \texttt{type(x)}
instruction (where \texttt{x} happens to be an integer).

\texttt{type(x).min} returns the smallest value representable by the
type \texttt{x}. Similarly, \texttt{type(x).max} primitive returns the
largest value that is representable by the type \texttt{x}. So for
example the \texttt{type(uint8).max} returns the maximum value
representable by the unsigned integer of size 8 bits, and in this case
it happens to be $255$ which is $2^8 - 1$.

\subsection{Arrays}\label{arrays}

Array types are something that are very common in most programming
languages, in the case of \texttt{Solidity} they come in two types

\begin{itemize}
\tightlist
\item
  \textbf{Static arrays}: where the size of the array is known at
  compile time. They are represented as \texttt{T{[}k{]}} (a static
  array of size \texttt{k}).
\item
  \textbf{Dynamic size arrays}: where the size of the array is known at
  compile time and its size may vary dynamically. They are represented
  as \texttt{T{[}{]}}
\end{itemize}

The elements of these arrays can be of any type that is supported by
\texttt{Solidity}. The indices that are used with these arrays are 0
based (the first array element is stored at \texttt{T{[}0{]}} and not
\texttt{T{[}1{]}}).

If these arrays are accessed by the logic past their length, then
\texttt{Solidity} automatically reverts that access and creates an
exception, which causes a failing assertion in the context of the
contract doing such an access.

From a security perspective, arrays are very commonly used in smart
contracts, \textbf{so the things to pay attention to are to check if the
correct index is being used especially in the context of indices being
zero based and to check if arrays have an off by 1 error, where they're
being accessed either beyond or below their supported indices}, in which
case such an \textbf{access could lead to an exception and the
transaction would revert.}

The other aspect to keep in mind with arrays is if the length of the
array that is being accessed is really long and if the types are
complicated underneath, then the amount of gas that is used for the
processing of such arrays could end up in what is known as a denial of
service attack (DoS) where those transactions revert because not enough
gas can be supplied as part of the transaction so you would end up with
no processing happening because a transaction would revert.

\subsubsection{Array Members}\label{array-members}

The members that are supported for array types there are four:

\begin{itemize}
\tightlist
\item
  \texttt{length} returns the number of elements in the array.
\item
  \texttt{push()} appends a 0 initialized element at the end of the
  array and it returns a reference to that element.
\item
  \texttt{push(x)} appends the specified element \texttt{x} to the end
  of the array and it returns nothing.
\item
  \texttt{pop} on the other hand removes an element from the end of the
  array and implicitly calls delete on that remote element.
\end{itemize}

\subsubsection{Memory Arrays}\label{memory-arrays}

\textbf{Memory arrays are arrays that are created in memory, they can
have dynamic length and can be created using the} \texttt{new}
\textbf{operator}. But as opposed to storage arrays, \textbf{it's not
possible to resize them}. So the \texttt{push()} \textbf{member
functions are not available} for such memory arrays.

So the options are for the developer to either \textbf{calculate the
required size in advance} and use that appropriately during the creation
of these arrays, \textbf{or create a new memory array and copy every
element of the older memory array into the new one} an example is shown
here.

\begin{lstlisting}[language=Solidity,numbers=none]
uint[] memory a = new uint[](7);
\end{lstlisting}

\subsubsection{Array Literals}\label{array-literals}

They are another type that is supported by \texttt{Solidity}. They are a
comma separated list of one or more expressions, enclosed in square
brackets (which is how arrays are represented in \texttt{Solidity}).

These are always statically sized memory arrays, whose length is the
number of expressions used within them. The base type of the array is
the type of the first expression of that list, such that all other
expressions can be converted to the first expression. If that is not
possible then it is a type error indicated by \texttt{Solidity}.

Fixed size memory arrays cannot be assigned to dynamically sized memory
arrays within \texttt{Solidity}, so these are some aspects to be kept in
mind when evaluating contracts that have array literals.

\subsubsection{Array Gas Costs}\label{array-gas-costs}

Arrays have \texttt{push} and \texttt{pop} operations. Increasing the
length of a storage array by calling \texttt{push}, has constant Gas
cost because storage is zero initialized. Whereas if you use
\texttt{pop} on such arrays to decrease their length, the Gas cost
associated with that operation depends on the size of the element being
removed. If the element being removed happens to be an entire array,
then it can be very costly because it includes explicitly clearing the
removed elements, which is similar to calling \texttt{delete} on each
one of them.

\subsubsection{Array Slices}\label{array-slices}

\texttt{Solidity} supports the notion array slices. Array slices are
views that are supported on contiguous array portions of existing
arrays. They are not a separate type in \texttt{Solidity}, but they can
be used in intermediate expressions to extract useful portions of
existing arrays as required by the logic within the smart contracts.
These are written as

\begin{lstlisting}[language=Solidity,numbers=none]
X[start:end]
/** This expression takes the array from element X[start]
up to element X[end-1]
*/
\end{lstlisting}

From an error checking perspective if
\texttt{start\ \textgreater{}\ end} or if
\texttt{end\ \textgreater{}\ n} (where \texttt{n} is the size of the
array) then an exception is thrown. Both these \texttt{start} and
\texttt{end} values are optional, where \texttt{start} defaults to
\texttt{0} and \texttt{end} defaults to the length of the array
\texttt{n}. Array slices do not have any members that are supported, and
for now \texttt{Solidity} only supports array slices for call data
arrays.

\subsection{Byte Arrays}\label{byte-arrays}

\textbf{Byte array types are used to store arrays of raw bytes}. There
are two kinds here:

\begin{itemize}
\tightlist
\item
  \textbf{Fixed size byte arrays}: we can use them if we know what the
  size of the byte array is going to be in advance. They come in 32
  kinds: \texttt{bytes1} for storing 1 byte, all the way to
  \texttt{bytes32} for storing 32 bytes, which is the full word size in
  the context of EVM.
\item
  \textbf{Dynamic size byte arrays}: we must use them if we do not know
  the fixed size in advance. They are indicated by \texttt{byte{[}{]}},
  but due to padding rules of EVM it wastes 31 bytes of space for every
  element that is stored in it. So, if we have a choice, then it's
  better to use the \texttt{bytes} type instead of the \texttt{byte}
  type for these byte arrays. This is something that you will commonly
  come across in smart contracts for storing raw bytes example in case
  of hashes.
\end{itemize}

\subsubsection{\texorpdfstring{\texttt{bytes} \&
\texttt{string}}{bytes \& string}}\label{bytes-string}

\texttt{bytes} are used to stir arbitrary byte data of arbitrary length.
Remember that \textbf{if we know beforehand the size of the byte array},
then we can \textbf{use the fixed size byte arrays} to store those
number of bytes.

But if you do not know what the size is beforehand, then we can use the
\texttt{bytes} type, and even there we have a choice of bytes or the
byte array we talked about earlier.

Remember that the byte array uses 31 bytes of padding for every element
stored and leads to waste of that space so \textbf{it's preferable to
use bytes over the byte array}.

\textbf{String type is equivalent to the byte style except that it does
not allow accessing the length of the string and the index of a
particular byte in that string}, so it does not have those members.
\texttt{Solidity} \textbf{does not yet have inbuilt string manipulation
functions but there are third party string libraries} that one can use.

\subsection{Function Types}\label{function-types}

\textbf{Function types are types used to indicate that variables
represent actual functions}. These variables can be used just like any
other variables: they can be assigned from functions because they are of
the function type, and they can be sent as arguments to other functions
and can also be used to return values from other functions.

They come in two types: \texttt{internal} and \texttt{external}. -
\textbf{Internal functions} can only be called inside the current
contract. - \textbf{External functions} consist of the address of the
contract where they're relevant and a function signature along with it.
They can be passed and returned from external function calls.

The usage of function types is somewhat minimal in most of the common
smart contracts.

\subsection{Structs}\label{structs}

From a data structure perspective, structs are custom data structures
that can group together several variables of the same or different types
to create something very unique to the contract as required by the
developer. So these are used extensively within smart contracts, they're
very commonly encountered. The various members of the structs are
accessed as follows

\begin{lstlisting}[language=Solidity,numbers=none]
// Create a struct
struct Book { 
   string title;
   string author;
   uint book_id;
}
// Fill in some info
Book my_book = Book("El Quixote", "Miguel de Cervantes", 1);
// Access a member
my_book.author
\end{lstlisting}

which returns\ldots{}

\begin{lstlisting}[language=Solidity,numbers=none]
"Miguel de Cervantes"
\end{lstlisting}

Some of the properties of \texttt{struct} types are that they can be
used inside mappings, arrays and they themselves can contain mappings
and arrays. All these different complex reference types can be used in a
very interrelated manner and allows for a versatile usage of these data
structures to support different kinds of encapsulation logic when it
comes to the different data types within a smart contract.
\textbf{There's one exception: \texttt{struct} types cannot contain
members of the same \texttt{struct} type.}

\subsection{Enums}\label{enums}

\textbf{Enums are a way to create user defined types in}
\texttt{Solidity}. \textbf{They can have members anywhere: from 1 member
all the way to a maximum of 256 members, and the default value of an}
\texttt{enum} \textbf{is that of the first member}. This is something
that you see sometimes in smart contracts where \texttt{enums}
\textbf{are used to represent the names of the various states} within
the context of the contract logic or the transitions in some cases. This
is something that helps to \textbf{improve readability} instead of using
the underlying integers that the \texttt{enums} really represent. As
they represent the underlying integer values, they can be explicitly
converted to and from integers. An example looks as follows

\begin{lstlisting}[language=Solidity,numbers=none]
enum ActionChoices{GoLeft, GoRight};
ActionChoices choice = ActionChoices.GoRight;
\end{lstlisting}

So here \texttt{choice} is a variable of \texttt{ActionChoices} and it
can be assigned the members of \texttt{ActionChoices}. Here we are
assigning \texttt{ActionChoices.GoRight} to \texttt{choice} and during
the course of the contact function, different members can be assigned to
that variable and it can be read from. This is used to improve
readability, so instead of using integer values one can use specific
names that correspond to those integer values in the context of what
makes sense from that contract and its underlying logic.

\subsection{Mappings}\label{mappings}

It is an interesting reference type somewhat unique to
\texttt{Solidity}. Mapping types define \texttt{(key,\ value)} pairs,
they're declared using the following syntax

\begin{lstlisting}[language=Solidity,numbers=none]
mapping(_key => _value) _Var
\end{lstlisting}

The \texttt{key} type in a \texttt{mapping} can be really any built-in
value type: \texttt{byte}, \texttt{string} or any contract or enum type
even. Other user defined or complex types, such as
\texttt{mapping\ structs} or \texttt{array} types are not allowed to be
used as the key type, so there are some restrictions here.

On the other hand, the \texttt{value} type of that
\texttt{(key,\ value)} pair, can be any type including mappings, arrays
and structs. There are some interesting aspects of how mappings are
created and maintained by \texttt{Solidity}: the \texttt{key} data is
not stored in the \texttt{mapping}, it is only used to look up the value
by taking a Keccak-256 hash of that \texttt{key} data.

They also do not have a concept of length nor a concept of a
\texttt{key} or \texttt{value} being set in the mapping. They can only
have a storage data location, so they are only allowed for state
variables. They cannot be used as parameters or return values of contact
functions that are publicly visible.

These restrictions are also true for arrays and structs that contain
mappings, not just mappings themselves. Also one cannot iterate over the
mappings, you cannot enumerate their keys and get the resulting values.
This is not supported by default but it is possible where required by
implementing another data structure on top of mappings and iterating
over them. So very versatile type in \texttt{Solidity} again, very
commonly encountered in smart contracts to store associations between
different data structures that are used in that contract logic.
