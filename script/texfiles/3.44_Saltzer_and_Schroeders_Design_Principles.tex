\section{Saltzer \& Schroeder's Design
Principles}\label{saltzer-schroeders-design-principles}

We will now discuss the 10 principles from Saltzer and Schroeder's
secure design principles, which were proposed by them in 1975 and have
been widely cited and used in various aspects of information security
ever since.

\subsection{Principle 1}\label{principle-1}

The first one is that of \textbf{least privilege} which states that
every program and every user of the system should operate using the
least set of privileges necessary to complete the job which means that
we should ensure that various system actors have the least amount of
privilege granted as required by their roles to execute their specific
tasks. Because granting excess privilege that what is absolutely
required is prone to misuse or abuse when trusted actors misbehave or
their access is hijacked by malicious entities privileges. Therefore
should be need-based.

\subsection{Principle 2}\label{principle-2}

The second principle is about \textbf{separation of privileges} which
states that where feasible a protection mechanism that requires two keys
to unlock it is more robust and flexible than one that allows access to
the presenter of only a single key.

This means that we should ensure critical privileges are separated
across multiple actors, so that, there are no single points of failure
or abuse. A good example of this is the use of a multisigs address
versus an EOA for privileged actors such as contract Owner, admin or
governance who control key contract functionalities such as pause and
pause shutdown, emergency fund, drain, upgradability of contracts,
allow, deny lists and critical parameters.

The multisig address should be composed of entities that are different
and mutually distrusting or verifying because such a privilege
separation prevents single points of failure.

\subsection{Principle 3}\label{principle-3}

The third principle is that of \textbf{least common mechanism}. Which
states that we should minimize the amount of mechanism common to more
than one user and depended on by all users.

This means that we should ensure that only the least number of security
critical modules or paths as required, are shared amongst the different
actors of code, so that impact from any vulnerability or compromise and
shared components is limited and contained to the smallest possible
subset.

In other words common points or parts of failure are minimized, there
are pros and cons of this approach that need to be made in depending on
the context.

\subsection{Principle 4}\label{principle-4}

The fourth principle is that of \textbf{fail-safe defaults} which states
that we need to base access decisions on permission rather than
exclusion, so we need to ensure that variables or permissions are
initialized to fail-safe default values which deny access by default,
but can later be made more inclusive or permissive, if and when
necessary.

Instead of opening up the system to everyone by default which may
include untrusted actors. We have discussed this in the context of
guarded launch for assets actors and actions. Such fail-safe initial
defaults could apply to function visibility critical parameter,
initializations and permissions of assets actors and actions, there are
again pros and cons of this approach that need to be considered as it
applies to open or closed systems given the emphasis of web3 on aspects
of openness permissionless participation and composability among other
things.

\subsection{Principle 5}\label{principle-5}

The fifth principle is that of \textbf{complete mediation} which states
that every access to every object must be checked for authority. Which
means that we should ensure that any required access control is enforced
along all access paths to the object or function being protected.
Examples are missing modifiers, permissive visibility or missing
authorization flows. Complete mediation, therefore requires access
control enforcement on every asset after action along all paths and at
all times.

\subsection{Principle 6}\label{principle-6}

The sixth principle is that of \textbf{economy of mechanism}, which says
keep the design as simple and small as possible. Which in this context
can be applied to ensure that contracts and functions are not overly
complex or large, so as to reduce readability maintainability or even
auditability. This embodies the keep it simple and stupid or KISS
Principle in some ways because complexity typically leads to insecurity
and hence should be kept as low as possible.

\subsection{Principle 7}\label{principle-7}

The seventh principle is that of \textbf{open design} which states that
the design should not be secret. This is especially relevant to the web3
space as we have discussed earlier because smart contracts are expected
to be open-sourced, verified and accessible to everyone for
permissionless participation and composability. Security by obscurity of
code or underlying algorithms is not an option. Security should be
derived from the strength of the design and implementation under the
assumption that Byzantine attackers will study their details and try to
exploit them in arbitrary ways.

\subsection{Principle 8}\label{principle-8}

The eighth principle is that of \textbf{psychological acceptability}
which states that it is essential that the human interface be designed
for ease of use, so that users routinely and automatically apply the
protection mechanisms correctly. Which in our context means that we need
to ensure that security aspects of smart contract interfaces and system
designs flows are human friendly and in queue them, so that we can
program them or use them with ease and with minimal risk. This is a
significant challenge in the web3 space today where, there is a lot of
early and experimental software undergoing rapid changes, but something
to be kept in mind from a security perspective as things evolve and
systems get more mass adoption.

\subsection{Principle 9}\label{principle-9}

The ninth principle is \textbf{work factor}, which recommends to
\textbf{compare the cost of circumventing the mechanism with the
resources of a potential attacker}. Which is very relevant and perhaps
at an extreme in the case of smart contracts in web3 because given the
magnitude of value managed by smart contracts it is safe to assume that
Byzantine attackers will risk the greatest amounts of the resources
possible across intellectual social and financial capital to support
such systems. And given the general state of current smart contracts the
cost of circumventing is not very high, relative to hardened software or
systems in the Web2 space for various reasons that we have discussed
earlier.

The rewards from exploiting them are in tens or even hundreds of
millions of dollars in some cases, so the risk versus reward is
extremely skewed here. Therefore the mitigation mechanisms must
appropriately factor in the highest levels of threat and risk.

\subsection{Principle 10}\label{principle-10}

The final tenth principle is about \textbf{compromise recording} which
states that mechanisms that reliably record that a compromise of
information has occurred can be used in place of more elaborate
mechanisms that completely prevent loss.

One way to interpret this is to say that achieving improving bug-free
code is theoretically and practically impossible for real world smart
contracts. Therefore one should strive for the best in performing all
security due diligence and reduce the attack surface as much as
possible. While in the same time, anticipate residual risk to exist in
the deployed system. Anticipate that there will be potential incidents
that exploit them and therefore have an instant response plan ready for
that.

For doing that we can ensure that smart contracts and their accompanying
operational infrastructure can be monitored and analyzed at all times
for minimizing loss from any compromise due to vulnerabilities and
exploits. As a concrete example critical operations in contracts should
emit events to facilitate off-chain monitoring at runtime, where the
available monitoring tools are used on smart contracts of interest to
analyze not only such events, but also transactions interacting with
them their Side-effects and potential security impacts.
