\section{\texorpdfstring{\texttt{ecrecover}}{ecrecover}}\label{ecrecover}

This security pitfall is related to the use of the \texttt{ecrecover}
primitive in EVM and supported by \texttt{Solidity}.

The specific pitfall is that it is susceptible to what is known as
signature malleability or non-unique signatures. Remember that elliptic
curve signatures in Ethereum have three components \texttt{v},
\texttt{r} and \texttt{s}. The \texttt{ecrecover} function takes in a
message hash the signature associated with that message hash and returns
the Ethereum address that corresponds to the private key that was used
to create that signature.

In the context of this pitfall, if an attacker has access to one of
these signatures, then they can create a second valid signature without
having access to the private key to generate that signature.

This is because of the specific range that the \texttt{s} value or the
\texttt{s} component of that signature can be in it can be in an upper
range or a lower range and both ranges are allowed by this primitive
which results in the malleability.

This depending on the logic of the smart contract, the context in which
it is using these signatures can result in replay attacks, so the
mitigation is to check that the \texttt{s} component is only in the
lower range and not in the higher range, this mitigation is enforced in
OpenZeppelin's \texttt{ECDSA} library which is the recommended best
practice.
