\section{Block Explorer}\label{block-explorer}

If we want to take a look at what has happened in the past in terms of
the transactions on Ethereum, the contracts that they interacted with,
then the application that allows us to look at all this data is what is
known as a block explorer: it lets us \textbf{explore the various blocks
and their contents on the blockchain}.

It's implemented as an application, a web portal if you will, and it
gives us real-time on-chain data about all the transactions, the blocks,
the Gas and everything that we have discussed so far. All this rich
information is available in a transparent manner on the blockchain and
can be accessed by everyone via this block explorer application.

In the case of Ethereum we have several block explorers. The most
popular one is \href{https://etherscan.io/}{Etherscan}. We also have
\href{https://etherchain.org/}{Etherchain},
\href{https://ethplorer.io/}{Ethplorer},
\href{https://blockchair.com/}{Blockchair} or
\href{https://blockscout.com/}{Blockscout}.
