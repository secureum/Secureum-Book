\section{EVM (Ethereum Virtual Machine) in
Depth}\label{evm-ethereum-virtual-machine-in-depth}

The EVM is the execution component of the Ethereum blockchain: it is the
runtime environment where all the smart contracts run. Recall that EVM
is a quasi-Turing complete machine: it's turing complete because the
underlying programming language supports arbitrary logic unbounded
complexity, but it's also bounded by the amount of Gas provided as part
of every transaction.

The Ethereum code runs within the EVM and it is written in a low level
stack based language referred to as the EVM Machine Code. This code
consists of a series of bytes (therefore referred to as a bytecode)
where every byte represents a single operation. So the opcodes are very
simple and each of them is a single byte.

\subsection{EVM Arquitecture}\label{evm-arquitecture}

Computer architectures are typically classified into either \textbf{von
Neumann architecture} or \textbf{Harvard architecture}. This depends on
how code and data are handled within the architecture: Are they stored
together? Are they transported over the buses together? How are they
cached? And so on\ldots{}

In the case of the EVM, the code is stored separately in a virtual ROM
and there is a special instruction to access the EVM code.

EVM is a very simple stack based architecture: the operands for EVM
instructions are placed on the stack and the output of those
instructions is also returned on the stack. There's no concept of
registers, virtual registers or anything like that.

Every architecture has a concept of a word and in the case of the EVM,
the word size is \textbf{256 bits}. It's believed that this was chosen
to facilitate some of the fundamental operations around the 256 hash
scheme and the elliptic curve computations.

The architecture is made up of four fundamental components:

\begin{enumerate}
\def\labelenumi{\arabic{enumi}.}
\item
  \textbf{The stack} The EVM has 1024 elements in the stack and each of
  those elements is 256 bits in length (equal to the word size). EVM
  instructions are allowed to operate with the top 16 stack elements.
  Most EVM instructions operate with the stack (because it's a stack
  based architecture) and there are also stack specific operations.
\item
  \textbf{The volatile memory}: in EVM, data placed in memory is not
  persistent across transactions on the blockchain. It is also linear
  (it's a byte array and therefore addressable at byte level) and zero
  initialized.\\

  There are three specific instructions that operate with memory, such
  as \texttt{MLOAD} which loads a word from memory and puts it onto the
  stack; \texttt{MSTORE} which stores a word in memory from the stack;
  and \texttt{MSTORE8} which stores a single byte in memory from the
  stack. These instructions (and more) will be reviewed in more detail
  in the following sections.
\item
  \textbf{The non-volatile storage}. Unlike memory, storage in EVM is
  non-volatile: data put in storage is persistent across transactions on
  the blockchain. It is implemented as a \texttt{(key,\ value)} store
  between 256 bit keys and 256 bit values, and it is also zero
  initialized.\\

  To understand how storage fits in within the concept of accounts and
  the blocks on the blockchain, recall that every account has a
  \texttt{storageRoot} field. This \texttt{storageRoot} field,
  implemented as a modified Merkle-Patricia tree, captures all the
  storage associated with that account. This is relevant for contract
  accounts that have associated storage. These \texttt{storageRoots}
  within the account are further captured as part of
  \texttt{the\ stateRoot}, which is one of the fields in the block
  header.\\

  There are two instructions that operate specifically on storage:
  \texttt{SLOAD} which loads a word from the storage and puts it onto
  the stack; and \texttt{SSTORE} which takes a word from the stack and
  puts it into storage.
\item
  \textbf{\texttt{Calldata}}: it is used specifically for data
  parameters of transactions and message calls. It is read only (it
  cannot be written to) and it's also bite addressable.\\

  There are three specific instructions that operate with call data:
  \texttt{CALLDATASIZE} which gives the size of the supplied call data
  and puts it onto the stack; \texttt{CALLDATALOAD} which loads the call
  data supplied onto the stack; and \texttt{CALLDATACOPY} that copies
  the supply call data to specific region of memory.
\end{enumerate}

\subsection{EVM Ordering}\label{evm-ordering}

Another concept typically associated with architectures is the concept
of ordering: \textbf{big-endian} ordering versus \textbf{little-endian
ordering}. In the case of the EVM, \textbf{it uses the big-endian
ordering}: the most significant byte of a word is stored at the smallest
memory address while the least significant byte is stored at the largest
address.

\subsection{Instruction Set}\label{instruction-set}

\emph{(A more complete and detailed description of the EVM opcodes can
be found in}
\href{https://www.evm.codes/?utm_source=tldrnewsletter}{\emph{evm.codes}}\emph{.
If there is any discrepancy/ambiguity of the contents found here, refer
to}
\href{https://www.evm.codes/?utm_source=tldrnewsletter}{\emph{evm.codes}}\emph{)}

Summarized walkthrough of the EVM opcodes

All the instructions supported by the EVM can be classified into 11
categories. Instructions that are found in categories \textbf{a} to
\textbf{i} operate on the stack.

The format for each of these instructions will be as follows:

\begin{lstlisting}[language=Solidity,numbers=none]
OPCODE MNEMONIC INPUTS OUTPUTS
\end{lstlisting}

\hfill\break
Let's see an example:\\

The opcode is the hex representation of the instruction. You will see
that the \texttt{0x00} opcode is used for the stop instruction. In
addition, the word \texttt{STOP} is the mnemonic of the instruction. and
then the two numbers that you see after the mnemonic refer to the number
of stack items placed for this instruction (inputs) and the number of
stack items removed (outputs).

So the stop opcode is \texttt{0x00}, thus it's the first instruction in
the instruction set mapping. The mnemonic is \texttt{STOP} (makes
sense), 0 items are placed and 0 items are removed from the stack.

In the case of \texttt{ADD}, you will see that it has 2 items placed
onto the stack (the 2 operands) and the computed result (the addition)
is placed back onto the stack.

That's why you see that there is one item placed onto the stack, which
is the result the addition of the two inputs. The same thing holds good
for multiplication, and so on\ldots{}

\hfill\break
\textbf{a. Stop \& Arithmetic}

\begin{lstlisting}[language=Solidity,numbers=none]
0x00 STOP 0 0
0x01 ADD 2 1
0x02 MUL 2 1
0x03 SUB 2 1
0x04 DIV 2 1
0x05 SDIV 2 1
0x06 MOD 2 1
0x07 SMOD 2 1
0x08 ADDMOD 3 1
0x06 MOD 2 1
0x07 SMOD 2 1
0x08 ADDMOD 3 1
0x09 MULMOD 3 1
0x0a EXP 2 1
0x0b SIGNEXTEND 2 1
\end{lstlisting}

\hfill\break
\textbf{b. Comparison \& Bitwise Logic}

\begin{lstlisting}[language=Solidity,numbers=none]
0x10 LT 2 1
0x12 SLT 2 1
0x20 GT 2 1
0x13 SGT 2 1
0x14 EQ 2 1
0x15 ISZERO 1 1
0x16 AND 2 1
0x17 OR 2 1
0x18 XOR 2 1
0x19 NOT 1 1
0x1a BYTE 2 1   
0x1b SHL 2 1
0x1c SHR 2 1
0x1d SAR 2 1
\end{lstlisting}

\hfill\break
\textbf{c.~SHA3 Instruction}

\begin{lstlisting}[language=Solidity,numbers=none]
0x20 SHA3 2 1
\end{lstlisting}

\emph{This single instruction is critical to Ethereum because it
computes the Keccak-256 Hash. The formal notation for how the Keccak-256
hash is calculated is:}

\begin{align*}
\mu_s'[0]&=\text{KEC}\left(\mu_m\left[\mu_s[0]\left(\mu_s[0]+\mu_s[1]-1\right)\right]\right)\\
\mu_i'&=\text{M}\left(\mu_i,\mu_s[0],\mu_s[1]\right)
\end{align*}

\emph{This is explained with more detail in the Ethereum Yellowpaper.}

\hfill\break
\textbf{d.~Environmental Information Instructions}

\emph{These set of instructions give information about the environment
or the execution context of the smart contract executing them.}

\begin{lstlisting}[language=Solidity,numbers=none]
0x30 ADDRESS 0 1
0x31 BALANCE 1 1
0x32 ORIGIN 0 1
0x33 CALLER 0 1
\end{lstlisting}

\begin{itemize}
\tightlist
\item
  \emph{The \texttt{ADDRESS} instruction gives the address of the
  currently executing account.}
\item
  \emph{\texttt{BALANCE} gives the ether balance of the currently
  executing account.}
\item
  \emph{\texttt{ORIGIN} gives the address of the originator of the
  transaction that actually led to the execution of the code within the
  EVM.}
\item
  \emph{\texttt{CALLER} gives the caller's address in the context of
  \texttt{Solidity}, these would be transaction origin
  (\texttt{tx.origin}) and message sender (\texttt{msg.sender})
  respectively.}
\end{itemize}

\begin{lstlisting}[language=Solidity,numbers=none]
0x34 CALLVALUE 0 1
0x35 CALLDATALOAD 1 1
0x36 CALLDATASIZE 0 1
0x37 CALLDATACOPY 3 0
\end{lstlisting}

\begin{itemize}
\tightlist
\item
  \emph{\texttt{CALLVALUE} in the context of \texttt{Solidity} would be
  the message value (\texttt{msg.vale}) that you would see in the smart
  contracts.}
\end{itemize}

\begin{lstlisting}[language=Solidity,numbers=none]
0x38 CODESIZE 0 1
0x39 CODECOPY 3 0
0x3a GASPRIZE 0 1
0x3b EXTCODESIZE 1 1
\end{lstlisting}

\begin{itemize}
\tightlist
\item
  \emph{\texttt{CODESIZE} gives the size of the code running in the
  current environment.}
\item
  \emph{\texttt{CODECOPY} lets you copy the code running in the current
  environment to memory.}
\item
  \emph{\texttt{GASPRICE} in the context of \texttt{Solidity}; you would
  see this as \texttt{transaction.gasPrice}, which gives you the price
  of the Gas in the current environment.}
\end{itemize}

\begin{lstlisting}[language=Solidity,numbers=none]
0x3b EXTCODESIZE 1 1
0x3c EXTCODECOPY 4 0
0x3d RETURNDATASIZE 0 1
0x3e RETURNDATACOPY 3 0
0x3f EXTCODEHASH 1 1
\end{lstlisting}

\emph{This set of instructions lets you query an external contract
account.}

\begin{itemize}
\tightlist
\item
  \emph{\texttt{EXTCODESIZE} gives you the size of the specified
  account's code.}
\item
  \emph{\texttt{EXTCODECOPY} copies the specified accounts code to
  memory.}
\item
  \emph{\texttt{RETURNDATASIZE} gives the size of the output data from
  the previous call in this current environment.}
\item
  \emph{\texttt{RETURNDATACOPY} copies that return data.}
\item
  \emph{\texttt{EXTCODEHASH} gives the hash of the external account's
  code.}
\end{itemize}

\hfill\break
\textbf{e. Block Information Instructions}

\emph{Similar to environment key information, EVM also has a set of
instructions that gives information about transactions block.}

\begin{lstlisting}[language=Solidity,numbers=none]
0x40 BLOCKHASH 1 1
0x41 COINBASE 0 1
0x42 TIMESTAMP 0 1
0x43 NUMBER 0 1
0x44 DIFFICULTY 0 1
0x45 GASLIMIT 0 1
\end{lstlisting}

\begin{itemize}
\tightlist
\item
  \emph{\texttt{BLOCKHASH} gives the hash of one of the specified 256
  most recent complete blocks. If the specified block is not one of the
  most recent 256 ones, then this instruction returns zero,
  \textbf{which is something that has a security implication}.}
\item
  \emph{\texttt{COINBASE} gives the block's beneficiary address (the
  address to which the block reward and transaction fees are credited
  to).}
\item
  \emph{\texttt{TIMESTAMP} gets the block's timestamp.}
\item
  \emph{\texttt{NUMBER} gets the block's number.}
\item
  \emph{\texttt{DIFFICULTY} gets the block's diffiulty.}
\item
  \emph{\texttt{GASLIMIT} gets the block's gas limit.}
\end{itemize}

\hfill\break
\textbf{f.~Stack, Memory, Storage and Flow Instructions}

\emph{The next category of instructions are related to the stack memory
and storage; load and store operations and also those that affect the
control flow.}

\begin{lstlisting}[language=Solidity,numbers=none]
0x50 POP 1 0
0x51 MLOAD 1 1
0x52 MSTORE 2 0
0x53 MSTORE8 2 0
0x54 SLOAD 1 1
0x55 SSTORE 2 0
\end{lstlisting}

\begin{itemize}
\tightlist
\item
  \emph{\texttt{POP} pops an element of the stack.}
\item
  \emph{\texttt{MLOAD} and \texttt{MSTORE} load and store from memory.}
\item
  \emph{\texttt{MSTORE8} stores a single byte to memory instead of the
  whole word.}
\item
  \emph{\texttt{SLOAD} and \texttt{SSTORE} load and store words from and
  to the storage.}
\end{itemize}

\emph{The next set of instructions affect the control flow.}

\begin{lstlisting}[language=Solidity,numbers=none]
0x56 JUMP 1 0
0x57 JUMPI 2 0
0x58 PC 0 1
0x59 MSIZE 0 1
0x5a GAS 0 1
0x5b JUMPDEST 0 0
\end{lstlisting}

\begin{itemize}
\tightlist
\item
  \emph{\texttt{JUMP} jumps to the specific location.}
\item
  \emph{\texttt{JUMPI} is a conditional jump that jumps depending on the
  value specified.}
\item
  \emph{\texttt{PC} gives you the value of the program counter.}
\item
  \emph{\texttt{MSIZE} gives the size of active memory in bytes as of
  this instruction.}
\item
  \emph{\texttt{GAS} gives the amount of available Gas as of this
  instruction. This is in the context of the Gas that is supplied with
  the transaction: how much gets consumed and how much is left.}
\item
  \emph{\texttt{JUMPDEST} has no effect on the machine state: it does
  not affect the control flow but it marks a particular destination as
  being a valid destination for the jump instructions that we talked
  about.}
\end{itemize}

\hfill\break
\textbf{g. Push Operations}

\emph{The next set of instructions are specific to the stack. These
instructions push operands or place items onto the stack. Depending on
the number of items placed, there are 32 such instructions.}

\begin{lstlisting}[language=Solidity,numbers=none]
0x60 PUSH1 0 1
0x61 PUSH2 0 1
 .     .
 .     .
 .     .
0x7f PUSH32 0 1
\end{lstlisting}

\emph{\texttt{PUSH1} pushes a single byte onto the stack, \texttt{PUSH2}
pushes 2 bytes and it goes all the way up to \texttt{PUSH31}. The
\texttt{PUSH32} instruction pushes a full word (32 bytes or 256 bits)
onto the stack.}

\hfill\break
\textbf{h. Duplication Operations}

\emph{The next category of instructions that operate on the stack are
the duplication operations, which duplicate items that are already on
the stack.}

\begin{lstlisting}[language=Solidity,numbers=none]
0x80 DUP1 1 2
0x81 DUP2 1 2
 .     .
 .     .
 .     .
0x8f DUP16 1 2
\end{lstlisting}

\emph{\texttt{DUP1} for example duplicates the first stack item,
\texttt{DUP2} duplicates the first two items on the stack, and it goes
all the way up to \texttt{DUP16}.}

\textbf{i. Exchange Operations}

\emph{The final set of instructions that operate on stack items are the
exchange operations}. These exchange or swap items that are already on
the stack.

\begin{lstlisting}[language=Solidity,numbers=none]
0x90 SWAP1 2 2
0x91 SWAP2 3 3
 .     .
 .     .
 .     .
0x9f SWAP16 17 17
\end{lstlisting}

\emph{\texttt{SWAP1} exchanges the first and second stack items,
\texttt{SWAP2} exchanges the first and third and so on all the way up to
\texttt{SWAP16} which exchanges the first and 17 \(^\text{th}\) stack
items.}\\

\textbf{j. Logging Operations}

\emph{These operations append log records from within the execution
context of the contract onto the blockchain. We talked about this a bit
in the context of the bloom filter in the block header.}

\begin{lstlisting}[language=Solidity,numbers=none]
0xa0 LOG0 2 0
0xa1 LOG1 3 0
0xa2 LOG2 4 0
0xa3 LOG3 5 0
0xa4 LOG4 6 0
\end{lstlisting}

\emph{These instructions differ in the number of topics that are
specified as being part of the log. So the log itself refers to the
event that is fired from within the context of the contract and in the
event. The different parameters can be specified as either being indexed
or non-indexed.}

\textbf{Indexed} parameters go into the \textbf{topics} part of the log
and the \textbf{non-indexed} parameters go into the \textbf{data} part
of the log.

\emph{This differentiates how fast the the parameters or the records can
be queried, searched and looked. These instructions are critical to how
the contracts actually communicate some of their state to the off-chain
interfaces or the off-chain monitoring tools.}

\hfill\break
\textbf{k. System Operations}

\emph{The next set of instructions include instructions that are
critical to how the system functions. They allow one to create new
contract accounts, call from one account to another in different ways,
revert from the current executing context, invalidate some of the things
that have happened and so on.}

\begin{lstlisting}[language=Solidity,numbers=none]
0xf0 CREATE 3 1
0xf1 CALL 7 1
0xf2 CALLCODE 7 1
\end{lstlisting}

\begin{itemize}
\tightlist
\item
  \emph{\texttt{CREATE} is used to create a new contract account that
  has associated code and storage with it. Recall that contract accounts
  can be created from an EOA by sending a special transaction to the
  zero address (\texttt{0x0}) or they can also be created from other
  contracts when they're executed. The address of the newly created
  account depends on the sender's address and the \texttt{nonce} of that
  account. So this makes the newly created contracts address dependent
  on the previous transactions that have executed from the sender's
  account. This becomes interesting when we talk about the related
  instruction called \texttt{CREATE2}.}
\item
  \emph{\texttt{CALL} allows the current executing context to do a
  message call into another account. So now there is a caller account
  that is doing a message call into a callee account. This is
  interesting because it lets contracts call each other in the executing
  context.}
\item
  \emph{\texttt{CALLCODE} is another call related instruction which lets
  the caller account call a callee account and lets the callee account
  execute its code in the context of the state of the caller's account.
  This distinction is really critical and it has big security
  implications in some of the future instructions we'll talk about.}
\end{itemize}

\begin{lstlisting}[language=Solidity,numbers=none]
0xf3 RETURN 2 0
0xf4 DELEGATECALL 6 1
0xf5 CREATE2 4 1
\end{lstlisting}

\begin{itemize}
\tightlist
\item
  \emph{\texttt{RETURN} holds execution and returns the output data.}
\item
  \emph{\texttt{DELEGATECALL} is a very interesting instruction part of
  the call family of instructions which acts very similar to
  \texttt{CALLCODE}. There is a caller account that calls into the
  callee account, and the callee account executes its code in the
  context of the caller's state. The difference here between
  \texttt{CALLCODE} is that in the case of \texttt{DELEGATECALL}, in the
  context of \texttt{Solidity}, \texttt{msg.sender} and
  \texttt{msg.value} of the caller's account is used in the execution
  context of the callee account.}
\item
  \emph{\texttt{CREATE2} is similar to \texttt{CREATE} and is used to
  create new contract accounts with associated code and storage. The
  difference here is that \texttt{CREATE2} allows you to create accounts
  with a predictable addresses, unlike \texttt{CREATE} where the address
  of the newly created contact account dependeds on the \texttt{nonce}.
  So \texttt{CREATE2} removes all the transactions that happened from
  the sender's account so that the address of contracts being generated
  are predictable. Again, this has big implications in security.}
\end{itemize}

\begin{lstlisting}[language=Solidity,numbers=none]
0xfa STATICCALL 6 1
0xfd REVERT 2 0
0xfe INVALID NaN NaN
\end{lstlisting}

\begin{itemize}
\tightlist
\item
  \emph{\texttt{STATICCALL} is another instruction in the call family
  which allows the callee account (that is being called into) to only
  read the state of the caller account without letting it modify it.
  This has security implications as well.}
\item
  \emph{\texttt{REVERT} holds execution of the current executing
  context, it returns the data and it returns the remaining Gas that's
  left behind after consuming all the Gas that was supplied as part of
  the triggering transaction so far.}
\item
  \emph{\texttt{INVALID} (\texttt{0xfe}) is again a special instruction
  in EVM. It consumes all the Gas that's been supplied as part of the
  triggering transaction and it is used in the context of some of the
  static analysis tools that we'll touch upon in later chapters.}
\end{itemize}

\begin{lstlisting}[language=Solidity,numbers=none]
0xff SELFDESTRUCT 1 0
\end{lstlisting}

\emph{The final instruction (\texttt{0xff}) in the EVM instruction set
is a special instruction called \texttt{SELFDESTRUCT}. As you can
imagine, this holds execution but it also destroys the account of the
executing context: the account is registered for later deletion.}

\emph{This has huge security implications because the contract account
that is executing will not exist after the transaction finishes. This is
something we will touch upon and some of the security aspects as when we
talk about some of the findings and security pitfalls in later
chapters.}

\subsection{Gas Costs}\label{gas-costs}

We have talked about Gas in the context of transaction, in the context
of the block Gas limit and so on\ldots{} But where it really matters is
in the context of Turing complexity and quasi-Turing completeness. The
boundedness imposed on the EVM programming language is stemming from the
Gas costs that are associated with each of the different EVM
instructions

All these instructions have different Gas costs and the reason for that
is because each of them has a different requirements when it comes to
the computation processing power of the executing Ethereum node, and
also the storage requirements, memory accesses and the disk accesses on
the real physical hardware that's running the Ethereum node in the
context of a miner or anyone else.

When we look at the Gas costs, the simplest instructions like
\texttt{STOP}, \texttt{INVALID} and \texttt{REVERT} (that only affect
the executing context in a very special way, without having a very high
demand or no demand on the processing or the storage of that executing
physical hardware), the Gas cost is zero.

For most of the arithmetic, logic and stack instructions, the Gas costs
vary between 3 to 5 Gas units. Let's contrast this with some of the more
demanding instructions like the call family of instructions, the
\texttt{BALANCE}, the \texttt{EXTCODEHASH}, \texttt{EXPORT},
\texttt{COPY}\ldots, those kinds of instructions have a much greater
processing requirement from the Ethereum node: these now cost 2600 Gas
units.

This again contrast with the memory instructions like \texttt{MLOAD} and
\texttt{MSTORE}, which within the context of the EVM are very simple
instructions that operate on EVM's internal data structures. These
memory instructions cost only 3 Gas units.

However the storage instructions like \texttt{SLOAD} and
\texttt{SSTORE}, because they deal with persistent state and have to
access the disk or the persistent state within the physical machine of
that Ethereum node, cost much more than the memory instructions:
\texttt{SLOAD} costs 2100 Gas and \texttt{SSTORE} costs 20000 Gas units.

To set a storage slot costs 20000 Gas. To change that storage value from
zero to a non-zero value (and there are optimizations here) costs only
5000 Gas in some of the other situations.

These Gas costs have changed over the duration of the last 5 to 6 years
as Ethereum has evolved. These changes happen to prevent some denial of
service attacks that have also happened in the past.

This can be researched in the documentation by looking at some of the
EIPs that have been created specifically to address the Gas cost of
these instructions in some of the most recent upgrades (like the Berlin
upgrade) to see why these costs Gas costs were changed for some of these
more demanding instructions and the rationale behind it.

These become important because not only they address the optimization
aspect when somebody is deploying a contract (Gas usage becomes
important because it affects the user experience of the user working or
interfacing with these contracts) but from a security perspective (these
Gas costs become important from the denial of service context as well).

The final set of instructions where the Gas costs are really high are
the \texttt{CREATE} instruction (which is probably the most expensive
instruction with a cost of 32000 Gas units; and as you can imagine this
is because create results in a new contract account being created, so a
lot of the data structures within the EVM context are created,
registered, have to be made persistent and so on\ldots) and
\texttt{SELFDESTRUCT} (it costs 5000 Gas units).
