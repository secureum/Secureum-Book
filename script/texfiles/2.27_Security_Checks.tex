\section{Security Checks}\label{security-checks}

\subsection{Zero-address Check}\label{zero-address-check}

\textbf{Also known as ``\emph{the first category of security checks}''.}

Remember that Ethereum addresses are 20 bytes in length and, if those 20
bytes all happen to be zeros (which is referred to as a zero address)
that is treated specially in \texttt{Solidity} contracts and also in the
context of the EVM (because the private key for a zero address is
unknown so, if Ether or tokens are transferred to the zero address, then
it is effectively the same as burning them or not being able to retrieve
them in the future).

Similarly, setting access control roles within the context of smart
contracts to the zero address will also not work because transactions
can't be signed with the private key of the zero address, because nobody
knows the private key.

Therefore zero addresses should be treated with a lot of extra care
within smart contracts, and from a security perspective zero address
checks should be implemented for all address parameters specifically for
those that are user supplied in external or public functions.

\subsection{\texorpdfstring{\texttt{tx.origin}
Check}{tx.origin Check}}\label{tx.origin-check}

\textbf{Also known as ``\emph{the second category of security
checks}''.}

Again, remember that Ethereum has two types of accounts: EOA and
contract accounts. Transactions in Ethereum can only originate from
EOAs, so \texttt{tx.origin} is representative of the EOA address where
the transaction originated from, so in situations where smart contracts
would like to determine if the message sender was a contract or whether
it was an EOA, then checking if the message sender is equal to
\texttt{tx.origin} is an effective way to do it.

There are some nuances in the usage of this, but at a high level this is
a check that you may encounter in smart contracts and has security
implications.

\subsection{Arithmetic Check}\label{arithmetic-check}

\textbf{Also known as ``\emph{the second category of security
checks}''.}

We have talked about the concept of overflows and underflows. Just to
refresh: where arithmetic is used with integers in \texttt{Solidity}, if
the value of that integer variable exceeds the maximum that can be
represented by that integer or goes below the lowest value that can be
represented by that integer type, then it results in what is known as
wrapping, where the value overflows from the maximum integer value of
the type and becomes zero or underflows below the lowest value
representable (which is typically zero) and becomes equal to the maximum
value representable.

This can have big security implications because the values of those
variables (maybe it's representing the balance of that account or
something else within the context of the application logic) wraps around
and becomes either zero or the maximum value, which can totally change
the application logic that is working with them. So such overflows are
underflows of balances or other accounting aspects related to such
variables can and have resulted in critical vulnerabilities.

These checks until \texttt{Solidity\ 0.8.0} had to be implemented by the
developers themselves, either within their own smart contracts or by
using \texttt{OpenZeppelin}'s \texttt{SafeMath} library, which provided
various arithmetic library functions for addition, subtraction,
multiplication, division and so on\ldots{} that implemented these checks
in the library functions.

\texttt{Solidity} recognize this aspect of arithmetic checks and how
they are used all over in most of the smart contracts because nearly
every contract deals with such integers and therefore these checks, as
of version \texttt{0.8.0}, are checked by default for integer types.
Furthermore, they can be overridden by the developer where that those
checks are not necessary.

To sum it up: arithmetic checks are one of the most critical checks that
one would encounter in \texttt{Solidity} contracts, and until version
\texttt{0.8.0} you would see a an extensive use of \texttt{SafeMath}
library from \texttt{OpenZeppelin} for doing so. From \texttt{0.8\ 0}
onwards, these are implemented by default.
