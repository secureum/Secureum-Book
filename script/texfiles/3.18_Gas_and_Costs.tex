\section{Gas and Costs}\label{gas-and-costs}

\subsection{Costly Operations}\label{costly-operations}

Certain operations in \texttt{Solidity} are considered costly or
expensive in terms of the amount of Gas units they use.

If such operations are used inside loops they end up consuming a lot of
Gas which could result in unexpected behavior.

The best example of a costly operation in \texttt{Solidity} is that of
state variable updates. Remember that in \texttt{Solidity} state
variables are stored in the storage area of the EVM. Updates to such
state variables use the \texttt{SSTORE} instruction of the EVM which are
one of the most Gas expensive.

As of the latest upgrade from Berlin, \texttt{SSTORE} costs 20000 Gas
units, if they are a cold store where the state variable is being
updated for the first time in the context of this transaction. Or they
cost 5000 Gas units if it is a warm store, in which case this variable
has already been updated in the context of this transaction.

So either 5000 or 20000 Gas units are consumed every time a state
variable is updated, so as you can imagine, if such updates are done
inside loops, then they could end up consuming a lot of Gas and result
in an OOG error, if the amount of Gas supplied in this transaction is
less than what is required.

The solution here is to use local variables instead of state variables
as much as possible. The reason is because local variables are allocated
in memory, and memory updates using \texttt{MSTORE} only cost 3 Gas
units compared to the 5000 or 20000 that storage updates cost.

So this notion of costly operations being used inside the loops leading
to OOG errors and in the worst case leading to a denial of service (DoS)
can be mitigated by caching and using local variables as much as
possible instead of storage variables.

\subsection{Costly Calls}\label{costly-calls}

Similar to state variable updates, external calls inside loops should
also be used very carefully. The reason is external calls cost 2600 Gas
as of the latest upgrade. This is more of a concern if the index of the
loop is controlled by the user, because in that case the number of
iterations of the loop is also user controlled.

That could result in a denial of service, if one of the calls inside the
loops reverts or if the execution runs Out-of-Gas because the Gas
applied in the transaction wasn't enough.

So the mitigation here is to avoid or reduce a number of external calls
made inside loops and also check that the loop index can't be user
controlled or that it is bounded to a small number of iterations, this
again is in the context of preventing opportunities for denial of
service.

\subsection{Block Gas Limit}\label{block-gas-limit}

Costly operations such as state variable updates and external calls
especially made inside loops are also relevant in the context of the
block Gas Limit.

Remember that Ethereum blocks have a notion of a block Gas Limit which
limits the total amount of Gas units consumed by all the transactions
included in the block to a maximum upper bound. This upper bound until
recently was 15 million Gas units. This has changed significantly in how
it works because of EIP1559, but the notion of a block Gas Limit still
remains.

The reason why this is relevant is because, if expensive operations are
used inside loops where the loop index may be user controlled. Then such
expensive operations may result in an Out-of-Gas error, this Out-of-Gas
could not only come from the amount of Gas units supplied in the
transaction that resulted in all this execution, but it could also arise
because of the Gas consumed by this transaction exceeding the Gas Limit
for this block.

So the mitigation here is again to evaluate the loops and make sure that
a lot of these expensive operations are not used inside the loops and
also to check if the loop index is user controlled, and if it can be
bounded to a small finite number, so that opportunities for denial of
service are prevented.

\subsection{Gas Griefing}\label{gas-griefing}

Gas Griefing is a security concept that becomes interesting in the
context of transaction relayers.

Remember that on Ethereum, users can submit transactions to the smart
contracts on the blockchain or alternatively they can submit what are
known as meta-transactions which are sent to the transaction relayers,
where they do not need to be paid for Gas. The relayers in turn, forward
such transactions to the blockchain with the appropriate amount of Gas.

In this scenario the users typically compensate the relays for the Gas
out of that. In such situations it becomes necessary for the users, to
trust the transaction relayers, to submit those transactions or forward
their transactions with a sufficient amount of Gas, so that their
transactions do not fail.
