\section{Proxy Pitfalls}\label{proxy-pitfalls}

The next set of security pitfalls and best practices that we'll discuss
is related to Proxy-based contracts.

Remember that Proxy-based architectures are used for upgradability and
other aspects desired in smart contract applications. In this Proxy
setup, there is typically a Proxy contract that does a
\texttt{delegatecall} to a logic contract, and because of the
\texttt{delegatecall}, the logic contract gets to implement logic that
executes on the state of the Proxy contract.

Under this specific setup, due to the \texttt{delegatecall} the data,
the logic aspects has specific requirements that need to be met by both
the Proxy as well as the logic contract. These lead to some security
pitfalls and best practices.

\subsection{Initializers}\label{initializers}

In this particular pitfall, initializer functions should not be callable
multiple times: they should be callable only once, and by the authorized
Proxy contract as soon as the logic contract has been deployed.

Remember that under a Proxy setup, the implementation contract can't use
a constructor to initialize its state, because it is working with the
state of the Proxy contract that does a \texttt{delegatecall}. So
instead, implementation contract is expected to declare an initializer
function which does all the required initializations for it. Such
functions need to have an \texttt{external} or \texttt{public}
visibility because they need to be callable from an external contract
(which is the Proxy contract).

The deployment is typically done from a deploy script or from a factory
contract. Preventing multiple invocations is critical because such
invocations could happen from unauthorized contracts (or unauthorized
users). In those cases, they could re-initialize the contract with
values that let them exploit some of the contract functionality.

The best practice here is to use \texttt{OpenZeppelin}'s
\texttt{Initializable} library, which provides an initializer modifier
that can be applied to the initialize function, preventing it from being
called multiple times.

\subsection{State Variables}\label{state-variables}

This is a pitfall related to the previous one discussed. This
specifically applies to initializing state variables in the Proxy-based
setup.

Constructors should not be used in the implementation (or logic
contracts) to initialize its state, but using an initializer function
instead. State variables in the implementation contract, similarly,
should not be initialized in their declarations themselves because such
initializations will not be reflected when the Proxy contract makes a
\texttt{delegatecall} to this implementation.

So instead, the state variables should be initialized within the
initializer function because otherwise they would not be set when the
\texttt{delegatecall} happens.

\subsection{Import Contracts}\label{import-contracts}

The contracts used in a Proxy setup may also derive from other libraries
or other contracts within the project itself, which could be defined in
other files. In this case they are imported to be used in the Proxy
contract.

These imported contracts that the Proxy contracts derive from, should
also adhere to the same rules discussed: the base contracts should also
not use a constructor, they should be using an initializer function. In
addition, such contacts should also not initialize state variables
during declaration.

The best practice is to make sure that the imported contracts also
follow those rules, because if not, the state would be uninitialized and
using that state could result in undefined behavior or potentially even
serious vulnerabilities.

\subsection{\texorpdfstring{\texttt{selfdestruct}}{selfdestruct}}\label{selfdestruct}

If a Data Proxy calls a logic implementation contract that has a
\texttt{selfdestruct} call in it, then that logic contract would end up
getting destroyed and thereafter all calls to that logic contract will
end up delegating calls to an address without any code.

Similarly, the use of \texttt{delegatecall} may also cause issues
because the logic implementation works with the state of the Data Proxy.

The best practice is to avoid entirely the use of \texttt{selfdestruct}
or \texttt{delegatecall}s with Proxy-based contracts.

\subsection{State Variables}\label{state-variables-1}

In Proxy-based contracts, the order layout type and mutability of state
variables declared in the proxy, the corresponding implementation (or
different versions of the implementation), should be preserved exactly
while upgrading. This is to prevent storage layout mismatch errors.
These ones can lead to very critical errors if they are inherited.

The best practice is to make sure that these aspects of state variables
are exactly the same in Proxy-based setups.

\subsection{Function Id}\label{function-id}

Remember that \texttt{Solidity} and EVM have the notion of a function
selector which is the \texttt{keccak256()} hash of the function
signatures.

These selectors are used to determine which contact function is being
called, so at runtime the function dispatcher in the contract byte code
should determine (by looking at the function selector) if one of the
functions in the proxy is being called or if this call needs to be
delegated to the implementation contract.

In Proxy-based setups, a malicious Proxy contract may declare a function
such that its function Id collides (is the same as) with one of the
Proxy Functions. So even though the call was targeting an implementation
function, the malicious proxy, hijacks that call and could lead to the
execution of a function that can cause an exploit.

The best practice here is to pay attention to the proxy, any trust
assumptions related to the proxy and implementation contracts. Also, to
check if the proxy has or can declare a function whose Id might collide
with one of the implementation contract functions.

\subsection{Shadowing}\label{shadowing}

Instead of the Proxy contract trying to hijack calls meant for the
implementation by declaring functions whose Id collides with the
implementation contract functions, they could simply shadow the
functions in the implementation contract.

This means that a Proxy contract can declare functions that have the
same name, the same parameter numbers and types as functions in the
implementation contract. In such a scenario, the function dispatcher
would simply call the proxy contact function instead of forwarding it to
the implementation contract.

This way, a malicious proxy can intercept (or hijack) calls instead of
delegating it to the implementation contract and exploit this aspect to
cause malicious behavior.

The best practice here is again to pay attention to the Proxy contract,
the implementation contract, look at all the functions declared in both
these contracts to see if any of the Proxy Functions are indeed
Shadowing those in the implementation contract. If that is the case,
recognize that such functions will be executed in the context of the
proxy without being forwarded to the implementation contract.
