\section{Security Tools}\label{security-tools}

Having discussed audit techniques at a high level, let's now talk a bit
about the tooling that is used in this space. Smart contract security
tools are critical in assisting both smart contract developers as well
as auditors with detecting potentially exploitable vulnerabilities,
highlighting dangerous programming styles or surfacing common patterns
of misuse.

None of these however replace the need for manual review today to
evaluate contract specific business logic and other complex control
flow, data flow and value flow aspects, so these tools at best
complement manual analysis today.

We can think of tools in the space under different categories such as
tools for testing, test coverage linting, static analysis, symbolic
checking, Fuzzing, formal verification and visualization disassemblers.
Finally, there are also monitoring and incident response tools.

Let's now discuss some of the widely used tools in these categories. We
will only dive into a few of them in some detail and only touch upon the
others.

It is encouraged to explore these tools; installing them (most of them
are open source and freely available), playing around with their options
to understand how they work, how effective they are and how they would
fit within the smart contract auditor toolbox.

\subsection{Slither}\label{slither}

So let's start with Slither which is a static analysis tool from Trail
of Bits and one of the most widely used tools in this space. Slither is
a static analysis framework written in \texttt{python3} for analyzing
smart contracts written in \texttt{Solidity}. It runs a suite of
vulnerability detectors prints visual information about contact details.

Also provides an API to easily write custom analyses. This helps
developers and auditors find vulnerabilities, enhance their code
comprehension and also quickly prototype any custom analysis that they
would like. It implements 75+ detectors in the publicly available free
version

\subsubsection{Features}\label{features}

At a high level, Slither implements vulnerability detectors and contact
information printers. It claims to have a low rate of false positives,
the runtime is typically less than one second per contract and it is
designed to integrate into CI/CD frameworks.

It implements built-in printers that quickly report crucial smart
contract information and also supports a detector API to write custom
analysis in \texttt{python3}. It uses an intermediate representation
known as SlithIR which enables simple and high precision analysis.

\subsubsection{Detectors}\label{detectors}

As mentioned, Slither implements 75+ detectors, each of which detects a
particular type of vulnerability. Slither can run on Truffle, Embark,
Dapp, Etherlime or Hardhat applications, or on a single
\texttt{Solidity} file. By default, Slither runs all its detectors.

To run only selected detectors from within its suite, there is a
\texttt{detect} option to specify the names of detectors to run.
Similarly, to exclude certain detectors, one can use the
\texttt{exclude} option to specify the names of detectors to exclude.
Two specific examples of detectors are \texttt{reentrancy-eth} and
\texttt{unprotected-upgrade}. One can also exclude detectors based on
the severity level associated with them.

So for example, to exclude all those detectors that are classified as
informational or low severity one can use the
\texttt{exclude\ informational} or \texttt{exclude\ low} options. On
this tool, one can list all available detectors using the
\texttt{list\ detectors} option, so it is encouraged to take a look at
this tool, the various options and configurations that it supports.

\subsubsection{Printers}\label{printers}

Besides the detectors, Slither has a concept of printers that allow
printing different types of contract information using the print
options. This helps in contract comprehension and gives us visibility
into a lot of different aspects of the contract that's being analyzed.
The various print options include things like the control flow graph.
the call graph, the contract summary data, dependencies of variables,
summary of the functions, inheritance relationships between contracts,
modifiers, require and assert calls, and storage order of the state
variables.

There are also many other details even from the Slither intermediate
representation. At the EVM level, all these could be very helpful in
quickly understanding the contract structure, getting a snapshot and
zooming in on key aspects that are relevant from a security perspective.

\subsubsection{Upgradability}\label{upgradability}

We've discussed in the security modules about how there are many
security challenges with Proxy-based upgradability and a lot of them
were inspired by checks implemented by Slither along with documentation
from OpenZeppelin on this topic.

Slither has a specific tool called the Slither check upgradeability,
which reviews contracts that use the delegateCall Proxy pattern to
detect potential security issues with upgradability.

These include initialized state variables missing or extra state
variables, different state variable ordering between the Proxy and
implementation contracts or different versions of the implementation
contracts itself. This also includes missing initialize function and
initialize function that is present but that can be called multiple
times because of the missing initializer modifier. Finally, function id
collision and shadowing.

All these upgradeability aspects are conveniently packaged into a
smaller tool which makes it very handy for checking that aspect.

\subsubsection{Code Similarity}\label{code-similarity}

Slither has a code similarity detector which can be used to detect
similar \texttt{Solidity} functions, based on machine learning. It uses
a pre-trained model using Etherscan verified contracts that is generated
from more than 60000 smart contracts and more than 850000 functions.
This can be a useful tool to detect vulnerabilities from code clones
forks or copies.

\subsubsection{Flat}\label{flat}

Slither also has a contract flattening tool which produces a flattened
version of the codebase. It supports three strategies:

\begin{itemize}
\tightlist
\item
  Most derived: for exporting all the most derived contracts.
\item
  One file: helps us export all the contracts in one standalone file.
\item
  Local import: exports every contract in one separate file.
\end{itemize}

This tool handles circular dependency and also supports many compilation
platforms such as Truffle, Hardhat, Etherlime and others.

\subsubsection{Format}\label{format}

Slither also has a formatting tool which automatically generates patches
or fixes for a few of its detectors. Patches are compatible with
\texttt{git}. The detectors supported with this tool are a
\texttt{unused-state}, \texttt{solc-version}, \texttt{pragma},
\texttt{naming-convention}, \texttt{external-function},
\texttt{constable-states} and \texttt{constant-function}.

The patches generated by this tool should be carefully reviewed before
applying them just so that you're comfortable with what those patches
look like and there are no bugs in it.

\subsubsection{ERC Conformance}\label{erc-conformance}

Slither has an ERC conformance tool called Slither check ERC which takes
conformance for various ERC standards such as \texttt{ERC20},
\texttt{ERC721}, \texttt{ERC777}, \texttt{ERC165}, \texttt{ERC223} and
\texttt{ERC1820}, some of which we have discussed in earlier chapters.

Examples of these checks are to see if functions are present, return the
correct type, have \texttt{view} mutability if events are present
emitted and parameters of such events are indexed as per the ERC
specification. This is again handy for consolidating all ERC specific
checks into one single tool.

\subsubsection{Prop}\label{prop}

Finally, Slither also has a property generation tool called the Slither
prop which generates code properties or invariants that can then be used
for testing with unit tests or Echidna. The \texttt{ERC20} scenarios
that can be tested with this tool are things like checking for correct
transfer, the possible functionality or that no one can incorrectly mint
or burn tokens.

\subsubsection{New Detectors}\label{new-detectors}

Besides the various detectors, printers and tools of Slither that we
just discussed, Slither also supports an extensible architecture that
allows one to integrate new detectors into the tool.

The skeleton for such a detector implementation has things like
arguments, help, impact, confidence, link to the wiki for that detector
and a placeholder for the most important part of the detector logic
itself. This extensible architecture can help with creating application
specific detectors and also enables the community to contribute new
detectors to the Slither codebase.

Those are all the Slither features that we're going to cover here and as
we see it is an extensive tool with support for 75+ detectors and
multiple other helpful features as well. For those reasons, it is a
widely referenced and used tool across projects in the space.

\subsection{Manticore}\label{manticore}

Let's now move on to another tool from Trail of Bits called Manticore
which is a symbolic execution tool. This again helps with analysis of
Ethereum smart contracts. Manticore can execute a program with symbolic
inputs and explore all possible states it can reach. It can
automatically produce concrete inputs that result in any desirable
program state, it can detect crashes and other failures in smart
contracts and provide instrumentation capabilities.

Finally, a programmatic interface to its analysis engine via
\texttt{python} API similar to Slyther.

\subsection{Echidna}\label{echidna}

Another tool from Trail of Bits is Echidna, which is a Fuzzing tool and
complements Slither and Manticore. This tool is written in haskell and
it performs grammar based Fuzzing campaigns based on a contracts' ABI to
falsify user defined predicates or even \texttt{Solidity} assertions in
the smart contract code.

\subsubsection{Features}\label{features-1}

Echidna has many notable features:

\begin{itemize}
\tightlist
\item
  it generates inputs tailored to the actual code.
\item
  has an optional corpus collection of predefined campaigns.
\item
  it supports mutations and coverage guidance for deeper bugs.
\item
  it can be powered by the Slither prop tool to extract useful
  information before the Fuzzing campaign.
\item
  it has source code integration to help identify which lines are
  covered after the Fuzzing campaign.
\item
  it has support for multiple user interfaces.
\item
  it has automatic test case minimization for quick triage.
\item
  it has seamless integration into the development workflow.
\end{itemize}

\subsubsection{Usage}\label{usage}

As for Echidna's usage, it is recommended looking up Echidna's
documentation and available tutorials on Trail of Bits' website for such
details.

At a high level, the usage involves three aspects:

\begin{enumerate}
\def\labelenumi{\arabic{enumi}.}
\item
  Executing the test runner where the core Echidna functionality is part
  of an executable called echidnatest that takes a contract and a list
  of invariants as inputs.\\

  For each invariant, it generates random call sequences to the contract
  and checks if the invariant holds, if it can find some way to falsify
  the invariant and it prints the call sequence that does so. These are
  typically referred to as counter examples in this terminology.\\

  If it can't find counter examples, then we have some assurance that
  the contract is safe with respect to that invariant.
\item
  Writing invariants, which are expressed as \texttt{Solidity} functions
  with names that begin with ``\texttt{echidna\_}''. They have no
  arguments and return a boolean.
\item
  Collecting and visualizing coverage after finishing the Fuzzing
  campaign, as Echidna can save the coverage maximizing corpus in a
  special directory which will contain two entries: a directory with
  JSON files that can be replayed by Echidna later, and a plain text
  file that contains a copy of the source code with coverage
  annotations.
\end{enumerate}

\subsection{Eth Security Toolbox}\label{eth-security-toolbox}

Trail of Bits has combined the three tools we just discussed into a
tools package which is a Docker container called Eth Security Toolbox
where they are pre-installed and pre-configured. This makes it very
handy and very easy to start off with using these tools. Besides these
three, it also has Rattle and Ethno tools which we will touch upon
later.

\subsection{Ethersplay}\label{ethersplay}

Ethersplay is a Binary Ninja plugin from Trail of Bits that enables an
EVM disassembler and related analysis tools. For those who aren't aware,
Binary Ninja is a widely used extensible reverse engineering platform
which can disassemble a binary and display it in various ways, so
Ethersplay effectively extends that to work with EVM bytecode: this
takes EVM byte code in raw library format as input and generates a
control flow graph of all functions. It can also be used to display
Manticore's coverage.

\subsection{Pyevmasm}\label{pyevmasm}

Pyevmasm is another security tool from Trail of Bits which provides an
assembler and disassembler library for the EVM. This includes a command
line utility for doing the assembling and disassembling and also
includes a python API for extensibility.

\subsection{Rattle}\label{rattle}

Rattle is another security tool from Trail of Bits. It is an EVM binary
static analysis framework that is designed to work with deployed smart
contracts. It takes EVM byte strings as inputs and uses a flow sensitive
analysis to recover the control flow graph.

In static analysis terminology, flow sensitive refers to an analysis
that considers the control flow of statements. Similarly, there is
context sensitive and path sensitive analysis as well. Rattle further
converts the control flow graph into a single static assignment (SSA)
form with infinite registers and optimizes this SSA by removing stacked
instructions of dups, swaps, pushes and pops (remember that EVM is a
stack based machine and there are typically many such stacked
instructions in the bytecode as operands are pushed onto the stack and
results are popped).

Rattle, by converting the byte code instructions from a stack machine to
SSA form removes more than 60\% of all EVM instructions. Because of
that, it presents a user-friendly interface for analyzing smart contract
bytecode.

For anyone interested in programming language analysis, it is encouraged
to look up these concepts of (SSA and sensitivity analysis).

\subsection{EVM CFG Builder}\label{evm-cfg-builder}

EVM CFG builder is another tool from Trail of Bits that is used to
extract the control flow graph (CFG) from EVM bytecode. It also recovers
function names and their attributes such as \texttt{payable},
\texttt{view}, \texttt{pure}, etc\ldots{} It outputs the CFG to a DOT
file. This EVM CFG builder tool is used by Ethersplay, Manticore and
some other tools from Trail of Bits.

\subsection{Crytic Compile}\label{crytic-compile}

Crytic compile is another tool from Trail of Bits. It is a smart
contract compilation library that is used in the security tools from
Trail of Bits. It supports Truffle, Embark, Etherscan, Brownie, Waffle,
Hardhat and other development environments.

\subsection{Solc-Select}\label{solc-select}

Solc-select is a security helper tool from Trail of Bits. It is a script
that is used to quickly switch between different \texttt{Solidity}
compiler versions. It manages installing and setting different salc
compiler versions using a wrapper around salc which picks the right
version according to what was said via solc-select. The solc binaries
are downloaded from the official \texttt{Solidity} language repository.

This tool is very helpful while analyzing different smart contact
projects because there is often a need to switch between different
\texttt{Solidity} compiler versions depending on which version is being
used by the project that is being analyzed. So this tool is very handy
in such situations and helps us work with other security tools that
depend on the \texttt{Solidity} compiler version.

\subsection{Etheno}\label{etheno}

Etheno is a testing tool referred to as the Ethereum testing Swiss Army
knife, again from Trail of Bits. It's a JSON RPC multiplexer analysis
tool wrapper and test integration tool, all bundled into one for
multiplexing.

It runs a JSON RPC server that can multiplex calls to one or more
Ethereum clients with an API for modifying such JSON RPC calls. It
enables differential testing by sending JSON RPC sequences to multiple
Ethereum clients and further helps with the deployment and interaction
with multiple networks at the same time.

For the analysis tool wrapping part, it provides a JSON RPC client for
advanced analysis tools such as Maticore, which makes it much easier to
work with such tools because there is now no need for custom scripts for
them.

For what it concerns integration with test frameworks such as Ganache
and Truffle, it helps run a local test network with a single command and
enables the use of Truffle migrations to bootstrap Manticore analysis.
So for all these reasons, it is referred to as the Swiss Army knife for
Ethereum testing.

\subsection{MythX}\label{mythx}

Now moving on to tools from ConsenSys Diligence, MythX may be considered
as their flagship tool. MythX is a powerful security analysis service
that finds vulnerabilities in Ethereum smart contact code during the
development lifecycle. It is a paid API based service that uses several
tools in the backend, these include Maru (a static analyzer), Mythril (a
symbolic analyzer) and Harvey (a gray box fuzzer). In combination among
these three tools, MythX implements a total of 46+ detectors. While Maru
and Harvey are closed source as of now, Mythril is open source. We'll
talk more about different aspects of MythX in the forthcoming sections.

\subsubsection{Process}\label{process}

So how does the MythX process work? Remember that MythX is an API based
service, so it does not run locally on the user's machines, but it runs
in the cloud.

So the first step is for the project to submit the code to the MythX
service.The analysis requests are encrypted with TLS, the code one
submits can only be accessed by them and one is expected to submit both
the source code and the compiled byte code of the smart contract for
best results

The second step is to activate the full suite of analysis techniques
behind MythX. The longer it runs, the more security weaknesses it can
detect. This is because the precision of the symbolic checker: the
Fuzzing components of MythX can get better with more iterations.

The third and final step is to receive a detailed analysis report from
the MythX service. This report lists all the weaknesses found in the
submitted code, including the exact location of those issues. The
reports that are generated can only be accessed by the submitter. MythX
here offers three scan modes: quick standard and deep, for differing
levels of analysis depth and provides a user-friendly dashboard for
analyzing the results returned.

\subsubsection{Tools}\label{tools}

Now let's talk about the tools used by the MythX service. When a project
submits their code to the MythX API, it gets analyzed by multiple
microservices in parallel where three tools cooperate to return a more
comprehensive set of results in the execution time decided by the type
of scan chosen.

The first of the three tools is a static analyzer called Maru that
parses the \texttt{Solidity} AST (Abstract Syntax Tree) for the project.

The second tool is a symbolic analyzer called Mythril that detects all
the possible vulnerable states in the contract.

Finally, the third tool is Harvey which is a grey box fuzzer that
detects vulnerable execution paths in the smart contract. Compared to
traditional black box Fuzzing, gray box Fuzzing is guided by coverage
information which is made possible by using program instrumentation to
trace the code coverage reached by each input during Fuzzing.

So these three tools are used in combination by the MythX service to
provide a comprehensive analysis of the vulnerabilities within the smart
contract being analyzed.

\subsubsection{Coverage}\label{coverage}

The coverage that is provided by MythX extends to most of the smart
contract weaknesses found in the smart contract weakness registry (SWC
registry) which we will talk more about in one of the forthcoming
sections. This comprehensive coverage addresses 46+ detectors as of
today.

\subsubsection{Security-as-a-Service}\label{security-as-a-service}

MythX is based on a security-as-a-service (SaaS) platform with the
premise that this approach is better because of three main reasons:

\begin{enumerate}
\def\labelenumi{\arabic{enumi}.}
\tightlist
\item
  With this approach, one can expect higher performance compared to
  running the security tools locally because the compute power in the
  cloud is typically much much higher than what may typically be
  expected at the user's end on a laptop or a desktop.
\item
  We can expect a higher vulnerability coverage with three tools than
  running any single standalone.
\item
  Continuous improvements to security analysis technology with new or
  improved security tests methodologies and tools can be adopted as the
  smart contract security landscape evolves with different types of
  vulnerabilities and exploit vectors emerging as the compiler revisions
  change, new coding patterns emerge, new dependencies start getting
  used, new protocols start getting used and even the Ethereum protocol
  upgrades over time.
\end{enumerate}

For these three reasons the SaaS or API based approach of MythX is
considered as being better than running any one of those tools locally
on the user's end.

\subsubsection{Privacy}\label{privacy}

It's understandable that project teams may have concerns uploading their
smart contract code to a SaaS like MythX, so MythX provides a privacy
guarantee the smart contract code submitted using their sas APIs.

The first one is that the code analysis requests are encrypted with TLS,
and to provide comprehensive reports and improve performance, the MythX
service stores some of the contract data in its database, including
parts of the source code and bytecode, but that data never leaves their
secure server and is not shared with any outside parties. It keeps the
results of the analysis so that it can be retrieved later, but the
reports can be accessed only by the project team: the service enforces
authorized access to such results.

\subsubsection{Performance}\label{performance}

Performance is usually a concern with security tools that perform deep
analysis such as with symbolic checking or Fuzzing because they may
require a lot of compute resources and proportionately longer amounts of
time for running through their analysis to get good coverage and
position.

In this case, MythX can be configured for three types of scans depending
on the time expectation. Quick scans run for five minutes, standard
scans run for 30 minutes while deep scans run for 90 minutes. As you can
imagine, standard scans gives better results than quick scans and deep
scans better than standard ones, so one can customize this the type of
scans according to the development phase and time available.

For example, quick scans can be perhaps run by developers during their
code comments and standard scans can be run at certain project
milestones while deep scans, that take a much longer time, can be run on
the nightly builds.

\subsection{Versions}\label{versions}

MythX comes in different versions, so that it can be accessed via
multiple ways. There is a command line interface version that provides a
unified tool access to MythX, there is MythXjs which is a library to
integrate detects in javascript or typescript projects, there is a
python library called pythex to integrate methods in python projects and
finally, there is a visual studio code extension for MythX that allows a
project to scan smart contracts and view the results directly from the
code editor.

\subsubsection{Pricing}\label{pricing}

As for pricing, MythX has four pricing plans.

\begin{enumerate}
\def\labelenumi{\arabic{enumi}.}
\tightlist
\item
  On-demand pricing plan that costs \$9.99 for three scans and all three
  scan modes are available as part of this plan.
\item
  Development plan that costs \$49 a month. This gives access to quick
  and standard scan modes only and it allows 500 scans a month.
\item
  Professional plan which costs \$249 a month and gives access to all
  scan modes and 10000 scans a month.
\item
  Enterprise pricing plan that allows for custom pricing, where custom
  plans can be decided between a project team and ConsenSys Diligence
  that meets the team's specific needs.
\end{enumerate}

\subsection{Scribble}\label{scribble}

Let's now move on to another tool from ConsenSys Diligence called
Scribble. Scribble is a verification language and a runtime verification
tool that translates high level specifications into \texttt{Solidity}
code. It allows one to annotate a \texttt{Solidity} smart contract with
specific properties. There are four goals with Scribble:

\begin{enumerate}
\def\labelenumi{\arabic{enumi}.}
\tightlist
\item
  Specifications should be easy to understand by developers and smart
  contract security auditors.
\item
  Specifications should be simple to reason about.
\item
  specifications should be efficiently checked using off-the-shelf
  analysis tools.
\item
  A small number of core specification constructs should be sufficient
  to express and reason about more advanced constructs.
\end{enumerate}

So Scribble transforms annotations made within smart contract code using
its specification language into concrete assertions, then with those
instrumented contracts (that are equivalent to the original ones) one
can use other tools from ConsenSys Diligence such as Mythril, Harvey or
MythX to leverage these assertions for performing deeper checks. So
Scribble is a relatively newer tool from ConsenSys Diligence and sounds
very powerful in its capabilities, so it is strongly encouraged to take
a look at the documentation of Scribble to get more insights on the
motivations, the underlying concepts driving the tool and to test it out
and exploring all its capabilities.

\subsection{Fuzzing-as-a-Service}\label{fuzzing-as-a-service}

Fuzzing as a service is a service that has been recently launched by
ConsenSys Diligence where projects can submit their smart contracts
along with embedded inline specifications or properties written using
the Scribble language that we just talked about. These contracts are run
through the Harvey fuzzer which uses the specified properties to
optimize Fuzzing campaigns and any violations from such Fuzzing are
reported back from the servers for the project to fix.

\subsection{Karl}\label{karl}

Karl is another security tool from ConsenSys Diligence, which is used to
monitor the Ethereum blockchain for newly deployed smart contracts that
may be vulnerable in real time. Karl checks for security vulnerabilities
using the Mythril detection engine. This can be an interesting
monitoring tool for detecting vulnerable deployed smart contracts, but
not during security auditing or reviews for projects that have yet to be
launched.

\subsection{Theo}\label{theo}

Another security tool from ConsenSys Diligence that is not specifically
meant for auditing, but interesting nevertheless is Theo. Theo is an
exploitation tool with a Metasploit like interface and provides a python
REPL console from where one can access a long list of interesting
features such as automatic smart contact scanning (which generates a
list of possible exploits), sending transactions to exploit a smart
contract, transaction pool monitoring, Front-running, backlining
transactions and many others.

\subsection{Visual Auditor}\label{visual-auditor}

A tool that could be very handy in the manual analysis phase of smart
contact auditing is the visual auditor. This is a visual studio
extension again from ConsenSys Diligence that provides security aware
syntax and semantic highlighting for \texttt{Solidity} and
\texttt{Vyper} languages.

Examples of things that are highlighted include modifiers, visibility
specifiers, security relevant built-ins (such as a global,
\texttt{tx.origin}, \texttt{msg.data} and so on\ldots), storage access
modifiers (indicating if a variable lives in memory or storage),
developer notes in comments (such as to do's, fix me, hack, etc\ldots),
invocations, operations, constructor, fallback functions, state
variables\ldots{}

It has support for review specific features such as audit annotations
and bookmarks, exploring dependencies and inheritance function signature
hashes. It also supports graph and reporting features such as
interactive call graphs with call flow highlighting diagrams and access
to Surya features, which we'll talk about in the next section.

It also supports code augmentation features where additional information
is displayed when hovering over Ethereum account addresses that allow
one to download the bytecode or open it in the browser, hovering over
assembly instructions to show the signatures and hovering over the state
variables to show their declaration information. So overall the visual
auditor is almost a must have tool while manually reviewing
\texttt{Solidity} or \texttt{Vyper} code during audits.

\subsection{Surya}\label{surya}

Surya is a visualization tool from ConsenSys Diligence that helps
auditors in understanding and visualizing \texttt{Solidity} smart
contracts by providing information about their structure and generating
call graphs and inheritance graphs that can be very useful.

It also supports querying the function call graph in many ways to help
during the manual inspection of contracts. his is integrated with the
visual auditor tool that we discussed in the previous section. Surya
supports several commands such as graph function trace, flatten,
inheritance, dependencies, parts, generating a report in the markdown
format, etc\ldots{}

\subsection{SWC Registry}\label{swc-registry}

It is always helpful to have a registry of unique vulnerabilities, so
that everyone can refer to a single source, keep it updated and use them
in interesting ways. One such effort is the smart contract weakness
classification registry (SWC registry).

This is an implementation of the weakness classification scheme proposed
in EIP1470. It is loosely aligned to the terminologies and structure
used in the common weakness enumeration (CWE) from web2 while being
specific to smart contracts. the goals of this project are three fold:

\begin{enumerate}
\def\labelenumi{\arabic{enumi}.}
\tightlist
\item
  To provide a way to classify security issues in smart contract
  systems.
\item
  To define a common language for describing security issues in smart
  contract systems, architecture design and code.
\item
  To serve as a way to train and improve smart contact security analysis
  tools.
\end{enumerate}

This repository is currently maintained by ConsenSys Diligence and
contains 36 entries as of now.

\subsection{CTFs}\label{ctfs}

Let's now talk about a related concept called capture the flag (or CTF
as it is popularly known as). CTFs are fun and educational challenges
where participants have to hack different dummy smart contracts that
have vulnerabilities in them. They help understand the complexities
around how such vulnerabilities may be exploited in the white.

The popular CTFs in the space of Ethereum smart contracts include
\href{https://capturetheether.com/}{Capture the Ether} which is a set of
20 challenges created by Steve Marks which tests knowledge of Ethereum
concepts of contracts, accounts and math among other things.

Then there is \href{https://ethernaut.openzeppelin.com/}{Ethernaut}
which is a web3 or \texttt{Solidity} based war game from OpenZeppelin
that is played in the Ethereum virtual machine, and each level is a
smart contract that needs to be hacked. The game is completely open
source and all levels are contributions made by players themselves.

Then we have \href{https://www.damnvulnerabledefi.xyz/}{Damn vulnerable
DeFi} which is a set of 15 DeFi related challenges created by
\href{https://twitter.com/tinchoabbate}{Tincho Abbate} security
researcher. Depending on the challenge one should either stop the system
from working, steal as much funds as they can or do some other
unexpected things.

Finally, we have
\href{https://twitter.com/paradigm_ctf?lang=es}{Paradigm CTF} which is
an annual CTF challenge created by Paradigm.

So CTFs can be a fun way to practically test out some of the things that
you've learned in these chapters, so it is encouraged to take a look at
some of these and see how well you do with them.

\subsection{Securify}\label{securify}

Securify is a security scanner developed by ChainSecurity. It's a static
analysis tool for Ethereum smart contracts written in Datalog and
supports 38+ vulnerabilities. We won't go into the details of this tool.

\subsection{VerX}\label{verx}

VerX is a formal verification tool, again from the ChainSecurity, that
can automatically prove temporal safety properties of Ethereum smart
contracts. The verifier is based on a combination of three ideas:

\begin{enumerate}
\def\labelenumi{\arabic{enumi}.}
\tightlist
\item
  Reduction of temporal safety verification to reachability checking.
\item
  A symbolic execution engine used to compute precise symbolic states
  within a transaction.
\item
  the concept of delayed abstraction, which approximates symbolic states
  at the end of transactions into abstract states.
\end{enumerate}

The details of this tool are out of scope over here. For more
information, it is encouraged to look at their website for documentation
and their academic paper for greater details behind the theory of this
tool.

\subsection{Smart Check}\label{smart-check}

Smart check is a security tool from SmartDec. It is another static
analysis tool for discovering vulnerabilities and other code issues in
Ethereum smart contracts written in \texttt{Solidity}. An interesting
implementation aspect here is that it translates \texttt{Solidity}
source code into an \texttt{xml} based intermediate representation, then
checks it against XPath patterns. For context, XPath stands for
\texttt{xml} path language, which uses a path notation for navigating
through the hierarchical structure of an \texttt{xml} document.

\subsection{K-framework}\label{k-framework}

K-framework is a verification framework from RuntimeVerification. It
includes kEVM which is a model of EVM in the K-framework. It is the
first executable specification of the EVM that completely passes the
official EVM test suites and so, could serve as a platform for building
a wide range of verbal analysis tools for EVM. Again we won't go into
any level of details for this framework, but it is encouraged to look at
the documentation to get a better understanding of its capabilities.

\subsection{Certora Prover}\label{certora-prover}

Certora Prover is a formal verification tool from Certora. It checks
that a smart contract satisfies a set of rules written in a language
called CVL (Certora Verification Language). Each rule is checked on all
possible transactions not by explicitly enumerating them of course, but
rather through symbolic techniques.

The prover provides complete path coverage for a set of safety rules
provided by the user. For example, a rule might want to check that a
bounded number of tokens can be minted in an \texttt{ERC20} contract.
The prover either guarantees that such a rule holds on all paths and all
inputs or produces a test input known as a counter example that
demonstrates a violation of this rule.

This problem addressed by Certora prover is going to be undecidable,
which means that there will always be some pathological programs or
rules for which the prover will time out without a definitive answer.

This prover takes as input the smart contract (either the bytecode or
the \texttt{Solidity} source code) along with a set of rules written in
CVL, and then automatically determines whether or not the contract
satisfies all the rules provided using a combination of two fundamental
computer science techniques known as abstract interpretation and
constraint solving.

\subsection{HEVM}\label{hevm}

DappHub's HEVM is an implementation of the EVM made specifically for
unit testing and debugging smart contracts. It can help run unit tests,
property tests and also help interactively debug contracts while showing
the \texttt{Solidity} source code, or also run arbitrary EVM code.

With this we have touched upon the various security tools that you may
come across in this space. There are likely others that we haven't
covered here purely for constraints of time and scope and some like SMT
checker which we have covered in the \texttt{Solidity} chapter earlier.

For all these tools, the best way to understand their capabilities and
specific use cases is to install and experiment with them.

In summary smart content security tools are useful in assisting auditors
while reviewing smart contracts they automate many of the tasks that can
be codified into rules with different levels of coverage correctness and
precision these tools are fast cheap scalable and deterministic compared
to manual analysis however they are also susceptible to false positives
they are therefore especially well suited correctly to detect common
security pitfalls and best practices at disability and EVM levels and
with varying degrees of manual assistance they can also be programmed to
check for application level business logic constraints.
