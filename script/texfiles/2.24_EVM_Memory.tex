\section{Memory}\label{memory}

Remember that the EVM is a stack based architecture. It has calldata,
the volatile memory and the non-volatile storage.

EVM memory has a linear layout, which means that all the memory
locations are stored linearly next to each other, and memory locations
can be addressed at byte level. The EVM instructions that are used to
access memory are \texttt{MLOAD}/\texttt{MSTORE} that operate on the
word size (256 bits) and, if one wants to store a single byte from the
stack to memory, one can use the \texttt{MSTORE8}. All locations in
memory are zero initialized.

\subsection{Reserved Memory and the Free Memory
Pointer}\label{reserved-memory-and-the-free-memory-pointer}

The first two 32 byte slots (from \texttt{0x0} to \texttt{0x40}) are
reserved by \texttt{Solidity} as a scratch space for the hashing
methods.

The third slot (again 32 bytes; from \texttt{0x40} to \texttt{0x60}) is
used for the free memory pointer, so this points to the next byte of
memory within \texttt{Solidity} that is considered as ``\emph{free}'' or
in effect this also indicates the amount of allocated memory currently
within \texttt{Solidity}.

The fourth slot (32 bytes; from \texttt{0x60} to \texttt{0x80}) is
referred to as a zero slot and is used by \texttt{Solidity} as an
initial value for dynamic memory arrays. We'll talk about that shortly.

Therefore, it makes sense that the initial value of the free memory
pointer and \texttt{Solidity} is \texttt{0x80} (the fifth slot) because
the first four 32 byte slots are reserved by \texttt{Solidity}. The free
memory pointer effectively points to memory that is allocatable in the
context of \texttt{Solidity} at any point in time and whenever memory is
allocated by the compiler, it updates the free memory pointer.

These concepts should be familiar, if you have looked at memory
allocation of any other programming languages, these just happen to be
the specific ways in which \texttt{Solidity} handles memory allocation
using the familiar concept of the free memory pointer.

\subsection{Memory Layout}\label{memory-layout}

\texttt{Solidity} places new memory objects at the free memory pointer
and all this memory that is allocated is never freed or deallocated. All
these concepts related to memory layout matter from a security
perspective only: if the developer is manipulating this memory directly
in the assembly language support provided by \texttt{Solidity} because,
if one is using \texttt{Solidity} as a high-level language without using
assembly, then all this is automatically handled by the
\texttt{Solidity} compiler itself.

\subsection{Memory Arrays}\label{memory-arrays}

For memory, every element within arrays in \texttt{Solidity} occupy 32
bytes. This is something we mentioned in the context of the byte array
and how every element occupying 32 bytes wastes a lot of space. Despite
the type of the memory array, this is not true for \texttt{bytes} and
\texttt{string} types.

For multi-dimensional memory arrays, those are pointers to memory
arrays.

For dynamic arrays, it is very similar to the storage even within
memory: these are stored by maintaining the length of the dynamic array
in the first slot of that array in memory followed by the array elements
themselves.

\subsection{Zeroed Memory}\label{zeroed-memory}

With respect to zeroed memory (memory containing zero bytes), there are
no guarantees made by the \texttt{Solidity} compiler that the memory
being allocated has not been used before, so one can't assume that the
memory contents contain zero bytes.

The reason for this is that there is no built-in mechanism to
automatically release or free allocated memory in \texttt{Solidity}. As
you can imagine, this has a security impact because if one is using
memory allocated objects, those are not guaranteed to be zeroed memory.
Then the default values may not be zeros. These again are relevant only
if memory is being manipulated directly in assembly within
\texttt{Solidity}. This should not be much of a concern if one is not
using assembly.
