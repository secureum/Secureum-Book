\section{Address Type}\label{address-type}

It's a type that is specific to \texttt{Solidity} and Ethereum, and it
is critical to security: the \texttt{address} type. To highlight its
importance, we dedicate a section especially to \texttt{address}.

\texttt{address} refers to the underlying Ethereum account address, the
EOA or the contract account. This is different from the addresses that
you might have encountered in other programming languages such as
\texttt{C} and \texttt{C++}, where they refer to variables' memory
address when you're dealing with pointers or references. Here address
signifies something very different: an account address.

The \texttt{address} types are 20 bytes in size, because remember that
is the size of the Ethereum address. They come in two types they can be
\textbf{plain} \texttt{address} types or they can have a
\texttt{payable} \textbf{specifier, and referred to as an
\texttt{address\ payable} type}, where it indicates that this
\texttt{address} type can receive Ether. There are different operators
that operate on address types, such as shown here

\begin{itemize}
\tightlist
\item
  Operators \texttt{==}, \texttt{!=}, \texttt{\textless{}},
  \texttt{\textless{}=}, \texttt{\textgreater{}} and
  \texttt{\textgreater{}=}
\item
  Implicit/Explicit Conversions
\end{itemize}

There are conversions that can be performed on \texttt{address} types.
Some of them are implicit and others are explicit. For converting
\texttt{address\ payable} types to \texttt{address} types, implicit
conversions can be used because it is safe. Whereas the other way
around, where an \texttt{address} type is converted to an
\texttt{address\ payable} type, that should be an explicit conversion
because now this address becomes capable of receiving Ether.

From a security perspective, \texttt{address} types play a critical role
in contracts. These addresses are used in different types of access
control: some may be considered as more privileged than the others in
the context of the contract logic.

Addresses can also hold Ether balances and token balances, so using the
right addresses in the right places and making sure that the correct
access control logic or the balances accounting logic is applied on
them, becomes very critical from a security perspective to make sure
there are no undefined behavior or unintended side effects leading to
security vulnerabilities.

\subsection{Address Members}\label{address-members}

Address types have different members that can give different aspects of
the underlying \texttt{address} types:

\begin{itemize}
\tightlist
\item
  \textbf{\texttt{balance}}: gives (as the name may suggest) the balance
  of that address in wei.
\item
  \textbf{\texttt{code}}: gives (surprise\ldots) the code of that
  address.
\item
  \textbf{\texttt{code\ hash}}: gives the hash of the code.
\end{itemize}

There are also the \texttt{transfer} and \texttt{send} members that
\textbf{are applicable to the} \texttt{address\ payable} \textbf{types}.
In addition, the \texttt{call}, \texttt{delegate\ call} and
\texttt{static\ call} members can be applied on plain \texttt{address}
types. We are going to elaborate on all of these shortly\ldots{}

So as you can imagine, these \texttt{address} members play a huge role
when it comes to the security aspects, because they deal with the
balances, look at the code, the code hash, the reentrancy aspects of
\texttt{send} and \texttt{transfer}, making external calls using
\texttt{call}, \texttt{delegatecall}, and \texttt{staticcall}, which are
critical when it comes to the trustworthiness of the contracts that are
being called at these addresses.

\subsubsection{Transfer \& Send}\label{transfer-send}

They make calls to the addresses that are specified by supplying a
limited gas stipend of only 2300 gas units. This is not adjustable: it
is something hard coded in \texttt{Solidity} to address the category of
reentrancy attacks on addresses. We'll take a look at the reentrancy
aspect later on in the security modules, but it's something to keep in
mind for now.

The \texttt{transfer} member is used for transferring Ether to the
destination address. This transfer triggers the \texttt{receive}
function, or the \texttt{fallback} function of the target contract. From
a security perspective this primitive affects reentrancy attacks: where
the target contract, if it is untrusted, could potentially call back
into the caller contract and lead to undesired behavior that could
affect token balances or other contract logic in in a very critical way.
The 2300 gas assumption is critical when you look at how contracts use
transfer and whether that transaction could fail and revert and lead to
undefined behavior.

Similar to \texttt{transfer}, there is \texttt{send} member
\texttt{Solidity}'s \texttt{address} type, which is somewhat a lower
level counterpart for \texttt{transfer}. It is used for Ether transfers:
it triggers the same receiver \texttt{fallback} functions like
\texttt{transfer}, but \textbf{it does not result in a failure if the
target contract uses more than 2300 gas unlike transfer}. In the case of
\texttt{send}, it does not revert but it just sends back a boolean
return value that indicates a either success (\texttt{true}) or failure
(\texttt{false}).

So if the \texttt{send} primitive is used to transfer value, then from a
security perspective it means that the return value of that
\texttt{send} primitive must be checked by the caller to make sure that
the transfer happened successfully or not, depending on what was
returned. Again, the \texttt{send} primitive affects reentrancy, which
is again why the \texttt{send} primitive was introduced as a mitigation
in \texttt{Solidity}. The return value check that is critical and and
different from its \texttt{transfer} counterpart.

\subsection{External Calls}\label{external-calls}

These are used to make low level calls to their specific address that is
specified. We talked about some of these calls in the context of the
underlying instructions, \texttt{call}, \texttt{delegatecall}, and
\texttt{staticcall} instructions, where we talked about how the callee
account in the case of \texttt{delegatecall} executes with its logic but
with the state of the caller account. Similarly in the case of
\texttt{staticcall} we talked about how the callee contract address can
access the state but cannot modify the state.

These instructions are used to interface with contracts that do not
adhere to the ABI or where the developer wants more direct control over
such calls. They all take single bytes memory parameter, return the
success condition as a boolean and return data in a bytes memory. They
can also use \texttt{abi.*} functions, such as \texttt{encode},
\texttt{codepath}, encode with selector, encoded signature\ldots{} to
encode structured data as part of the arguments.

They can also use gas and value modifiers to specify the amount of gas
and Ether for these calls. The latter is applicable for the
\texttt{call} primitive but not \texttt{delegatecall} or
\texttt{staticcall}.

To summarize: the \texttt{delegatecall} is used where the caller
contract wants to use the logic specified by the callee contract but
with the state and other aspects of the caller contract itself. So while
the code of the given address is used, all other aspects such as
\texttt{storage}, \texttt{balance}, \texttt{message}, \texttt{sender}
are taken from the current caller contract. The purpose of
\texttt{delegatecall} is to enable use cases such as libraries or proxy
upgradability, where the logic code is stored in the callee contract but
that operates on the state of the caller contract.

\texttt{staticcall} is used where we want the called function in the
callee contract to look at or to read the state of the caller contract,
but not modify it in any way.

The use of external calls have different types of security implications,
these are low level calls that should be avoided in most cases unless
absolutely required and there are no alternatives available, because
these are calling out to external contracts that may be untrusted, in
the context of the current applications threat model or trust model.

These external contracts could result in undefined behavior, use more
gas than expected, cause re-entrances to the caller contract and might
also return failures where if the return value is not checked, could
result in undefined behavior as well.

A counter-intuitive aspect of these low-level calls is that \textbf{if
these calls are made to contract accounts that do not exist for some
reason they still return \texttt{true} based on the design of the EVM}.

This can have some serious side effects if the contract logic assumes
that the external call was successful and executed the logic that it
expected it to execute because it got a value of \texttt{true} from
these primitives. The mitigation for this aspect of low level calls is
to check for contract existence before these calls are made, and have
the logic handle it appropriately if they do not exist.

This has resulted in some serious security vulnerabilities being
reported in various high-profile smart contract projects, so something
to be kept in mind when analyzing the security of smart contracts that
use these low level calls.

\subsection{Contract Type}\label{contract-type}

Every contract that's declared is its own type and these contract types
\textbf{can be explicitly converted to and from address types}. That is
what they represent underneath. These contract types do not have any
operators supported, and the only members of these types are external
functions declared in the contract along with any state variables.

Solidity supports some contract related primitives that need to be
understood:

\begin{itemize}
\item
  The keyword \texttt{this} refers to the current contract. Remember:
  \textbf{it can be converted to an} \texttt{address} \textbf{type
  explicitly}.
\item
  The \texttt{selfdestruct()} primitive in \texttt{Solidity}, related to
  the \texttt{SELFDESTRUCT} instruction in the EVM. This primitive is a
  high level wrapper on top of that instruction, it takes in a single
  argument: an \texttt{address} type specifying the recipient.\\

  This recipient will receive all the funds in the contract when it is
  destroyed (the Ether balance in that destroyed contract, when the
  execution ends). There are some specifics of \texttt{selfdestruct}
  that need to be kept in mind from a security perspective for several
  reasons.\\

  One of them is that the recipient \texttt{address} specified in this
  primitive does not execute the \texttt{receive()} function when it is
  triggered.\\

  \emph{Recall that contracts can specify a receive function that gets
  triggered on Ether transfers or under other conditions}.\\

  In the context of the \texttt{selfdestruct()} primitive, the recipient
  \texttt{address} happens to be a contract and it specifies a
  \texttt{receive()} function that does not get triggered when
  \texttt{selfdestruct()} happens.\\

  This is critical because any logic that might be within that
  \texttt{receive()} function might have been anticipated by the
  developer to be triggered anytime ether is received by the contract,
  but \texttt{selfdestruct()} is an exception to that logic.\\

  Also the contract gets destroyed by \texttt{selfdestruct()} only at
  the end of the transaction that has triggered this flow. What this
  means is that if there is any other logic after
  \texttt{selfdestruct()} that may revert for various reasons, then that
  revert undoes the destruction of the contract itself. So just because
  we see a \texttt{selfdestruct()} in the control flow does not mean
  that the contract gets destroyed, because logic after that might
  revert and really not result in the destruction of this contract.
\end{itemize}

Other primitives specific to the contract type supported by
\texttt{Solidity} are beased on the \texttt{type(x)} instruction (where
\texttt{x} is a contract type).

The primitives supported are \texttt{type(C).name} that returns a name
of the contract, \texttt{type(C).creationCode} and
\texttt{type(C).runtimeCode} primitives return the creation and runtime
byte codes of that contract.

These are interesting details that are best examined by writing a simple
contract and looking at what these primitives return.

There's also the interface id primitive (\texttt{type(I).interfaceId})
that returns the identifier for the interface specified. We'll take a
look at the differences between interfaces and contracts later on, but
these are primitives that are supported by \texttt{Solidity} specific to
the contract or interface type.
