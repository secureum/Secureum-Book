\section{Mathematical \& Cryptographic
Functions}\label{mathematical-cryptographic-functions}

Solidity supports the addition and multiplication operations with
modulus: \texttt{addmod()} and \texttt{mulmod()}.

It obviously supports the Keccak-256 hashing function that is
fundamental to Ethereum and used extensively within Ethereum and smart
contracts themselves.

It also supports the standardized SHA-256 algorithm (related to
Keccak-256), but the standardized version, it further supports one of
the older hashing function, the ripe message digest
\texttt{ripemd160(bytes\ memory)} for historical reasons.

Finally it supports what is known as the \texttt{ecrecover} primitive.
This is the elliptic curve recover function that takes in the hash of a
message as an argument along with the signature components, the ECDSA
signature components of \texttt{v}, \texttt{r} and \texttt{s}.
\texttt{ecrecover} takes in these arguments and returns the address (or
recovers the address) associated with the public key from the elliptic
curve signature that is specified in the parameters. This is used in
various smart contracts and it is used for different types of logic
within them.

\subsection{\texorpdfstring{\texttt{ecrecover}
Malleability}{ecrecover Malleability}}\label{ecrecover-malleability}

\texttt{ecrecover} is susceptible to malleability, or in other words
\textbf{non-uniqueness}. In the context of signatures this means that a
valid signature, can be converted into \textbf{a second valid signature
without requiring knowledge of the private key} to generate those
signatures.

This, depending on how signatures are used within the contract logic,
can result in replay attacks, where the second valid signature can be
used by the user or even by the attacker to bypass the contract logic
that is using these signatures.

The reason for this malleability is the math behind how elliptic curve
cryptography works, so the signature components of \texttt{v},
\texttt{r} and \texttt{s}. The \texttt{s} value can either be in the
lower order range or in the higher order range, and \texttt{ecrecover}
does not prevent the \texttt{s} value from being in one of these two
ranges. This is what allows the malleability.

If the smart contract logic using \texttt{ecrecover} requires the
signatures to be unique, then currently the best practice is to use the
ECDSA wrapper from OpenZeppelin, that enforces the \texttt{s} value to
be in the lower range (it forces there to be a single valid signature
for these signature components).
