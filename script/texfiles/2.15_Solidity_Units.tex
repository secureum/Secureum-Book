\section{\texorpdfstring{\texttt{Solidity}
Units}{Solidity Units}}\label{solidity-units}

\subsection{Ether Units}\label{ether-units}

Ether is $18$ decimals, the smallest unit is a wei. There are various
names given for different numbers of weis: $1$ gwei $= 10^9$ wei,
$1$ Ether $= 10^{18}$.

In the case of the \texttt{Solidity} types, a literal number can be
given a suffix of a wei, or a gwei (gigawei) or an Ether. These are used
to specify sub denominations of Ether, as we see here, which are used
when contracts want to manipulate different denominations of Ether in
the context of the logic.

\subsection{Time Units}\label{time-units}

As you can imagine contracts might want to work with different notions
of time for various types of logic that they want to encode.
\texttt{Solidity} supports different suffixes that represent time, and
these can be applied to literal numbers and these suffixes are:
\texttt{seconds}, \texttt{minutes}, \texttt{hours}, \texttt{days} and
\texttt{weeks}.

The base unit for time is \texttt{seconds}, so literally when 1 is used
it is the same as representing \texttt{1\ seconds}. The suffixes cannot
be directly applied onto variables, so if you want to apply time units
to certain variables, then one needs to multiply that variable with that
time unit.

So as an example shown, if we have a \texttt{daysafter} variable and we
wanted to represent the number of days then we have to proceed like
follows

\begin{lstlisting}[language=Solidity,numbers=none]
daysafter * 1 days
\end{lstlisting}

That is the only way how Solidity allows one to use these units with
variables.
