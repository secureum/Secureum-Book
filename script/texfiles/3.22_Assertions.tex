\section{Assertions}\label{assertions}

\subsection{\texorpdfstring{\texttt{assert()}}{assert()}}\label{assert}

This security best practice is related to the use of \texttt{assert}s
within smart contracts. \texttt{assert()} should be used only to check
or verify program invariants within the smart contracts. They should not
be used to make any state changes within their predicates and they
should also not be used to validate any user inputs.

The reason for this is because, if any state changes are made as the
side-effects of the predicates within asserts, then those could be
missed both by developers during maintenance or when they are trying to
do any testing.

They could also be missed by auditors because these state changes are
not expected to happen within a search and similarly, asserts should not
be used to validate user inputs because that should be done using
\texttt{require()} statements.

As a general rule we do not expect to see any failures from asserts
during normal contract functioning and therefore these best practices
become very relevant.

\subsection{\texorpdfstring{\texttt{assert()}
vs.~\texttt{require()}}{assert() vs.~require()}}\label{assert-vs.-require}

This best practice is related to the use of \texttt{assert()} versus
\texttt{require()} and the specific conditions in which they should be
used.

These two aspects are related, but they have different usages.

\texttt{assert}s should be used to check or verify invariants where
these invariants are expected to be held during normal contract
functioning, so we do not expect any of these asserts to fail during the
contract execution and any failures are critical \texttt{panic} errors
that need to be caught and dealt with in a very serious manner.

On the other hand \texttt{require()} is meant to be used for input
validation of arguments that are supplied by users to various public or
external functions where we do expect failures to happen because the
user provided values may be the zero address in some cases or maybe
values that are out-of-range or do not make sense from the smart
contracts perspective.

So this difference is something to be kept in mind, the best practice is
to use \texttt{assert()} or \texttt{require()} appropriately as the
situation demands. This had a more significant impact until
\texttt{Solidity} compiler version \texttt{0.8.0}. Until then,
\texttt{require()} used the \texttt{REVERT} opcode which refunded the
remaining Gas and failure, whereas \texttt{assert} used the
\texttt{INVALID} opcode which consumed all the supplied Gas.

So until that version the usage of \texttt{assert()} or
\texttt{require()} incorrectly, would result in different Gas semantics.
This is because in one situation the remaining Gas would be refunded,
whereas in the other case all of it would be consumed.

So this affected user experience as well, but this has changed since
version \texttt{0.8.0} where both \texttt{require()} and
\texttt{assert()} use the \texttt{REVERT} opcode and refund all the
remaining Gas on failures.
