\section{Variables}\label{variables}

\subsection{Variable Names}\label{variable-names}

There's a security best practice related to variable names. This ties to
the programming style guidelines that we discussed in the
\texttt{Solidity} module.

The names of variables should be as distinct and unique from each other
as possible because if they are very similar (if they differ by only a
few characters or one character), then it could be confusing both to the
developer as well as to the security reviewer.

From a developer's perspective, it could lead to replaced usages where
the developer uses a different variable than what was intended. As you
can imagine, this can have disastrous effects to the functioning of the
smart contract.

So variable naming affects readability it affects maintainability and
auditability of the code. The best practice is to use very distinct
names for the variables meaningful names for the variables, so that
errors are avoided.

\subsection{Uninitialized Variables}\label{uninitialized-variables}

Another security pitfall related to variables is the use of
uninitialized state or local variables. Remember that in
\texttt{Solidity} the default values of uninitialized variables such as
address, \texttt{bool} or \texttt{uint} is \texttt{0}, string is
\texttt{""} and so on.

This results in address variables ending up as the zero address and
boolean variables taking the value of \texttt{false} because 0 is
effectively \texttt{false} (we have talked about the risks from 0
addresses and \texttt{bool}s being \texttt{false} by default will result
in the conditionals taking a different branch than what was intended).

The best practice is to make sure that state and local variables are
initialized with reasonable values, so that errors are avoided from
having the default values being used.

\subsection{Constants}\label{constants}

There is a best practice related to the use of the \texttt{constant}
specifier for state variables in \texttt{Solidity}.

State variables whose values do not need to change for the duration of
the lifetime of the contract can be declared as \texttt{constant}. This
saves Gas because the compiler replaces all the occurrences of such
state variables with the constant value. This effectively means that
reading such state variables no longer requires the expensive
\texttt{SLOAD} instructions.

So the best practice is to identify such state variables whose values do
not need to change over the lifetime of the contract and declare them as
constant. This also has an additional side effect on improving security
because such state variables can no longer be accidentally changed
within the different functions of the contract.

\subsection{Unused State/Local
Variables}\label{unused-statelocal-variables}

Another aspect that needs to be paid attention in the use of variables
inside contracts is that these could be state variables or local
variables. The specific aspect is that if these variables are declared
but are never used within the contract.

It could be indicative of missing logic that is expected to be there
(that uses these variables in certain ways). It may be missing because
the developer forgot to add it or it could simply be indicative of some
optimization opportunity where such variables can actually be removed
and reduce the size of the byte curve, and therefore reduce the amount
of Gas that is used either during deployment or during runtime.

The best practice here is to pay attention to all the variables that are
declared (in functions, in the contract, state variables\ldots), see if
they are used and, if they are never used, then determine if they need
to be removed for optimization, or if there is any logic that is missing
that needs to be added that uses those variables.
