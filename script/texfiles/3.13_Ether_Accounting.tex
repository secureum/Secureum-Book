\section{Ether Accounting}\label{ether-accounting}

\subsection{Contract Balance}\label{contract-balance}

This security pitfall is related to the Ether balance of a smart
contract and how that can change unexpectedly outside the assumptions
made by the developer.

Remember that smart contracts can be created to begin with a specific
Ether balance. Also there are also functions within the smart contract
that can be specified as being payable, which means that they can
receive Ether via message value.

These two ways can be anticipated by the developer to change the Ether
balance of the contract. But there are also other ways in which the
Ether balance of the contract can change.

One such way is the use of \texttt{coinbase} transactions. These are the
beneficiary addresses used in the block headers where the miner
typically specifies the address to which the block rewards and all the
transaction Gas fees should go to. That \texttt{coinbase} address could
point to a specific smart contract where all the rewards, the Gas fees
go to, if that block is successfully included in the blockchain.

The other unexpected way could be via the \texttt{selfdestruct}
primitive where, if a particular smart contract is specified as the
recipient address of \texttt{selfdestruct()}, then upon that executing
the balance of the contract being destructed would be transferred to the
specified recipient contract.

So these two ways the \texttt{coinbase} and \texttt{selfdestruct}
although very unusual and unexpected could in theory change the Ether
balance of any smart contract, this could be well outside the
assumptions made by the developer or the team behind the smart contract.

So what this means is that, if the application logic implemented by a
smart contract makes assumptions on the balance of Ether in this
contract and how that can change, then those assumptions could become
invalid because of these extreme situations in which it can be changed.
So this is something to be paid attention to while analyzing the
security for contract from the perspective of the Ether balance that it
holds.

\subsection{\texorpdfstring{\texttt{fallback}
vs.~\texttt{receive}}{fallback vs.~receive}}\label{fallback-vs.-receive}

This security consideration is related to the use of \texttt{fallback}
and \texttt{receive} functions within a smart contract.

Remember from our discussion in the \texttt{Solidity} modules, there are
differences between these two functions, there are some similarities.
These are related to the visibility, the mutability and \textbf{the way
that Ether transfers are handled by these two different functions}.

So from a security perspective, if these functions are used in a
contract, then one should check that the assumptions are valid and if
not, what are the implications thereof.

\subsection{Locked Ether}\label{locked-ether}

Locked Ether refers to the situation where the contract has an Ether
balance that gets locked because Ether can be sent to that contract via
\texttt{payable} functions, but there's no way for users to withdraw
that Ether from that contract (the contract contains no functionality to
withdraw Ether).

The obvious solutions are to remove the \texttt{payable} attributes from
functions to prevent Ether from being deposited via them or adding
withdrawal capabilities to the smart contract. The simple situations for
this particular pitfall can be easily recognized and fixed. But also,
there could be complex scenarios where the contract can be taken to a
particular state (either accidentally or maliciously) where the Ether or
the token balance of the contract gets locked and can't be withdrawn.
